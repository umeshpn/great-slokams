
\clearpage

\section{ശ്ലോകങ്ങൾ: മലയാളം/മണിപ്രവാളം}
\label{sec:slokams:mal}

\subsection{അ, ആ}

\begin{enumerate}

\begin{slokam}{\VSv}{\PG}{അക്കാലം സഖി, മാഞ്ഞുപോയൊരു} % dummy 1
അക്കാലം സഖി! മാഞ്ഞുപോയൊരു ദിനം മൂക്കുത്തു വെയ്ക്കുമ്പൊള്‍ നീ\\
വക്കാണിച്ചു വലിച്ചെറിഞ്ഞിതൊടുവെൻ നേർക്കങ്ങു കൽച്ചൂതുകള്‍\\
ത്വക്കാഴത്തിൽ മുറിഞ്ഞൊരെന്റെ നിടിലേ മായാതെയുണ്ടിന്നുമാ-\\
ദ്ധിക്കാരം തൊടുവിച്ച പൊൻ തിലകമാം ത്വദ്രാഗമുദ്രാങ്കുരം
\end{slokam} % dummy 2

\Letter{അ}{ത} % dummy 3

\Book{നാൽക്കാലികൾ}. % dummy 4


\begin{slokam}{\VSv}{\KA}{അംഗച്ഛേദനതുല്യമാണു}
അംഗച്ഛേദനതുല്യമാണു ഭവതിക്കേതാദൃശ്യന്മാർ നിജോ-\\
ത്സംഗം വിട്ടു പിരിഞ്ഞുപോവതു, വിധിക്കിമ്മട്ടു കീഴ്പ്പെട്ടിനി,\\
തുംഗപ്രാഭവമാർന്നിടും ത്രിപഥഗാസംഗത്തിനാൽ സ്തുത്യമാം\\
‘വംഗ’ദ്യോവിലുദിച്ചുയർന്ന ‘രവി’യെ സ്നേഹിക്ക വിശ്വംഭരേ!
\end{slokam}

\Letter{അ}{ത}

\Book{പ്രരോദനം}.



\begin{slokam}{\VSr}{\Unk}{അംഗാരാക്ഷായ സാക്ഷാലിനിയ}
അംഗാരാക്ഷായ സാക്ഷാലിനിയ നിയമിനാം ചിത്തതാരിൽക്കളിക്കും\\
ഭൃംഗായാദ്രീന്ദ്രകന്യാഹൃദയകുമുദിനീ ശീതധാമ്നേ നിതാന്തം\\
തുംഗേ കോടീരഭാരേ വിലസിന വിധിമുണ്ഡായ നിത്യം നമസ്തേ\\
ഗംഗാകല്ലോലസംക്ഷാളിതലളിതശശാങ്കായ തെങ്കൈലനാഥ!
\end{slokam}

\Letter{അ}{ത}


\begin{slokam}{\VPv}{\UN}{അടങ്ങണമൊതുങ്ങണം}
"അടങ്ങണമൊതുങ്ങണം, പണികൾ മൊത്തവും ചെയ്യണം,\\
വണങ്ങണമവന്റെ കാലടി, യുയർത്തൊലാ ശബ്ദവും,\\
ഇറങ്ങരുതൊരിക്കലും തനിയെ" -- യീവിധം പെണ്ണിനെ--\\
ത്തടങ്കലിലടച്ചിടൊല്ലിനിയുമുള്ള നാളെങ്കിലും!
\end{slokam}

\Letter{അ}{ഇ}

\begin{slokam}{\VSk}{\KKT}{അടങ്ങാതന്തിയ്ക്കങ്ങ്}
അടങ്ങാതന്തിയ്ക്കങ്ങലർശരരിപുസ്വാമി നടനം\\
തുടങ്ങുമ്പോള്‍ നോക്കിസ്സരസമഥ കൊണ്ടാടുമവനെ\\
മുടങ്ങാതെപ്പോഴും നവരസമൊലിയ്ക്കുന്ന മിഴിയാൽ\\
കൊടുങ്ങല്ലൂരമ്മേ! കുശലമടിയങ്ങള്‍ക്കു തരണേ!
\end{slokam}

\Letter{അ}{മ}

\begin{slokam}{\VMl}{\Ull}{അടിയനിനിയുമുണ്ടാം}
അടിയനിനിയുമുണ്ടാം ജന്മമെന്നാലതെല്ലാ-\\
മടിമുതൽ മുടിയോളം നിന്നിലാകട്ടെ തായേ!\\
അടിമലരിണവേണം താങ്ങുവാൻ മറ്റൊരേട-\\
ത്തടിയുവതു ഞെരുക്കം മുക്തി സിദ്ധിക്കുവോളം.
\end{slokam}

\Letter{അ}{അ}

\Book{ഉമാകേരളം}.


\begin{slokam}{\VSr}{\SVL}{അണ്ണാക്കിൽ തങ്ങി}
"അണ്ണാക്കിൽ തങ്ങി വെണ്ണക്കഷണ,മതലിവാൻ തെല്ലു പാ"ലെന്നു കള്ള-\\
ക്കണ്ണീരോടും യശോദയ്ക്കുടയൊരുടുതുകിൽത്തുമ്പു തൂങ്ങിപ്പിടിച്ചു്‌\\
തിണ്ണം ശാഠ്യം പിടിക്കും കപടമനുജനാം കണ്ണനുണ്ണിക്കെഴും തൃ-\\
ക്കണ്ണിൻ കാരുണ്യപൂരം കവിത പൊഴിയുമെൻ നാക്കു നന്നാക്കിടട്ടെ!
\end{slokam}

\Letter{അ}{ത}

\begin{slokam}{\VSr}{\Unk}{അത്യുഗ്രാടോപമോടും}
അത്യുഗ്രാടോപമോടും പടനടുവിലടുക്കുന്ന ദൈത്യേന്ദ്രദേഹം\\
കുത്തിക്കീറിപിളർന്നിട്ടുടനെ ചുടു കടുംചോര കോരിക്കുടിച്ചും\\
മത്തോടേറ്റം ചിരിച്ചും മുഹുരപി മദമോടട്ടഹാസം പൊഴിച്ചും\\
നൃത്തം വച്ചും കളിച്ചീടിന കൊടിയ കൊടുംകാളിയെക്കൈതൊഴുന്നേൻ!
\end{slokam}

\Letter{അ}{മ}



\begin{slokam}{\VSr}{ചങ്ങനാശ്ശേരി രവിവർമ്മ കോയിത്തമ്പുരാൻ}{അന്തർഭാഗത്തു ചേർത്ത}
അന്തർഭാഗത്തു ചേർത്തമ്മുരമഥനനെ, യമ്മട്ടു നാട്ടാർ രസം വ-\\
ന്നന്തം കൂടാതഞ്ഞും, ജനഹിതമതിനായ്‌ വേലയിൽത്താനിരുന്നും,\\
ചന്തം പൂണ്ടുല്ലസിക്കും മഹിതഗുണ, ഭവാൻ സിന്ധുവിന്നൊപ്പമത്രേ,\\
കിം തു ശ്രീമൻ! ഭവാനിൽ ക്ഷിതിവര! നിലയില്ലായ്കയെന്നുള്ളതില്ല.
\end{slokam}

\Letter{അ}{ച}

\begin{slokam}{\VSv}{\HM}{അന്തിപ്പന്തലിലമ്പിളിയ്ക്കു ഭുവനം}
അന്തിപ്പന്തലിലമ്പിളിക്കു ഭുവനം കൈമാറി, യാദിത്യന-\\
ത്യന്തം ക്ഷീണിതനായ് മറഞ്ഞു, വെയിലിൻ പായച്ചുരുൾക്കെട്ടുമായ്\\
വെന്തങ്ങേറെവലഞ്ഞു ഭൂമി പിടയും നേരം നിലാപാൽക്കുടം\\
സ്വന്തം കൈകളിലേന്തിയിന്ദു പകരും ദൃശ്യം മനോമോഹനം!
\end{slokam}

\Letter{അ}{വ}

\begin{slokam}{\VKm}{\KJ}{അന്നഗാമിനി, നതാംഗി}
അന്നഗാമിനി, നതാംഗി, ഫുല്ലനളിനായതാക്ഷി, നഗരാട്സുതേ!\\
കുന്നിനൊത്ത കുചകുംഭവാഹിനി, കുമാരധാത്രി, കുലദേവതേ !\\
പന്നഗാഭരണപാണിപത്മപരിലാളിതേ! ലളിതശോഭിതേ!\\
മന്നിലിന്നു മമ മാലകന്നു മരുവാൻ കൊതിച്ചു  നതി ചെയ്‌വു തേ
\end{slokam}

\Letter{അ}{പ}

\begin{slokam}{\VSr}{\VKG}{അപ്പീലിക്കണ്ണു ചൂടും}
അപ്പീലിക്കണ്ണു ചൂടും തിരുമുടി, വലജിദ്രത്നസങ്കാശഫാലം,\\
മുപ്പാരെല്ലാം മയക്കും പുരികലത, ദയാലോലമാം നീലനേത്രം,\\
നൽപീതോദ്യദ്ദുകൂലം, മുരളിയുടെ മുഖത്തുമ്മവെയ്ക്കുന്ന വക്ത്രം,\\
ചിൽപാരമ്യപ്രഭാരഞ്ജിതമൃദുഹസിതം, കണ്ണ, ഞാൻ കാണ്മതെന്നോ!
\end{slokam}

\Letter{അ}{ന}

\begin{slokam}{\VKm}{കോഴിക്കോട് ബാലകൃഷ്ണൻ നായർ}{അപ്രജാപതി ഭവത്പദാബ്ജ}
അപ്രജാപതി ഭവത്പദാബ്ജഭവമാം സുസൂക്ഷ്മതരപാംസുവാ- \\
ലിപ്രപഞ്ചമഖിലം ചമച്ചു പരിപൂർണ്ണമായ്; ജനനി! ശൗരിയോ \\
സപ്രയാസമവ താങ്ങിടുന്നു തനതായിരം തലകളാലുമേ;  \\ 
ക്ഷിപ്രമായവ പൊടിച്ചഹോ! ഭസിതമാക്കിടുന്നു ഹരശങ്കരൻ. 
\end{slokam}

\Letter{അ}{സ}

\Book{സൗന്ദര്യലഹരി പരിഭാഷ}.
\OSlRef{തനീയാംസം പാംസും}.

\begin{slokam}{\VSr}{\UN}{അമ്പത്തൊന്നക്ഷരത്താൽ}
അമ്പത്തൊന്നക്ഷരത്താൽഗ്ഗജഹയരഥവൈചിത്ര്യവൃത്തങ്ങളേകും \\
വമ്പോടും, കൃത്യമായാ യതിയെയരിഗളം പോലെ നന്നായ് മുറിച്ചും,\\
അമ്പെയ്യും പോലെ പദ്യാവലി തുരുതുരെ വിട്ടൂക്കിലായ് പ്രാസമൂന്നി-\\
ഡ്ഡംഭം മറ്റോർക്കു ഭേദിച്ചലറിയണകയായക്ഷരശ്ലോകസൈന്യം!
\end{slokam}

\Letter{അ}{അ}

\begin{slokam}{\VSr}{\Mazha}{അമ്പത്തൊന്നക്ഷരാളീ}
അമ്പത്തൊന്നക്ഷരാളീകലിതതനുലതേ! വേദമാകുന്ന ശാഖി-\\
ക്കൊമ്പത്തൻപോടു പൂക്കും കുസുമതതിയിലേന്തുന്ന പൂന്തേൻകുഴമ്പേ!\\
ചെമ്പൊൽത്താർബാണഡംഭപ്രശമനസുകൃതോപാത്തസൗഭാഗ്യലക്ഷ്മീ-\\
സമ്പത്തേ! കുമ്പിടുന്നേൻ കഴലിണ വലയാധീശ്വരീ വിശ്വനാഥേ!
\end{slokam}

\Letter{അ}{ച}

\Book{ഭാഷാനൈഷധം ചമ്പു}.

\begin{slokam}{\VSv}{\HM}{അമ്പത്തൊന്നു മനോഹരാക്ഷര}
അമ്പത്തൊന്നു മനോഹരാക്ഷരദളപ്പൂചൂടി മുഗ്ദ്ധാംഗിയാ-\\
യെമ്പാടും കവനോത്സവദ്യുതിയുമായെത്തുന്ന കാവ്യാംഗനേ!\\
അമ്പേറ്റന്നൊരു പക്ഷി വീണു പിടയും നേരം മഥിക്കും മന-\\
ക്കൊമ്പിൽപ്പാടിയുണർന്ന നിൻ ചരിതമിന്നേകട്ടെയുത്തേജനം!
\end{slokam}

\Letter{അ}{അ}


\begin{slokam}{\VSv}{\Poonthanam}{അമ്പാടിക്കൊരു ഭൂഷണം}
അമ്പാടിക്കൊരു ഭൂഷണം, രിപുസമൂഹത്തിന്നഹോ ഭീഷണം,\\
പൈമ്പാല്‍ വെണ്ണ തയിര്‍ക്കു മോഷണ, മതിക്രൂരാത്മനാം പേഷണം,\\
വന്‍പാപത്തിനു ശോഷണം, വനിതമാര്‍ക്കനന്ദസംപോഷണം,\\
നിന്‍പാദം മതി ഭൂഷണം - ഹരതു മേ മഞ്ജീരസങ്ഘോഷണം
\end{slokam}

\Letter{അ}{പ}

\Book{ശ്രീകൃഷ്ണകർണ്ണാമൃതം}.

\begin{slokam}{\VSr}{\VKG}{അമ്മാമൻ തന്റെ നെഞ്ഞത്ത്}
അമ്മാമൻ തന്റെ നെഞ്ഞത്തമരിലമരവേ പോർമിടുക്കിൻ തിളപ്പാൽ,\\
നിർമ്മായം കാളിയൻ തൻ തലയിൽ വിലസവേ ലാസ്യമേളക്കൊഴുപ്പാൽ,\\
സമ്മോദം ഗോപകന്യാരതികളിൽ വിഹരിച്ചീടവേ കാമവായ്പാൽ,\\
ചെമ്മേ തത്തിപ്പുളച്ചോരിടയനുടെയരക്കെട്ടറുക്കട്ടെ ദുഃഖം!
\end{slokam}

\Letter{അ}{സ}

\begin{slokam}{\VSv}{\VenM}{അയ്യോ, നല്ലൊരരങ്ങണഞ്ഞു}
അയ്യോ നല്ലൊരരങ്ങണഞ്ഞഭിനയത്തിന്നൊത്ത കാലോചിത-\\
ക്കയ്യോരോന്നു തുടർന്നു നല്ല നടനെന്നാപ്പേരു കേൾപ്പാനഹം.\\
പീയൂഷാംശുകലാകലാപദയിതേ, ഭക്തപ്രിയേ, നിൻ കൃപാ-\\
പീയൂഷത്തിനു കുമ്പിടുന്നു ജഗദാലംബേ കുരുംബേശ്വരീ
\end{slokam}

\Letter{അ}{പ}



\begin{slokam}{\VPc}{\UN}{അരിയ്ക്കകത്തു വല്ലതും വരയ്ക്കലല്ല}
അരിയ്ക്കകത്തു വല്ലതും വരയ്ക്കലല്ല, മൂന്നു നാൾ\\
അടച്ചു വെച്ച പുസ്തകം തുറക്കലല്ല, വിദ്യയെ\\
ഗ്രഹി, ച്ചതിന്റെയപ്പുറം മഥിച്ചു നീ ജഗത്തിനായ്\\
തിരിച്ചു നൽക, കിട്ടിടും സരസ്വതീകടാക്ഷവും!
\end{slokam}

\Letter{അ}{ഗ}

\begin{slokam}{\VSv}{\Unk}{അല്ലേ സ്നേഹിതരേ}
അല്ലേ സ്നേഹിതരേ! കുറച്ചുസമയം മറ്റുള്ളതെല്ലാം മറ-\\
ന്നുല്ലാസം ഹൃദയത്തിനേകുവതിനായൊന്നിച്ചിരുന്നങ്ങനെ\\
ചൊല്ലേറും കവിതാരസം നുകരുവാനിച്ഛിച്ചു വന്നെത്തിയോർ-\\
ക്കെല്ലാം നന്ദി,യിടയ്ക്കിയ്ക്കിടയ്ക്കിതു തുടർന്നീടാൻ ശ്രമിച്ചീടണം.
\end{slokam}

\Letter{അ}{ച}


\begin{slokam}{\VSv}{\Unk}{അല്ലോളം തവ മന്ദഹാസനികടേ}
അല്ലോളം തവ മന്ദഹാസനികടേ കോലും നിലാവും കറു;-\\
പ്പല്ലിന്നുണ്ടു നിലാവൊളം വെളുവെളുപ്പുല്ലാസി കേശാന്തികേ;\\
കല്ലോളം കടുതെന്നു തോന്നുമൊരിളം പൂ, മെയ്‌ തൊടുന്നോര്‍ക്കഹോ!;\\
കല്ലും പല്ലവകോമളം തവ മനം ചിന്തിക്കിലേണേക്ഷണേ!
\end{slokam}

\Letter{അ}{ക}

\Book{ലീലാതിലകം}.

\begin{slokam}{\VDv}{\VCBP}{അവനമാ വനമാലി}
അവനമാ വനമാലി നടത്തുമെ-\\
ന്നിഹ മുദാഹമുദാര നിനച്ചു ഞാന്‍\\
ഭുവനപാവന, പാര്‍ക്കുവതെന്നതോര്‍-\\
ക്കണമിതാണമിതാദരമാശ മേ.
\end{slokam}

\Letter{അ}{ഭ}

\Topic{യമകം (ദ്രുതവിളംബിതം, നാലു വരിയിലും)}. \NextSlRef{ഭവനമാ വനമാക്കി}.




\begin{slokam}{\VSr}{ശങ്കുണ്ണിക്കുട്ടൻ}{അശ്വത്ഥത്തിന്നിലയ്ക്കും}
അശ്വത്ഥത്തിന്നിലയ്ക്കും തൊഴുക ശിശുനിലയ്ക്കും നിലയ്ക്കും നിലയ്ക്കും\\
വിശ്വം താനാഹരിയ്ക്കും വ്രജഭുവി വിഹരിക്കും ഹരിക്കും ഹരിക്കും\\
ശശ്വദ്ഭക്തങ്കലാപത്സമയമണികലാപത്കലാപത്കലാപത്‌-\\
പാർശ്വം സ്വഃ പാദപായാസകരശുഭദ! പായാദപായദപായാഃ
\end{slokam}

\Letter{അ}{ശ}

\Topic{അന്ത്യപ്രാസവും യമകവും}.  \PrevSlRef{ഭക്തർക്കിഷ്ടം കൊടുക്കും},
\NextSlRef{ശർമ്മത്തെസ്സൽക്കരിക്കും}


\begin{slokam}{\VSr}{\VNM}{ആകമ്രം മദ്ധ്യമുദ്യന്മണി}
ആകമ്രം മദ്ധ്യമുദ്യന്മണിഗുണനികരം, നീലനീരന്ധ്രമേഘ-\\
ശ്രീ കക്കും വേണി നൽസ്രഗ്ദ്ധര, പരിചിയലും ശ്രോണിയോ പൃഥ്വി തന്നെ,\\
ശ്രീകണ്ഠങ്കൽ പ്രഹർഷിണ്യയി ഭഗവതി, നിൻ ദൃഷ്ടി ഹാ ഹന്ത, ചിത്രം!\\
നൈകച്ഛന്ദോവിശേഷാകൃതിയിലമരുവോളാര്യയാണെങ്കിലും നീ!
\end{slokam}

\Letter{അ}{ശ}

\Book{ഭഗവത്സ്തോത്രമാല}.

\begin{slokam}{\VSv}{പണ്ഡിറ്റ് കെ പി കറുപ്പൻ}{ആചാരപ്പടി പാർത്തു കാൺകിൽ}
ആചാരപ്പടി പാർത്തു കാൺകിലടിയൻ സദ്ധീവരൻ, നല്പെഴു-\\
ന്നാചാന്താഗമവാർദ്ധിമുത്തണികയാ, ലമ്മട്ടിലും ധീവരൻ\\
ആചാര്യത്വവുമുണ്ടു, ചാരുധിഷണാനിസ്തേജിതാമർത്ത്യലോ-\\
കാചാര്യോത്തമ, പിന്നെയെന്തടിയനെക്കാപ്പിയ്ക്കു തോൽപ്പിക്കുവാൻ?
\end{slokam}

\Letter{അ}{അ}

ധീവരജാതിയിൽപ്പെട്ട തന്നെ കാപ്പിയ്ക്കു ക്ഷണിക്കാഞ്ഞതിനെ ചോദ്യം ചെയ്ത് പണ്ഡിറ്റ്
കെ. പി. കറുപ്പൻ കൊച്ചി മഹാരാജാവിന് അയച്ച കത്ത്. 

% Sreelakshmi: ആടീ മൃണ്മയവേദിയിൽ, 

\begin{slokam}{\VSv}{\YK}{ആടീ മൃണ്മയവേദിയിൽ}
ആടീ മൃണ്മയവേദിയിൽ ശ്രുതി പിഴച്ചെന്തൊക്കെയോ വ്യർത്ഥമായ് \\
പാടീ, പാടിയതില്ല പോലിവിടെ ഞാൻ പാടേണ്ട പാട്ടൊന്നുമേ\\
പാടീരം പവമാനനെത്തിരയുമോ, പ്രാണന്റെ കാറ്റേ, ഭവാൻ\\
തേടീലുള്ളിലെ നിത്യസത്പരിമളച്ചെപ്പൊന്നിനെ, ച്ചിത്തിനെ?
\end{slokam}

\Letter{അ}{പ}

\begin{slokam}{\VSr}{\GSK}{ആ ദിക്കിൽ കാത്തു നിൽപ്പൂ ജലധര}
ആ ദിക്കിൽ കാത്തു നിൽക്കൂ ജലധര, പകലേ തന്നെ ചെന്നെത്തി നീ, പോ-\\
യാദിത്യൻ കണ്ണിൽ നിന്നും മറവതു വരെയും ശ്രീമഹാകാളമല്ലോ;\\
ആദിശ്രീശൂലപാണിത്തിരുവടിയുടെ മൂവന്തിവേലയ്ക്കു ഭേരീ-\\
വാദിത്വം ശ്ലാഘ്യമത്രേ, ഫലമവികലമാർന്നാവു നിൻ മന്ദ്രനാദം!
\end{slokam}

\Letter{അ}{അ}


\Book{മേഘസന്ദേശം പരിഭാഷ}.
\OSlRef{അപ്യന്നസ്മിൻ ജലധര}.



\begin{slokam}{\VSv}{\PCM}{ആദ്യം വന്നതു കാലബോധം}
ആദ്യം വന്നതു കാലബോധ, മതിനോടൊപ്പം പദാര്‍ത്ഥാദിസ-\\
മ്പാദ്യം, ഹൃത്തിനു തെല്ലൊരാര്‍ദ്രത, വെളിച്ചത്തോടവിദ്വേഷവും\\
ഉദ്യത്‌പത്രകരോപനീതമുകുളശ്ലോകം നിവേദ്യങ്ങളായ്‌\\
പ്രദ്യോതാര്‍പ്പണമാക്കിടാമിവിടെ നാ, മുദ്യാനവിദ്യാര്‍ത്ഥികള്‍!
\end{slokam}

\Letter{അ}{ഉ}


\begin{slokam}{\VSv}{\TMV}{ആനക്കമ്പമൊരുത്ത, നാനനടയാൾ}
ആനക്കമ്പമൊരുത്ത, നാനനടയാള്‍ക്കമ്പം പര, ന്നീശ്വര-\\
ദ്ധ്യാനക്കമ്പമൊരാള്‍ക്കു, നൽക്കഥകളിക്കമ്പം മുറയ്ക്കന്യനും\\
ഗാനക്കമ്പമതാണു പിന്നെയൊരുവ, ന്നീയുള്ളവന്നക്ഷര-\\
ശ്ലോകക്കമ്പവുമാട്ടെ, യെന്തപകടം? ഭ്രാന്താലയം കേരളം!
\end{slokam}

\Letter{അ}{ഗ}

\begin{slokam}{\VSv}{\KJ}{ആനന്ദം പകരുന്നതുണ്ടൊരുവന്}
ആനന്ദം പകരുന്നതൊണ്ടൊരുവനഗ്ഗാനം, പരന്നാകിലോ \\
നൂനം സാഹിതി, ശില്പമേക, നിതരന്നൂനം പെടാ നർത്തനം, \\
മാനം താൻ ധനമേക, നന്യനതു വിജ്ഞാനം; സ്മരാരേ! തവാ-\\
ധീനം പാർ കളിവീടു തേ; മനുജരന്യൂനം കളിപ്പാട്ടവും. 
\end{slokam}

\Letter{അ}{മ}

\Topic{അഷ്ടപ്രാസം}. 

\begin{slokam}{\VSv}{\VRV}{ആ മൺമെത്തകളാറ്റുനോറ്റ}
ആ മൺമെത്തകളാറ്റുനോറ്റ മധുരസ്വപ്നങ്ങള്‍ തൻ ജീവിത-\\
പ്രേമം പാടിയ സാമഗാനലഹരീഹർഷാഞ്ചിതാത്മാക്കളായ്‌\\
ഹാ! മന്വന്തരഭാവശിൽപികള്‍ നമുക്കെന്നേക്കുമായ്‌ത്തന്നതാ-\\
ണോമൽക്കാർത്തികനെയ്‌വിളക്കെരിയുമീയേകാന്തയാഗാശ്രമം!
\end{slokam}

\Letter{അ}{ഹ}

\Book{സർഗ്ഗസംഗീതം}.


\begin{slokam}{\VSv}{\GRT}{ആ മന്ദാരമരങ്ങളാണു}
ആ മന്ദാരമരങ്ങളാണു തനയന്മാരായ നാലഞ്ചു പേര്‍,\\
പൂമാതാ മക, ളെന്നുമല്ല ഭുവനത്രാതാവു നാരായണന്‍\\
ജാമാതാവുമെടോ നിനക്കു കടലേ, നീ തന്നെ രത്നാകരം,\\
ശ്രീമത്വം കുറവി -- ല്ലലച്ചിലിനിയും തീര്‍ന്നില്ലതാണദ്ഭുതം!
\end{slokam}

\Letter{അ}{ജ}



\begin{slokam}{\VSv}{\UN}{ആരക്ഷീണതപസ്യയാൽ}
ആ, രക്ഷീണതപസ്യയാ, ലഖിലലോകാധീശദത്തം കലാ--\\
സാരം ചിപ്പിയിൽ മുത്തുപോ, ലസുലഭാനന്ദാഭമാക്കുന്നുവോ,\\
ആരാൽ കേരളനാടു മന്നിലഭിമാനാഗാരമാകുന്നുവോ,\\
ആ രാഗാങ്കണരാജപൂജിതമഹാഗന്ധർവ്വ, തേ സ്വാഗതം!
\end{slokam}

\Letter{അ}{അ}

യേശുദാസിനെപ്പറ്റി.

\begin{slokam}{\VSr}{\VNM}{ആരംഗം സർവമാച്ഛാദിതമഹഹ}
ആരംഗം സർവമാച്ഛാദിതമഹഹ, ചിരാൽ കാലമാം ജാലവിദ്യ-\\
ക്കാരൻ തൻ പിഞ്ഛികോച്ചാലന, മുലകിൽ വരുത്തില്ലയെന്തെന്തു മാറ്റം?\\
നേരമ്പോക്കെത്രകണ്ടൂ ഭവതിയിഹ പദം തോറു? മെന്തൊക്കെ മേലിൽ\\
സ്വൈരം കാണും, പുരാണപ്രഥിതനദി നിളാ ദേവി, നിത്യം നമസ്തേ!
\end{slokam}

\Letter{അ}{ന}


\begin{slokam}{\VMk}{\MPN}{ആരാധിപ്പാ, നരുണപദ}
ആരാധിപ്പാ, നരുണപദമാ, രൻപരൊ, ത്താരമാണ്ടോ,-\\
രാരായേണ്ടോരരയൊ, ടടിപെട്ടാരവാർകൊങ്കവായ്പ്പാൽ,\\
ആരാജശ്രീമുഖവു, മലസാ, ക്ഷ്യഭ്രകേശങ്ങളും ചേർ,-\\
ന്നാരാൽക്കാണാ, മബലകളെയ, ങ്ങാലവട്ടങ്ങളോടും.
\end{slokam}

\Letter{അ}{അ}


\Book{ശുകസന്ദേശം പരിഭാഷ}.
\OSlRef{തത്സേവാർത്ഥം തരുണസഹിതാഃ}. 
\Topic{ആദിപ്രാസം}.


\begin{slokam}{\VSv}{\VRV}{ആരണ്യാന്തരഗഹ്വരോദര}
ആരണ്യാന്തരഗഹ്വരോദരതപസ്ഥാനങ്ങളിൽ, സൈന്ധവോ-\\
ദാരശ്യാമമനോഭിരാമപുളിനോപാന്തപ്രദേശങ്ങളിൽ\\
ആ, രന്തർമുഖമിപ്രപഞ്ചപരിണാമോദ്ഭിന്നസർഗക്രിയാ-\\
സാരം തേടിയലഞ്ഞു പ, ണ്ടവരിലെച്ചൈതന്യമെൻ ദർശനം
\end{slokam}

\Letter{അ}{അ}

\Book{സർഗ്ഗസംഗീതം}.

\begin{slokam}{\VSv}{\Unk}{ആരോ ചെയ്തൊരു പുണ്യ}
ആരോ ചെയ്തൊരു പുണ്യകർമ്മഫലമോ, യീ മർത്ത്യജന്മം, ഭവത്- \\
കാരുണ്യാമൃതവർഷമോ തിരുമനസ്സെന്നിൽ കടാക്ഷിച്ചതോ? \\
നേരോതാമിനി, ആശയില്ല പലതും നേടാമെനിക്കെൻ പ്രഭോ!\\
ആരായാലിനിയെന്തഹോ, ശിവപദത്താരായ് ലയിക്കാം സ്വയം!
\end{slokam} 
 
\Letter{അ}{ന}

\begin{slokam}{\VKm}{ഇ. എൻ. വി. നമ്പൂതിരി}{ആറ്റിലെത്തിയെതിരേ}
ആറ്റിലെത്തിയെതിരേ കുളിച്ചു കുറിയിട്ടു കണ്ണെഴുതി വൃത്തിയായ്\\
മാറ്റെടുത്തു ഞൊറിയിട്ടുടുത്തു, മുടി കോതി, ചൂടി ദശപുഷ്പവും,\\
ആറ്റുദർഭ പല പൂക്കൾ കായ്കളിവയൊക്കെയേന്തി, ശശിവാരവും \\
നോറ്റുകൊണ്ടുമ തുടങ്ങിയുഗ്രതപമാപ്പുരാരി വരനായ് വരാൻ.
\end{slokam}

\Letter{അ}{അ}

\end{enumerate}

\subsection{ഇ, ഈ}

\begin{enumerate}

\begin{slokam}{\VSv}{\KKN}{ഇന്നാകുന്നു ശകുന്തളാ}
ഇന്നാകുന്നു ശകുന്തളാഗമനമെന്നോർത്തുള്ളിൽ വല്ലായ്മയു-\\
ണ്ടെന്നായശ്രു തടഞ്ഞടഞ്ഞിതു ഗളം, ചിന്താജഡം കാഴ്ചയും,\\
എന്നേ! കാനനവാസി ഞാനുമിതുപോൽ സ്നേഹാത്തിനാലാർത്തനാം; \\
എന്നാലെങ്ങനെ താങ്ങിടും ഗൃഹി സുതാവിശ്ലേഷദുഃഖം നവം? 
\end{slokam}

\Letter{ഇ}{എ}


\Book{അഭിജ്ഞാനശാകുന്തളം പരിഭാഷ}. 
\OSlRef{യാസ്യത്യദ്യ ശകുന്തളേതി}.


\begin{slokam}{\VSr}{\Unk}{ഇന്നാളോളം നിനക്കായ്.}
ഇന്നാളോളം നിനക്കായുടലുതകിയതാം സാധനത്തെപ്പുറന്തൊ-\\
ണ്ടെന്നാക്കിത്തൊട്ടു കൂടാത്തൊരു നിലയിൽ നിറുത്തീലയോ നീ പുളിങ്ങേ?\\
ഇന്നാ നീ വംശബീജങ്ങളെയുടനുടനെച്ചുഴ്ന്നെടുക്കുന്ന കൈ മേൽ\\
നന്നായ് പറ്റിപ്പിടിച്ചോ പരമതുരലിലിട്ടിന്നി നിന്നെ ചതയ്ക്കും.
\end{slokam}

\Letter{ഇ}{ഇ}

\begin{slokam}{\VSv}{\Ottoor}{ഇന്ദുശ്രീമുഖമിന്ദ്രനീല}
ഇന്ദുശ്രീമുഖമിന്ദ്രനീലരുചിരാകാരം, കരിം കൂന്തലിൻ\\
വൃന്ദം തൂങ്ങിയ നെറ്റി, കുഡ്മളിതമാം നേത്രാബ്ജമീ വേഷമായ്,\\
മന്ദം നിദ്ര തെളിഞ്ഞ നേരമൊരു നാൾ കാലത്തു പൂമെത്ത മേൽ\\
നന്ദശ്രീ കണി കണ്ട പൂർണ്ണപരമാനന്ദത്തെ നണ്ണട്ടെ ഞാൻ!
\end{slokam}

\Letter{ഇ}{മ}


\begin{slokam}{\VSr}{\VNM}{ഇപ്പാരാം നാട്യരംഗ}
ഇപ്പാരാം നാട്യരംഗത്തറയിലിവനിതാ വേഷവും കെട്ടി, നൃത്തം\\
വെപ്പാനെത്തേണ്ടിവന്നൂ ചെറുതൊഴികഴിവില്ലാത്ത നിന്നാജ്ഞമൂലം.\\
തപ്പാതെൻ ചൊല്ലിയാട്ടം ജനരസകരമായിട്ടു തീരേണമെങ്കിൽ\\
തൃപ്പാദം പിൻതുണച്ചീടണമിവിടെ നടസ്വാമിതൻ വാമമെയ്യേ!
\end{slokam}

\Letter{ഇ}{ത}


\begin{slokam}{\VSr}{\RV}{ഇപ്പാഴ്പ്പാതപ്പരപ്പിൽ‌}
ഇപ്പാഴ്പ്പാതപ്പരപ്പിൽ‌പ്പെരുകിന മുരടും മുള്ളു, മേറുന്ന കല്ലും\\
മൽ‌പ്പാദങ്ങൾക്കു ശല്യം പകരരുതതിനായൊപ്പമുണ്ടെങ്കിലും നീ\\
എപ്പോൾ മിന്നുന്ന വെൺകല്പടവുകളണയുന്നെന്റെ കാൽച്ചോട്ടിലെന്നാ-\\
ലപ്പോഴേ നിന്നെയൂരിപ്പുറമെയെറിയുമേ ഞാനുറപ്പായ് ചെരിപ്പേ.
\end{slokam}

\Letter{ഇ}{എ}


\begin{slokam}{\VDv}{\VCBP}{ഇവനിതാ വനിതാ}
ഇവനിതാ വനിതാവിഷയാഗ്രഹ-\\
ക്കടലിലാടലിലാണ്ടുലയുന്നു ഹാ!\\
സുരവിഭോ! രവിഭോജ്ജ്വല, നിൻമിഴി-\\
പ്രകരമേ കരമേലണയിക്കുവാൻ.
\end{slokam}

\Letter{ഇ}{സ}

\Topic{യമകം (ദ്രുതവിളംബിതം, നാലു വരിയിലും)}. \PrevSlRef{വിമലമാമലമാനിനി}


\begin{slokam}{\VMb}{\AUK}{ഇളകാത്ത ഹൃത്തൊടിള}
ഇളകാത്ത ഹൃത്തൊടിളകാത്തവന്റെ വൻ \\
കളവാണിതെന്നു കളവാണി കേള്‍ക്കവേ \\
പരമാർത്ഥമോർത്തു പരമാർത്തചിത്തനാ- \\
യരി കത്തുമുള്ളൊടരികത്തു നിന്നുപോയ്‌.
\end{slokam}

\Letter{ഇ}{സ}

\Topic{യമകം (മഞ്ജുഭാഷിണി, നാലു വരിയിലും)}.

\begin{slokam}{\VSr}{\RV}{ഈടേറും മേടതോറും}
ഈടേറും മേടതോറും ചുടല, ചുടലയിൽ‌പ്പോലുമോലും വെളിച്ചം,\\
വാടാവെട്ടത്തിനുള്ളിൽക്കരി, കരിവിറകിൽ കായ്ച്ചുനിൽക്കുന്ന വൃക്ഷം,\\
കീടത്തിൽ മെയ്യിലും നിൻ തിരുവുട, ലിരുളിൽ പാതയും കാട്ടിടും നിൻ\\
കേടറ്റുള്ളോരു നേത്രാഞ്ജനമകമിഴിയിൽ‌ പൂശണേ ദേവി, മായേ
\end{slokam}

\Letter{ഇ}{ക}

\begin{slokam}{\VSv}{\UN}{ഈയെൻ ജീവിതവീഥി}
ഈയെൻ ജീവിതവീഥി തന്നിലിടറും കാലെന്നു മോഹി, ച്ചുറ--\\
പ്പായെന്നും പരിഹാസമേകി മരുവുന്നോരാ ഖലക്കൂട്ടമേ,\\
തീയെന്നോർത്തു   കരിഞ്ഞു പണ്ടു,  വെയിലിൽ വാടാതെ നിൽക്കാൻ ബലം\\
നീയോരോന്നുമെനിക്കു നൽകി, യതിനാലോതട്ടെ ഞാൻ നന്ദിയെ!
\end{slokam}

\Letter{ഇ}{ത}


\begin{slokam}{\VMk}{\KV}{ഈ ലോകത്തിൽ സുഖമസുഖവും}
ഈ ലോകത്തിൽ സുഖമസുഖവും മിശ്രമായ്ത്താനിരിക്കും\\
മാലോകർക്കും മതിമുഖി! വരാറില്ലയോ മാലനേകം?\\
ആലോചിച്ചീവിധമവിധവേ! ചിത്തമാശ്വസ്തമാകി-\\
ക്കാലോപേതം കദനമതിനിക്കാണി കൂടി ക്ഷമിക്ക.
\end{slokam}

\Letter{ഇ}{അ}

\Book{മയൂരസന്ദേശം}.

\end{enumerate}

\subsection{ഉ, ഊ}

\begin{enumerate}

\begin{slokam}{\VBh}{\PKV}{ഉടുത്തുള്ള പട്ടൊന്നു}
ഉടുത്തുള്ള പട്ടൊന്നു മേൽപോട്ടൊതുക്കി-\\
ത്തിടുക്കെന്നരക്കെട്ടു ധൃഷ്ടം മുറുക്കി\\
മിടുക്കോടിടങ്കൈ മടക്കീട്ടു മുട്ടിൽ-\\
ക്കടുക്കുന്ന കോപത്തൊടാഞ്ഞൊന്നടിച്ചു
\end{slokam}

\Letter{ഉ}{മ}

\begin{slokam}{\VSr}{\Kund}{ഉണ്ടോ നേരത്തുടുക്കും}
ഉണ്ടോ നേരത്തുടുക്കും തളിരൊടമരടിക്കും ചൊടിക്കും, ചൊടിക്കും\\
കൊണ്ടല്ലേറെക്കടുക്കുന്നഴകുമൊരു മിടുക്കും മുടിക്കും മുടിക്കും\\
കണ്ടാലുള്‍ക്കാമ്പിടിക്കുന്നഴലു കിടപിടിക്കും പിടിക്കും പിടിക്കും\\
കൊണ്ടാടേണ്ടും നടയ്ക്കും മുടിയഴിയുമിടയ്ക്കൊന്നടിക്കുന്നടിക്കും.
\end{slokam}

\Letter{ഉ}{ക}

\Topic{അന്ത്യപ്രാസവും യമകവും}.  \NextSlRef{കൊണ്ടൽച്ചായൽക്കറുപ്പും}


\begin{slokam}{\VSv}{\Unk}{ഉണ്ടോ കൂരിരുളിന്നുടപ്പിറവി}
ഉണ്ടോ കൂരിരുളിന്നുടപ്പിറവി പ, ണ്ടാദൗ കലാനായക-\\
ന്നുണ്ടോ സന്തതി, കേളിരട്ടപെറുമോ മാമേരു കാലാന്തരേ?\\
ഉണ്ടാമോ മദഹസ്തിമകസ്തകമിളംതൂണ്മേൽ, നറുംകുപ്പിപോ-\\
ന്നുണ്ടാമോ പുനരാമേലനുപമേ! നീ ചൊല്ലു കൗണോത്തരേ !
\end{slokam}

\Letter{ഉ}{ഉ}

\Book{പദ്യരത്നം}.

\begin{slokam}{\VSv}{\AUK}{ഉണ്ണാനമ്മ വിളിച്ചിടുന്ന}
ഉണ്ണാനമ്മ വിളിച്ചിടുന്ന സമയം മണ്ണാഹരിക്കുന്നതായ്\\
കണ്ണാല്‍ കണ്ടുപിടിക്കെ,യശ്രു നിറയും കണ്ണാല്‍ കടാക്ഷിച്ചുടന്‍\\
അണ്ണാക്കോളമകത്തി വായിലുലകും വിണ്ണാകെയും കാട്ടിടും\\
കണ്ണാ!വിസ്മിതയായി നിന്ന ജനനിക്കെണ്ണാവതോ നിന്‍ പൊരുള്‍? 
\end{slokam}

\Letter{ഉ}{അ}

\Book{സഹസ്രദളം}.

\Topic{അഷ്ടപ്രാസം}.

\begin{slokam}{\VVt}{\VNM}{ഉണ്ണിഗ്ഗണേശ്വരനു പമ്പരമായ്}
 ഉണ്ണിഗ്ഗണേശ്വരനു പമ്പരമായ് ചമഞ്ഞു \\
വിണ്ണില്‍ക്കിടന്നു തിരിയുന്ന മുനിക്കകാണ്ഡേ, \\
മണ്ണിന്‍ ചുവട്ടിലമരും ഭുവനങ്ങള്‍ കൂടി \\
കണ്ണില്‍പ്പതിഞ്ഞു പലവട്ടമിതെന്തു മായം?
\end{slokam}

\Letter{ഉ}{മ}

\Book{ശിഷ്യനും മകനും}.


\begin{slokam}{\VSr}{\KothJ}{ഉണ്ണിത്തൃക്കാലിണയ്ക്കും}
ഉണ്ണിത്തൃക്കാലിണയ്ക്കും, പനിമതികിരണം പോന്നൊളിക്കുന്നൊളിക്കും,\\
വെണ്ണയ്ക്കൊക്കുന്ന മെയ്ക്കും, കനകമണിയരഞ്ഞാൺ തുടയ്ക്കും തുടയ്ക്കും,\\
എണ്ണം തീരാ വണക്കം, തിരുമരിയസുതപ്പൂഞ്ചൊടിക്കും, ചൊടിക്കും\\
കണ്ണിൻ കോണിൽക്കളിക്കും ഭുവനദുരിതമെല്ലാമൊഴിക്കും മൊഴിക്കും.
\end{slokam}

\Letter{ഉ}{എ}

\Topic{അന്ത്യപ്രാസവും യമകവും}.

\begin{slokam}{\VKm}{\PCM}{ഉമ്മ വെച്ചിടണമെങ്കിൽ}
ഉമ്മ വെച്ചിടണമെങ്കിൽ നീ തരിക വെണ്ണ, മാലയിതു ചൂടുവാൻ\\
സമ്മതിപ്പതിനു വെണ്ണ, ഞാൻ മുരളിയൂതുവാനുരുള വേറെയും\\
അമ്മയോടു മണിവർണ്ണനോതിയതറിഞ്ഞു ദേവമുനിസംകുലം\\
ബ്രഹ്മസാധന വെടിഞ്ഞു വല്ലവഴി തേടി വല്ലവികളാകുവാൻ!
\end{slokam}

\Letter{ഉ}{അ}

\begin{slokam}{\VSv}{\UV}{ഊണിന്നാസ്ഥ കുറഞ്ഞു}
ഊണിന്നാസ്ഥ കുറഞ്ഞു, നിദ്ര നിശയിങ്കൽപോലുമില്ലാതെയായ്‌,\\
വേണുന്നോരോടൊരാഭിമുഖ്യമൊരുനേരം നാസ്തി നക്തം ദിവം,\\
കാണും, പോന്നു പുറത്തുനിന്നു കരയും ഭൈമീ - നളന്നന്തികേ\\
താനും പുഷ്കരനും തദീയ വൃഷവും നാലാമതില്ലാരുമേ. 
\end{slokam}

\Letter{ഉ}{ക}

\Book{നളചരിതം ആട്ടക്കഥ - രണ്ടാം ദിവസം}.

\begin{slokam}{\VSv}{മാങ്കുഴി കൃഷ്ണൻ നമ്പൂതിരിപ്പാട്}{ഊണെന്നും പണമെന്നുമേണമിഴിമാർ}
ഊണെന്നും പണമെന്നുമേണമിഴിമാരെന്നും നിനച്ചെപ്പൊഴും\\
കേണയ്യോ കഴിയുന്നു ഹന്ത, പകലും രാവും കൃപാവാരിധേ! \\
പ്രാണൻ പോമളവിൽ കിടന്നഴലുമന്നേരം വൃഷാരൂഢനായ്\\
കാണേണം വൃഷഭാദ്രിനാഥ,   തിരുമെയ് നേരേ പുരാരേ, മമ!
\end{slokam}

\Letter{ഉ}{പ}

\end{enumerate}

\subsection{ഋ, ൠ}

\begin{enumerate}

\begin{slokam}{\VDv}{\Kund}{ഋതുവിലംഗജദീപനമാം}
ഋതുവിലംഗജദീപനമാം സുമ-\\
പ്പുതുമ പോലെയശോകതരുക്കളില്‍\\
സുതളിര്‍ കാതിലതാ പ്രിയ ചേര്‍ത്ത ചാ-\\
രുത വിടാതവിടാര്‍ത്തി വിടര്‍ത്തിടും
\end{slokam}

\Letter{ഋ}{സ}

\Book{രഘുവംശം പരിഭാഷ}. 
\Topic{യമകം (ദ്രുതവിളംബിതം)}. \PrevSlRef{വിവിധനര്‍മ്മഭിരേവം}.

\end{enumerate}

\subsection{എ, ഏ, ഐ}

\begin{enumerate}

\begin{slokam}{\VSv}{\VNM}{എങ്കൽ, പ്രാണപതേ,}
"എങ്കൽ, പ്രാണപതേ, കനിഞ്ഞരുളുകെൻ കുറ്റം പൊറു" -  ത്തെന്നുഷ- \\
സ്സിങ്കൽത്തങ്ങളിൽ നോക്കി നിന്നു സഖിമാരാത്തസ്മിതം കേൾക്കവേ,\\
തങ്കക്കൂട്ടിലിരുന്നു തത്ത തെളിവായ് ചൊല്ലും വിധൗ ലജ്ജയാൽ \\ 
തങ്കൽത്തന്നെ ലയിച്ചു പോയ് തരളയാം തണ്ടാർമിഴിത്തയ്യലാൾ. 
\end{slokam}

\Letter{എ}{ത}

\Book{വിലാസലതിക}


\begin{slokam}{\VSk}{\RV}{എടുത്തിട്ടൂക്കേറും കരമിരുപതാൽ}
എടുത്തിട്ടൂക്കേറും കരമിരുപതാൽ തൻ നിലയനം\\
കിളർത്തിപ്പന്താടും ദശവദനനിൽ പ്രീതി പെരുകി\\
കരുത്തേറും വാളും വരവുമരുളിപ്പോന്നു ചുടല-\\
ക്കളത്തെപ്പുക്കോരാപ്പുരരിപു തരേണം രിപുജയം.
\end{slokam}

\Letter{എ}{ക}

\begin{slokam}{\VSk}{\VenM}{എനിക്കില്ലാ പദ്യാവലിയെഴുതുവാൻ}
"എനിക്കില്ലാ പദ്യാവലിയെഴുതുവാൻ നൈപുണമഹോ!\\
മിനക്കെട്ടാലുണ്ടാകിലുമതു പിഴച്ചീടു, മതിനാൽ\\
കനക്കുന്നാക്ഷേപം കവികളിലുരയ്ക്കാ" മിതി സദാ\\
നിനയ്ക്കുന്നുണ്ടിപ്പോള്‍ ചില വിരുതരീർഷ്യാവസതികള്‍.
\end{slokam}

\Letter{എ}{ക}

\begin{slokam}{\VSv}{\HM}{എന്നാണെത്തിയ,  തെത്രനാളവധി}
"എന്നാണെത്തിയ,  തെത്രനാളവധി?, ഹേ! കുഞ്ഞുങ്ങളും പത്നിയും \\
വന്നില്ലേ? സുഖമല്ലയോ? തനു ലവം ശോഷിച്ചുവോ?, തോന്നലോ?" \\
ഒന്നൊന്നായ് കുശലങ്ങളോതിയിതുപോൽ പണ്ടൊക്കെയെൻ ഗ്രാമമേ! \\
മുന്നിൽത്തേടിയണഞ്ഞ നിൻ പതിവുകൾ വ്യാധിക്കു കീഴ്പ്പെട്ടുവോ?
\end{slokam}

\Letter{എ}{ഒ}


\begin{slokam}{\VSr}{\Balendu}{എന്നായാലും മരിക്കും, വിധിയുടെ}
എന്നായാലും മരിക്കും, വിധിയുടെ വിഹിതം പോലെയല്ലോ നടക്കും,\\
പിന്നീടെങ്ങാന്‍ ജനിക്കാം, മധുഹരകരുണാപാത്രമായെങ്കിലാവാം;\\
ഒന്നേ മോഹിപ്പു - വീണ്ടും ധരണിയില്‍ വരുവാനാണു മേ യോഗമെന്നാ-\\
ലെന്നും നന്ദാത്മജന്‍ തന്‍ പദയുഗമകമേ വാഴണം വാഴുവോളം.
\end{slokam}

\Letter{എ}{ഒ}



\begin{slokam}{\VSr}{\KKT}{എന്നാലും താതനല്ലേ?}
എന്നാലും താതനല്ലേ? പുനരവിടെ നടക്കുന്നതും യാഗമല്ലേ?\\
ചെന്നാലും നിങ്ങളല്ലേ? പരമവിടെ വിശേഷിച്ചു ചെല്ലേണ്ടതല്ലേ?\\
ഇന്നെന്താണീഷ്ടമില്ലേ? തവ തിരുവെഴുനള്ളത്തിനിബ്ഭാവമില്ലേ?\\
നന്നല്ലേ മട്ടു, വല്ലെങ്കിലുമിഹ മമ വാക്കിന്നു സിദ്ധാന്തമല്ലേ?
\end{slokam}

\Letter{എ}{ഇ}

\Topic{അന്ത്യപ്രാസം}.

\begin{slokam}{\VSr}{\VKG}{എന്നിട്ടെല്ലാവരും പോ, യറയിലിരുവരായ്‌}
"എന്നിട്ടെല്ലാവരും പോ, യറയിലിരുവരായ്‌ - ഞാനുമദ്ദേഹവും", "ചൊ-\\
ല്ലെന്നി?", "ട്ടെൻ തോഴിമാരേ, ദയിതനൊടൊരുമിച്ചത്ര ഞാൻ പോയ്‌ക്കിടന്നു",\\
"എന്നി?", "ട്ടെന്നിട്ടു തേങ്ങാക്കുല", "പറയു സഖീ" യെന്നു നിർബ്ബന്ധമായി-\\
ച്ചൊന്നപ്പോളുള്‍ത്രപാബന്ധുരമുഖകമലം പൊത്തിയസ്സാധു കേണാള്‍!
\end{slokam}

\Letter{എ}{എ}

\begin{slokam}{\VSv}{\CKP}{എന്നെപ്പോലുമെനിക്കു നേർവഴി}
എന്നെപ്പോലുമെനിക്കു നേർവഴി നയിക്കാനൊട്ടുമാകാത്ത ഞാ-\\
നന്യന്മാരെ നയിച്ചു നായകപദപ്രാപ്തിക്കു ദാഹിക്കയോ?\\
കന്നത്തത്തിനുമുണ്ടു മന്നിലതിരെന്നോർക്കാതെ തുള്ളുന്നു ഞാ-\\
നെന്നെത്തന്നെ മറന്നു -- കല്ലുകളെറിഞ്ഞെൻ കാലൊടിക്കൂ വിധേ!
\end{slokam}

\Letter{എ}{ക}


\begin{slokam}{\VSv}{\UN}{എന്നോടിന്നിവളെന്തു ചെയ്യും}
  "എന്നോടിന്നിവളെന്തു ചെയ്യു"മിതുമോർത്തേറെക്കടുപ്പിച്ചു ഞാൻ; \\
  "ഒന്നും ചൊല്ലുവതില്ലയെന്തിവ"നിതോർത്തെൻ കാന്തയും കോപമായ്;\\
  കൺകൾ വല്ല വഴിയ്ക്കുമായുഴറവേ, പെട്ടെന്നു   കള്ളം നിറ- \\
  ഞ്ഞെന്നിൽ പുഞ്ചിരി പൊട്ടി, യപ്പൊഴവളോ കേണാൾ, അലിഞ്ഞാനിവൻ!
\end{slokam}
  
\Letter{എ}{ക}

\Book{അമരുകശതകം പരിഭാഷ}. \OSlRef{പശ്യാമോ മയി കിം പ്രപദ്യത}.


\begin{slokam}{\VSr}{\Unk}{എൾപ്പൂ കീഴ്പ്പോട്ടു നില്ക്കിൻറിതു}
എൾപ്പൂ കീഴ്പ്പോട്ടു നില്ക്കിൻറിതു തവ തനിയേ ചാരുനാസാപുടത്തോ-\\
ടൊപ്പം വാരാ; ഞ്ഞൊളിക്കിൻറിതു കിമപി കിളിച്ചുണ്ടുമക്കൂടുതോറും;\\
നല്പാലിന്നും നറുംതേനിനുമിനിയ സുധാവേണുവീണാദികൾക്കും\\
കല്പിക്കും ദീനഭാവം ദിനമനു നിതരാം വാണി കൗണോത്തരേ തേ.
\end{slokam}

\Letter{എ}{ന}

\Book{പദ്യരത്നം}.

\begin{slokam}{\VSv}{\UN}{എല്ലാമീശ്വരനിശ്ചയം, പനി വരാ}
"എല്ലാമീശ്വരനിശ്ചയം, പനി വരാ നല്ലോർക്കു", "രോഗാണുവെ--\\
ന്നില്ലാ ഭൂമിയിലൊന്നു", "മുഷ്ണമതിനെക്കൊല്ലും", "കുടിക്കിഞ്ചിനീർ",\\
"പുല്ലാണിന്നു കൊറോണ" - യെന്നു വെറുതേ ചൊല്ലാതെ കൈ സോപ്പൊടായ്\\
നല്ലോണം കഴുകേണ, മന്യരൊടു കൂട്ടെല്ലാം വെടിഞ്ഞീടണം.
\end{slokam}

\Topic{അഷ്ടപ്രാസം}.

\Letter{എ}{പ}

\begin{slokam}{\VSr}{\VenM}{എല്ലായ്പോഴും കളിപ്പാൻ ചുടല}
എല്ലായ്പോഴും കളിപ്പാൻ ചുടല, വിഷമഹോ ഭക്ഷണത്തിന്നു, മെന്ന-\\
ല്ലുല്ലാസത്തോടു മെയ്യാഭരണമരവമാ, യിങ്ങനേ തീർന്നു കാന്തൻ,\\
ചൊല്ലേറും മക്കളാനത്തലവനൊരു മകൻ, ഷണ്മുഖൻ മറ്റൊരാ, ളി-\\
ന്നെല്ലാം നോക്കുന്ന നേരം തവ \sam{മലമകളേ, ജാതകം ജാതി തന്നെ}!
\end{slokam}

\Letter{എ}{ച}

സമസ്യാപൂരണം. മറ്റു പൂരണങ്ങൾ: \SlRef{പെറ്റോരാ മക്കളെല്ലാമപകടം}, \SlRef{മുപ്പാരും കാക്കുവാനില്ലപരൻ}, \SlRef{മെയ്യിൽപ്പാമ്പുണ്ടനേകം}.


\begin{slokam}{\VVt}{\Ull}{ഏകത്ര കൊങ്കകള്‍, പരത്ര}
ഏകത്ര കൊങ്കകള്‍, പരത്ര നിതംബബിംബം,\\
പാകത്തിലീയവയവങ്ങള്‍ തടിച്ചിടുമ്പോള്‍,\\
ശോകത്തൊടക്കഥ നിനച്ചു ചടച്ചു മദ്ധ്യം;\\
ലോകത്തിലേവനുമസൂയ കൃശത്വമേകും.
\end{slokam}

\Letter{എ}{ശ}


\begin{slokam}{\VMk}{\KV}{ഏടാകൂടം വളരെ}
ഏടാകൂടം വളരെയിടനാടോടെ പോകുന്നതാകിൽ\\
കൂടാ കൂടപ്രകൃതികൾ കുടിയ്ക്കാരിടയ്ക്കേറെയില്ലാ\\
വാടാവല്യാം വിടപികളിലും പാർത്തു പാരാതെ പോകാം\\
വാടാവള്ളിക്കുടിലുകളിലും വിശ്രമിച്ചശ്രമം താൻ
\end{slokam}

\Letter{എ}{വ}

\Book{മയൂരസന്ദേശം}.


\begin{slokam}{\VSv}{\VNM}{ഏണപ്പെണ്മണിയോടടുത്തു}
ഏണപ്പെണ്മണിയോടടുത്തു കണവൻ കൈ തോളിൽ വെച്ചപ്പൊ, "ഴെ- \\
ന്താണങ്ങയ്ക്കധുനാ വിയർപ്പതയി ഞാൻ വീശാം കുറ", ച്ചെന്നുടൻ \\ 
നാണം ഹന്ത! പൊറായ്കയാൽ വിശറിയൊന്നപ്പൊന്മണീകങ്കണ-\\ 
ക്വാണം പൂണ്ട കരത്തിലേന്തി, യതിനാൽ ദീപം കെടുത്താളവൾ. 
\end{slokam}

\Letter{എ}{ന}

\Book{വിലാസലതിക}

\begin{slokam}{\VSv}{\VKG}{ഏതോടക്കുഴലിൻ നിനാദമധുര}
 ഏതോടക്കുഴലിൻ നിനാദമധുരസ്രോതസ്സു ഗോപാംഗനാ-\\
വ്രാതത്തിന്‍ ഭവബന്ധനങ്ങളവസാനിപ്പിച്ചു മാത്രയ്ക്കകം\\
ഏതിന്‍ പൂര്‍ണ്ണരസാനുഭൂതി മുനിമാരാലും സമാസാദ്യമ-\\
ശ്രീതാവും മുകില്‍വര്‍ണ്ണ മുഗ്ദ്ധമുരളീ, നിന്നെബ്ഭജിയ്ക്കുന്നു ഞാന്‍!
\end{slokam}

\Letter{എ}{എ}



\begin{slokam}{\VSr}{\Unk}{ഏറിക്കൊള്ളായിരുന്നൂ പുരഹര}
ഏറിക്കൊള്ളായിരുന്നൂ പുരഹര,സുഖമേ തോല്‍ പൊളിപ്പാന്‍ കനത്തോ-\\
രൂഷത്തം നീയൊരാനത്തലവനെ വെറുതേ കൊന്റതെന്തിന്ദുമൌലേ?\\
ഏറെ പ്രേമോദയംപൂണ്ടഴകിയ തിരുമെയ്യംബികയ്ക്കായ്ക്കൊടുപ്പാ-\\
നാരപ്പോ! ചൊന്നതാലം പെരുകിന ശിവനേ! പോറ്റി ചെല്ലൂര്‍പ്പിരാനേ!
\end{slokam}

\Letter{എ}{എ}

\Book{ചെല്ലൂർനാഥസ്തവം}. 

\end{enumerate}

\subsection{ഒ, ഓ, ഔ}


\begin{enumerate}


\begin{slokam}{\VSv}{\UN}{ഒക്കത്തേറ്റിയൊരമ്മ}
ഒക്കത്തേറ്റിയൊരമ്മ, സൈക്കിൾ കയറാൻ കൂട്ടായ താതൻ, വഴി--\\
യ്ക്കൊപ്പം വന്നവർ, പത്രമാസികകളും ഗ്രന്ഥങ്ങളും, കൂട്ടുകാർ,\\
മക്കൾ, ഭാര്യ, സദാ കണക്കു പറയുന്നാശാരി, യദ്ധ്യാപകർ,\\
വിക്കിപ്പീഡിയ, ഗൂഗി, ളിങ്ങനെ ഗുരുശ്രേഷ്ഠർക്കു  കൈ കൂപ്പിടാം!
\end{slokam}

\Letter{ഒ}{മ}


\begin{slokam}{\VSr}{\Unk}{ഒന്റിന്മേലൂന്റിനാലത്തൊഴിലൊരുവനു}
ഒന്റിന്മേലൂന്റിനാലത്തൊഴിലൊരുവനു മാറ്റീടുവാൻ വേല; വേല-\\
പ്പെണ്ണിൻ പുണ്യൗഘമേ! മന്മനമഗതി വധൂമണ്ഡലേ മഗ്നമല്ലോ;\\
എന്റാലൊന്റുണ്ടു യാചേ തിരുവടിയൊടു ഞാൻ ഉത്തമാം മുക്തിനാരീ-\\
മിന്റേ പൂണായ്വരേണം മമ, തവ കൃപയാ ദേവ! നാവാമുരാരേ!” 
\end{slokam}

\Letter{ഒ}{എ}


\begin{slokam}{\VSv}{\UN}{ഒപ്പം നിന്ന സഹോദരർക്കശനി}
ഒപ്പം നിന്ന സഹോദരർക്കശനിയായ് കെട്ടും മതിൽക്കെട്ടുകൾ,\\
സ്പർദ്ധയ്ക്കായതു നിർത്തിടും വളരെ നാൾ, സ്വാതന്ത്ര്യഘോഷത്തൊടേ\\
മൊത്തം തച്ചു തകർത്തിടും, വിധി വിധിച്ചാഘോഷമായ് പിന്നെയും\\
കെട്ടിപ്പൊക്കിടുമന്യമാം പലതുമീ മർത്യന്റെ നെഞ്ചത്തു താൻ!
\end{slokam}

\Letter{ഒ}{മ}

ബെർലിൻ മതിൽ പൊളിച്ചതിന്റെ (1989-11-09) മുപ്പതാം വാർഷികമായ 2019-11-09-നു്
എഴുതിയതു്.

\begin{slokam}{\VSv}{\UN}{ഒറ്റയ്ക്കല്ല ജനിപ്പതും മരണവും}
ഒറ്റയ്ക്കല്ല ജനിപ്പതും മരണവും ജീവിപ്പതും -- നമ്മളെ---\\
ച്ചുറ്റിപ്പറ്റിയനേകരുണ്ടു കരുണാഗാരങ്ങ, ളീ ജീവിതം\\
തെറ്റിപ്പോയി മുടിഞ്ഞിടാതെ വഴി നേരാക്കിത്തരുന്നോർ, കടം\\
പറ്റിത്താൻ മമ ജന്മമീ നിലയിലായ് -- എല്ലാമറിഞ്ഞിന്നു ഞാൻ!
\end{slokam}

\Letter{ഒ}{ത}

\begin{slokam}{\VSv}{\UN}{ഒറ്റയ്ക്കാണു ജനിപ്പതും മരണവും}
ഒറ്റയ്ക്കാണു ജനിപ്പതും മരണവും,  ജീവിപ്പതും - തൻ മനം \\
തെറ്റിക്കാൻ വഴി നോക്കിടും,  വെറുതെ  "ഞാൻ നീ തന്നെ"യെന്നോതിടും, \\
ഉറ്റോരെന്നു പറഞ്ഞിടുന്ന ജനമോ സ്വാർത്ഥത്തിനായെപ്പൊഴും\\
പറ്റിയ്ക്കും, നിജമേതു? നല്ല വഴിയേ, തൊട്ടും തിരിഞ്ഞില്ല മേ!
\end{slokam}

\Letter{ഒ}{ഉ}


\begin{slokam}{\VSv}{\PG}{ഒറ്റയ്ക്കേ ജനി മർത്ത്യ}
ഒറ്റയ്ക്കേ ജനി മർത്ത്യനാ മൃതിയുമിങ്ങൊറ്റയ്ക്കു തന്നേ, മറി-\\
ച്ചൊറ്റയ്ക്കിത്തിരി നേരമെങ്കിലുമവന്നാവില്ല ജീവിക്കുവാൻ;\\
ഉറ്റസ്നേഹിത, കൂട്ടിനന്യർ ചിലരേ വേണം മരിക്കും വരേയ്-\\
ക്കൊറ്റച്ചക്രരഥത്തിലെത്തിടുകയില്ലെത്തേണ്ടിടത്തേവനും! 
\end{slokam}

\Letter{ഒ}{ഉ}

\Book{നാൽക്കാലികൾ}.


\begin{slokam}{\VSv}{\VNM}{ഓങ്കാരാബ്ജമരന്ദമേ}
ഓങ്കാരാബ്ജമരന്ദമേ, മുനിമനോഭൃംഗവ്രജങ്ങള്‍ക്കു നൽ-\\
പ്പൂങ്കാവേ, പുരുഷാശനപ്പരിഷയാം വേനൽക്കു കാളാഭ്രമേ,\\
തേൻ കാൽ കൂപ്പിന വാണിമാർക്കൊരു മുടിക്കല്ലായ പൂമങ്കയാള്‍\\
താൻ കാമിച്ചു വളർത്ത പുണ്യതരുവിൻ കായേ, വണങ്ങുന്നു ഞാൻ!
\end{slokam}

\Letter{ഒ}{ത}

\begin{slokam}{\VSv}{\VRV}{ഓണക്കോടി ഞൊറിഞ്ഞുടുത്തു}
ഓണക്കോടി ഞൊറിഞ്ഞുടുത്തു കമുകിൻ പൊൻപൂക്കുലച്ചാർത്തുമായ്‌\\
പ്രാണപ്രേയസി കാവ്യകന്യ കവിളത്തൊന്നുമ്മവെച്ചീടവേ\\
വീണക്കമ്പികള്‍ മീട്ടി മാനവ മനോരാജ്യങ്ങളിൽ ച്ചെന്നു ഞാൻ\\
നാണത്തിന്റെ കിളുന്നുകള്‍ക്കു നിറയെപ്പാദസ്വരം നൽകുവാൻ
\end{slokam}

\Letter{ഒ}{വ}


\Book{സർഗ്ഗസംഗീതം}.

\begin{slokam}{\VSv}{\VNM}{ഓമൽച്ചെഞ്ചൊടി മുന്തിരിങ്ങ}
ഓമൽച്ചെഞ്ചൊടി മുന്തിരിങ്ങ, മൃദുവാം കൈത്തണ്ട പൂവൻപഴം,\\
ശ്യാമഭ്രൂ മലരമ്പവി, ല്ലുരസിജം ചെന്തെങ്ങിളന്നീരു താൻ, \\
ഈ മട്ടൊക്കെയുമൊന്നു പോലെ മധുരം നിന്നംഗ, മിന്നെന്തുവാൻ \\
ഹാ, മൈക്കണ്ണി, കഷായമായി പവിഴം പോലേ തവാക്ഷിദ്വയം?
\end{slokam}

\Letter{ഒ}{ഇ}

\Book{വിലാസലതിക}.


\begin{slokam}{\VMk}{\KV}{ഓമൽപിച്ചിച്ചെടിലത}
ഓമൽപിച്ചിച്ചെടിലത മരുല്ലോളിതാ വർഷബിന്ദു-\\
സ്തോമക്ലിന്നാ പുതുമലർ പതുക്കെ സ്ഫുടിപ്പിച്ചിടുമ്പോൾ\\
പ്രേമക്രോധക്ഷുഭിത ഭവതീ ബാഷ്പധാരാവിലാംഗീ\\
ശ്രീമന്മന്ദസ്മിതസുമുഖിയാകുന്നതോർമ്മിച്ചിടുന്നേൻ!
\end{slokam}

\Letter{ഒ}{പ}

\Book{മയൂരസന്ദേശം}.


\begin{slokam}{\VSv}{\VRV}{ഓരോ ജീവകണത്തിനുള്ളിലും}
ഓരോ ജീവകണത്തിനുള്ളിലുമുണർന്നുദ്ദീപ്തമായ്‌, ധർമ്മസം-\\
സ്കാരോപാസനശക്തിയായ്‌, ചിരതപസ്സങ്കൽപ്പസങ്കേതമായ്‌,\\
ഓരോ മാസ്മരലോകമുണ്ടതിലെനിക്കെന്നന്നന്തരാത്മാവിനെ-\\
ത്തേരോടിക്കണമെന്റെകാവ്യകലയെക്കൊണ്ടാകുവോളം വരെ!
\end{slokam}

\Letter{ഒ}{ഒ}

\Book{സർഗ്ഗസംഗീതം}.


\begin{slokam}{\VSv}{\KJ}{ഓരോ ജീവകണത്തിലും സുദൃഢമായ്}
ഓരോ ജീവകണത്തിലും സുദൃഢമായ് വേരോടി നിൽപ്പൂ, ഭവത്-\\
സാരോത്കൃഷ്ടമഹത്വമെന്നു പുകഴും സാരോക്തിയോർത്തീടവേ, \\
ധീരോദാത്തനെവൻ നിനയ്ക്കിലതിനിസ്സാരോപസൃഷ്ടാംഗനാ-\\
രോരോന്നും പ്രതിഭിന്നമ, ല്ലഖിലവും സ്ഫാരോദ്യദാരാദ്ധ്യമാം. 
\end{slokam}

\Letter{ഒ}{ധ}

\Topic{അഷ്ടപ്രാസം}. 


\begin{slokam}{\VSv}{\UN}{ഓരോ വർഷവുമെത്തവേ പുതിയതാം}
ഓരോ വർഷവുമെത്തവേ പുതിയതാം സ്വപ്നങ്ങൾ നെയ്യുന്നു നാം,\\
പാരിൻ നന്മ പെരുത്തു തന്നെ വരുമെന്നോർത്താശ്വസിക്കുന്നു നാം,\\
നേരാകാം, പൊളിയായിടാം, പല വിധം പോകാം, ശരിക്കെന്തിലും\\
നേരേയുള്ള പദങ്ങൾ വെച്ചഴകിൽ മുന്നേറാൻ ശ്രമിച്ചീടണം!
\end{slokam}

\Letter{ഒ}{ന}

\begin{slokam}{\VSv}{\VRV}{ഓരോ സൂക്ഷ്മവുമീയപാരതയിലെ}
ഓരോ സൂക്ഷ്മവുമീയപാരതയിലെ സ്ഥൂലത്തെയുള്‍ക്കൊള്ളുവാന്‍ \\
വേരോടിച്ചുവളര്‍ന്നുവന്ന പരിണാമങ്ങള്‍ക്കു ദൃക്സാക്ഷിയായ്‌, \\
ഈ രോഗാതുരമാം യുദ്ധത്തിനമൃതും കൊണ്ടെത്തുമെന്‍ ചന്ദന- \\
ത്തേരോടും വഴിവിട്ടുമാറുകകലേ മിഥ്യാഭിമാനങ്ങളേ! 
\end{slokam}

\Letter{ഒ}{ഇ}

\Book{അദ്ധ്വാനത്തിൻ വിയർപ്പാണു ഞാൻ}

\begin{slokam}{\VSr}{\UN}{ഓലക്കാൽ പോലെ വെച്ചേൻ}
ഓലക്കാൽ പോലെ വെച്ചേൻ കുറുകെ നെടുകെയും ജോലിയും വീടു, മെന്നി-\\
ട്ടാലംബം വിട്ട വാഴത്തടയൊടു സമമായ് ജീവിതം ഛിന്നമാക്കി,\\
ചാലേ കേളിക്കൊരുങ്ങീ, വ്യഥകളരശു പോൽ നീക്കമോരോന്നിലും, പ-\\
ണ്ടാലോചിച്ചില്ല, യിന്നീയടിയറവൊഴിവാക്കീടുവാൻ മാർഗ്ഗമുണ്ടോ?
\end{slokam}

\Letter{ഒ}{ച}



\end{enumerate}

\subsection{ക}

\begin{enumerate}

\begin{slokam}{\VSv}{\Naduv}{കട്ടിന്മേൽ മൃദുമെത്തയിട്ടതിനു}
കട്ടിന്മേൽ മൃദുമെത്തയിട്ടതിനുമേലേറെഗ്ഗുണം ചേർന്നിടും\\
പട്ടും മറ്റുവിശേഷമുള്ളവകളും നന്നായ്‌ വിരിച്ചങ്ങിനെ\\
ഇഷ്ടം പോലെ കിടന്നുറങ്ങുമവരാപ്പാറപ്പുറത്തേറ്റവും\\
കഷ്ടപ്പെട്ടു കിടന്നതോർത്തധികമായുള്‍ത്താരു കത്തുന്നു മേ.
\end{slokam}

\Letter{ക}{ഇ}



\begin{slokam}{\VSr}{\VKG}{കണ്ടാലെന്താണു, കാലത്ത്}
കണ്ടാലെന്താണു കാലത്തസിതഘനമനോഹാരി തൻ വാക ചാർത്തി-\\
ക്കൊണ്ടുള്ളാ നിൽപ്പു, മന്ത്രാന്വിതസലിലപയഃപൂരകുംഭാഭിഷേകം! \\
കണ്ടാലെന്താണു നൽച്ചന്ദനസുരഭിലമായ് പട്ടണിഞ്ഞാഭ കോലും \\
തണ്ടാർക്കണ്ണന്റെ രൂപം! ഹരിചരണമുപാസിക്കുവാനോ ഞെരുക്കം?
\end{slokam}

\Letter{ക}{ക}

\begin{slokam}{\VSv}{\PG}{കണ്ടിട്ടുള്ള ദിനം മറന്നു}
കണ്ടിട്ടുള്ള ദിനം മറന്നു, കുശലോദന്തങ്ങള്‍ തമ്മിൽച്ചെവി-\\
ക്കൊണ്ടിട്ടിപ്പോളിരുണ്ടുനീണ്ടൊരിരുപന്തീരാണ്ടു തീരാറുമായ്‌\\
ഉണ്ടിന്നും പ്രിയതോഴി മത്‌സ്മരണയിൽ പൊന്നിൻകിനാവായിരം\\
ചെണ്ടിട്ടീടിന രണ്ടിളം കരളുചേർന്നൊന്നായൊരന്നാളുകള്‍.
\end{slokam}

\Letter{ക}{ഉ}

\Book{നാൽക്കാലികൾ}.

\begin{slokam}{\VSr}{\VKG}{കണ്ടോരുണ്ടോ? കഴുത്തിൽ}
കണ്ടോരുണ്ടോ? കഴുത്തിൽ പരിമളതുളസീദാമ, മാ നീലവണ്ടിൻ-\\
തണ്ടാറ്റും മെയ്യു, ചെന്താമരദളനയനം, ഗോപവാടം സ്വഗേഹം\\
തെണ്ടും വൃന്ദാവനത്തിൽ തപനതനയ തൻ കൂലകുഞ്ജാന്തരത്തിൽ,\\
കണ്ടെത്താനായ്‌ സഹായിപ്പവനു മമ നമസ്കാരമാജീവനാന്തം!
\end{slokam}

\Letter{ക}{ത}


\begin{slokam}{\VSr}{\VKG}{കണ്ടോരുണ്ടോ? തപശ്ശാന്തത}
കണ്ടോരുണ്ടോ? തപശ്ശാന്തത നിറയുമകക്കാമ്പിലേക്കെത്തിനോക്കാ-\\
റുണ്ടത്രേ, ഗോപിമാർതൻസ്മരമഥിതമനസ്സിങ്കലും തങ്ങുമത്രേ;\\
ഉണ്ടത്രേ നാമമോരായിര, മുപനിഷദുക്തിക്കെഴും യുക്തിയേയും\\
തിണ്ടാടിപ്പിച്ച മായാവിയെ, യൊരുകുറി കാണിക്കുമോ കാണിനേരം?
\end{slokam}

\Letter{ക}{ഉ}

\begin{slokam}{\VSr}{\VKG}{കണ്ടോരുണ്ടോ? തുറുങ്കിൽ}
കണ്ടോരുണ്ടോ? തുറുങ്കിൽപ്പിറവി, തനിനിറം കണ്ടതില്ലാരു, മാടി-\\
ക്കൊണ്ടൽക്കാന്തിപ്പകിട്ടു, ണ്ടിടയരുടെ നടുക്കാണു കൗമാരകാലം,\\
തെണ്ടും മാടിന്റെ പിന്നിൽ, പകലിരവു കവർന്നുണ്ണു, മെന്നാലുമുള്ളിൽ-\\
ക്കണ്ടാലാനന്ദമേകും രസികനെയൊരുനോക്കെങ്കിലും കാട്ടിടാമോ?
\end{slokam}

\Letter{ക}{ത}

\begin{slokam}{\VSr}{\VKG}{കണ്ടോരുണ്ടോ? വ്രജത്തിൻ}
കണ്ടോരുണ്ടോ? വ്രജത്തിൻ വ്രതസുകൃതഫലക്കാമ്പിനെ, പ്പാമ്പിനെ, ക്കാ-\\
ളിന്ദിത്തണ്ണീരു നഞ്ഞാക്കിയ കുടിലനെയോടിച്ചൊരെൻ തമ്പുരാനെ?\\
കണ്ടോരുണ്ടോ തകർക്കും പെരുമഴ തടയാൻ കുന്നിനെപ്പൊക്കിനിർത്തി-\\
ത്തണ്ടറ്റുള്ളണ്ടർകോൻ തൻ മിഴികളിൽ മഴപെയ്യിച്ച കാളാംബുദത്തെ?
\end{slokam}

\Letter{ക}{ക}

\begin{slokam}{\VSr}{\AUK}{കണ്ണാ കണ്ണാകണം നീ}
കണ്ണാ കണ്ണാകണം നീ, കലിതകലിതമസ്സിൻ നടുക്കും നടുക്കും \\
മാറാ മാറാടിടും ദുഷ്കൃതി തകൃതി തകർക്കേ തിരിയ്ക്കാതിരിയ്ക്ക \\
പോരാ പോരാടുവാൻ ഞാൻ, വിനതവിന തവാർദ്രേക്ഷണത്താൽ ക്ഷണത്താൽ \\
മാറ്റും മാറ്റുള്ളതാമാ മമത മമ തരം നൽക കാലേക കാലേ 
\end{slokam}

\Letter{ക}{പ}

\Topic{യമകം (സ്രഗ്ദ്ധര, 12 എണ്ണം)}.


\begin{slokam}{\VSv}{വെളുത്തേരി കേശവൻ വൈദ്യൻ}{കണ്ണോടൊത്ത കറുത്ത താമരയിതാ}
 കണ്ണോടൊത്ത കറുത്ത താമരയിതാ തണ്ണീരിൽ മുങ്ങി പ്രിയേ!\\
 തുണ്ഡത്തോടെതിരായ വെണ്മതി ഘനേ തിണ്ണെന്നു മങ്ങീടിനാൻ\\
 നിന്നോടൊത്തു നടക്കുമന്നനിരയും ചെന്നെത്തിനാർ മാനസേ\\
 പൊന്നേ! നിന്നൊടു തുല്യവസ്തുവപി മേ ദൈവം പൊറുത്തീലഹോ!
\end{slokam}

\Letter{ക}{ന}.

പരിഭാഷ.  \OSlRef{യത്ത്വന്നേത്രസമാനകാന്തി}.

\begin{slokam}{\VMl}{\UN}{കതിരവനെ വിഴുങ്ങും}
കതിരവനെ വിഴുങ്ങും സർപ്പമ, ല്ലന്ധകാരം \\
മതിയുടെ നിഴലാണെന്നോതി ശാസ്ത്രം; മനുഷ്യർ \\
ഗതി ഗുണമിയലാനായ് രാഹുകാലം ഗണിപ്പൂ; \\
\sam{ക്ഷിതിയിലധികമാവുന്നന്ധവിശ്വാസമിന്നും}!
\end{slokam}

\Letter{ക}{ഗ}


സമസ്യാപൂരണം. 


\begin{slokam}{\VSv}{\KA}{കഷ്ടം, സ്ഥാനവലിപ്പമോ}
കഷ്ടം സ്ഥാനവലിപ്പമോ പ്രഭുതയോ സജ്ജാതിയോ വംശമോ\\
ദൃഷ്ടശ്രീ തനുധാടിയോ ചെറുതുമിങ്ങോരില്ല ഘോരാനലൻ\\
സ്പഷ്ടം മാനുഷഗർവ്വമൊക്കെയിവിടെപ്പുക്കസ്തമിക്കുന്നിത-\\
ങ്ങിഷ്ടന്മാർ പിരിയുന്നു, ഹാ! ഇവിടമാണദ്ധ്യാത്മവിദ്യാലയം!
\end{slokam}

\Letter{ക}{സ}

\Book{പ്രരോദനം}.

\begin{slokam}{\VSk}{\KND}{കഴിഞ്ഞേ പോകുന്നൂ}
കഴിഞ്ഞേ പോകുന്നൂ പകലുമിരവും ജർജ്ജരിതമായ്‌\\
കൊഴിഞ്ഞേ വീഴുന്നൂ നിറമുടയൊരെൻ പീലികള്‍ വൃഥാ\\
ഒഴിഞ്ഞേ കാണുന്നൂ ദിനമനു, നഭസ്സീ, മയിലിനൊ-\\
ന്നഴിഞ്ഞാടാനെന്താണൊരു വഴി? വരൂ നീലമുകിലേ!
\end{slokam}

\Letter{ക}{ഒ}

\begin{slokam}{\VSv}{\VRV}{കാടത്തത്തെ മനസ്സിലിട്ടു}
കാടത്തത്തെ മനസ്സിലിട്ടു കവിയായ്‌ മാറ്റുന്ന വല്‌മീകമു-\\
ണ്ടോടപ്പുൽക്കുഴലിന്റെ ഗീതയെഴുതിസ്സൂക്ഷിച്ച പൊന്നോലയും\\
കോടക്കാർനിര കൊണ്ടുവന്ന മനുജാത്മാവിന്റെ കണ്ണീരുമായ്‌\\
മൂടൽമഞ്ഞിൽ വിരിഞ്ഞു നിൽക്കുമിവിടെപ്പൂക്കും വനജ്യോത്സ്നകള്‍.
\end{slokam}

\Letter{ക}{ക}


\Book{സർഗ്ഗസംഗീതം}.


\begin{slokam}{\VSv}{\NNM}{കാടത്തത്തൊടെതിർത്തു}
കാടത്തത്തൊടെതിർത്തു തോറ്റൊരുവനേ ഗീതാർത്ഥസാരം ഗ്രഹി-\\
ച്ചീടത്തക്കവനാകയുള്ളു ദൃഢ, മിത്തത്വം സമസ്താർത്ഥദം\\
നേടട്ടേ ``നര''നെന്നു പാർത്ഥനൊടടർക്കായിക്കനിഞ്ഞെത്തിയാ\\
വേടൻ കൂടകിരാതമൂർത്തി തുണ നിൽക്കേണം നമുക്കെപ്പൊഴും!
\end{slokam}

\Letter{ക}{ന}

\begin{slokam}{\VSr}{\VenM}{കാടല്ലേ നിന്റെ ഭർത്താവിനു}
"കാടല്ലേ നിന്റെ ഭർത്താവിനു ഭവന?" -- "മതേ, നിന്റെയോ?"; "നിന്മണാളൻ\\
ചൂടില്ലേ പന്നഗത്തെ?" -- "ശ്ശരി, തവ കണവൻ പാമ്പിലല്ലേ കിടപ്പൂ?";\\
"മാടല്ലേ വാഹനം നിൻ ദയിത" -- "നതിനെയും നിൻ പ്രിയൻ മേയ്പ്പതില്ലേ?";\\
"കൂടില്ലേ തർക്ക" - മെന്നങ്ങുമ രമയെ മടക്കും മൊഴിയ്ക്കായ്‌ തൊഴുന്നേൻ!
\end{slokam}

\Letter{ക}{മ}

\Topic{ഉമാരമാസംവാദം}. \SeeAlso{പിച്ചക്കാരൻ ഗമിച്ചാനെവിടെ}, 
\SeeAlso{നന്നോ മെയ്യണിവാനുമേ}, \SeeAlso{കുന്നിൻനാട്ടിലെ ബാന്ധവം}.

\begin{slokam}{\VSv}{\KND}{കാണാനെന്തൊരു മോഹം}
കാണാനെന്തൊരു മോഹമെപ്പൊഴുമെനിക്കെന്നോ?  സുനീലാഞ്ജന-\\
ച്ചേണാളും തനുകാന്തിയേന്തിയെഴുമപ്പുല്ലാങ്കുഴൽക്കാരനെ?\\
കാണാതേ ചില നേരമെന്റെ പിറകിൽ കൺ പോത്തി, യെന്നെത്തുലോം \\
നാണിപ്പിച്ചു വിടാൻ വരുന്ന കുസൃതിക്കൂടായ ഗോവിന്ദനെ!
\end{slokam}

\Letter{ക}{ക}


\begin{slokam}{\VSr}{\KCKP}{കാതില്‍ക്കത്തുന്ന കാന്തി}
കാതില്‍ക്കത്തുന്ന കാന്തിപ്രചുരിമ തിരളും തോടയോ, മോടിയാടി-\\
ക്കോതിബ്ബന്ധിച്ച കൂന്തല്‍ക്കുലമതില്‍ വിലസും മാലതീമാല താനോ,\\
പാതിത്തിങ്കള്‍പ്രകാശം തടവുമളികമോ കാന്തിയേന്തുന്നതില്ലി-\\
പ്പാതിവ്രത്യാഖ്യമാകും സുമഹിതമണി താന്‍ ഭൂഷണം യോഷമാരില്‍.
\end{slokam}

\Letter{ക}{പ}




\begin{slokam}{\VSr}{\KKT}{കാന്തന്മാരൊത്തു}
കാന്തന്മാരൊത്തു, കാൽത്താർ, കടി കടുകളവിൽ, ക്ലാന്തമധ്യം, കനത്തിൽ-\\
ക്കാന്തിപ്പിട്ടുള്ള കൊങ്കക്കുട, മഴകു കലർന്നാടിടും കമ്രഹാരം,\\
കാന്തത്തിങ്കള്‍പ്രഭാസ്യം, കളിയുടയ കയൽക്കണ്ണു, കാർകൂന്തലേവം\\
കാന്ത്യാ കല്യാണിമാർ കൈവിശറിയൊടവിടെദ്ദേവസേവയ്ക്കു കൂടും.
\end{slokam}

\Letter{ക}{ക}


\Book{ശുകസന്ദേശം പരിഭാഷ}.
\OSlRef{തത്സേവാർത്ഥം തരുണസഹിതാഃ}.
\Topic{ആദിപ്രാസം}.

\begin{slokam}{\VSv}{\VKG}{കാരുണ്യക്കടലേ}
കാരുണ്യക്കടലേ, പരർക്കു ചുടലേ, ഗോപാംഗനാമാനസ-\\
ത്താരേന്തും തുടലേ, തുനിഞ്ഞു തുടരെക്കാക്കൂ മറക്കാതലേ,\\
സാരസ്യത്തളിരേ, മനസ്സു കുളിരെത്തൃക്കണ്ണയച്ചാഗമ-\\
ത്തേരോട്ടും കരമേ, കനിഞ്ഞു കുരു മേ സാഹായ്യ, മോങ്കാരമേ!
\end{slokam}

\Letter{ക}{സ}

\begin{slokam}{\VOth}{അത്തിപ്പറ്റ രവി}{കാരോലപ്പവുമട}
കാരോലപ്പവുമട, മല, രവിലൊടു രസഗുള ലഡു മധു നവനീതം തൈ-\\
രോരോന്നും യദുകുലപതിവര! തവ തിരുമലരടികളിലിഹ നേദിപ്പേൻ,\\
സ്ഫാരോത്കർഷവുമസുലഭനിരുപമസുകൃതവുമനുദിനമമലം ചേർന്നി-\\
ട്ടാരോഗ്യത്തൊടു ധരണിയിലെഴുവതിനടിയനു തവ തുണയരുളീടേണം
\end{slokam}

\Letter{ക}{സ}

\begin{slokam}{\VSr}{\VNM}{കാലൻ കാളായസാത്യുത്ക്കട}
കാലൻ കാളായസാത്യുത്ക്കടമുസലവുമായ്‌ കാണികള്‍ക്കുള്‍നടുങ്ങും\\
കോലം കോലുന്ന കൂട്ടാളികളൊടുമൊരുമിച്ചാർത്തടുത്തെത്തിടുമ്പോള്‍\\
കാലച്ചെന്തീക്കനൽച്ചാർത്തെതിർമുനയൊടു നിൻ കൈത്തലത്തിൽത്തിളങ്ങും\\
ശൂലം താനാണു മാഹേശ്വരി, ശരണമെനിക്കാ ഭയപ്പാടൊഴിക്കാൻ.
\end{slokam}

\Letter{ക}{ക}

\begin{slokam}{\VSr}{പാലൂര്}{കാലം മാറിക്കഴിഞ്ഞൂ,}
"കാലം മാറിക്കഴിഞ്ഞൂ, കവിതയെഴുതിയാലാർക്കുവേണം? ഭവാനി-\\
ക്കാലത്തെക്കാവ്യമാകും കഥകളെഴുതണം, നോവലായാൽ വിശേഷം!" \\
കാലംപോൽ ചൊല്ലിടുന്നൂ പലരുമിതുവിധം, പത്നിയും, കാലമാണി-\\
ക്കോലം കെട്ടിച്ചിടുന്നൂ കുശവനതു തിരുത്തീടുവാനാകുമെന്നോ?
\end{slokam}

\Letter{ക}{ക}


\begin{slokam}{\VSr}{\VKG}{കാലം വന്നാൽ മരിയ്ക്കും}
കാലം വന്നാൽ മരിയ്ക്കും; ദുരിതമനുഭവിയ്ക്കാം, ഭവിയ്ക്കാതിരിയ്ക്കാം\\
നീളാം, നീളാതിരിയ്ക്കാം, മരണനിലയിരുന്നോ കിടന്നോ നടന്നോ\\
ആലോചിയ്ക്കേണ്ട, യാവർത്തനജനിമരണക്ലേശവിഭ്രാന്തി നീങ്ങാൻ\\
ചേലിൽച്ചിന്തിച്ചുറപ്പിയ്ക്കുക രഹസി രമാകാന്തകാന്തസ്വരൂപം!
\end{slokam}


\Letter{ക}{അ}

\begin{slokam}{\VSv}{\SVL}{കാലാരാതി കനിഞ്ഞിടുന്നതു വരെ}
കാലാരാതി കനിഞ്ഞിടുന്നതുവരെക്കാളും തപം ചെയ്തു തൽ-\\
ക്കോലം പാതി പകുത്തെടുത്തൊരു കുളുർക്കുന്നിന്റെ കുഞ്ഞോമനേ!\\
കാലൻ വന്നു കയർത്തുനിന്നു കയറെൻ കാലിൽ കടന്നിട്ടിടും-\\
കാലത്താക്കഴുവേറിതൻ കഥ കഴിക്കേണം മിഴിക്കോണിനാൽ
\end{slokam}

\Letter{ക}{ക}

\begin{slokam}{\VSr}{\VNM}{കാലേതാനും മടക്കി}
കാലേതാനും മടക്കി, ത്തുട തുടയുടെ മേൽ ചേർ, ത്തതിന്നബ്ജമാല്യം\\
പോലേ വാരിക്കു മേൽ നീട്ടിയ രുചിരവലംകയ്യലങ്കാരമാക്കി,\\
മേലേ വൻ പോർമുലപ്പൊന്നണിചിതറുമിടംകൈ കവിള്‍ത്തട്ടിനേകി-\\
ച്ചേലേറും കണ്ണടച്ചെൻ ശശികലികയിതാ വെൺനഭസ്സിൽ ശയിപ്പൂ!
\end{slokam}

\Letter{ക}{ക}

\Book{വിലാസലതിക}.


\begin{slokam}{\VKm}{\KN}{കാളമേഘമിട തിങ്ങി}
കാളമേഘമിടതിങ്ങിവിങ്ങി ബത ഭംഗിയേറുമിടിമിന്നലും \\
മേളമോടു ജലധാര മാരികളുമെത്രയും ബഹുമനോഹരം \\
അർദ്ധരാത്രിസമയം സമാഗത മുദിച്ചു ചന്ദ്രനു മഹോ തദാ \\
സിദ്ധചാരണ സുരാവലി സ്തുതികളുദ്ധതം ദിവി മഹോത്സവം! 
\end{slokam}

\Letter{ക}{അ}

\begin{slokam}{\VSv}{\Unk}{കാളാംഭോദകമായ കമ്രരുചിയാൽ}
കാളാംഭോദകമായ കമ്രരുചിയാൽ, കാളാഞ്ജനക്കല്ലിനെ-\\
ക്കാളാരമ്യകളേബരേ കഠിനമക്ഷ്വേളാഹി കൊത്തുമ്പൊഴും\\
കോളായെന്നു നിനച്ചുകൊണ്ടഥ മനം കാളാതെ കാളിന്ദിയിൽ-\\
ക്കാളാഹീനശയൻ തുടർന്നു മധുരം താളാത്മകം നർത്തനം!
\end{slokam}

\Letter{ക}{ക}


\begin{slokam}{\VSv}{\Unk}{കാളാംഭോധരപാളി താളി}
കാളാംഭോധരപാളി താളി പിഴിയും കറ്റക്കരിമ്പൂങ്കുഴൽ-\\
ക്കാലംബായ മുഖാവലോകസമയേ നെയ്‌ വെയ്ക്കുമിച്ചന്ദ്രമാഃ\\
കോലത്താർചരഭൂമിപാലകനകക്കുംഭം തൊഴും പോർമുലയ്‌-\\
ക്കോലക്കത്തൊടു നിന്നെ വാഴ്ത്തുമതിനാന്റാമല്ല കൗണോത്തരേ!
\end{slokam}

\Letter{ക}{ക}

\Book{പദ്യരത്നം}.


\begin{slokam}{\VSv}{\VKG}{കാളിന്ദിപ്പുഴവക്കിലുണ്ട്}
കാളിന്ദിപ്പുഴവക്കിലുണ്ടൊരരയാൽവൃക്ഷം, കണിക്കൊന്നയെ-\\
ക്കാളും മഞ്ജുളമായ മഞ്ഞവസനം ചാർത്തുന്നൊരാളുണ്ടതിൽ,\\
കാളാബ്ദാഞ്ചിതകോമളാകൃതികലാപാലംകൃതോഷ്ണീഷനാ-\\
ണാ, ളെൻ നിർഭരഭാഗ്യമേ, മദനഗോപാലൻ മദാലംബനം!
\end{slokam}

\Letter{ക}{ക}


\begin{slokam}{\VSr}{\VKG}{കാറോടിക്കും വപുസ്സും}
കാറോടിക്കും വപുസ്സും, രമ നിജ കുചകുംഭത്തിലെക്കുങ്കുമത്താ-\\
ലാറാടിപ്പോരുരസ്സും, തിറമൊടു മയിലിൻ പീലി ചൂടും ശിരസ്സും,\\
കൂറാളും സന്മനസ്സും, നളിനപദരജസ്സും, സ്വഭക്തന്നുവേണ്ടി-\\
ത്തേരോടിക്കും യശസ്സും, കരുതുക മനമേ! സാർത്ഥമാം നിൻ ജനുസ്സും!
\end{slokam}

\Letter{ക}{ക}

\begin{slokam}{\VSr}{കൂനേഴത്തു പരമേശ്വരമേനോൻ}{കാറോടും കാന്തി, കാർശ്യം,}
കാറോടും കാന്തി, കാർശ്യം, കുറിയൊരുടൽ, കുറും കൗശലം കൂടവേ താൻ\\
പേറീടുന്നോരമാന്തം, മലിനത, ഫലിതം, ചുണ്ടിറുക്കിച്ചിരിക്കൽ, \\
ആരോടും ചേർന്നിണങ്ങും പ്രകൃതി, യമനെയും കിട്ടിയാൽ കൊട്ടയാട്ടി-\\
പ്പോരാടിത്താൻ കുലുങ്ങാതൊരു നിലയിവ താൻ വെണ്മണിച്ഛായയോർത്താൽ. 
\end{slokam}

\Letter{ക}{അ}


\begin{slokam}{\VSv}{\UN}{കാറ്റിൽ തീപ്പൊരി പോലെ}
കാറ്റിൽ തീപ്പൊരി പോലെ നിൻ കൃപ പടർന്നീടേണ, മെൻ ഹൃത്തതിൽ\\
മാറ്റേറിക്കറതീർന്നു ഭക്തി നിറയും പൊൻപാത്രമായീടണം\\
നൂറ്റൊന്നപ്പമനേകകോടി ഹൃദയാനന്ദാശ്രുവൊത്തിന്നിതാ\\
പോറ്റീ, ഈശ്വരമംഗലാധിപ, സമർപ്പിക്കുന്നു തൃക്കാൽക്കൽ ഞാൻ!
\end{slokam}

\Letter{ക}{ന}


\begin{slokam}{\VKm}{എം. ആർ. മാടപ്പള്ളി}{കാറ്റുലച്ച ചെറുവഞ്ചി പോലെ}
കാറ്റുലച്ച ചെറുവഞ്ചിപോലെ മമ ജീവിതം ദുരിതപർണ്ണമാ-\\
യാറ്റിലൂടെ ഗതിവിട്ടു പോകുവതു കാണ്മതില്ല ജഗദംബിക\\ 
ഏറ്റിടാൻ കരമയച്ചുതന്നു ദുരിതങ്ങളാറ്റി സുഖമേകുകെ-\\
ന്നാറ്റുകാലമരുമംബികേ ത്രിപുരസുന്ദരീ തരിക മംഗളം
\end{slokam}

\Letter{ക}{എ}

\begin{slokam}{\VSr}{\Unk}{കാറ്റേൽക്കുമ്പോള്‍ തിളങ്ങും}
കാറ്റേൽക്കുമ്പോള്‍ തിളങ്ങും തൊടുകുറി, കുറിയിൽച്ചേർത്തുവെച്ചൂതിയെന്നാൽ\\
മാറ്റേറും വില്ലു, വില്ലിൻ മുകളിലമരുവോർക്കല്ലൽ തീർപ്പോരു ബാണം,\\
പോറ്റീ! ബാണം കിടക്കും മണിമയസദനം കങ്കണം, കങ്കണത്തി-\\
ന്നൂറ്റം കാ, റ്റെത്ര നന്നിത്തൊഴിലുകള്‍, ശിവപേരൂരെഴും തിങ്കള്‍മൗലേ!
\end{slokam}

\Letter{ക}{പ}

\begin{slokam}{\VPv}{\UN}{കിടത്തി ജടയിൽപ്പിടിച്ചൊരുവളെ}
കിടത്തി ജടയിൽപ്പിടിച്ചൊരുവളെ, പ്പരയ്ക്കേകി ത-\\
ന്നിടത്തുവശമാകവേ - പരിഭവങ്ങള്‍ തീർത്തിട്ടു, താൻ\\
കൊടുത്തൊരു വരത്തിനാൽ വലയവേ, സഹായത്തിനാ-\\
യടുത്തവളൊടൊത്തൊരാ മദനവൈരിയെക്കൈതൊഴാം!
\end{slokam}

\Letter{ക}{ക}



\begin{slokam}{\VHr}{\KKK}{കിഴവനെ യുവാവാക്കും}
കിഴവനെ യുവാവാക്കും വാക്കും തിലപ്രസവപ്രഭയ്‌-\\
ക്കഴലനുദിനം മൂക്കും മൂക്കും മിനുത്തൊരു ഗണ്ഡവും\\
മിഴികളടിയാലാക്കും ലാക്കും തകർപ്പൊരു കാറണി-\\
ക്കുഴലിയിവള്‍ തൻ നോക്കും നോക്കും തരുന്നൊരു കൗതുകം.
\end{slokam}

\Letter{ക}{മ}


\begin{slokam}{\VSv}{\PCM}{കീര്‍ത്തിയ്ക്കാം തിരുനാമം}
കീര്‍ത്തിയ്ക്കാം തിരുനാമ, മക്ഷരലസത്ക്കീര്‍ത്തേ, വിചാരങ്ങളാല്‍\\
ചാര്‍ത്തിയ്ക്കാം മലര്‍മാല, യെന്‍ ഹൃദി വിളങ്ങീടുന്ന നിന്‍മൂര്‍ത്തിമേല്‍\\
ഭക്ത്യുന്മത്തഘനം പൊഴിച്ചു മിഴിനീരാറാട്ടുമാ, മെങ്കിലീ-\\
മര്‍ത്ത്യത്വം പരദേവതേ, ക്ഷണികമായാലെ,ന്തെനിയ്ക്കുത്സവം!
\end{slokam}

\Letter{ക}{ഭ}



\begin{slokam}{\VSr}{\VKG}{കിർത്തിക്കേണ്ടുന്ന ഭക്തപ്രവരരെ}
കിർത്തിക്കേണ്ടുന്ന ഭക്തപ്രവരരെ, വിഷയഭ്രാന്തരെ, കൂട്ടുകാരെ,-\\
പ്പേർത്തുംപേർത്തും സ്മരിപ്പോർകളെ, യൊരുകുറിയും പണ്ടു കാണാത്ത പേരെ, \\
ഓർത്താൽ മാത്രം ചിരിക്കും ലഘുപരിചയമുള്ളോരെ, നാം മാറിമാറി-\\
പ്പാർക്കും പൂരപ്പറമ്പിൽ പ്രകൃതിയുടെ പകർപ്പക്ഷരശ്ലോകലോകം. 
\end{slokam}

\Letter{ക}{ഒ}

\begin{slokam}{\VSv}{\VNM}{കുട്ടിക്കാലമതെത്ര തുഷ്ടികരം}
കുട്ടിക്കാലമതെത്ര തുഷ്ടികര? മന്നദ്ദേഹമെന്നോടു വേർ-\\
പെട്ടിട്ടുള്ള ദിനം ചുരുങ്ങു, മൊരുമിച്ചല്ലാതെയില്ലൊന്നുമേ\\
കിട്ടില്ലൊട്ടിടയിപ്പൊഴസ്സുഭഗനെക്കാണാനുമെന്നായി - പാർ-\\
ത്തട്ടിൽ ദുഃസ്ഥിതിഹേതുവിങ്ങു ഹതമാമീ യൗവനം താനഹോ!
\end{slokam}

\Letter{ക}{ക}

\Book{വിലാസലതിക}.

\begin{slokam}{\VVt}{\VenM}{കുട്ടിക്കുരംഗമിഴിയാമുമ തന്റെ}
കുട്ടിക്കുരംഗമിഴിയാമുമ തന്റെ ചട്ട \\
പൊട്ടിക്കുരുത്തിളകുമക്കുളുർകൊങ്ക രണ്ടും \\
മുട്ടിക്കുടിക്കുമൊരു കുംഭിമുഖത്തൊടൊത്ത \\
കുട്ടിയ്ക്കു ഞാൻ കുതുകമോടിത കൈതൊഴുന്നേൻ!
\end{slokam}

\Letter{ക}{മ}

\begin{slokam}{\VKm}{\VNM}{കുന്തളാവലി വിയർത്ത}
കുന്തളാവലി വിയർത്ത പൂങ്കവിളിൽ മിന്നിയും, കുളുർമുലക്കുടം\\
ചന്തമോടു നടമാടിയും, പൃഥുനിതംബമണ്ഡലമുലഞ്ഞുമേ,\\
ചെന്തളിർത്തനു തളർന്നിടും പടി, സലീലമദ്രിജ നിജാന്തികേ\\
പന്തടിക്കെ, നിടിലാക്ഷനാർന്ന പുളകം നമുക്കരുള്‍ക മംഗളം!
\end{slokam}

\Letter{ക}{ച}


\Book{വിലാസലതിക}.


\begin{slokam}{\VSv}{\SVL}{കുന്നിയ്ക്കും കുറയാതെ}
കുന്നിയ്ക്കും കുറയാതെ കുന്നൊടു കുശുമ്പേറും കുചം പേറിടും\\
കുന്നിൻനന്ദിനി കുന്ദബാണനു കൊലക്കേസൺനു പാസ്സായതിൽ\\
ഒന്നാം സാക്ഷിണിയായ നീ കനിവെഴും വണ്ണം കടക്കണ്ണെടു-\\
ത്തൊന്നെന്നിൽ പെരുമാറണേ പെരുവനത്തപ്പന്റെ തൃപ്പെൺകൊടീ!
\end{slokam}

\Letter{ക}{ഒ}


\begin{slokam}{\VSv}{\RV}{കുന്നിൻനാട്ടിലെ ബാന്ധവം}
"കുന്നിൻനാട്ടിലെ ബാന്ധവം കഠിനമോ?" "തണ്ണീരിലും മെച്ചമാ-"\\
"ണുണ്ണിക്കുമ്പ നിറഞ്ഞിടാത്തൊരഴലോ?" "വന്ധ്യത്വമോർത്താൽ സുഖം."\\
"പെണ്ണിൻ മാതിരി പാതിമേനിയഴകോ?" "പെൺവേഷമോ?"യെന്നു താർ-\\
ക്കണ്ണൻ വമ്പിനെ വെന്ന വാണിയൊടു മുക്കണ്ണൻ തുണച്ചീടണം.
\end{slokam}

\Letter{ക}{പ}

\SlRef{കാടല്ലേ നിന്റെ ഭർത്താവിനു} എന്ന ശ്ലോകത്തിന്റെ പുരുഷപക്ഷം.  

\begin{slokam}{\VPv}{\Kund}{കുലാദ്യഖിലസദ്ഗുണ}
കുലാദ്യഖിലസദ്ഗുണക്കിളികളൊത്തു കൂത്താടിടും \\
കുലായമരമേ! കരിംകുഴലിമാർ കുലോത്തംസമേ! \\
കുലാദ്രികുലകുഞ്ജരപ്രിയകുമാരികേ! പാപസം-\\
കുലാദ്രികുലിശാമലക്കഴലിണയ്ക്കു കൂപ്പുന്നു ഞാൻ! 
\end{slokam}

\Letter{ക}{ക}

\begin{slokam}{\VPv}{\RV}{കുളിർത്ത മണിമാറു}
കുളിർത്ത മണിമാറു ചേർത്തമൃതമൂട്ടി ശാസ്താവുമായ്‌-\\
ക്കളിച്ചു പ്രണയാർദ്രമാം മിഴികളീശനിൽത്തൂകിയും\\
കിളർന്ന മദനാഗ്നിയിൽ മദനവൈരിയെച്ചുട്ടു നീ\\
വിളങ്ങുക രമാപതേ മനസി മോഹിനീരൂപനായ്‌!
\end{slokam}

\Letter{ക}{ക}

\begin{slokam}{\VSv}{\Unk}{കൂറും നന്മയുമേറിടും}
കൂറും നന്മയുമേറിടും പ്രിയതമേ, പൊന്‍പണ്ടമുണ്ടാക്കിയാല്‍\\
മാറും സങ്കടമെങ്കിലോ പുനരിതാ നാലുണ്ടിതൊറ്റപ്പവന്‍\\
ഏറും ഭംഗി കലര്‍ന്നു കാണ്‍, പവനിതാ കാല്‍കാലതായ്‌ കയ്യിലി-\\
\samd{ന്നാറും പിന്നെയൊരാറുമെന്നിവ ഗണിച്ചീടുമ്പൊളേഴായ്‌ വരും}{ആറും പിന്നെയൊരാറുമെന്നിവ ഗണിച്ചീടുമ്പൊളേഴായ്‌ വരും}.
\end{slokam}

\Letter{ന}{സ}


സമസ്യാപൂരണം. 



\begin{slokam}{\VSr}{\VNM}{കേറാനെന്തേ മടിക്കുന്നതു}
കേറാനെന്തേ മടിക്കുന്നതു മമ കരളിൽ? കാമലോഭാദിയാകും\\
ചേറാണിങ്ങൂന്നി വെച്ചീടുകിലടിവഴുതിത്തെറ്റി വീണേക്കുമെന്നോ?\\
കൂറാളും നീ വിചാരിക്കുകിലിഹ ചളി കൊണ്ടുള്ള കേടാകമാനം\\
മാറാനുണ്ടോ പ്രയാസം? മകുടജിതലസത്കോടിസൂര്യപ്രകാശേ!
\end{slokam}

\Letter{ക}{ക}

\begin{slokam}{\VSv}{\VNM}{കേറുന്നൂ മുകളിൽ കഥഞ്ചന}
കേറുന്നൂ മുകളിൽ കഥഞ്ചന, പതിച്ചീടുന്നു കീഴ്പോട്ടു താൻ\\
ചേറും മുള്ളുമിയന്ന പാഴ്ക്കുഴികളിൽ ചാടുന്നു ദുർവാരമായ്,\\
ആറുൾപ്പുക്കൊഴുകുന്നു പൂർണ്ണതയിലേക്കെൻ നീരമേ നീ പദം\\
തോറും കാട്ടുവതീ പ്രപഞ്ചപഥികന്മാർ തൻ പ്രയാണക്രമം. 
\end{slokam}

\Letter{ക}{അ}

\begin{slokam}{\VSr}{\HM}{കൊക്കായ് പാറിപ്പറക്കാം}
കൊക്കായ് പാറിപ്പറക്കാ, മൊഴുകുമരുവിയിൽച്ചേലെഴും മത്സ്യമാകാം,\\
പൊക്കത്തിൽ വ്യോമമെത്താൻ തുനിയുമൊരു മുളംകൂട്ടമായ് കാറ്റിലാടാം,\\
അർക്കൻ ചിത്രംവരയ്ക്കും കളരിയിലണയാം വിദ്യയൊന്നഭ്യസിക്കാ,-\\
മിക്കാണുന്നാറ്റുവക്കിൻ പടവിലിതുവിധം ചിത്തമേ! വന്നു നിന്നാൽ!
\end{slokam}

\Letter{ക}{അ}

\begin{slokam}{\VSr}{\KND}{കൊഞ്ചും മഞ്ജീരമോമൽ തരിവള}
കൊഞ്ചും മഞ്ജീരമോമൽ തരിവള, യിളകും കിങ്കിണിച്ചാർത്തു, ഗന്ധം\\
തഞ്ചും വ്യാലോലശോഭാലളിതമിളദളിശ്രീലസദ്വന്യമാല്യം\\
ചെഞ്ചുണ്ടിൽ പുഞ്ചിരിപ്പാലമൃതു, മഷിയണിത്തൃക്കടക്കണ്ണിലോജ-\\
സ്സഞ്ചും കാരുണ്യപൂരം - സതു ജയതു ഘനശ്യാമളൻ കോമളാംഗൻ!
\end{slokam}

\Letter{ക}{ച}




\begin{slokam}{\VSr}{\VKG}{കൊണ്ടൽച്ചായൽക്കറുപ്പും}
 കൊണ്ടൽച്ചായൽക്കറുപ്പും, സ്തനയുഗമദനച്ചെപ്പുറപ്പും, വെടിപ്പും,\\
ചുണ്ടിൻ ചോപ്പും, കരിംകൂവളചകിതമിഴിച്ചഞ്ചലിപ്പും, നടപ്പും,\\
കൊണ്ടാടും പട്ടുടുപ്പും, സരസമിയലുമിപ്പെൺകിടാവിൻ പൊടിപ്പെ-\\
ക്കണ്ടാൽ തണ്ടാർശരന്നും സരഭസമുളവാം നെഞ്ചിടിപ്പും ചടപ്പും!
\end{slokam}

\Letter{ക}{ക}

\Topic{അന്ത്യപ്രാസം}.  \PrevSlRef{ഉണ്ടോ നേരത്തുടുക്കും}.


\begin{slokam}{\VKm}{\ARRV}{കൊണ്ടല്‍വേണിയൊരു രണ്ടുനാലടി}
 കൊണ്ടല്‍വേണിയൊരു രണ്ടുനാലടി നടന്നതില്ലതിനുമുമ്പു താന്‍\\
കൊണ്ടു ദര്‍ഭമുന കാലിലെന്നു വെറുതെ നടിച്ചു നിലകൊണ്ടുതേ\\
കണ്ഠവും ബത തിരിച്ചുനോക്കിയവള്‍ വല്‍ക്കലാഞ്ചലമിലച്ചിലില്‍-\\
ക്കൊണ്ടുടക്കുമൊരു മട്ടു കാട്ടി വിടുവിച്ചിടുന്ന കപടത്തൊടേ
\end{slokam}

\Letter{ക}{ക}

\Book{അഭിജ്ഞാനശാകുന്തളം പരിഭാഷ}.
\OSlRef{ദർഭാങ്കുരേണ ചരണഃ ക്ഷതഃ}.


\begin{slokam}{\VSv}{\VRV}{കോടക്കാറ്റിലഴിഞ്ഞുലഞ്ഞ ചിടയും}
കോടക്കാറ്റിലഴിഞ്ഞുലഞ്ഞ ചിടയും ചിക്കിക്കിടന്നീടുമാ\\
ക്കാടങ്ങിങ്ങു ചവച്ചെറിഞ്ഞ തളിരും പൂവും പിടഞ്ഞീടവേ\\
നാടന്തഃപ്രഹരങ്ങളേറ്റു കിടിലം കൊള്‍കേ, മുലപ്പാലുമായ്‌\\
പാടം നീന്തിവരുന്ന പൌര്‍ണ്ണമി, നിനക്കാവട്ടെ ഗീതാഞ്ജലി.
\end{slokam}

\Letter{ക}{ന}


\Book{സർഗ്ഗസംഗീതം}.


\begin{slokam}{\VSr}{\VenA}{കോടക്കാർവർണ്ണനോടക്കുഴലൊടു}
കോടക്കാർവർണ്ണനോടക്കുഴലൊടു കളി വിട്ടോടിവന്നമ്മ തന്റേ\\
മാടൊക്കും പോർമുലപ്പാലമിതരുചി ഭുജിച്ചാശ്വസിക്കും ദശായാം\\
ഓടി ക്രീഡിച്ചു വാടീടിന വദനകലാനാഥഘർമ്മാമൃതത്തെ-\\
ക്കൂടെക്കൂടെത്തുടയ്ക്കും സുകൃതനിധി യശോദാകരം കൈതൊഴുന്നേൻ!
\end{slokam}

\Letter{ക}{ഒ}

\begin{slokam}{\VSv}{\SVL}{കോപം, വാശി, കുശു, മ്പസൂയ,}
കോപം, വാശി, കുശു, മ്പസൂയ, ദുര, ദുർമ്മന്ത്രം, മരു, ന്നുന്മദാ-\\
ലാപം, ലോഭ, മല, ട്ടുരുട്ടു, നുണ, സിദ്ധാന്തം, മൊശോടത്തരം,\\
വ്യാപാദം, ചതി, വാദ, മേഷണി, പണക്കു, ത്തൂറ്റ -- മെന്നീ വക-\\
ച്ചാപല്യങ്ങളിലൊന്നു പോലുമറിയെപ്പേറുന്ന പെണ്ണല്ലിവള്‍!
\end{slokam}

\Letter{ക}{വ}


\begin{slokam}{\VSv}{\VNM}{കോരിക്കൂട്ടിയ പാഴ്ക്കരിക്കിടയിലെ}
കോരിക്കൂട്ടിയ പാഴ്ക്കരിക്കിടയിലെത്തീക്കട്ടയോ, പായലാൽ\\
പൂരിച്ചുള്ള ചെളിക്കുളത്തിലുളവാം പൊന്താമരപ്പുഷ്പമോ,\\
മാരിക്കാറണിചൂഴുമിന്ദുകലയോ, പോലേ മനോജ്ഞാംഗിയാ-\\
ളാരിക്കാണ്മൊരിരുണ്ട കൊച്ചുപുരതൻ കോലായിൽ നിൽക്കുന്നവള്‍?
\end{slokam}

\Letter{ക}{മ}
\Book{ഒരു സന്ധ്യാപ്രണാമം}.

\begin{slokam}{\VSr}{\KN}{കോൽത്തേനോലേണമോരോ പദ, മതി}
കോൽത്തേനോലേണമോരോ പദ, മതിനെ നറും പാലിൽ നീരെന്ന പോലേ\\
ചേർത്തീടേണം, വിശേഷിച്ചതിലുടനൊരലങ്കാരമുണ്ടായ്‌ വരേണം,\\
പേർത്തും ചിന്തിക്കിലർത്ഥം നിരുപമരുചി തോന്നേണ, മെന്നിത്ര വന്നേ\\
തീർത്തീടാവൂ ശിലോകം - ശിവ ശിവ! കവിതാരീതി വൈഷമ്യമത്രേ!
\end{slokam}

\Letter{ക}{പ}

\begin{slokam}{\VSr}{\AR}{കൗശേയം പിഞ്ഞി, കൗമോദകിയുടെ}
കൗശേയം പിഞ്ഞി, കൗമോദകിയുടെ തല പോയ്, കൗസ്തുഭം ക്ലാവിൽ മുങ്ങീ,\\
ഏശാതായ് നാന്ദകക്കൂരലകു, മതു തുരുമ്പിച്ചു, ശാർങ്‌ഗം നുറുങ്ങീ,\\
കൂസാതായ് പാഞ്ചജന്യധ്വനമരിക, ളനങ്ങാതെയായ് ചക്രം, മെന്തീ \\
ക്ലേശം കാണാതെയിന്നും കവികൾ ജരഠകാവ്യോപമാനങ്ങൾ കോർപ്പൂ? 
\end{slokam}

\Letter{ക}{ക}


\begin{slokam}{\VSr}{\VenM}{ക്രീഡിച്ചും കീരവാണീ}
ക്രീഡിച്ചും കീരവാണീമണികളൊ, ടിടയിൽ കയ്യിൽ നെയ്‌ പാലിതെല്ലാം\\
മേടിച്ചും, കട്ടശിച്ചും, പ്രണതരിലലിവിൻ നീർ തുളിച്ചും, തുണച്ചും,\\
കൂടിച്ചും പാണ്ഡവർക്കുന്നതി, കുരുനിരയെത്തക്കമോർത്തങ്ങു കുണ്ടിൽ-\\
ച്ചാടിച്ചും വാണ ഗോപീജനസുകൃതസുഖക്കാതലേ, കൈതൊഴുന്നേൻ!
\end{slokam}

\Letter{ക}{ക}

\begin{slokam}{\VSv}{\VNM}{ക്ഷീണാപാണ്ഡുകപോലമാം}
ക്ഷീണാപാണ്ഡുകപോലമാം മുഖവുമായ്‌, തൻ മന്ദിരത്താഴ്‌വര-\\
ത്തൂണാലൊട്ടു മറഞ്ഞുനിന്നു, നെടുതാം വീർപ്പിട്ടുകൊണ്ടങ്ങനെ\\
"കാണാം താമസിയാതെ" യെന്നൊരുവിധം ബന്ധുക്കളോടോതിടും\\
പ്രാണാധീശനെ, യശ്രുപൂർണ്ണമിഴിയായ്‌ നോക്കുന്നു മൈക്കണ്ണിയാള്‍.
\end{slokam}

\Letter{ക}{ക}

\Book{വിലാസലതിക}.



\begin{slokam}{\VSv}{\KA}{ക്ഷീണിക്കാത്ത മനീഷയും}
ക്ഷീണിക്കാത്ത മനീഷയും മഷിയുണങ്ങീടാത്ത പൊൻപേനയും\\
വാണിക്കായ്‌ തനിയേയുഴിഞ്ഞു വരമായ്‌ നേടീ ഭവാൻ സിദ്ധികള്‍\\
കാണിച്ചൂ വിവിധാത്ഭുതങ്ങള്‍ വിധിദൃഷ്ടാന്തങ്ങളായ്‌, വൈരിമാർ\\
നാണിച്ചൂ, സ്വയമംബ കൈരളി തെളിഞ്ഞീക്ഷിച്ചു മോക്ഷത്തെയും
\end{slokam}

\Letter{ക}{ക}

\Book{പ്രരോദനം}.

\begin{slokam}{\VSr}{\Unk}{ക്ഷീരശ്രീ ചേർന്ന വാചാ}
ക്ഷീരശ്രീ ചേർന്ന വാചാ സരസമിഹ മുളപ്പിച്ച നിൻ കീർത്തിബീജം\\
പാരെപ്പേരും വിതച്ചിൻറഴകൊടു വിളയിച്ചീടുവൻ നൂനമിഞ്ഞാൻ;\\
കാരുണ്യം നൽക്കവീനാം വളമിതിനു ; മിനക്കെട്ടു കാക്കിൻറതെല്ലാം\\
മാരക്ഷ്മാപാല; നേതും വരതനു, കുറവില്ലത്ര കൌണോത്തരേ! മേ.
\end{slokam}

\Letter{ക}{ക}

\Book{പദ്യരത്നം}.

\end{enumerate}

\subsection{ഖ}

\begin{enumerate}

\begin{slokam}{\VVt}{\KA}{ഖേദിച്ചിടൊല്ല കളകണ്ഠ}
ഖേദിച്ചിടൊല്ല കളകണ്ഠ! വിയത്തില്‍ നോക്കി\\
രോദിച്ചിടേണ്ട, രുജയേകുമതിജ്ജനത്തില്‍\\
വേദിപ്പതില്ലിവിടെയുണ്മ തമോവൃതന്മാ-\\
രാദിത്യലോകമറിയുന്നിതു നിന്‍ ഗുണങ്ങള്‍.
\end{slokam}

\Letter{ഖ}{വ}

\Book{ഗ്രാമവൃക്ഷത്തിലെ കുയിൽ}.

\begin{slokam}{\VVt}{\KA}{ഖേദിയ്ക്കകൊണ്ടു ഫലമില്ല}
ഖേദിയ്ക്കകൊണ്ടു ഫലമില്ല, നമുക്കതല്ല\\
മോദത്തിനും ഭുവി വിപത്തു വരാം ചിലപ്പോള്‍;\\
ചൈതന്യവും ജഡവുമായ്‌ കലരാം ജഗത്തി-\\
ലേതെങ്കിലും വടിവിലീശ്വര വൈഭവത്താല്‍.
\end{slokam}

\Letter{ഖ}{ച}

\Book{വീണ പൂവ്}.

\end{enumerate}

\subsection{ഗ}

\begin{enumerate}


\begin{slokam}{\VSv}{\CKP}{ഗാനത്താലവനീപതേ, മധുരമാം}
ഗാനത്താലവനീപതേ, മധുരമാം ചെമ്മുന്തിരിച്ചാറിനാ-\\
ലാനന്ദക്കതിര്‍ വീശിടുന്നു നിയതം ഹര്‍മ്മ്യാന്തരത്തില്‍ ഭവാന്‍\\
ആ നല്‍ച്ചെമ്പനിനീരലര്‍പ്പുതു വികാരത്തില്‍പ്പുഴുക്കുത്തിയ-\\
റ്റാനല്ലാതുതകുന്നതില്ലണുവുമെന്‍ ദുര്‍വ്വാരഗര്‍വ്വാങ്കുരം!
\end{slokam}

\Letter{ഗ}{അ}



\begin{slokam}{\VSv}{\VKG}{ഗൂഢം പാതിരയിൽ}
ഗൂഢം പാതിരയിൽ പ്രസിദ്ധമഥുരാകാരാഗൃഹത്തിങ്കൽ നി-\\
ന്നോടിപ്പോയ്‌, ശിശുമോഷണക്രിയയിലും കയ്യിട്ടു, ഗോപാലനായ്‌\\
ആടിപ്പാടി നടന്നു, വല്ലവികള്‍തൻ ചേതസ്സുമച്ചേലയും\\
കൂടിക്കട്ടുമുടിച്ചൊരത്തടവുപുള്ളിക്കായ്‌ നമിക്കാദ്യമായ്‌.
\end{slokam}

\Letter{ഗ}{അ}

\begin{slokam}{\VSr}{\Unk}{ഗോവൃന്ദം മേച്ചു വൃന്ദാവനഭുവി}
ഗോവൃന്ദം മേച്ചു വൃന്ദാവനഭുവി ഭുവനം മൂന്നിനും മൂലമാകും\\
ഗോവിന്ദൻ പന്തടിച്ചും പലവക കളിയാൽ ക്ഷീണനായ്‌ മാറിടുമ്പോള്‍\\
ആവിർമോദാലശോകച്ചെറുതളിരുകളാലാശുവീശിത്തലോടി-\\
ജ്ജീവിപ്പിക്കുന്ന ഗോപീജനനിര നിരയം നീക്കണം നിത്യവും മേ.
\end{slokam}

\Letter{ഗ}{അ}
\end{enumerate}
\subsection{ഘ}

\begin{enumerate}

\begin{slokam}{\VSr}{\KA}{ഘോരാകാരാട്ടഹാസപ്രകടിതകലഹം}
ഘോരാകാരാട്ടഹാസപ്രകടിതകലഹം മൃത്യു വന്നെത്തി നോക്കും\\
നേരം നാരീജനത്തിൻ കളികളുമിളിയും നോക്കുമൂക്കുള്ള വാക്കും\\
പോരാ പോരിൽത്തടുപ്പാൻ; പരമശിവപദാംഭോജരേണുപ്രസാദം\\
പോരും പോരും കൃതാന്തപ്രതി ഭയമകലത്താക്കുവാനാർക്കുമെന്നും!
\end{slokam}

\Letter{ഘ}{പ}

\begin{slokam}{\VSr}{\Unk}{ഘോരാണാം ദാനവാനാം}
ഘോരാണാം ദാനവാനാം നിരുപമപൃതനാഭാരഖിന്നാം ധരിത്രീ-\\
മോരോ ലീലാവതാരൈരഴകിനൊടു സമാശ്വാസയന്തം നിതാന്തം\\
ക്ഷീരാംബോധൗ ഭുജംഗാധിപശയനതലേ യോഗനിദ്രാമുദാരാം\\
നേരേ കൈക്കൊണ്ടു ലക്ഷ്മീകുളുർമുല പുണരും പത്മനാഭം ഭജേഥാഃ
\end{slokam}

\Letter{ഘ}{ക}

\begin{slokam}{\VVt}{\KA}{ഘോരായുധവ്രണിതകാന്ത}
ഘോരായുധവ്രണിതകാന്തകളേബരം കൈ-\\
ത്താരാൽക്കനിഞ്ഞഹഹ, തൊട്ടുതലോടിടുമ്പോള്‍\\
ശ്രീരാജകന്യകള്‍ കൊതിച്ചുവരുന്ന വീര-\\
ദാരാസ്പദത്തിലുമുഷയ്ക്കു വിരക്തി തോന്നി!
\end{slokam}

\Letter{ഘ}{ശ}


\end{enumerate}
\subsection{ച}

\begin{enumerate}

\begin{slokam}{\VKm}{\UN}{ചഞ്ചലങ്ങളളകങ്ങൾ ചിന്നി}
ചഞ്ചലങ്ങളളകങ്ങൾ ചിന്നി, യിളകുന്ന കുണ്ഡലമുലഞ്ഞുമാ--\\
ക്കുഞ്ഞുവേർപ്പുകണമാകെ മൂടി വടിവിൽ കുറച്ചു കുറി മാഞ്ഞുമേ,\\
പഞ്ചബാണരതി തീർന്നു കണ്ണുകൾ തളർന്നു, സൗഭഗമണയ്ക്ക നിൻ\\
നെഞ്ചിലേറുമവൾ തൻ മുഖം, ഹരവിരിഞ്ചവിഷ്ണുജനമെന്തിനോ?
\end{slokam}

\Letter{ച}{പ}

\Book{അമരുകശതകം പരിഭാഷ}. \OSlRef{ആലോലാമളകാവലീം വിലുളിതാം}

\begin{slokam}{\VSr}{\CN}{ചഞ്ചൽച്ചില്ലീലതയ്ക്കും, പെരിയ}
ചഞ്ചൽച്ചില്ലീലതയ്ക്കും, പെരിയ മണമെഴും പൂമുടിക്കും തൊഴുന്നേൻ;\\
അഞ്ചിക്കൊഞ്ചിക്കുഴഞ്ഞിട്ടമൃതു പൊഴിയുമപ്പുഞ്ചിരിക്കും തൊഴുന്നേൻ;\\
അഞ്ചമ്പൻ ചേർന്ന യൂനാം മനസി ഘനമുലയ്ക്കും മുലയ്ക്കും തൊഴുന്നേൻ;\\
നെഞ്ചിൽ കിഞ്ചിൽക്കിടയ്ക്കും കൊടിയ കുടിലതയ്ക്കൊന്നു വേറേ തൊഴുന്നേൻ!
\end{slokam}

\Letter{ച}{അ}

\begin{slokam}{\VSr}{\VKG}{ചന്തം ചിന്തുന്ന ചന്ദ്രോത്സവം}
ചന്തം ചിന്തുന്ന ചന്ദ്രോത്സവ, മനുഭവരാസിക്യ സമ്പന്നമുക്താ-\\
വൃന്ദം നാരായണീയം, പുനമഹിഷകൃതോൽകൃഷ്ട ചമ്പൂകദംബം,\\
സന്ദേശച്ചാർത്തു മേഘഭ്രമരശുകമയൂരാദി സാഹിത്യമൂല്യം\\
സ്പന്ദിച്ചീടും തരംഗോജ്ജ്വലതരളിതമാണക്ഷരശ്ലോകസിന്ധു!
\end{slokam}

\Letter{ച}{സ}


\begin{slokam}{\VSr}{\VenM}{ചാടിൻ ചട്ടം ചവിട്ടി}
ചാടിൻ ചട്ടം ചവിട്ടിച്ചിതറിയതിൽ മുതിർന്നോരോമനക്കാലു പൊക്കി-\\
ച്ചാടുമ്പോള്‍ ചന്തി കുത്തിച്ചതുപുതയഥ വീണേറെ മേൽച്ചേറണിഞ്ഞും\\
ചാടുന്തിപ്പിച്ചവയ്ക്കും ചതിയുടയ ചലൽക്കണ്ണനാം കണ്ണനെച്ചാ-\\
ഞ്ചാടി\samd{ച്ചാരത്തു ചാരുസ്മിതരുചി ചിതറിക്കൊണ്ടു കണ്ടീടണം മേ}{ചാരത്തു ചാരുസ്മിതരുചി ചിതറിക്കൊണ്ടു കണ്ടീടണം മേ}!
\end{slokam}

\Letter{ച}{ച}


സമസ്യാപൂരണം. 



\begin{slokam}{\VSv}{\VKG}{ചായപ്പീടികയിൽ പ്രഭാത}
ചായപ്പീടികയിൽ പ്രഭാതസമയം പാലേകുവാൻ പോമിളം-\\
പ്രായക്കാരികൾ പിച്ചളത്തമല പൂഞ്ചായൽക്കു മേൽ വെച്ചിതാ\\
ഞായം വെച്ചു നടന്നിടുന്നു കുതുകാലമ്പാടിയിൽപ്പോൽ, ഘന-\\
ച്ഛായാസുന്ദരബാലകാനുഗമനോത്സാഹം പ്രതീക്ഷിപ്പു ഞാൻ! 
\end{slokam}

\Letter{ച}{ഞ}

\begin{slokam}{\VSr}{\VNM}{ചാരം, വെള്ളം, കരിത്തോൽ,}
ചാരം, വെള്ളം, കരിത്തോൽ, മഴു, വെരു - തിവയു, ണ്ടെപ്പൊഴും ഭൂതജാലം \\
ചാരത്തു, ണ്ടീ നിലയ്ക്കുള്ളൊരു പടുമലയൻ കെട്ടിയോളായ തായേ! \\
സ്വൈരം ത്രൈലോക്യവിത്താം തിരുമിഴിയെ വിത, ച്ചെൻ മനസ്സാം നിലത്തി-\\
ന്നേരം ഭക്തിക്കൃഷിക്കായ്ത്തുടരു, കിതതിനിജ്ജന്മി ചാർത്തിത്തരുന്നൂ!
\end{slokam}

\Letter{ച}{സ}

\begin{slokam}{\VSv}{\ARRV}{ചാലേ മാലിനിയും}
ചാലേ മാലിനിയും, മരാളമിഥുനം മേവും മണല്‍ത്തിട്ടയും,\\
ചോലയ്ക്കപ്പുറമായ്‌ മൃഗങ്ങള്‍ നിറയും ശൈലേന്ദ്രപാദങ്ങളും,\\
ചീരം ചാര്‍ത്തിന വൃക്ഷമൊന്നതിനടിയ്ക്കായിട്ടു കാന്തന്റെ മെയ്‌\\
ചാരി, ക്കൊമ്പിലിടത്തുകണ്ണുരസുമാ മാന്‍പേടയും വേണ്ടതാം.
\end{slokam}

\Letter{ച}{ച}

\Book{അഭിജ്ഞാനശാകുന്തളം പരിഭാഷ}. \OSlRef{കാര്യാ സൈകതലീനഹംസമിഥുനാ}.

\begin{slokam}{\VSr}{\Punam}{ചിന്നീടും കാന്തി}
ചിന്നീടും കാന്തി പൊന്നിൻ തളികയിൽ വലമായ് മാലയും വെച്ചു താരിൽ-\\ 
ത്തന്വംഗീഭംഗി കൈക്കൊ, ണ്ടരുവയർ നടുവേ മിന്നൽ പോലേ വിലോലാ \\
കണ്ണിന്നാനന്ദധാരാ കമലശരപരബ്രഹ്മവിദ്യേവ മൂർത്താ \\
ധന്യാ വന്നാളരങ്ങത്തതിമൃദുഹസിതാ മൈഥിലീ മന്ദമന്ദം. 
\end{slokam}

\Letter{ച}{ക}


\Book{ഭാഷാരാമായണചമ്പു}.

\begin{slokam}{\VSk}{\UN}{ചിരിക്കാം ചെന്തീയിൽ}
ചിരിക്കാം ചെന്തീയിൽ ഹൃദയമെരിയുമ്പോഴു, മിടറാ--\\
തിരിക്കാം പാദങ്ങൾ മണലടിയിലൂടൂർന്നിടുകിലും;\\
മരിക്കാനില്ലൊട്ടും വിഷമ, മൊരുവൻ പോലുമറിയാ;--\\
തൊരിക്കൽ കൂടിന്നിപ്പിറകുവിളി നൽകാതെ മനമേ!
\end{slokam}

\Letter{ച}{മ}

\begin{slokam}{\VSv}{\HM}{ചുറ്റും കെട്ടിയുയർത്തിവച്ച}
ചുറ്റും കെട്ടിയുയർത്തിവച്ച വെയിലിൻ ചൂളക്കളത്തിൽക്കൊടും-\\
കുറ്റം ചെയ്തവരെന്നപോൽ കതിരവൻ തള്ളുന്നു മാലോകരെ \\
പറ്റാതായൊരുമാത്രപോലുമിനിയും താപം സഹിച്ചീടുവാൻ,\\
വറ്റുംമുമ്പു, നിറഞ്ഞനീർക്കുടമുടൻ പോയ്കൊണ്ടുവാ കൊണ്ടലേ!
\end{slokam}

\Letter{ച}{പ}


\begin{slokam}{\VSr}{\PVRI}{ചുറ്റും മുണ്ടില്ല, ചീറ്റും}
ചുറ്റും മുണ്ടില്ല, ചീറ്റും ചില ഫണികളണിക്കോപ്പു, ഭൂതങ്ങളാണേ\\
ചുറ്റും, ചെന്തീയു ചിന്തും മിഴി, ചിത നടുവിൽ കേളി, ഗംഗയ്ക്കു ചിറ്റം,\\
ചുറ്റും നീയെന്നു താൻ ചൊന്നൊരു വടുവടിവായ്ച്ചൊന്ന വാക്കോടു ചിത്തം\\
ചെറ്റും ചേരാത്ത ഗൗരീനില കരളലിയെക്കണ്ട കണ്ണേ ജയിക്ക!
\end{slokam}

\Letter{ച}{ച}

\begin{slokam}{\VSv}{\VenV}{ചൂടായ്കിൽ തുളസീദളം}
ചൂടായ്കിൽ തുളസീദളം, യമഭടത്തല്ലിങ്ങു ചൂടായ്‌വരും;\\
പാടായ്കിൽ തിരുനാമ, മന്തകഭടന്മാരിങ്ങു പാടായ്‌വരും;\\
കൂടായ്കിൽ സുകൃതങ്ങള്‍ ചെയ്‌വതിനഹോ പാപങ്ങള്‍ കൂടായ്‌വരും;\\
വീടായികിൽ കടമേവനും നരകമാം നാടിങ്ങു വീടായ്‌വരും
\end{slokam}

\Letter{ച}{ക}

\begin{slokam}{\VSr}{\Unk}{ചൂടില്ലാത്തോരു ഫാലം}
ചൂടില്ലാത്തോരു ഫാലം, ചുടലയിൽ നടമാടാത്ത ചീലം, മതിത്തെൽ\\
ചൂടീടാത്തൊരു ചൂഡം, പരമൊരു പുഴകൂടാത കോടീരഭാരം,\\
ഓടും മാൻപേട തേടാതൊരു കരകമലം, ചാരുതെങ്കൈലയിൽപ്പോയ്‌\\
നീടാർന്നീടാത നാഥം, തരുണിയൊടയുതം,ദൈവതം നൈവ ജാനേ.
\end{slokam}

\Letter{ച}{ഒ}

\begin{slokam}{\VSv}{\VKG}{ചൂടും പൂവിനു ശണ്ഠകൂടുമിരു}
ചൂടും പൂവിനു ശണ്ഠകൂടുമിരുപേർദ്ദാരങ്ങള്‍, കള്ളും കുടി-\\
ച്ചാടും ജ്യേഷ്ഠ, നുഴന്നു കാട്ടിലലയും ബന്ധുക്കളും തോഴരും,\\
കൂടും പെൺകൊതിയാൽ പരന്റെ തടവിൽ പർത്തോരു പൗത്രൻ, ഹരേ!\\
വേടൻ തൻ കണ ശാഖിയിൽ തവ ശവം തൂങ്ങാതെ രക്ഷിച്ചതോ?
\end{slokam}

\Letter{ച}{ക}


\begin{slokam}{\VSr}{\CKP}{ചെന്നായിൻ ഹൃത്തിനും ഹാ}
ചെന്നായിൻ ഹൃത്തിനും ഹാ, ഭുവി നരഹൃദയത്തോളമയ്യോ, കടുപ്പം\\
വന്നിട്ടില്ലാ, ഭുജിപ്പൂ മനുജനെ മനുജൻ, നീതി കൂർക്കം വലിപ്പൂ,\\
നന്നാവില്ലിപ്രപഞ്ചം, ദുരയുടെ കൊടിയേ പൊന്തു, നാറ്റം സഹിച്ചും\\
നിന്നീടാനിച്ഛയെന്നോ? മഠയ, മനുജ, നീ പോകു, മിണ്ടാതെ പോകൂ!
\end{slokam}

\Letter{ച}{ന}


\begin{slokam}{\VSv}{\UN}{ചൊല്ലൂ, നിൻ വരവെപ്പൊൾ}
  "ചൊല്ലൂ, നിൻ വരവെപ്പൊ? ഴുച്ച കഴിയും മുമ്പോ? കഴിഞ്ഞീടുമോ\\
  തെല്ലും കൂടി? യതോ പകൽക്കൊടുവിലോ?" ചോദിച്ചു മൈക്കണ്ണിയാൾ;\\
  ഇല്ലം വിട്ടൊരു നൂറു നാൾകളകലെപ്പോകുന്ന ഭർത്താവിനെ--\\
  ത്തല്ലീട്ടിങ്ങനെ കണ്ണുനീരൊടു മുടക്കീടാൻ തുനിഞ്ഞീടിനാൾ.
\end{slokam}
  
\Letter{ച}{ഇ}

\Book{അമരുകശതകം പരിഭാഷ}. \OSlRef{പ്രഹരവിരതൌ മദ്ധ്യേ വാഹ്നസ്തതോ}


\begin{slokam}{\VSr}{\Unk}{ചേരുന്നീലാരുമായെൻ ശ്രുതി}
ചേരുന്നീലാരുമായെൻ ശ്രുതി, പിരിമുറുകിപ്പൊട്ടിടുന്നൂ വലിയ്ക്കും-\\
തോറും, താളം പിഴയ്ക്കുന്നിതു പലകുറിയും, കാലുറപ്പീല നിൽപ്പിൽ\\
മാരാരേ, ചെണ്ട കൊട്ടിക്കരുതിതു വിധമങ്ങെന്നെയും ശിഷ്യനാക്കീ-\\
ട്ടാരാലെൻ തെറ്റു തീർത്താ, ലുലകുമുഴുവനും കേളി കേള്‍പ്പിച്ചിടാം ഞാൻ!
\end{slokam}

\Letter{ച}{മ}


\begin{slokam}{\VSr}{\VKG}{ചേലക്കള്ളൻ ചിലപ്പോള്‍}
ചേലക്കള്ളൻ ചിലപ്പോള്‍, ചില സമയമൊടുങ്ങാതരക്കെട്ടു ചുറ്റാൻ\\
നീളത്തിൽപ്പട്ടു നൽകുന്നവ; നിടയനിട, യ്ക്കെപ്പൊഴും രാജരാജൻ;\\
ലീലാലോലൻ ചിലപ്പോ, ളഖിലസമയവും നിർഗ്ഗുണബ്രഹ്മ; - മെന്നെ-\\
പ്പോലുള്ളോരെന്തറിഞ്ഞൂ പുരഹരവിധിമാർ പോലുമോരാത്ത തത്ത്വം!
\end{slokam}

\Letter{ച}{ല}

\begin{slokam}{\VSv}{\VKG}{ചേലായാൽ മതി പെൺകുളിക്കടവിലെ}
ചേലായാൽ മതി പെൺകുളിക്കടവിലെച്ചേലങ്ങള്‍ കക്കും, മുല-\\
പ്പാലായാൽ മതിയാസ്വദിക്കണമലം ഹാലാഹലം ചേരിലും,\\
നീലാറ്റിൻ കരയാകിലും മതി രതിക്രീഡയ്ക്കു, കാട്ടോട തൻ\\
കോലായാൽ മതി പാടുവാൻ - ചതുരനോ തെമ്മാടിയോ നീ ഹരേ!
\end{slokam}

\Letter{ച}{ന}

\begin{slokam}{\VBh}{\PKV}{ചൊടിച്ചുഗ്രമാം കണ്ണു}
ചൊടിച്ചുഗ്രമാം കണ്ണു തിണ്ണെന്നുരുട്ടി\\
ത്തടിച്ചുള്ള കയ്യിൽ ഗദാ ദണ്ഡു മേന്തി\\
പിടിച്ചൂക്കു കൂടുന്നൊരഭ്യാസി, പല്ലും\\
കടിച്ചശു ഭീമൻ രണാഗ്രത്തിലെത്തി
\end{slokam}

\Letter{ച}{പ}


\end{enumerate}


\subsection{ഛ}
\begin{enumerate}

\begin{slokam}{\VSr}{\VKG}{ഛന്ദസ്സിൻ താളമാത്രാ}
ഛന്ദസ്സിൻ താളമാത്രാഗുരുലഘുയതിവിന്യാസസൗഭാഗ്യമുള്ളിൽ-\\
ത്തന്നത്താനേ വളർത്തി, സ്സഹൃദയസമുദായത്തിലംഗത്വമേകി,\\
അന്യൂനോച്ചാരണാർത്ഥസ്ഫുടത പരിചയം കൊണ്ടുറപ്പിച്ചു, ശിഷ്യർ-\\
ക്കന്നന്നായ്‌ പാഠമേകുന്നൊരു ഗുരുവരനാണക്ഷരശ്ലോകസൂരി!
\end{slokam}

\Letter{ഛ}{അ}


\begin{slokam}{\VSv}{\UN}{ഛായാഗ്രാഹകപൃഷ്ഠദർശനം} 
ഛായാഗ്രാഹകപൃഷ്ഠദർശന, മലർച്ചെണ്ടിന്റെ ചീയും മണം,\\
തീയൊക്കും വെയിലത്തു മേനികള്‍ വിയർത്തീടുന്നതിൽ  സ്പർശനം,\\
മായം ചേർത്തൊരു ഭക്ഷണം, ചെകിടടച്ചീടും വിധം ഭാഷണം,\\
നായന്മാർക്കു വിവാഹഘോഷണ, മഹോ! \sam{പഞ്ചേന്ദ്രിയാകർഷണം}!
\end{slokam}

\Letter{ഛ}{ത}


സമസ്യാപൂരണം. മറ്റു പൂരണങ്ങൾ: \SlRef{പുഞ്ചപ്പാടവരമ്പിലാടി}.



\end{enumerate}

\subsection{ജ}
\begin{enumerate}




\begin{slokam}{\VSr}{\Punam}{ജംഭപ്രദ്വേഷിമുമ്പിൽ}
ജംഭപ്രദ്വേഷിമുമ്പിൽ സുരവരസദസി ത്വദ്‌ഗുണൗഘങ്ങള്‍ വീണാ-\\
ശുംഭത്‌പാണൗ മുനൗ ഗായതി സുരസുദൃശാം വിഭ്രമം ചൊല്ലവല്ലേ\\
കുമ്പിട്ടാളുർവശിപ്പെ, ണ്ണകകമലമലിഞ്ഞൂ, മടിക്കുത്തഴിഞ്ഞൂ\\
രംഭ, യ്ക്കഞ്ചാറുവട്ടം കബരി തിരുകിനാള്‍ മേനകാ മാനവേദ!
\end{slokam}

\Letter{ജ}{ക}



\begin{slokam}{\VSv}{\UN}{ജാതിബ്ഭ്രാന്തരടക്കി വെച്ചൊരമലം}
ജാതിബ്ഭ്രാന്തരടക്കി വെച്ചൊരമലം ശാസ്ത്രീയസംഗീതമാം\\
സ്ഫീതാഭാന്വിതപേടകത്തിലെ മണിച്ചിത്രക്കൊടുംതാഴിനെ\\
വീതായാസമഴി, ച്ചതിങ്കൽ നിധി പോൽ വാഴും സ്വരങ്ങൾക്കെഴും \\
രീതിയ്ക്കൊത്തവയെയ്തിടുന്ന വിരുതേറീടുന്ന യോദ്ധാവു  നീ!
\end{slokam}

\Letter{ജ}{വ}

യേശുദാസിനെപ്പറ്റി.

\begin{slokam}{\VSr}{\KVPR}{ജാതീ, ജാതാനുകമ്പാ ഭവ}
ജാതീ, ജാതാനുകമ്പാ ഭവ, ശരണമയേ! മല്ലികേ, കൂപ്പുകൈ തേ\\
കൈതേ, കൈതേരി മാക്കം കബരിയിലണിവാൻ കയ്യുയർത്തും ദശായാം\\
ഏതാ, നേതാൻ മദീയാനലർശരപരിതാപോദയാ, നാശു നീ താൻ\\
നീ താൻ, നീ താനുണർത്തീടുക ചടുലകയൽക്കണ്ണി തൻ കർണ്ണമൂലേ!
\end{slokam}

\Letter{ജ}{എ}


\begin{slokam}{\VSv}{\DSN}{ജ്ഞാനത്തിന്നു നിധാനമായ്}
ജ്ഞാനത്തിന്നു നിധാനമായ്, നിഗമവിദ്യയ്ക്കുള്ളടിസ്ഥാനമായ്,\\
കാണപ്പെട്ട സമസ്തവും മധുരമായ് മേളിച്ച മഞ്ജൂഷയായ്,\\
ക്ഷേമത്തിന്റെ വിധാനമായ്, പരമതത്ത്വത്തിൻ മണിക്കോവിലായ്,\\
ശോഭിക്കുന്നു പരാശരാത്മജമഹദ്ഗ്രന്ഥം മഹാഭാരതം!
\end{slokam}

\Letter{ജ}{ക}

\begin{slokam}{\VSv}{\KA}{ജ്ഞാനം താൻ ക്ഷണവൃത്തി}
ജ്ഞാനം താൻ ക്ഷണവൃത്തി; ബുദ്ധിയിമവെട്ടുമ്പോളിടയ്ക്കാത്തമഃ-\\
സ്ഥാനം കാണുവതില്ലയാതവലയത്തോടൊത്ത വേഗത്തിനാൽ;\\
നൂനം ഹാ! ക്ഷണമൃത്യുവിങ്ങനുദിനം നിദ്രാഖ്യമായ് ദീർഘമാ-\\
മൂനം വിട്ടതു നീണ്ടനന്തരമിതാം വിശ്രാന്തി ജന്തുക്കളിൽ.
\end{slokam}

\Letter{ജ}{ന}

\Book{പ്രരോദനം}.


\begin{slokam}{\VSv}{\VRV}{ജ്ഞാനം, - വിശ്വവികാസസംസ്കൃതി}
ജ്ഞാനം, - വിശ്വവികാസസംസ്കൃതി,യുമ- ന്നെത്തിച്ച വിജ്ഞാന, - മെന്‍ \\ 
പ്രാണസ്പന്ദനതാളമാക്കി വിടരും ചൈതന്യമായ്‌ വന്നു ഞാന്‍. \\
ഞാനദ്ധ്വാന, - മെനിക്കുവേണ്ടിയുണരും നുറ്റാണ്ടിലെ സംക്രമ-\\
സ്ഥാനത്തെന്തിതൊരന്ധകാരശിഖരം നീ നട്ടു വേദാന്തമേ? 
\end{slokam}

\Letter{ജ}{ഞ}

\Book{അദ്ധ്വാനത്തിൻ വിയർപ്പാണു ഞാൻ}

\end{enumerate}

\subsection{ഞ}
\begin{enumerate}

\begin{slokam}{\VSr}{\KKT}{ഞാനെന്നാൽ ഞായരക്ഷയ്ക്കൊരു}
ഞാനെന്നാൽ ഞായരക്ഷയ്ക്കൊരു ഗുണവഴിയേ പോകയാം തിങ്കളൊക്കും\\
മാനം ചൊവ്വായ്‌ വഹിക്കും ബുധനതിമതിമാൻ വ്യാഴതുല്യപ്രഭാവൻ\\
നൂനം പൊൻവെള്ളിയെന്നീവക ശനിനിയതം വിദ്യതാൻ വിത്തമെന്നാ\\
ജ്ഞാനം മേ തന്നൊരച്ഛൻ കനിയണമിഹമേ വെണ്മണിക്ഷ്മാസുരേന്ദ്രൻ.
\end{slokam}

\Letter{ഞ}{ന}



\begin{slokam}{\VSv}{\VRV}{ഞാനീ ഗ്രീഷ്മസരോവരത്തിൽ}
ഞാനീ ഗ്രീഷ്മസരോവരത്തിൽ വിടരും ചെന്താമരപ്പൂവിലും,\\
നാണിച്ചീവഴി നൃത്തമാടിയൊഴുകും കാട്ടാറിലും, കാറ്റിലും,\\
ധ്യാനിക്കുന്ന കലാചലത്തിലലിയും മൗനത്തിലും, കണ്ടു നിൻ\\
വീണക്കമ്പിയിലംഗുലീമുനകളാൽ നീ തീർത്ത കാവ്യോത്സവം.
\end{slokam}

\Letter{ഞ}{ധ}

\Book{ഗ്രാമദർശനം}

\begin{slokam}{\VSv}{\VRV}{ഞാനീ ജാലകവാതിലിൽ}
ഞാനീ ജാലകവാതിലിൽ ചെറുമുളന്തണ്ടിൽ ഞൊറിഞ്ഞിട്ടതാ-\\
ണീ നീലത്തുകിൽ ശാരദേന്ദുകലയെപ്പാവാട ചാർത്തിക്കുവാൻ \\
ഹാ, നിത്യം ചിറകിട്ടടിച്ചു ചിതറിക്കീറിപ്പറപ്പിച്ചുവോ\\
ഞാനിസ്സർഗ്ഗതപസ്സമാധിയിലിരിക്കുമ്പോൾ കൊടുങ്കാറ്റുകൾ?
\end{slokam}

\Letter{ഞ}{ഹ}


\Book{സർഗ്ഗസംഗീതം}.


\begin{slokam}{\VSv}{\KKT}{ഞാനോ മാനിനിമാര്‍ക്കു}
ഞാനോ മാനിനിമാര്‍ക്കു മന്മഥനഹോ! ശാസ്ത്രത്തിലെന്നോടെതിര്‍-\\
പ്പാനോ പാരിലൊരുത്തനില്ല, കവിതയ്ക്കൊന്നാമനാകുന്നു ഞാന്‍;\\
താനോരോന്നിവയോര്‍ത്തുകൊണ്ടു ഞെളിയേണ്ടെന്‍ ചിത്തമേ! നിശ്ചയം\\
താനോ ജീവനൊരസ്ഥിരത്വമതിനാല്‍ നിസ്സാരമാണൊക്കെയും.
\end{slokam}

\Letter{ഞ}{ത}



\begin{slokam}{\VSv}{\KJ}{ഞെട്ടും താവകനാമമൊന്നു}
ഞെട്ടും താവകനാമമൊന്നു ചെവിയിൽ തട്ടുമ്പൊഴേ ദുഷ്ടർ, കു-\\
മ്പിട്ടും കൊണ്ടു വണങ്ങിടും ഗിരിശ! ദിക്കൊട്ടുക്കു കാത്തീടുവോർ, \\
പൊട്ടും കോൾമയിരംബിക, യ്ക്കസുവിനെക്കെട്ടുന്ന പാശം യമൻ \\
കെട്ടും പിൻ തിരിയും, മുറയ്ക്കു സുകൃതം കിട്ടും വിടാതേവനും. 
\end{slokam}

\Letter{ഞ}{പ}

\Topic{അഷ്ടപ്രാസം}. 

\end{enumerate}

\subsection{ട}
\begin{enumerate}

\begin{slokam}{\VSv}{\EP}{ടിക്കറ്റിന്നു തപസ്സുചെയ്യണം}
ടിക്കറ്റിന്നു തപസ്സുചെയ്യണ, മൊരോ കഷ്ടം സഹിക്കേണ, മ-\\
പ്പെട്ടിക്കെട്ടുകള്‍, മെത്ത, കൂജ, പലതും കെട്ടിപ്പെറുക്കീടണം;\\
മുട്ടിത്തട്ടി മുഷിഞ്ഞു, കാശു മുഴുവൻ ദീപാളി, കോമാളിയായ്‌\\
നാട്ടിൽപ്പോക്കു നടത്തിടുന്ന മലയാളത്താനു കൈകൂപ്പണം!
\end{slokam}

\Letter{ട}{മ}


\end{enumerate}

\subsection{ഠ}
\begin{enumerate}

\begin{slokam}{\VSv}{\VKG}{ഠാണാവിൽജ്ജനനം}
ഠാണാവിൽജ്ജനനം, വളർന്നതിടയപ്പെണ്ണിന്റെ കൈത്തൊട്ടിലിൽ,\\
പ്രാണൻ കാത്തതൂ പെൺകൊലക്കൊടുമയാൽ, വിദ്യാലയം ഗോഗൃഹം, \\
ക്ഷോണീരംഭകളാം വ്രജാംഗനകൾ തൻ ജാരത്വമുദ്യോഗ, മെ- \\
ന്താണാവോ തവ മേന്മ?  കംസവധമോ പാർത്ഥന്റെ സാരഥ്യമോ?
\end{slokam}

\Letter{ഠ}{ക}


\end{enumerate}

\subsection{ത}
\begin{enumerate}


\begin{slokam}{\VSv}{\HM}{തട്ടിൻമേലെയിടയ്ക്കു മൂഷികഗണം}
തട്ടിൻമേലെയിടയ്ക്കു മൂഷികഗണം പാറാവിനായ് കൃത്യമായ് \\
തട്ടിത്തട്ടിയണഞ്ഞു രാപ്പകുതിയിൽ പ്പിന്നെപ്രഘോഷങ്ങളായ് \\
ഒട്ടിച്ചേർന്നുകിടന്ന നിദ്രയുടനെപ്പേടിച്ചുണർന്നെന്നെയി- \\
ങ്ങിട്ടിട്ടോടി മറഞ്ഞുപോയ്, ഒടുവിലീ നാൽക്കാലിയെൻ കൂട്ടുമായ്!
\end{slokam}

\Letter{ത}{ഒ}



\begin{slokam}{\VSr}{\CN}{തണ്ടാർമാതാം രമയ്ക്കോ}
തണ്ടാർമാതാം രമയ്ക്കോ, തരളമിഴി മലത്തയ്യലാളാമുമയ്ക്കോ,\\
കൊണ്ടാടും മേനകയ്ക്കോ, സരസിജമുഖിയാമുർവ്വശിക്കോ, ശചിക്കോ,\\
വണ്ടാർപൂവേണിമാർ വന്നടിമലർ പണിയും ഭാരതിക്കോ, രതിക്കോ,\\
കണ്ടാൽ സൗന്ദര്യമേറുന്നതു മമ പനയഞ്ചേരി നാരായണിക്കോ?
\end{slokam}

\Letter{ത}{വ}

\begin{slokam}{\VKm}{\UN}{തന്തയുള്ളതു തണുപ്പനാണു}
തന്തയുള്ളതു തണുപ്പനാണു, സഹജൻ സ്വതേ കൊടിയ ഭീരുവും,\\
തൻ തലയ്ക്കസുഖമുള്ള രണ്ടു സുതർ, പാമ്പു മാത്രമണിയാനുമേ,\\
കാന്തനോ ചുടലവാസി, തെണ്ടി, വിഷമാണു ഭക്ഷണ, മൊരുത്തിയെ-\\
ക്കോന്തനായ് തലയിലേറ്റു, മെന്റെ \sam{ഗിരിജേ കടുത്തു തവ ജാതകം}!
\end{slokam}

\Letter{ത}{ക}


സമസ്യാപൂരണം. 


\begin{slokam}{\VSr}{\KJ}{തന്നില്ലല്ലോ വിധാതാ}
തന്നില്ലല്ലോ വിധാതാ മധുരിമ മമ വാക്കിന്നു, ദൂതൊന്നയയ്ക്കാൻ\\
വന്നില്ലല്ലോ മരാളം, മുകിൽ മയിലിവയും ദൂരെയാം ജീവനാഥേ!\\
നിർന്നിദ്രം ഞാൻ നയിയ്ക്കും നിശകളിലറിയാതെന്റെ കൺകോൺ കവിഞ്ഞി-\\
ങ്ങുന്നിദ്രം വീണ ബാഷ്പോദകകണികകൾ താനെന്റെ സന്ദേശകാവ്യം
\end{slokam}

\Letter{ത}{ന}

\begin{slokam}{\VVt}{\Ull}{തയ്യായ നാളിലലിവാർന്നൊരു}
തയ്യായ നാളിലലിവാർന്നൊരു തെല്ലു നീർ തൻ\\
കയ്യാലണപ്പവനു കാമിതമാക നൽകാൻ\\
അയ്യായിരം കുല കുലയ്പ്പൊരു തെങ്ങുകള്‍ക്കു-\\
മിയ്യാളുകള്‍ക്കുമൊരു ഭേദമശേഷമില്ല.
\end{slokam}

\Letter{ത}{അ}

\Book{ഉമാകേരളം}.


\begin{slokam}{\VSv}{\VKG}{തല്പത്തിന്നരികത്തണഞ്ഞു}
തല്പത്തിന്നരികത്തണഞ്ഞിടുകിലെന്നുള്ളം തുടിക്കുന്നു, വൻ\\
വീർപ്പോലുന്നു, മദീയനാഡി പരിശോധിക്കെപ്പനിക്കുന്നു മേ; \\
ഇപ്പാഥോജമനോഹരാനന വെറും നേഴ്സായിരിക്കും വരെ-\\
ക്കൈപ്പേറുന്ന മരുന്നിനാലൊഴികയില്ലെന്നാമയം നിശ്ചയം! 
\end{slokam}

\Letter{ത}{ഇ}


\begin{slokam}{\VKm}{\KV}{താമരയ്ക്കു ശശിയോടുമില്ലിഹ}
താമരയ്ക്കു ശശിയോടുമില്ലിഹ ശശിക്കു താമരയൊടും തഥാ\\
പ്രേമ, മെന്നതു നിമിത്തമേതുമൊരു ചേതമില്ലതിനു രണ്ടിനും\\
സാമരസ്യനിലയാണു വേണ്ടതഭിരാമരാമവരു തങ്ങളിൽ\\
കാമമിന്നതുളവായിടായ്കിലയശസ്സതീവ നിയതിക്കു താൻ!
\end{slokam}

\Letter{ത}{സ}

\Book{അന്യാപദേശശതകം പരിഭാഷ}.

\begin{slokam}{\VSr}{\VKG}{തായയ്ക്കും താതനും നിൻ}
തായയ്ക്കും താതനും നിൻ ജനനകഥയറിഞ്ഞന്നുതൊട്ടേ തുറുങ്കിൽ-\\
ച്ചായാറായീ, യശോദാദികളസുരഭടദ്രോഹഭീയാൽ വലഞ്ഞൂ\\
ആയർപ്പെണ്ണുങ്ങള്‍ വെണ്ണക്കളവിലുമലരമ്പിങ്കലും പമ്പരം പോ-\\
ലായീ കാർവർണ്ണ, നീയാർക്കഭയമരുളിയെന്നൊന്നു ചൊല്ലിത്തരാമോ?
\end{slokam}


\Letter{ത}{അ}

\begin{slokam}{\VSr}{\Ull}{തായാരാകട്ടെ, തൻ}
തായാരാകട്ടെ, തൻ നാടുടയ പുകളെഴും തമ്പുരാനാട്ടെ, യാരും\\
ന്യായാപേതം നടന്നാ, ലവരുടെ തലയിൽക്കണ്‌ഠകോടാലിയൊപ്പം\\
പായാറാക്കി, ബ്ഭയത്തെപ്പകലിരവൊരുപോൽ പാപികള്‍ക്കുള്ളിലേകും\\
മായാവിപ്രേന്ദുവിൽച്ചെന്നടിയുക മനമേ, മറ്റിടം ചുറ്റിടാതെ!
\end{slokam}

\Letter{ത}{പ}

\begin{slokam}{\VSr}{\ONN}{താരത്തിൽ കണ്ടിടുന്നൂ}
താരത്തിൽ കണ്ടിടുന്നൂ ചില പെരിയ ജനം, നിത്യവും നിന്നെ മൂലാ-\\
ധാരത്തിൽ പിന്നെ വേറേ ചില, രപരജനം താമരത്താരിനുള്ളിൽ, \\
സാരത്തെത്തേടിടുന്നോരൊഴു പൊഴുതുമഹോ!  നിന്നെയല്ലാതൊരന്യാ-\\
കാരത്തെക്കണ്ടിടുന്നില്ലയി, മധുരസമുദ്രോദ്ഭവേ, ഭൂർഭുവസ്വ! 
\end{slokam}

\Letter{ത}{സ}


\begin{slokam}{\VSr}{\VNM}{താരാപുഷ്പങ്ങൾ ചിന്നി}
താരാപുഷ്പങ്ങൾ ചിന്നി, പ്പല പറവകൾ തൻ കൊഞ്ചലാൽ സ്തോത്രമോതി,\\
സ്ഫാരാകാശച്ചെരാതിൽത്തരുണദിനകരപ്പൊൻവിളക്കും കൊളുത്തി,\\
ആരാഗക്കാവിവസ്ത്രത്തൊടു നിയതമുഷഃസന്ധ്യയാരെപ്പിഴയ്ക്കാ-\\
താരാധിക്കുന്നു, നമ്മൾക്കുണർവരുളണമസ്സർവ്വലോകൈകനാഥൻ!
\end{slokam}

\Letter{ത}{അ}

\begin{slokam}{\VSv}{\VenM}{താരാഹാരമലങ്കരിച്ചു}
താരാഹാരമലങ്കരിച്ചു, തിമിരപ്പൂഞ്ചായൽ പിന്നോക്കമി-\\
ട്ടാ രാകേന്ദുമുഖത്തിൽ നിന്നു കിരണസ്മേരം ചൊരിഞ്ഞങ്ങനെ\\
ആരോമൽ കനകാബ്ജകോരകകുചം തുള്ളിച്ചൊരാമോദമോ-\\
ടാരാലംഗനയെന്ന പോലെ നിശയും വന്നാളതന്നാളഹോ!
\end{slokam}

\Letter{ത}{അ}

\begin{slokam}{\VSr}{\Punam}{താരിൽത്തന്വീകടാക്ഷാഞ്ചല}
താരിൽത്തന്വീകടാക്ഷാഞ്ചലമധുപകുലാരാമ! രാമാജനാനാം\\
നീരിൽത്താർബാണ! വൈരാകരനികരതമോമണ്ഡലീചണ്ഡഭാനോ!\\
നേരെത്താതോരു നീയാം തൊടുകുറി കളകായ്കെന്നുമേഷാ കുളിക്കും\\
നേരത്തിന്നിപ്പുറം വിക്രമനൃവര! ധരാ ഹന്ത! കൽപാന്തതോയേ.
\end{slokam}

\Letter{ത}{ന}

\begin{slokam}{\VSr}{\Unk}{താരുണ്യാഭോഗഭാരത്തൊടും}
താരുണ്യാഭോഗഭാരത്തൊടു, മതിപൃഥുകാദിത്യബിംബത്തിനേക്കാ-\\
ളാരുണ്യം തേടുമംഗത്തൊടു, മുലകലിയും മന്ദഹാസങ്ങളോടും,\\
കാരുണ്യക്കൂത്തരങ്ങാകിയ കടമിഴിയാൽ കോഴവിട്ടേഴുമേഴും\\
പാരെണ്ണികാത്തുപോരും പശുപതിപകുതിദ്ദേഹമേ, ദേഹി സൗഖ്യം!
\end{slokam}

\Letter{ത}{ക}



\begin{slokam}{\VSv}{\HM}{താളം തെറ്റിയ ജീവിത}
താളം തെറ്റിയ ജീവിതക്രമവുമായ് മുന്നോട്ടു പോകുമ്പൊഴും,\\
നീളും തീവ്രമനിശ്ചിതത്ത്വമഴലായ് തൂങ്ങിക്കിടക്കുമ്പൊഴും\\
നാളം മങ്ങിയ മൺവിളക്കു പകരും പ്രത്യാശതൻ നാമ്പുമായ്\\
നാളത്തെപ്പുലരിക്കു,കൂരിരുളുമായ് മല്ലിട്ടു നീങ്ങട്ടെ ഞാൻ!
\end{slokam}

\Letter{ത}{ന}

\begin{slokam}{\VSr}{\KAM}{തിണ്ണം ചെന്നിട്ടു തീയിൽ}
തിണ്ണം ചെന്നിട്ടു തീയിൽ തെളിവിനൊടു തിളയ്ക്കുന്ന പാലൊട്ടു പൊന്നിൻ\\
കിണ്ണം കൊണ്ടമ്മ കാണാതളവിലുടനുടൻ മുക്കി, മുക്കിൽ പതുങ്ങി\\
കർണ്ണം പാർത്തങ്ങു നിന്നിട്ടതു ചൊടിയിണകൊണ്ടൂതിയൂതിക്കുടിക്കും\\
കണ്ണൻ കാരുണ്യപൂർണൻ കളകമലദളക്കണ്ണനെൻ കണ്ണിലാമോ?
\end{slokam}

\Letter{ത}{ക}


\begin{slokam}{\VSv}{\PCM}{തീരാഞ്ഞോ കൊതി, കട്ടവെണ്ണ}
തീരാഞ്ഞോ കൊതി, കട്ടവെണ്ണ കഴിയെക്കൈ നക്കിയും കന്നുതൻ\\
ചാരെപ്പിന്നെയണഞ്ഞു താട തടവിക്കൊഞ്ചിച്ചിരിച്ചങ്ങനെ\\
ചൗര്യത്തിൻ കഥ ചൊല്ലിടുന്ന ഹരിയെദ്ദർശിച്ചു ഹർഷാശ്രുവായ്‌\\
ദൂരത്തമ്മ, യടുത്തു നിന്നു പശു, ഞാൻ ഹൃത്താം തൊഴുത്തിങ്കലും.

\end{slokam}

\Letter{ത}{ച}

\begin{slokam}{\VSr}{\KochT}{തുങ്ഗശ്രീസിംഹവാഹേ}
തുങ്ഗശ്രീസിംഹവാഹേ! തുഹിനശിഖരിതൻ കന്യകേ! നിസ്തുലാഭേ!\\
ഭൃങ്ഗാളീകേശി! ചാപബ്‌ഭൃകുടി! മൃഗസമാനാക്ഷി! കുംഭസ്തനാഢ്യേ!\\
ഭങ്ഗം മീനാക്ഷി! തീർത്തീടുക മധുമഥനാജാദിസേവ്യേ! വൃഷാങ്കോ-\\
ത്സങ്ഗശ്രീസൗമ്യഗേഹേ! ഭഗവതി! കടകോല്ലാസിഹസ്തേ നമസ്തേ!

\end{slokam}

\Letter{ത}{ഭ}

\begin{slokam}{\VSv}{\KND}{തുമ്പീ തുള്ളുക, തുള്ളിയാർക്കുക,}
തുമ്പീ തുള്ളുക, തുള്ളിയാർക്കുക, രസം മുറ്റുന്ന കാറ്റേ, മലർ-\\
ത്തുമ്പേ, കമ്പിതകമ്രകുദ്മളരസാനമ്രേ, പതിഞ്ഞാടുക;\\
എൻ പച്ചക്കിളി, യൊന്നു വായ്ക്കുരയിടൂ; നിൽക്കുന്നു മുറ്റത്തതാ\\
മുൻപിൽ സ്വാർജ്ജിതനിർജ്ജരാർജ്ജുനയശോവൃദ്ധൻ, ബലിത്തമ്പുരാൻ!

\end{slokam}

\Letter{ത}{എ}

\begin{slokam}{\VSr}{\HM}{തുള്ളിപ്പാറിപ്പറക്കും സുമുഖി}
തുള്ളിപ്പാറിപ്പറക്കും സുമുഖി ശലഭമേ! വർണ്ണവൈവിദ്ധ്യമോലും\\
പുള്ളിപ്പൂമ്പട്ടുടുപ്പിന്നുറവിടമറിയാനുണ്ടെനിക്കൊട്ടു മോഹം\\
മുള്ളിൽത്തട്ടിത്തടഞ്ഞാലിഴകളകലുമോ? ശക്തമാണെങ്കിലൊന്നെൻ\\
പുള്ളിക്കാരിക്കുവാങ്ങാം തിരികെയവിടെ ഞാനെത്തിടുമ്പോൾ കൊടുക്കാൻ!
\end{slokam}

\Letter{ത}{മ}



\begin{slokam}{\VSv}{\VKG}{തൂവെണ്ണപ്രിയനാണു}
"തൂവെണ്ണപ്രിയനാണു, പാൽക്കൊതിയനാ, ണുത്തുംഗഗോവർദ്ധന-\\
ക്കുന്നേറ്റുന്നവനാണു പാവമിവ"നെന്നുള്ളോരപഖ്യാതിയെ\\
രാധേ! ഞാൻ കളയാൻ ശ്രമിച്ചു പഴുതേ - നിന്മേനി തൂവെണ്ണതാൻ,\\
ക്ഷീരം പുഞ്ചിരി, നിന്റെ പോർമുലകളോ ഞാനേറ്റിടും കുന്നുകള്‍!
\end{slokam}

\Letter{ത}{ര}



\begin{slokam}{\VSr}{\VKG}{തൂവെള്ളിപ്പർവ്വതത്തിൻ}
തൂവെള്ളിപ്പർവ്വതത്തിൻ കൊടുമുടിയിലിരിപ്പാ, ണിരപ്പാളിയാ, ണ-\\
പ്പൂവമ്പദ്വേഷിയാ, ണത്തനു പകുതിയുമയ്ക്കുള്ളതാണുള്ളൊടൊപ്പം,\\
വേവിക്കും തീയു കണ്ണിൽ, കരുണയുടെ കളിത്തട്ടു, മങ്ങെന്തുമേവം\\
ഭാവിക്കും പോലെയാടും നവനടനകലാദക്ഷ, മാം രക്ഷ രക്ഷ!
\end{slokam}

\Letter{ത}{വ}

\begin{slokam}{\VSv}{\TMV}{തെണ്ടേണം പല ദിക്കിൽ നാഥനു}
തെണ്ടേണം പല ദിക്കിൽ നാഥനു തുണയ്‌, ക്കെന്നാലുമന്നന്നു കോൽ\\
കൊണ്ടേറെ പ്രഹരം സഹിക്കണമഹോ പെട്ടത്തലയ്ക്കാണതും.\\
പണ്ടേ നീ പരതന്ത്രനാം, കയർ വരിഞ്ഞംഗങ്ങള്‍ ബദ്ധങ്ങളായ്‌,\\
ചെണ്ടേ നിന്റെയകത്തെ വേദന പുറത്താരുണ്ടറിഞ്ഞീടുവാൻ?

\end{slokam}

\Letter{ത}{പ}


\begin{slokam}{\VSv}{\VenM}{തെണ്ടീട്ടാണശനം}
തെണ്ടീട്ടാണശനം, തുണിയ്ക്കു പകരം തോലാണുടുത്തീടുവാൻ,\\
പണ്ടം പന്നഗമാണു, കണ്ട ചുടലക്കാടാണിരുന്നീടുവാൻ,\\
തണ്ടാർസായകവൈരിയാണു ഭഗവൻ! സർവ്വജ്ഞനാണെങ്കിലും\\
രണ്ടാളുണ്ടു കളത്രമെന്റെ ശിവനേ! ചിത്രം ചരിത്രം തവ!
\end{slokam}
 
\Letter{ത}{ത}

\begin{slokam}{\VKm}{\NV}{തെന്നലാം മണിരഥത്തിലേറി}
തെന്നലാം മണിരഥത്തിലേറി, മുഴുതിങ്കളാം കുട പിടിച്ചു, നൽ-\\
ക്കന്നലാം ചിലയിൽ മൂളിടുന്നൊരളി ഞാണിണക്കിയുലകൊക്കെയും,\\
തന്നടിക്കടിയിലാക്കുമാ, റലർശരങ്ങൾ തൂകി മധുര സ്മിതൻ\\
വന്നു ചിത്രയുടെ മുന്നിൽ നിന്നു മധുസംയുതൻ മദനനോതിനാൻ
\end{slokam}

\Letter{ത}{ത}

\begin{slokam}{\VSr}{\VKG}{തേരോടിക്കെ, ക്കടക്കണ്മുന}
തേരോടിക്കെ, ക്കടക്കണ്മുന കണവനിലർപ്പിച്ചതേയുള്ളു ധീരം\\
പോരാടിപ്പിക്കുവാൻ തൻ സ്വജനമഹിതമായ്‌ കണ്ടനേരം സുഭദ്ര;\\
തേരോടിക്കെക്കിരീടിക്കഖിലപതി മിനക്കെട്ടു വേദാന്ത ചിന്താ-\\
സാരം ചൊല്ലേണ്ടിവന്നൂ, \sam{കമനിയുടെ കടക്കണ്ണു ഗീതയ്ക്കു മീതെ}!
\end{slokam}

\Letter{ത}{ത}


സമസ്യാപൂരണം. 



\begin{slokam}{\VSv}{\VKG}{തേടിത്തേടി നടന്നു കാലടി}
 തേടിത്തേടി നടന്നു കാലടി കഴയ്ക്കട്ടേ, ഭവത്കീർത്തനം\\
പാടിപ്പാടി വരണ്ടൂണങ്ങുകിലുണങ്ങീടട്ടെ ജിഹ്വാഞ്ചലം\\
കൂടെക്കൂടെ നടത്തുമർച്ചന തളർത്തീടട്ടെ കൈ രണ്ടു, മി-\\
ക്കൂടാത്മാവു വെടിഞ്ഞിടും വരെ ഹരേ! നിന്നെ സ്മരിച്ചാവു ഞാൻ!

\end{slokam}

\Letter{ത}{ക}

\begin{slokam}{\VSv}{\KA}{തേനഞ്ചീടിന 'ഗാഥ'യാലൊരു}
 തേനഞ്ചീടിന 'ഗാഥ'യാലൊരു മഹാൻ താരാട്ടി മുമ്പമ്പിയ-\\
ന്നാനന്ദാശ്രുവിൽ മുക്കി മറ്റൊരു മഹാധന്യൻ 'കിളിക്കൊഞ്ചലാൽ'\\
ദീനത്വം കലരാതെയന്യസരസൻ 'തുള്ളിച്ചു' തൻ പാട്ടിനാൽ\\
നൂനം കൈരളിയമ്മയും ശിശുവുമായ്‌ നിന്നാളവർക്കന്നഹോ!

\end{slokam}

\Letter{ത}{ദ}

\Book{പ്രരോദനം}.


\begin{slokam}{\VKm}{\KV}{തേളു തുച്ഛമൊരു}
തേളു തുച്ഛമൊരു കീടകം പരമിതെന്തുചെയ്യുമൊരെറുമ്പിനെ-\\
ക്കാളുമില്ല പണി കൊല്ലുവാനിതിനെ വാഴുമെത്രയിതു വാഴ്കിലും;\\
ആളുകള്‍ക്കു പുനരെന്തുപേടി, യവർ പേരുകേട്ടുമുടനോടിടും;\\
കാളുമുഗ്രവിഷമുള്ള വാൽമുനയതിന്റെ തീവ്രത കഥിപ്പതോ!
\end{slokam}

\Letter{ത}{അ}

\Book{അന്യാപദേശശതകം പരിഭാഷ}.
\OSlRef{കീടഃ കശ്ചന വൃശ്ചികഃ,}.


\begin{slokam}{\VSv}{\VNM}{ത്രാസം നൽകിയുണർന്ന}
ത്രാസം നൽകിയുണർന്ന കാളിയഫണീന്ദ്രൻ തൻ ഫണത്തിങ്കലും,\\
രാസക്രീഡയിൽ വല്ലവീനടുവിലായ്‌ വൃന്ദാവനത്തിങ്കലും,\\
ഹാ, സർവ്വോത്തമയാകുമാ മുരളി തൻ പാട്ടേറ്റുപാടീ ധൃതോ-\\
ല്ലാസം, നർത്തനമാടി തത്ക്കവിതയാള്‍ ഗോവിന്ദനൊന്നിച്ചുതാൻ.
\end{slokam}

\Letter{ത}{ഹ}

\Book{കവിത}.

\begin{slokam}{\VVt}{\AUK}{ത്വരിതമുക്തിയിലാശ}
ത്വരിതമുക്തിയിലാശയുദിച്ചു നിൻ-\\
ചരിതമീശ്വരി വാഴ്ത്തിടുവോർക്കയേ\\
ഭരിതസൌഖ്യമെഴുന്നതിനില്ല പോൽ\\
പരിധി വാരിധി വാനിവ പോലവേ.
\end{slokam}

\Letter{ത}{ഭ}

\Book{സഹസ്രദളം}.

\Topic{യമകം (ദ്രുതവിളംബിതം)} 


\begin{slokam}{\VVt}{\VNM}{ത്വിട്ടോലുമക്ഷികള്‍}
ത്വിട്ടോലുമക്ഷികള്‍, നരച്ചു വളർന്നു മാറിൽ\\
തൊട്ടോരു താടി, ചുളിവീണു പരന്ന നെറ്റി\\
മുട്ടോളമെത്തിയ ഭുജാമുസലങ്ങളെന്നീ-\\
മട്ടോടവൻ വിലസി മേദുര ദീർഘകായൻ

\end{slokam}

\Letter{ത}{മ}

\Book{ബന്ധനസ്ഥനായ അനിരുദ്ധൻ}.



\end{enumerate}
\subsection{ദ}
\begin{enumerate}


\begin{slokam}{\VSr}{\NKD}{ദിക്കാലാതീതസദ്}
ദിക്കാലാതീതസദ്വാസന തവ ഹൃദയഗ്രന്ഥി ഭേദിച്ചു വാർന്നോ-\\
രക്കാലം കൊല്ലുകൊല്ലാശ്രമമൃഗമിതു താനെന്നു ലോകം കയർക്കേ,\\
കക്കാടേ, നീ പൊറുത്തൂ കവിതയെ വളയാൻ കാടു കേറിത്തിമിർക്കും\\
ധിക്കാരം പണ്ടു, പിന്നീടനവസരപുരസ്കാരസത്കാരപൂരം.
\end{slokam}

\Letter{ദ}{ക}

കവി കക്കാടിനോട്. 


\begin{slokam}{\VSr}{\ONN}{ദുഷ്ക്കർമ്മം ചെയ്തിരിയ്ക്കാമഹം}
ദുഷ്ക്കർമ്മം ചെയ്തിരിയ്ക്കാമഹ, മതുമുഴുവൻ ചിത്രഗുപ്തൻ കണക്കിൻ-\\
ബുക്കിൽ കൊള്ളിച്ചിരിയ്ക്കാം, യമനുമതിനു കണ്ടോട്ടെയെന്നായിരിയ്ക്കാം,\\
മുക്കണ്ണപ്രാണനാഥേ! ഭഗവതി! തവ തൃക്കാലെഴും കാലമാരും\\
മുഷ്ക്കെന്നിൽ ചെയ്യുമെന്നുള്ളൊരു ഭയമടിയന്നില്ല പുല്ലാണിതെല്ലാം.

\end{slokam}

\Letter{ദ}{മ}

\begin{slokam}{\VMk}{\KV}{ദൂനം ദൂരസ്ഥിതദയിതനായ്}
ദൂനം ദൂരസ്ഥിതദയിതനായേതുമാശ്വാസമില്ലാ-\\
തേനം ദീനം ജനമനു കനി‍ഞ്ഞൊന്നു ചെയ്താലുമിപ്പോൾ \\
സ്യാനന്ദൂരം പുരവരമതിൽ ചെന്നു മൽപ്രാണനാഥ-\\
യ്ക്കാനന്ദം നീയരുളുക പറഞ്ഞെന്റെ സന്ദേശവാക്യം.
\end{slokam}

\Letter{ദ}{സ}

\Book{മയൂരസന്ദേശം}.

\begin{slokam}{\VSv}{\VRV}{ദൈവം വന്നതു തത്ത്വചിന്തകൾ}
 ദൈവം വന്നതു തത്ത്വചിന്തകൾ തെളിച്ചെത്തിച്ച തേർത്തട്ടിലോ?\\
ഭാവങ്ങൾക്കഭിരാമരൂപമരുളും ശില്പീന്ദ്രശില്പത്തിലോ?\\
പൂവർപ്പിച്ച പുരോഹിതന്റെ ഭജനപ്പാട്ടിന്റെ മാറാപ്പിലോ?\\
പാവം മിഥ്യ പണിഞ്ഞുയർത്തിയ തമോരൂപപ്രപഞ്ചത്തിലോ?
\end{slokam}


\Letter{ദ}{പ}

\end{enumerate}

\subsection{ധ}
\begin{enumerate}

\begin{slokam}{\VSv}{\UN}{ധാതാവെത്ര മനോഹരാംഗനകളെ}
ധാതാവെത്ര മനോഹരാംഗനകളെസ്സൃഷ്ടിച്ചു പോൽ, ഏറ്റവും\\
ചേതോഹാരിണിമാർ മുഴുക്കെ സിനിമാതാരങ്ങളായ്ത്തീർന്നഹോ!\\
ഏതായാലുമതൊന്നു പോലെ, യൊരുവൾ കാന്താംഗിയെൻ മക്കൾ തൻ\\
\sam{മാതാവായിടുവാൻ തരൂ വിധി വരും ജന്മത്തിലെൻ നാന്മുഖാ}!
\end{slokam}


\Letter{ധ}{എ}


സമസ്യാപൂരണം. 



\begin{slokam}{\VSv}{ചുനക്കര ഉണ്ണിക്കൃഷ്ണവാരിയർ}{ധിഗ്ധിഗ്‌ രാക്ഷസരാജ}
ധിഗ്ധിഗ്‌ രാക്ഷസരാജ! ദുഷ്പരിഭവം വായ്പിച്ചു നിന്‍ ദോര്‍ബ്ബലം\\
വിദ്യുജ്ജിഹ്വവിപത്തി മാത്രമെളുതാമങ്ങേയ്ക്കു നീചപ്രഭോ!\\
കഷ്ടം, നിസ്ത്രപ! നോക്കു, കണ്ണിരുപതും ചേര്‍ക്കൂ, വെറും താപസന്‍\\
കുട്ടിക്രീഡയില്‍ വാളിളക്കിയതിനാല്‍ നിന്‍ പെങ്ങളീ മട്ടിലായ്‌!
\end{slokam}

\Letter{ധ}{ക}

\Book{നിരനുനാസികപ്രബന്ധം പരിഭാഷ}.

\begin{slokam}{\VSv}{\VNM}{ധ്യാനം നിർത്തി, വിശാലനീല}
 ധ്യാനം നിർത്തി, വിശാലനീലനയനദ്വന്ദ്വം തുറന്നപ്പൊഴേയ്‌-\\
ക്കാനന്ദോൽപുളകാംഗി തന്റെ കുടിലിൻ മീതേ നഭോവീഥിയിൽ\\
ദീനത്രാണപരായണം ത്രിജഗതീനാഥന്റെ തൃക്കയ്യതാ\\
നൂനം വാരി വിതച്ചിടുന്നു വളരെസ്സൗവർണ്യനാണ്യങ്ങളെ!
\end{slokam}

\Letter{ധ}{ദ}

\Book{ഒരു സന്ധ്യാപ്രണാമം}.


\end{enumerate}

\subsection{ന}
\begin{enumerate}

\begin{slokam}{\VSr}{\PG}{നഞ്ഞാളും കാളിയൻ തൻ}
 നഞ്ഞാളും കാളിയൻ തൻ തലയിലു, മതുപോലക്കുറൂരമ്മയാകും\\
കുഞ്ഞാത്തോൽ പാലുകാച്ചും കരികലമതുതന്നുള്ളിലും, തുള്ളിയോനേ!\\
ഇഞ്ഞാനെന്നുള്ള ഭാവക്കറയധികതരം പൂണ്ടു, മാലാണ്ടുപോമെൻ\\
നെഞ്ഞാം രങ്ഗത്തു തങ്കത്തളകളിളകി നീ നിത്യവും നൃത്തമാടൂ!
\end{slokam}

\Letter{ന}{ഇ}

\Book{നാൽക്കാലികൾ}.

\begin{slokam}{\VSr}{\VKG}{നന്ദിയ്ക്കെൻ നന്ദി നാഥാ}
"നന്ദിയ്ക്കെൻ നന്ദി നാഥാ, പഴനിയുടെ സമീപത്തിൽ നാമിത്ര വേഗം\\
വന്നല്ലോ, ചിത്ര, മുണ്ണിക്കുടയ മയിലതാ പാമ്പിനെത്തിന്നു നിൽപ്പൂ";\\
"വന്ദ്യം വൃന്ദാവനം താനിതു, കനകലതാകമ്രയാം രാധയെസ്സാ-\\
നന്ദം പിഞ്ഛാവതംസൻ..." ഗിരിജയുടെ മുഖം നമ്രമായ്‌, താമ്രമായീ!
\end{slokam}

\Letter{ന}{വ}


\begin{slokam}{\VSr}{\ARSK}{നന്നിശ്ശബ്ദപ്രഘോഷം വിരുത}
നന്നിശ്ശബ്ദപ്രഘോഷം വിരുത ജലനിധേ! നിന്നലച്ചാർത്തിതെല്ലാ-\\
മൊന്നിച്ചുച്ചണ്ഡവേഗത്തൊടു കടുതരനിർദ്ധ്വാനമൊത്താപതിയ്ക്കേ,\\
നിന്നിൽ ദാഹം കെടുത്താനൊരു തവി കുടിനീർപോലുമില്ലെന്നതോർക്കാ-\\
തുന്നിദ്രം ത്വന്മഹത്ത്വം കവികളെവിടെയും പാടിടും മോടിയിൽത്താൻ!
\end{slokam}

\Letter{ന}{ന}


\begin{slokam}{\VSv}{\KAM}{നന്നോ മെയ്യണിവാനുമേ}
 "നന്നോ മെയ്യണിവാനുമേ ഫണി?", "രമേ, മെത്തയ്ക്കു കൊള്ളാം!"; "കണ-\\
ക്കെന്നോ കാളയിതേറുവാനനുദിനം?", "മേച്ചീടുവാനുത്തമം!";\\
"എന്നാലെന്നുമിരന്നിടുന്നതഴകോ?", "കക്കുന്നതിൽ ഭേദമാ"-\\
ണെന്നാക്കുന്നലർമങ്കമാരുടെ കളിച്ചൊല്ലിങ്ങു താങ്ങാകണം!
\end{slokam}

\Letter{ന}{എ}

\Topic{ഉമാരമാസംവാദം}. \SeeAlso{പിച്ചക്കാരൻ ഗമിച്ചാനെവിടെ}, 
\SeeAlso{കാടല്ലേ നിന്റെ ഭർത്താവിനു}, \SeeAlso{കുന്നിൻനാട്ടിലെ ബാന്ധവം}.

\begin{slokam}{\VSv}{\VKG}{നാണിക്കുന്ന നവോഢയെപ്പരുഷമായ്‌}
നാണിക്കുന്ന നവോഢയെപ്പരുഷമായ്‌ കെട്ടിപ്പിടിക്കുന്നതും,\\
ഘ്രാണിക്കാന്‍ ത്വരയാര്‍ന്നു കൊച്ചുമുകുളം നുള്ളിപ്പൊളിക്കുന്നതും,\\
ആണത്തം പൊടിമീശയില്‍ തെളിയുവാന്‍ ചായം പുരട്ടുന്നതും,\\
കാണിപ്പൂ മധുരാനുഭൂതി തടയും മര്‍ത്ത്യക്ഷമാശൂന്യത.
\end{slokam}

\Letter{ന}{അ}

\begin{slokam}{\VSv}{\UN}{നാദബ്രഹ്മമഹാഗ്നിതന്നിൽ}
നാദബ്രഹ്മമഹാഗ്നിതന്നിലലിവോടാവിശ്വകർമ്മാവെടു--\\
ത്തൂതിക്കാച്ചിയ സ്വർണ്ണമേ, നിഖിലലോകത്തിന്റെ സായുജ്യമേ,\\
ശ്രോതാക്കള്‍ക്കമരത്വമെന്നുമരുളും പീയൂഷമേ, സാഹിതീ--\\
ശ്രീതാവും മലയാളഭാഷയുടെ സത്സൗഭാഗ്യമേ, സ്വാഗതം!
\end{slokam}

\Letter{ന}{ശ}

യേശുദാസിനെപ്പറ്റി.


\begin{slokam}{\VSv}{\VRV}{നാദം ശൂന്യതയിങ്കലാദ്യമമൃതം}
നാദം ശൂന്യതയിങ്കലാദ്യമമൃതം വര്‍ഷിച്ച നാളില്‍, ഗതോ-\\
ന്മാദം വിശ്വപദാര്‍ത്ഥശാലയൊരിടത്തൊന്നായ്‌ തുടിച്ചീടവേ,\\
ആ ദാഹിച്ചു വിടര്‍ന്ന ജീവകലികാജാലങ്ങളില്‍, കാലമേ,\\
നീ ദര്‍ശിച്ച രസാനുഭൂതി പകരൂ മത്‌ പാനപാത്രങ്ങളില്‍
\end{slokam}

\Letter{ന}{അ}

\Book{സർഗ്ഗസംഗീതം}.

\begin{slokam}{\VSr}{\VRV}{നാദം, താളം, വെളിച്ചം,}
നാദം, താളം, വെളിച്ചം, നിഴൽ, നിറമിവയാൽ നൃത്തശിൽപം രചിക്കും\\
കാലത്തിൻ കമ്രനാഭീനളിനകലികയിൽ വീണ തൂമഞ്ഞുതുള്ളി\\
നാളത്തെപ്പൊന്നുഷസ്സിൻ പ്രമദവനികയിൽ കൽപനാപത്മരാഗ-\\
ത്താലത്തിൽ കാഴ്ച വെക്കാൻ പ്രകൃതിയുടെ കലാശാല ഞാൻ തേടിവന്നൂ!
\end{slokam}

\Letter{ന}{ന}


\Book{ഗ്രാമദർശനം}


% Rajesh: നാദം, സ്പന്ദം, വെളിച്ചം, (Rajesh) 
\begin{slokam}{\VSr}{\RV}{നാദം, സ്പന്ദം, വെളിച്ചം,}
നാദം, സ്പന്ദം, വെളിച്ചം, സമയ, മളവെഴാപ്പഞ്ചഭൂതങ്ങളെല്ലാം\\
തോതിൽച്ചേർത്തിട്ടു വിശ്വം പണിവൊരു വിരുതൻ ബ്രഹ്മനാശാരിപോലും\\
ആർ തൻ നോക്കിൻ മുഴക്കോൽ ഞൊടിയിട വെടിയാതേ മുറുക്കെപ്പിടിക്കു-\\
ന്നായമ്മേ വന്നു ചെമ്മേ നടമരുളണമേ നാവിലമ്മേ ചിരം മേ.
\end{slokam}

\Letter{ന}{അ}


\begin{slokam}{\VSr}{\Unk}{നാലാമ്‌നായൈകമൂലം}
നാലാമ്‌നായൈകമൂലം, നതജനദിവിഷത്‌പാദപം, നേത്രവഹ്നി-\\
ജ്വാലാനിർദ്ദഗ്ദ്ധമീനദ്ധ്വജ, മചലസുതാരൂഢവാമാങ്കഭാഗം,\\
കാലാരാതിം, കപർദ്ദോദരകബളിതമന്ദാകിനീമാനനീയം,\\
കൈലാസാവാസലോലം, കനിവൊടു മനമേ ചന്ദ്രചൂഡം ഭജേഥാ!
\end{slokam}

\Letter{ന}{ക}


\begin{slokam}{\VSv}{\UN}{നാവെപ്പോള്‍ മുരളുന്നതും}
നാവെപ്പോള്‍ മുരളുന്നതും പരുഷമാം ഹുങ്കാരമാണെങ്കിലും,\\
ഭാവം താളമിതൊക്കെയെന്റെ ധിഷണയ്ക്കപ്രാപ്യമാണെങ്കിലും,\\
നീ വാഗ്വർഷിണി, നൂപുരധ്വനിയുതിർത്തെത്തീടവേ, കേള്‍ക്കുവാ-\\
നാവും മച്ഛ്രുതികള്‍ക്കു – ഞാനവനിയിൽ  സംഗീതമേ, ഭാഗ്യവാൻ!
\end{slokam}

\Letter{ന}{ന}


\begin{slokam}{\VSr}{\KKT}{നാളീകാക്ഷന്റെ മാറത്തരുളിന}
നാളീകാക്ഷന്റെ മാറത്തരുളിന രമ തൻ കയ്യിലാടുന്ന കേളീ-\\
നാളീകത്തിങ്കലെത്തേൻ ഹരിയുടെ തിരുനാഭിക്കകം വീണിടുമ്പോള്‍\\
നാളീകാവാസനാകും ശിശുവിനിവള്‍ ദരം കൊണ്ടു പാൽ നൽകിടുന്നെ-\\
ന്നാളീടുന്നാദരാൽ ദേവകള്‍ കരുതുമിതേകട്ടെ നമ്മള്‍ക്കു ശുദ്ധം.
\end{slokam}

\Letter{ന}{ന}

പരിഭാഷ.  \OSlRef{ശ്രീരാജീവാക്ഷവക്ഷസ്ഥല}


\begin{slokam}{\VSv}{\Balendu}{നിത്യം തെണ്ടുവതെത്ര നീചം}
"നിത്യം തെണ്ടുവതെത്ര നീച, മരുതേ" - യര്‍ത്ഥിച്ചുപോല്‍ ഷണ്മുഖന്‍,\\
"മറ്റെന്തുണ്ടൊരു മാര്‍ഗ്ഗ"മെന്നു കളിയായ്‌ ചോദിച്ചുപോലീശ്വരന്‍,\\
പെട്ടെന്നോതിയൊരാറു ജോലികള്‍ മുറയ്ക്കോരോന്നുമോരോ മുഖം:\\
"നൃത്തം, യുദ്ധ, മുടുക്കുകൊട്ടു, കഥനം, നീര്‍സേചനം, ശിക്ഷണം!"
\end{slokam}

\Letter{ന}{പ}
 
\SeeAlso{അംബാ കുപ്യതി താത}. 

\begin{slokam}{\VSv}{\UN}{നിത്യം ശ്ലോകസദസ്സിൽ}
നിത്യം ശ്ലോകസദസ്സിലോർമ്മയെയരിച്ചത്യന്തഹൃദ്യങ്ങളാം\\
പദ്യങ്ങള്‍ പരിചോടെടുത്തരുളിടും സ്തുത്യർഹരാം പണ്ഡിതർ\\
മുക്തന്മാർ മുനിമാരുമെന്നുമൊരുമിച്ചുള്‍ത്താരിലാശിച്ച പോ--\\
ലെത്തുന്നൂ പരമം പദം സകലദം സത്യം ശിവം സുന്ദരം!
\end{slokam}

\Letter{ന}{മ}

\Topic{അഷ്ടപ്രാസം}. 


\begin{slokam}{\VSv}{\VKG}{നിന്നാദ്യസ്മിത, മാദ്യചുംബനം}
നിന്നാദ്യസ്മിത, മാദ്യചുംബന, മനുസ്യൂതസ്ഫുരന്മാധുരീ-\\
മന്ദാക്ഷം, പുളകാഞ്ചിതസ്തനയുഗം, പ്രേമാഭിരാമാനനം,\\
കുന്ദാസ്ത്രോത്സവചഞ്ചലത്പൃഥുനിതംബശ്രീസമാശ്ലേഷസ-\\
മ്പന്നാനന്ദമഹോ മനോഹരി! മരിപ്പിക്കും സ്മരിപ്പിച്ചു നീ!
\end{slokam}

\Letter{ന}{ക}


\begin{slokam}{\VSv}{\ARRV}{നിന്‍ നേത്രത്തിനു തുല്യമാം}
നിന്‍ നേത്രത്തിനു തുല്യമാം കുവലയം വെള്ളത്തിനുള്ളത്തിലായ്‌\\
നിന്നാസ്യപ്രഭ തേടുമമ്പിളിയൊളിക്കപ്പെട്ടു കാര്‍കൊണ്ടലാല്‍\\
അന്നത്തന്വികള്‍ നിന്നൊടൊത്ത നടയുള്ളോരങ്ങുമണ്ടീടിനാര്‍,\\
നിന്നൌപമ്യവുമിന്നുകാണ്‍മതു പൊറുക്കുന്നില്ലഹോ ദുര്‍വിധി.
\end{slokam}

\Letter{ന}{അ}

\Book{ഭാഷാഭൂഷണം}.

പരിഭാഷ.  \OSlRef{യത്ത്വന്നേത്രസമാനകാന്തി}.

\begin{slokam}{\VSv}{\Unk}{നിൻ നേത്രത്തൊടു നേരെനിക്കു}
നിൻ നേത്രത്തൊടു നേരെനിക്കു രജനീ, വൈവശ്യമോമന്മുല-\\
ക്കുൻേറാടൊക്കു, മുറക്കമൂണുസുഖമെന്റിത്യാദി മദ്ധ്യോപമം,\\
ധന്യേ, മാരതുരാൽ നിതംബസദൃശം നിന്നെപ്പിരിഞ്ഞീടിനാൽ\\
നിന്നെക്കാണ്മതിനുണ്ടുപായമിവയോരോന്റോർത്തു കൌണത്തരേ!
\end{slokam}

\Letter{ന}{ധ}

\Book{പദ്യരത്നം}.

\begin{slokam}{\VSv}{വൈക്കം സി. എൻ. രാമൻ പിള്ള}{നിൻപത്രം മൃദുകോമളം}
നിൻപത്രം മൃദുകോമളം; മധുരമാണല്ലോ ഫലം വാസനാ–\\
സമ്പത്താർന്നു സുവർണ്ണഭംഗി കലരും നിൻകേസരം സുന്ദരം;\\
വൻപൊക്കും തനുകാന്തിയും നളിനമേ! യോഗ്യൻ ഭവാനെങ്കിലും\\
ജൃംഭിക്കുന്നൊരു കണ്ടകം നിറയുമിത്തണ്ടിണ്ടൽ ചേർക്കുന്നു മേ.
\end{slokam}

\Letter{ന}{വ}

\Book{അന്യാപദേശശതകം പരിഭാഷ}.


\begin{slokam}{\VSr}{\Vyl}{നിൻ ഭവ്യം പെയ്തു നൂറ്റാണ്ടരുളിയ}
നിൻ ഭവ്യം പെയ്തു നൂറ്റാണ്ടരുളിയ തിരുനാൾ,  കേരളവ്യാസ, വിദ്യാ- \\
രംഭം രണ്ടും നിമിത്തം കലയുടെ സഖിയാം കന്നി ഹാ ധന്യയായി;\\
ദംഭം വിട്ടിന്നു തൃക്കാൽപ്പൊടികൾ തടവുമീപ്പൂഴിയിൽക്കാവ്യവിദ്യാ-\\
രംഭം പെയ്യട്ടെ വീണ്ടും കവികളിവിടെ, യെന്നാലിനിത്തോലിയുണ്ടോ?
\end{slokam}

\Letter{ന}{ദ}

കൊടുങ്ങല്ലൂർ കുഞ്ഞിക്കുട്ടൻ തമ്പുരാനെപ്പറ്റി. 


\begin{slokam}{\VSr}{\KKT}{നിൽക്കട്ടേ പേറ്റുനോവിൻ കഥ}
നിൽക്കട്ടേ പേറ്റുനോവിൻ കഥ, രുചികുറയും കാല, മേറും ചടപ്പും\\
പൊയ്ക്കോട്ടേ, കൂട്ടിടേണ്ടാ മലമതിലൊരു കൊല്ലം കിടക്കും കിടപ്പും,\\
നോക്കുമ്പോള്‍ ഗർഭമാകും വലിയ ചുമടെടുക്കുന്നതിൻ കൂലി പോലും\\
തീർക്കാവല്ലെത്ര യോഗ്യൻ മകനു, മതു നിലയ്ക്കുള്ളൊരമ്മേ തൊഴുന്നേൻ!
\end{slokam}

\Letter{ന}{ന}

\OSlRef{ആസ്താം താവദിയം}.


\begin{slokam}{\VSv}{\VRV}{നീയാം സ്നേഹപയോധരത്തെ}
നീയാം സ്നേഹപയോധരത്തെയൊരുനാളെത്തിപ്പിടിച്ചേൻ, ഞൊറി-\\
ഞ്ഞീ യാഗാശ്രമമൺവിളക്കിനരുകിൽ രാമാംബരം നീർത്തുവാൻ;\\
മായാംഭോധി കടഞ്ഞുയർന്ന കവിതേ! നീ നിന്റെയന്തർമ്മുഖ-\\
ശ്രീയാലെന്നിൽ വിരിച്ച ദിവ്യസുരഭീപുഷ്പങ്ങളോർക്കുന്നു ഞാൻ
\end{slokam}

\Letter{ന}{മ}

\begin{slokam}{\VSv}{\VRV}{നീയിന്ത്യയ്ക്കൊരു ശാപമായി}
നീയിന്ത്യയ്ക്കൊരു ശാപമായിവരുമെന്നാരോർത്തു! യജ്ഞപ്പുക-\\
ത്തീയിൽപ്പണ്ടു കുരുത്ത മാനവമഹാസംസ്കാരമല്ലല്ലി നീ?\\
ചായില്യങ്ങള്‍ വരച്ച പൊയ്മുഖവുമായ്‌ നിൻ മന്ത്രവാദം നിന-\\
ക്കീയില്ലത്തു നിറുത്തുവാൻ സമയമായില്ലേ, സമൂഹാന്ധതേ?
\end{slokam}

\Letter{ന}{ച}

\Book{അദ്ധ്വാനത്തിൻ വിയർപ്പാണു ഞാൻ}

\begin{slokam}{\VVt}{\GSK}{നീരന്ധ്രനീലജലദ}
നീരന്ധ്രനീലജലദപ്പലകപ്പുറത്തു\\
വാരഞ്ചിടുന്ന വളർവില്ലുവരച്ചുമായ്ച്ചും\\
നേരറ്റ കൈവളകളാൽ ചില മിന്നൽ ചേർത്തും\\
പാരം ലസിക്കുമമലപ്രകൃതിക്കു കൂപ്പാം
\end{slokam}

\Letter{ന}{ന}

\begin{slokam}{\VSv}{\VNM}{നീരന്ധ്രാളകമിന്ദ്രനീല, മമലം}
നീരന്ധ്രാളകമിന്ദ്രനീല, മമലം പല്ലൊക്കെ മു, ത്തുത്സ്മിതം\\
ഹീരം, മൽപ്രിയ തന്റെ ചുണ്ടു പവിഴം, പൂമേനി ഗോമേദകം,\\
ആ രത്നങ്ങള്‍ വശത്തിലുള്ളവനിതാ സ്വൽപം ധനം നേടുവാൻ\\
ദൂരത്തേയ്ക്കു ഗമിക്കയാണു -- മഹിതം നിൻ പ്രാഭവം ലോഭമേ!
\end{slokam}

\Letter{ന}{അ}



\begin{slokam}{\VSv}{\VKG}{നീരാടും ജട, നീറണിഞ്ഞ}
നീരാടും ജട, നീറണിഞ്ഞ തിരുമെയ്‌, നീറുന്ന തൃക്ക, ണ്ണുമാ--\\
നീരന്ധ്രപ്രണയാഭിഷിക്തഹൃദയം, നഞ്ഞാണ്ട കണ്ഠസ്ഥലം,\\
കാളാഹിച്ചുരുള്‍ കങ്കണം, ശില ഗൃഹം, കാളപ്പുറം തേർത്തടം,\\
കാലാരേ! ചുടലക്കളക്കളരിയാശാനേ! നമിക്കുന്നു ഞാൻ!
\end{slokam}

\Letter{ന}{ക}


\begin{slokam}{\VSv}{\VRV}{നീലക്കണ്ണുകളോ, ദിനാന്ത}
നീലക്കണ്ണുകളോ, ദിനാന്തമധുരസ്വപങ്ങള്‍തൻ ചന്ദന-\\
ച്ചോലയ്ക്കുള്ളിൽ വിടർന്നു പാതിയടയും നൈവേദ്യപുഷ്പങ്ങളോ,\\
കാലം കൊത്തിയെടുത്ത ഹംസദമയന്തീശിൽപമിന്നും നള-\\
ന്നാലങ്കാരിക ഭംഗിയോടെയെഴുതും സന്ദേശകാവ്യങ്ങളോ?
\end{slokam}

\Letter{ന}{ക}

\begin{slokam}{\VSr}{\KJ}{നീലാകാശത്തിലേറും}
നീലാകാശത്തിലേറും പറവകളമലം, മാരിവില്ലപ്രമേയം,\\
ബാലാദിത്യപ്രകാശം പ്രതിനവസുഖദം, വർഷമാകർഷണീയം,\\
ജാലം ലോലംബജാലം, മലരൊളിമധുരം, മാതൃസംശുദ്ധശിക്ഷാ-\\
ലോലം, ഹാ ബാല്യകാലം! മതിവരെ നുകരാൻ കാത്തുനിന്നീല കാലം!
\end{slokam}

\Letter{ന}{ജ}


\begin{slokam}{\VSr}{\Unk}{നീലാകാശപ്പരപ്പോ തവ തനു}
നീലാകാശപ്പരപ്പോ തവ തനു, തിരുനട്ടത്തിലൊന്നായഴിഞ്ഞാ-\\
ലോലാഭോഗം ഭവപ്പൂങ്കുഴലഴകിൽ വിളങ്ങുന്നതോ മേഘജാലം?\\
കാലാരിപ്പെൺകിടാവേ, വിധുമുഖി, വിളയാട്ടത്തിൽ നിൻ മന്ദഹാസം\\
പാലാഴിക്കോളിളക്കം പടി വിലസുവതോ വെണ്ണിലാവാരറിഞ്ഞൂ?
\end{slokam}

\Letter{ന}{ക}


\begin{slokam}{\VSv}{\VRV}{നീ വന്നെത്തിയതീ യുഗപ്രതിഭതൻ}
നീ വന്നെത്തിയതീ യുഗപ്രതിഭതൻ തേജസ്സിലോ, ഞങ്ങളെ-\\
ദ്ദൈവങ്ങള്‍ക്കു നടയ്ക്കുവെച്ച വിധിതൻ മായാവിമാനത്തിലോ\\
പുവർപ്പിച്ച പുരോഹിതന്റെ ഭജനപ്പാട്ടിന്റെ മാറാപ്പിലോ\\
പാവം മിഥ്യ തെളിച്ചു തന്ന പൊളിയാറായോരു തേർത്തട്ടിലോ?
\end{slokam}

\Letter{ന}{പ}


\Book{ഗ്രാമദർശനം}

\begin{slokam}{\VSv}{\Unk}{നീറും തീപ്പൊരി കണ്ണിലും}
നീറും തീപ്പൊരി കണ്ണിലും, നിറമെഴും ചന്ദ്രൻ ശിരസ്സിങ്കലും,\\
ചീറും പാമ്പു കഴുത്തിലും ചെറുപുലിത്തോൽ നല്ലരക്കെട്ടിലും,\\
സാരംഗം മഴുവും കരങ്ങളിലുമങ്ങീശന്നു ചേരും പടി-\\
\samd{യ്ക്കാറും പിന്നെയൊരാറുമെന്നിവ ഗണിച്ചീടുമ്പൊളേഴായ്‌ വരും}{ആറും പിന്നെയൊരാറുമെന്നിവ ഗണിച്ചീടുമ്പൊളേഴായ്‌ വരും}.
\end{slokam}

\Letter{ന}{സ}


സമസ്യാപൂരണം. 


\begin{slokam}{\VSr}{\VKG}{നൂറ്റാണ്ടിൽപ്പാതിയോളം}
നൂറ്റാണ്ടിൽപ്പാതിയോളം പകലിരവുമഹങ്കാരചർക്കയ്ക്കുമേലേ\\
നൂറ്റേൻ ഹാ! പാപനൂലിൻ കഴികളതു കൃപാലോല! ഞാൻ നെയ്തെടുത്തു;\\
ചുറ്റിക്കാണുന്നൊരിജ്ജീവിതവസനമുപേക്ഷിച്ചു,നിൻ കാൽ തുടയ്ക്കാൻ\\
പേറ്റെടും തോർത്തുമുണ്ടൊന്നിവനിനി വിരചിച്ചീടുവാൻ നേരമുണ്ടോ?
\end{slokam}

\Letter{ന}{ച}

\begin{slokam}{\VSv}{\SVL}{നെഞ്ഞത്തിന്നലെ രാത്രി പൂച്ച}
"നെഞ്ഞത്തിന്നലെ രാത്രി പൂച്ച കടികൂടിച്ചാടി വീണോ, നിറം-\\
മാഞ്ഞെന്തീ വടു ചുണ്ടി"ലെന്നു സഖിമാരോതിച്ചിരിക്കും വിധൗ\\
കുഞ്ഞമ്മിഞ്ഞ കുറച്ചൊളിച്ചൊരു വിധം ചെഞ്ചുണ്ടു പൊത്തി, ഹ്രിയാ\\
ഞഞ്ഞമ്മിഞ്ഞ പറഞ്ഞിടുന്നൊരചലക്കുഞ്ഞേ, കനിഞ്ഞീടു നീ!
\end{slokam}

\Letter{ന}{ക}


\begin{slokam}{\VSr}{\Vyl}{നെഞ്ഞിൽക്കാണാവു നാം}
നെഞ്ഞിൽക്കാണാവു നാ, മക്കതിരൊളി തിരളും ധീരനെ, ബ്ഭസ്മരേഖാ-\\
മഞ്ജുശ്രീഫാലപാർശ്വേ കലയുടെ കുറി ചൂടുന്നൊരത്തമ്പുരാനെ,\\
കുഞ്ഞിക്കയ്യാൽ വലിക്കെ, ക്കവിത കസവു നീട്ടുന്ന തോള്‍മുണ്ടു ചേരും\\
കുഞ്ഞിക്കുട്ടാഖ്യനെ, സ്സുസ്മിതസിതമണിയാം വിസ്മയം തഞ്ചുവോനെ.
\end{slokam}

\Letter{ന}{ക}

കുഞ്ഞിക്കുട്ടൻ തമ്പുരാനെപ്പറ്റി.


\begin{slokam}{\VMl}{\Ull}{നെടിയ മല കിഴക്കും}
നെടിയ മല കിഴക്കും നേരെഴാത്താഴി മേക്കും\\
വടിവിലെലുകയായിത്തഞ്ചിടും വഞ്ചിനാടേ!\\
അടിയനിതറിയിക്കാമബ്ധികാഞ്ചിക്കു നീയേ\\
മുടിനടുവിൽ വിളങ്ങും മുഖ്യമാണിക്യരത്നം.
\end{slokam}

\Letter{ന}{അ}

\Book{ഉമാകേരളം}.


\begin{slokam}{\VVt}{\Kund}{നെന്മേനിവാകമലർമേനി}
നെന്മേനിവാകമലർമേനി വെടിഞ്ഞുവാഴ്‌ത്തും\\
തന്മേനി കണ്ടു മലർവില്ലനെ വെന്ന വമ്പൻ\\
തന്മേനി നേർപകുതി താനെഴുമൂരകത്തു-\\
ള്ളമ്മേ! നിനക്കുടയ കാലിണ കൈതൊഴുന്നേൻ
\end{slokam}

\Letter{ന}{ത}

\Topic{പച്ചമലയാളശ്ലോകം}

\begin{slokam}{\VSr}{തോലൻ}{നേന്ത്രയ്ക്കായ് നാലു കീറി}
നേന്ത്രയ്ക്കാ നാലു കീറി, പ്പുനരതു ചതുരാകാരഖണ്ഡം നുറുക്കി-\\
ച്ചന്തത്തിൽച്ചാരുമോരിൽത്തദനു കറ കളഞ്ഞുഷ്ണതോയത്തിലിട്ട്\\
നെയ്യിൽ ഭൂയോ വറുത്തിട്ടഴകൊടു ഗുളവും ജീരകം ചുക്കുമെല്ലാം\\
നന്നായ് ചേർത്തങ്ങു വെച്ചാ, ലമൃതിനു സമമാം ശർക്കരോപ്പേരി കൊണ്ട്വാ
\end{slokam}

\Letter{ന}{ന}

\begin{slokam}{\VSr}{\RV}{നേരമ്പോക്കായി മാത്രം}
നേരമ്പോക്കായി മാത്രം രതിയെ, മദിരയെ സ്വച്ഛമാം വെള്ളമായും\\
നാരിത്വത്തെച്ചരക്കായ്‌, സഹജ മനുജനെക്കേവലം കക്ഷിയായും\\
ധർമ്മത്തെബ്ഭിക്ഷയായും കവനമൊരു വെറും കൗതുകം മാത്രമായും\\
നേരംപോൽ പാർത്തു കാണാൻ നിപുണത തികയും കൈരളീഭാഷ വെൽക.
\end{slokam}

\Letter{ന}{ഭ}

\begin{slokam}{\VSv}{\NDK}{നോക്കിക്കാണുകിലെന്തു}
നോക്കിക്കാണുകിലെന്തു മെച്ചമറുപത്തഞ്ചാംവയസ്സിൻ പഴം-\\
ചാക്കിൽ തൂക്കിയ ലാഭചേതമിവിടെത്തട്ടിക്കിഴിച്ചീടുകിൽ?\\
നീക്കിക്കാട്ടുവതിന്നു മേൽഗതി നമുക്കെന്നാളുമേ കൈവശം\\
ബാക്കി, ബ്ലാങ്കൊരു ബാങ്കുചെക്കു കവിതേ നിന്നക്ഷരശ്ലോകമേ!
\end{slokam}

\Letter{ന}{ന}

\begin{slokam}{\VSv}{\KND}{നോവാറ്റും കുളിർതെന്നലൊന്നിളകിയാൽ}
നോവാറ്റും കുളിർതെന്നലൊന്നിളകിയാൽക്കൂടി, ക്കുഴഞ്ഞാടിടും\\
തൈവാകച്ചെറുചില്ലമേൽ സുഖമിരുന്നാലോലമാലോലമായ്,\\
പൂവാലൻകിളി,നീ കുലുക്കി വിടുമീ വാലിന്റെ തുമ്പിൽക്കിട-\\
ന്നാവാം ഭൂമിതിരിഞ്ഞിടുന്ന,തനുകമ്പാർഹം ഭവച്ചാപലം!
\end{slokam}

\Letter{ന}{പ}


\begin{slokam}{\VSr}{\VKG}{നോവിപ്പിക്കാതെ, ശസ്ത്ര}
നോവിപ്പിക്കാതെ, ശസ്ത്രക്രിയകളുടെ സഹായങ്ങളില്ലാതെ, തിക്തം\\
സേവിപ്പിക്കാതെ, പൂർവ്വാർജ്ജിതകവനകലാബോധബീജാങ്കുരത്തെ\\
ഭാവം നോക്കിത്തുടിപ്പി, ച്ചകമലർ വികസിപ്പിച്ചു സഞ്ജാതമാക്കും\\
പ്രാവീണ്യത്തിന്നു കേള്‍വിപ്പെടുമൊരു സുധിയാണക്ഷരശ്ലോകവൈദ്യൻ!
\end{slokam}

\Letter{ന}{ഭ}


\end{enumerate}
\subsection{പ}
\begin{enumerate}


\begin{slokam}{\VSv}{\VNM}{പങ്കാശ്മാദിപരീതമാകിയ}
പങ്കാശ്മാദിപരീതമാകിയ വഴിയ്ക്കിക്കൂരിരുട്ടത്തു നീ \\
കൺ കാണാഞ്ഞു കമിഴ്ന്നു വീ, ണധരമയ്യയ്യോ! മുറിപ്പെട്ടു പോയ്; \\
തൻ കാര്യത്തിനു നിന്നെയാക്കിയവനെക്കാണാനയച്ചീദൃശാ- \\
തങ്കാപ്തിയ്ക്കിടയാക്കി ഞാൻ; സഖി, മമ സ്വാർത്ഥസ്പൃഹയ്ക്കായ്ത്തൊഴാം! 
\end{slokam}

\Letter{പ}{ത}

\Book{വിലാസലതിക}

\begin{slokam}{\VSr}{\VNM}{പച്ചപ്പട്ടും തൊഴും നിൻ}
പച്ചപ്പട്ടും തൊഴും നിൻ തരളതനു തലോടുന്നു; തന്നങ്കഭാഗേ\\
വെച്ചംഭോജാക്ഷി വെള്ളിത്തളികയിലെ നറും പാലെടുത്തേകിടുന്നു;\\
നൽച്ചന്തം ചേർന്ന നർമ്മോക്തികളിടയിലുരയ്ക്കുന്നു - മുജ്ജന്മമേതോ\\
മെച്ചം ചേർന്നുള്ള പുണ്യം ശുകതരുണ, ഭവാൻ ചെയ്തിരിക്കുന്നു നൂനം!
\end{slokam}

\Letter{പ}{ന}

\begin{slokam}{\VUv}{\AUK}{പണിക്കു വന്നും വിപണിക്കു}
പണിക്കു വന്നും വിപണിക്കു വന്നും\\
ഹരിച്ചിടുന്നു വിഹരിച്ചിടുന്നു\\
ധരാസുരൻ കണ്ണധരാസുരൻ കാൺ\\
തവാലയത്തിൽ കിതവാലയത്തിൽ.
\end{slokam}

\Letter{പ}{ധ}

\Topic{യമകം (ഉപേന്ദ്രവജ്ര, നാലു വരിയിലും)}.

\begin{slokam}{\VSv}{\VRV}{പണ്ടങ്കക്കളരിയ്ക്കകത്തു}
പണ്ടങ്കക്കളരിയ്ക്കകത്തു പടവാളൂറ്റിപ്പയറ്റി, പ്പുകൾ\\
കൊണ്ടംഗങ്ങളിൽ രക്തചന്ദനമണിഞ്ഞെത്തുന്ന നൂറ്റാണ്ടുകൾ\\
കൊണ്ടെത്തന്ന ചുവന്ന പൂക്കൾ ചെവിയിൽ ചൂടിപ്പടച്ചട്ടയും\\
കൊണ്ടെത്തുന്നൊരെനിക്കു നീ വഴി മുടക്കീടാതെ വേദാന്തമേ! 
\end{slokam}

\Letter{പ}{ക}

\Book{ഗ്രാമദർശനം}.


\begin{slokam}{\VSv}{\KA}{പണ്ടേയുണ്ടു മനുഷ്യനി}
പണ്ടേയുണ്ടു മനുഷ്യനിഗ്‌ഗുണപുരോഭാഗിത്വ, മദ്ദുർഗ്‌ഗുണം\\
കണ്ടേറുന്ന വിവേകശക്തിയതിനെക്കൊന്നില്ലയിന്നേവരെ;\\
മിണ്ടേണ്ടാ കഥ, ഹന്ത! യിന്നതു വെറും മൂർഖത്വമോ മോഹമോ?\\
വണ്ടേ! നീ തുലയുന്നു; വീണയി വിളക്കും നീ കെടുക്കുന്നുതേ.
\end{slokam}

\Letter{പ}{മ}

\Book{പ്രരോദനം}.



\begin{slokam}{\VSv}{\KN}{പത്രം വിസ്തൃതമത്ര}
പത്രം വിസ്തൃതമത്ര തുമ്പമലർ തോറ്റോടീടിനോരന്നവും\\
പുത്തൻ നെയ്‌ കനിയെപ്പഴുത്ത പഴവും കാളിപ്പഴം കാളനും\\
പത്തഞ്ഞൂറുകറിയ്ക്കുദാസ്യമിയലും നാരങ്ങയും മാങ്ങയും\\
നിത്യം ചെമ്പകനാട്ടിലഷ്ടി തയിർമോർ തട്ടാതെ കിട്ടും ശുഭം
\end{slokam}

\Letter{പ}{പ}

\begin{slokam}{\VSr}{\VenM}{പയ്യെപ്പൈക്കുട്ടി തന്നെ}
പയ്യെപ്പൈക്കുട്ടി തന്നെപ്പരിചിനൊടു പിടിച്ചുന്തിനീക്കീട്ടു, തള്ള-\\
പ്പയ്യിൻ കാൽക്കൂടണഞ്ഞി, ട്ടകിടവിടവിടെത്താൻ പതുക്കെത്തലോടി,\\
തയ്യാറായ്‌ മുട്ടുകുത്തി, ത്തദനു മുഖമുയർത്തി, ച്ചുരത്തും നറുംപാ-\\
ലയ്യാ! മുട്ടിക്കുടിക്കും പശുപശിശുപദം കേവലം മേ\prash{}വലംബം!
\end{slokam}

\Letter{പ}{ത}

\begin{slokam}{\VSr}{\VNM}{പള്ളിക്കൈവില്ലു പൊൻകുന്ന്}
പള്ളിക്കൈവില്ലു പൊൻകു, ന്നലർമകള്‍പതിയാമമ്പു, തോഴൻ ധനേശൻ,\\
വെള്ളിക്കുന്നായ വീ, ടിപ്പെരുമകള്‍ കലരും പോറ്റി തൻ കെട്ടിലമ്മേ!\\
കൊള്ളിച്ചാലെന്തു തൃക്കണ്ണടിയനി, ലവിടേയ്ക്കിഷ്ടയാം ദാസിയായ്‌ പാർ-\\
പ്പുള്ളിശ്രീദേവി പോന്നെൻ പുരയിലധിവസിക്കേണ്ടി വന്നേക്കുമെന്നോ?
\end{slokam}

\Letter{പ}{ക}


\begin{slokam}{\VPv}{\UN}{പറഞ്ഞ കടുവാക്കുകൾ}
പറഞ്ഞ കടുവാക്കുകൾ, പടിയടച്ച ബന്ധങ്ങളും,\\
ചൊരിഞ്ഞ മിഴിനീർക്കണം, ചിരിയൊഴിഞ്ഞതാം ജീവിതം,\\
കരിഞ്ഞ ഹൃദയങ്ങളും, കവിത വറ്റിടും പേനയും\\
തിരിഞ്ഞു വരവില്ലിനിദ്ധര സമുദ്രമായ്ത്തീരിലും!
\end{slokam}

\Letter{പ}{ക}

\begin{slokam}{\VSr}{\VKG}{പറ്റെക്കീറിപ്പൊളിഞ്ഞോരുടു}
പറ്റെക്കീറിപ്പൊളിഞ്ഞോരുടുതുണിയിലിനിസ്സൂചികുത്തേണ്ടഴിക്കാന്‍\\
പറ്റില്ലീജീര്‍ണവാസസ്സുയിരിനൊടുരുകിച്ചേര്‍ന്നതാണെന്നു തോന്നും\\
പെറ്റും കൊന്നും കളിക്കും പ്രകൃതിയുടെ ഹിതത്തിന്നു കുമ്പിട്ടിടാനേ\\
പറ്റൂ, തോണിക്കകത്തോടിയ പഥിക, ഭവാനെത്ര ലാഭിച്ചു നേരം?
\end{slokam}

\Letter{പ}{പ}

\begin{slokam}{\VSv}{\CN}{പാടത്തുംകര നീളെ}
പാടത്തുംകര നീളെ നീലനിറമായ്‌ വേലിയ്ക്കൊരാഘോഷമാ-\\
യാടി,ത്തൂങ്ങി,യല,ഞ്ഞുലഞ്ഞു സുകൃതം കൈക്കൊണ്ടിരിയ്ക്കും വിധൗ\\
പാരാതെ വരികെന്റെ കയ്യിലധുനാ പീയൂഷഡംഭത്തെയും\\
ഭേദിച്ചൻപൊടു കയ്പവല്ലി തരസാ പെറ്റുള്ള പൈതങ്ങളേ!
\end{slokam}

\Letter{പ}{പ}

\begin{slokam}{\VSv}{വൈരശ്ശേരി കെ. എം. നമ്പൂതിരി}{പാടാനില്ലൊരു പാടവം}
പാടാനില്ലൊരു പാടവം, പല തരം ചിത്രം വരച്ചീടുവാ-\\
നേതും വൈഭവമില്ല, നല്ല നടനം ചെയ്യാനുമാകില്ല മേ,\\
പാകം വന്ന സദസ്യരുള്ള സഭയിൽ ചെന്നെത്തി നല്ലക്ഷര-\\
ശ്ലോകം ചൊല്ലി രസിച്ചു തൃപ്തിയടയാൻ മാത്രം കൊതിക്കുന്നു ഞാൻ! 
\end{slokam}

\Letter{പ}{പ}


\begin{slokam}{\VSv}{\VKG}{പാടിക്കേട്ടതു പാടുപെട്ടു}
പാടിക്കേട്ടതു പാടുപെട്ടു പറയാറാകുമ്പൊഴേയ്കും കിളി-\\
ക്കൂടിൻനേർക്കു കുതിച്ചിടുന്നു മരണം കള്ളക്കരിമ്പൂച്ചയായ്‌\\
ആടും കൂട്ടിനകത്തിരുന്നു കരുണം കേഴുന്നൊരിപ്പക്ഷിയെ-\\
ക്കൂടെക്കൂടെയലട്ടിടും പണി, സഖേ പാടില്ല പാടില്ല മേൽ!
\end{slokam}

\Letter{പ}{അ}


\begin{slokam}{\VSv}{\Unk}{പാടിപ്പാടിയനന്തമാധുരി}
പാടിപ്പാടിയനന്തമാധുരി ചൊരി, ഞ്ഞാലോലമെൻ ചന്ദന-\\
ക്കാടിൻ ശാദ്വല സാന്ദ്രകാന്തിയിലഴിഞ്ഞാടും കളാലാപിനി.\\
കൂടിക്കൂടിവരുന്ന രാഗമൊടു ഞാൻ, നിൻ പഞ്ചവർണ്ണക്കിളി-\\
ക്കൂടിൻ വാതിലിൽ വെയ്ക്കുമിപ്പഴയരിക്കാണിക്ക, കൈക്കൊള്ളുമോ?
\end{slokam}

\Letter{പ}{ക}


\begin{slokam}{\VSr}{\PG}{പാടില്ലാ നീലവണ്ടേ, സ്മരനുടെ}
പാടില്ലാ നീലവണ്ടേ, സ്മരനുടെ വളർവില്ലിന്റെ ഝങ്കാരനാദം\\
പാടിപ്പാടിപ്പറന്നെൻ പ്രിയയുടെ വദനാംഭോരുഹം ചുറ്റിനിൽക്കാൻ\\
പേടിച്ചിട്ടല്ല -- ഭർത്തൃപ്രണിഹിതമതിയാണെന്റെ ജീവേശി -- യെങ്കിൽ--\\
ക്കൂടി, ക്കാടൻ, കുരൂപൻ, കുമതി വിതറുമാവെണ്മയിൽ കന്മഷം നീ.
\end{slokam}

\Letter{പ}{പ}

\Book{നാൽക്കാലികൾ}.

\begin{slokam}{\VSv}{\KVIT}{പാതിക്കെട്ടു കൊതിച്ചു ഞാൻ}
പാതിക്കെട്ടു കൊതിച്ചു ഞാൻ പലതരം തൽപ്പാതിയിൽപ്പാതിയിൽ-\\
പ്പാതിത്വത്തൊടു പാതിയാടി പലതും പാഹീതി മുൻപായഹോ!\\
പാതിച്ചോർനടയാള്‍ക്കു പാതി നയനം പോലും വിടർന്നീല, യി-\\
പ്പാരുഷ്യത്തൊടു പാതിവിന്ദശരനും പാതിപ്പെടുത്തുന്നു മാം!
\end{slokam}

\Letter{പ}{പ}


\begin{slokam}{\VSr}{\Unk}{പാഥോരാശിപ്രഭാവേ}
പാഥോരാശിപ്രമാഥേ മുരരിപുവചസാ കാളകൂടം കുടിച്ച-\\
ന്നേതും ഖേദം വരാതേ പതിയുയിർ പരിപാലിച്ച മംഗല്യശീലേ!  \\
ആധാരം നിന്നടിത്താരിണയൊഴികെ ജഗത്തിങ്കലില്ലാർത്തിഭാജാം \\
നാഥേ! രോഗാതുരൻ ഞാൻ, തുണപെടുക കുമാരാലയം കോലുമമ്മേ!
\end{slokam}

\Letter{പ}{അ}

\begin{slokam}{\VSr}{\VNM}{പാരം പാരാകെ വേണ്ടും}
പാരം പാരാകെ വേണ്ടും പരിചിനു കടലാസ്സാക്കി, നീരാഴമേറും\\
പാരാവാരത്തെയെല്ലാം പരശിവദയിതേ, നന്മഷിപ്പാത്രമാക്കി,\\
പോരാ, നിശ്ശേഷപക്ഷിപ്പരിഷകളുടെയും തൂവലും പൂ, ണ്ടതന്ദ്ര-\\
ന്മാരായ്‌ ബാണാസുരന്മാർ പലരെഴുതുകിലും തീരുമോ നിൻ ഗുണങ്ങള്‍?
\end{slokam}

\Letter{പ}{പ}

\begin{slokam}{\VSr}{\VCBP}{പാരാവാരം കരേറിക്കരകള്‍}
പാരാവാരം കരേറിക്കരകള്‍ മുഴുവനും മുക്കിമൂടാത്തതെന്തോ?\\
താരാവൃന്ദങ്ങള്‍ തമ്മിൽ സ്വയമുരസി മറിഞ്ഞത്ര വീഴാത്തതെന്തോ?\\
നേരായാരാഞ്ഞു നോക്കീടുക മദമിയലും മർത്ത്യരേ, നിങ്ങളെന്നാ-\\
ലാരാൽ കണ്ടെത്തുമെല്ലാറ്റിനുമുപരി വിളങ്ങുന്ന വിശ്വസ്വരൂപം.
\end{slokam}

\Letter{പ}{ന}

\Book{ഒരു വിലാപം}.


\begin{slokam}{\VSv}{\UN}{പാരിൽ പ്രാർത്ഥന}
പാരിൽ പ്രാർത്ഥന, യൊത്തുചേരൽ, ജനനിയ്ക്കാലിംഗനം, തൊട്ടതെ--\\
ല്ലാരും നിർത്തി ശുചിത്വമന്ത്രമുരുവായ് സൂക്ഷിച്ചു മുന്നേറവേ,\\
ഓരോ ജല്പനമേന്തി വന്ന ജളർ നിൻ നെഞ്ചത്തു പൊങ്കാലയായ്\\
ആരോഗ്യപ്പെരുമേ, തകർക്കരുതു നിൻ കൂടും യശസ്സൊക്കെയും! 
\end{slokam}

\Letter{പ}{ഒ}

\begin{slokam}{\VSr}{\Mazha}{പാലംഭോരാശിമദ്ധ്യേ}
 പാലംഭോരാശിമദ്ധ്യേ ശശധരധവളേ ശേഷഭോഗേ ശയാനം,\\
മേളം കോലും കളായദ്യുതിയൊടു പടതല്ലുന്ന കാന്തിപ്രവാഹം,\\
നാളന്നേറിത്തുളുമ്പും നിരുപമകരുണാഭാരതിമ്യൽകടാക്ഷം\\
നാളീകത്താരിൽമാതിൻ കുളുർമുലയുഗളീഭാഗധേയം ഭജേഥാഃ.
\end{slokam}

\Letter{പ}{ന}

\Book{ഭാഷാനൈഷധചമ്പു}.


\begin{slokam}{\VSr}{\VKG}{പാലാഴിക്കോളിരമ്പത്തിലുമൊരു}
 പാലാഴിക്കോളിരമ്പത്തിലുമൊരു പൊഴുതും സ്വാപഭംഗം വരില്ലെ-\\
ന്നാലും മാതാവുഷസ്സിൽദ്ദധി കടയുമൊലിക്കാഞ്ഞുണർന്നേൽപതെന്തോ?\\
നീലക്കാർവർണ്ണ! ദേവർക്കനുപമമമൃതം നൽകുമത്താമരക്ക-\\
യ്യാലേ പാലും നറും വെണ്ണയുമടവിലെടുത്തുണ്മതെന്താരറിഞ്ഞു?
\end{slokam}

\Letter{പ}{ന}

\begin{slokam}{\VSr}{\Punam}{പാലാഴിത്തയ്യലാള്‍ തൻ}
പാലാഴിത്തയ്യലാള്‍ തൻ തിരുനയനകലാലോലലോലംബമാലാ-\\
ലീലാരംഗം, ഭുജംഗേശ്വര മണിശയനേ തോയരാശൗ ശയാനം,\\
മേലേ മേലേ തൊഴുന്നേൻ - ജഗദുദയപരിത്രാണസംഹാരദീക്ഷാ-\\
ലോലാത്മാനം പദാന്തപ്രണത സകലദേവാസുരം വാസുദേവം
\end{slokam}

\Letter{പ}{മ}

\Book{ഭാഷാരാമായണചമ്പു}.

\begin{slokam}{\VSv}{\KJ}{പാലാഴിത്തിര, വെണ്മതിപ്രഭ,}
പാലാഴിത്തിര, വെണ്മതിപ്രഭ, ലസത്തൂലാഞ്ചലം, ജാഹ്നവീ-\\
കീലാലം, ബിസതന്തുജാലമിവയുദ്വേലാഭ കൈക്കൊണ്ടുടൻ, \\
ചേലായ് ചേർന്നു ഘനീഭവിച്ചു ഭസിതശ്രീലാസ്യഭാസ്സാർന്ന പോൽ \\
കൈലാസം ധവളം ശിവാശിവലസല്ലീലാലയം കോമളം.
\end{slokam}

\Letter{പ}{ച}

\Topic{അഷ്ടപ്രാസം}. 

\begin{slokam}{\VSv}{\VKG}{പാലാഴിത്തിരമാല നാലുപുറവും}
പാലാഴിത്തിരമാല നാലുപുറവും തട്ടിക്കുലുക്കുമ്പൊഴും\\
വേലപ്പെണ്ണടിരണ്ടുമാത്തകുതുകം മെല്ലെത്തലോടുമ്പൊഴും\\
പാലിക്കാനമരർഷിമാർ സ്തുതികഥാഗീതം പൊഴിക്കുമ്പൊഴും\\
ചേലിൽ ചാഞ്ഞുകിടന്നുറങ്ങുമുടയോനേകട്ടെയുത്തേജനം!
\end{slokam}

\Letter{പ}{പ}


\begin{slokam}{\VMk}{\KV}{പാലിക്കാനായ് ഭുവനമഖിലം}
പാലിക്കാനായ് ഭുവനമഖിലം ഭ്രതലേ ജാതനായ-\\
ക്കാലിക്കൂട്ടം കലിതകുതുകം കാത്ത കണ്ണന്നു ഭക്ത്യാ\\
പീലിക്കോലൊന്നടിമലരിൽ നീ കാഴ്ചയായ് വച്ചിടേണം\\
മൗലിക്കെട്ടിൽ തിരുകുമതിനെത്തീൎച്ചയായ് ഭക്തദാസൻ.
\end{slokam}

\Letter{പ}{പ}

\Book{മയൂരസന്ദേശം}.



\begin{slokam}{\VSv}{\KJ}{പാളീടാത്ത മനീഷയും}
പാളീടാത്ത മനീഷയും, പല കലാപാളീവിലാസങ്ങളും, \\
നാളീകാലയനായികാനയനസ്ത്കേളീലസത്പ്രൗഢിയും, \\
കാളീകാന്ത! കലർന്ന സൂരിവരനും ചൂളീടുമെന്നും ഭവത്- \\
കേളീചാതുരി വാഴ്ത്തുവാൻ, വിഷമയവ്യാളീവരാലംകൃത!
\end{slokam}

\Letter{പ}{ക}

\Topic{അഷ്ടപ്രാസം}. 


\begin{slokam}{\VSr}{\OKM}{പിച്ചക്കാരൻ കുബേരൻ}
പിച്ചക്കാരൻ കുബേരൻ, പിതൃപതി സുകൃതാഘാനഭിജ്ഞൻ, വിരൂപൻ\\
പച്ചക്കാമൻ, പതംഗൻ പകൽമതി, പരവാൻ പദ്മനാഭാഗ്രജാതൻ,\\
അർച്ചിഷ്മാനപ്രഭാംഗൻ, ധിഷണനധിഷണൻ, നിന്റെ പാദാംബുജത്താ-\\
രർച്ചിക്കും മർത്ത്യനെക്കണ്ടറിയുമവനുമേ പാപനിഷ്ഠൻ വസിഷ്ഠൻ!
\end{slokam}

\Letter{പ}{അ}


\begin{slokam}{\VSr}{\ARRV}{പിച്ചക്കാരൻ ഗമിച്ചാനെവിടെ}
"പിച്ചക്കാരൻ ഗമിച്ചാനെവിടെ?", "ബലിമഖം തന്നിൽ"; "എങ്ങിന്നു നൃത്തം?",\\
"മെച്ചത്തോടാച്ചിമാർ വീടതിൽ"; "എവിടെ മൃഗം?", "പന്നി പാഞ്ഞെങ്ങു പോയോ?";\\
"എന്തേ കണ്ടില്ല മൂരിക്കിഴടിനെ?", "ഇടയൻ ചൊല്ലുമക്കാര്യമെല്ലാം"\\
സൗന്ദര്യത്തർക്കമേവം രമയുമുമയുമായുള്ളതേകട്ടെ മോദം.
\end{slokam}

\Letter{പ}{എ}

\OSlRef{ഭിക്ഷാർത്ഥീ സ ക്വ യാതഃ}.

\Topic{ഉമാരമാസംവാദം}. \SeeAlso{കാടല്ലേ നിന്റെ ഭർത്താവിനു}, 
\SeeAlso{നന്നോ മെയ്യണിവാനുമേ}, \SeeAlso{കുന്നിൻനാട്ടിലെ ബാന്ധവം}.

\begin{slokam}{\VKm}{\VKG}{പിച്ചനെല്ലവിലിടിച്ചുകെട്ടിയ}
പിച്ചനെല്ലവിലിടിച്ചുകെട്ടിയ പഴന്തുണിക്കിഴിയുമേന്തിയ-\\
ന്നച്യുതന്റെ തിരുമുൻപിൽ വന്നവനിരന്നതില്ല ധനമെങ്കിലും\\
അച്ഛസൗഹൃദമുറച്ച ഭക്തിയിലണച്ചു വേണ്ടവിഭവങ്ങള്‍ നി-\\
ന്നിച്ഛ തുച്ഛതരമർഹനെങ്കിലരുളാത്തതില്ല കരുണാകരൻ
\end{slokam}

\Letter{പ}{അ}

\begin{slokam}{\VSr}{\GRT}{പീലിക്കാർകൂന്തൽ കെട്ടി}
 പീലിക്കാർകൂന്തൽ കെട്ടിത്തിരുകിയതിൽ മയിൽപ്പീലിയും ഫാലദേശേ\\
ചാലേ തൊട്ടുള്ള ഗോപിക്കുറിയുമഴകെഴും മാലയും മാർവിടത്തിൽ\\
തോളിൽച്ചേർത്തുള്ളൊരോടക്കുഴലുമണികരേ കാലി മേയ്‌ക്കുന്ന കോലും\\
കോലും ഗോപാലവേഷം കലരുമുപനിഷത്തിന്റെ സത്തേ നമസ്തേ!
\end{slokam}

\Letter{പ}{ത}

\begin{slokam}{\VSr}{\Poonth}{പീലിക്കാർകൂന്തൽ കെട്ടീട്ടഴകൊടു}
പീലിക്കാർകൂന്തൽ കെട്ടീട്ടഴകൊടു നിടിലേ ചാരുഗോരോചനം ചേർ-\\
ത്തേലസ്സും പൊൻചിലമ്പും വളകളുമണിയിച്ചമ്മതന്നങ്കഭാഗേ\\
ലീലാഗോപാലവേഷത്തൊടു മുരളിയുമായ്‌ കാലി മേയ്ക്കുന്ന കോലും\\
ചലേ കൈക്കൊണ്ടു മന്ദസ്മിതമൊടു മരുവും പൈതലേ, കൈതൊഴുന്നേൻ!
\end{slokam}

\Letter{പ}{ല}



\begin{slokam}{\VKm}{\CVVB}{പീലി ചാർത്തിയൊരു കുന്തളം}
പീലി ചാർത്തിയൊരു കുന്തളം, വടിവിലാടിടും മകരകുണ്ഡലം,\\
ചാലിൽ മുത്തിഴകള്‍, വന്യമാല, കളഭാദി ചേർന്ന പുതുസൗരഭം,\\
കാലിൽ നൽത്തളകള്‍, പൊന്നിടഞ്ഞ തുകിൽ, പിന്നെ മീതെയരഞ്ഞാണുമീ-\\
ച്ചേലിൽ രാസനടനം തുടർന്ന തവ മൂർത്തി പേർത്തുമിവനോർത്തിടാം
\end{slokam}

\Letter{പ}{ക}


\Book{നാരായണീയം പരിഭാഷ}.
\OSlRef{കേശപാശധൃതപിഞ്ഛികാ}.


\begin{slokam}{\VSv}{\RV}{പുഞ്ചപ്പാടവരമ്പിലാടി}
പുഞ്ചപ്പാടവരമ്പിലാടി, യിളകിക്കൊഞ്ചിച്ചിരി, ച്ചാടതൻ\\
തുഞ്ചം കോട്ടിയ കുമ്പിളിൽപ്പുതുമണം തഞ്ചുന്ന പൂ നുള്ളിയും\\
നെഞ്ചിൽത്തൊട്ടു തലോടി, യെൻ ചൊടികളിൽ പഞ്ചാരമുത്തം തരും\\
മൊഞ്ചത്തിപ്പുതുമാരി തൻ വരവിതിൽ \sam{പഞ്ചേന്ദ്രിയാകർഷണം}!
\end{slokam}

\Letter{പ}{ന}

\Topic{അഷ്ടപ്രാസം}.
സമസ്യാപൂരണം. മറ്റു പൂരണങ്ങൾ: \SlRef{ഛായാഗ്രാഹകപൃഷ്ഠദർശനം}.


\begin{slokam}{\VSr}{\YK}{പുറ്റിൻ മൗനത്തിൽ വാചാലതയുടെ}
പുറ്റിൻ മൗനത്തിൽ വാചാലതയുടെ നിധി നീ തേടിയെത്തിപ്പിടിച്ചും,\\
നെറ്റിക്കണ്ണന്റെ ഢക്കാരവതടിനിയിൽ നീരാടി നീന്തിത്തുടിച്ചും,\\
മുറ്റിപ്പീയൂഷമോലും മുരഹരമുരളീരന്ധ്രകൽപം കഴിച്ചും,\\
ചെറ്റിമ്പം ശാരദേ! നീ തരുമളവിളയിൽ ജീവിതം ജീവിതവ്യം!
\end{slokam}

\Letter{പ}{മ}


\begin{slokam}{\VSr}{\VKG}{പൂണെല്ലുന്തിച്ചടച്ചാടിയ}
പൂണെല്ലുന്തിച്ചടച്ചാടിയ മമ കവിതപ്പയ്യിനേയന്തിനേര-\\
ത്താണല്ലൊ ഞാൻ കറക്കാൻ മുതിരുവതു ഭവാനിഷ്ടനൈവേദ്യമേകാൻ;\\
താണേൻ, നൂണേനകിട്ടിൽപ്പലകുറി, യൊടുവിൽച്ചെറ്റു കൈവന്ന ദുഗ്ദ്ധം\\
നാണം കെട്ടാണു വയ്ക്കുന്നതു പദമലരിൽ, ഗോകുലാനന്ദമൂർത്തേ!
\end{slokam}

\Letter{പ}{ത}


\begin{slokam}{\VSr}{\ONN}{പൂമാതല്ലേ കളത്രം?}
പൂമാതല്ലേ കളത്രം? ചപലകളിലവള്‍ക്കഗ്രഗണ്യത്വമില്ലേ?\\
പൂമെയ്‌ പാമ്പിന്മെലല്ലേ? വിഷമെഴുമവനൊന്നൂതിയാൽ ഭസ്മമല്ലേ?\\
ഭീമഗ്രാഹാദിയാദോഗണമുടയ കടൽക്കുള്ളിലല്ലേ നിവാസം?\\
സാമാന്യം പോലെയെന്തുള്ളതു പറക നിനക്കത്ര പൂർണ്ണത്രയീശ!
\end{slokam}

\Letter{പ}{ഭ}

\begin{slokam}{\VSv}{\Poonth}{പൂമെത്തേലെഴുനേറ്റിരുന്നു}
പൂമെത്തേലെഴുനേറ്റിരുന്നു "ദയിതേ, പോകുന്നു ഞാ"നെന്നു കേ--\\
ട്ടോമൽക്കണ്ണിണനീരണിഞ്ഞ വദനപ്പൂവോടു ഗാഢം തദാ\\
പൂമേനിത്തളിരൊന്നു ചേർ "ത്തഹമിനിക്കാണുന്നതെ"ന്നെന്നക-\\
പ്പൂമാലോടളിവേണി ചൊന്ന മധുരച്ചൊല്ലിന്നു കൊല്ലുന്നു മാം.
\end{slokam}

\Letter{പ}{പ}

\begin{slokam}{\VSv}{\KND}{പൂവിൻ ചുണ്ടു തൊടുന്നതില്ല}
പൂവിൻ ചുണ്ടു തൊടുന്നതില്ല പുതുതേൻ തെണ്ടും ദ്വിരേഫം; മണം\\
തൂവിത്തൂവിയലഞ്ഞലഞ്ഞു തിരിയാൻ വന്നില്ല മന്ദാനിലൻ;\\
മാവിൻകൊമ്പിലുറക്കമായ്‌ക്കുയിൽ; വെറും ജീവച്ഛവം മാത്രമി-\\
ബ്ഭൂവി,ന്നെന്തൊരു കഷ്ട,മിങ്ങനെവരാനെന്തേ വസന്തോത്സവം?
\end{slokam}

\Letter{പ}{മ}


\begin{slokam}{\VSr}{\Unk}{പെറ്റോരാ മക്കളെല്ലാമപകടം}
പെറ്റോരാ മക്കളെല്ലാമപകട, മൊരുവന്നാറു മോറുണ്ടു കഷ്ടേ!\\
മറ്റേവന്‍ കൂറ്റനാനത്തലയ, നയി മണാളന്‍ മഹാപിച്ചതെണ്ടി;\\
ചിറ്റം മറ്റൊന്നിനോടുണ്ടവ, നൊരുനനമുണ്ടെങ്കിലും ചുറ്റുവാനായ്‌–\\
പ്പറ്റീട്ടില്ലിത്രനാളും, തവ \sam{മലമകളേ, ജാതകം ജാതി തന്നെ}!
\end{slokam}

\Letter{പ}{ച}


സമസ്യാപൂരണം. മറ്റു പൂരണങ്ങൾ: \SlRef{എല്ലായ്പോഴും കളിപ്പാൻ ചുടല}, \SlRef{മുപ്പാരും കാക്കുവാനില്ലപരൻ}, \SlRef{മെയ്യിൽപ്പാമ്പുണ്ടനേകം}


\begin{slokam}{\VSr}{\Unk}{പേടിച്ചീടായ്ക വന്നീടരികിലിനിയും}
പേടിച്ചീടായ്ക വന്നീടരികിലിനിയുമവ്യാകുലം കൂകെടോ! നീ\\
മാടപ്രാവേ, മണംമേവിന മണിതകലാവിഭ്രമം മൽപ്രിയായാഃ;\\
ഊടപ്പാടെപ്പടിച്ചാറഴകുതിതു സഖേ, "മാരചേമന്തികപ്പെൺ"\\
ക്രീഡിച്ചീടിന്റ മാരോത്സവമണിനിലയം തങ്കലോ നിൻകുലായം?
\end{slokam}

\Letter{പ}{ഉ}

\Book{പദ്യരത്നം}.


\begin{slokam}{\VSr}{\ARRV}{പേടിച്ചോടേണ്ട നില്പിൻ}
പേടിച്ചോടേണ്ട നില്പിൻ, ചപലകപികളേ, തുംഗശക്രേഭകുംഭം\\
പാടിച്ചോരിശ്ശരം നിങ്ങടെയുടൽ തൊടുവാനേറെ നാണിച്ചിടുന്നൂ\\
സൗമിത്രേ! നിൽക്ക, നീയോ വിഷയമയി, രുഷാമിങ്ങിവൻ മേഘനാഥൻ\\
സ്വാമിഭ്രൂഭംഗസന്ദാനിതജലനിധിയാം രാമനെത്തേടിടുന്നു. 
\end{slokam}

\Letter{പ}{സ}

\begin{slokam}{\VSr}{\GSK}{പേരോർക്കുന്നീല, കൃത്യപ്പിഴ}
പേരോർക്കുന്നീല, കൃത്യപ്പിഴ പിണയുകയാലാദ്യമായിട്ടൊരാണ്ടേ-\\
യ്ക്കാരോമൽക്കാന്ത വേറിട്ടപഗതമഹിമാവായ് നിജസ്വാമിശാപാൽ\\
ഓരോരോ മാമരപ്പൂംതണലൊടവനിജാസ്നാനസംശുദ്ധമാം ത-\\
ണ്ണീരോലും രാമഗിര്യാശ്രമനിരയിലലഞ്ഞീടിനാൻ യക്ഷനേകൻ.
\end{slokam}

\Letter{പ}{ഒ}

\Book{മേഘസന്ദേശം പരിഭാഷ}.
\OSlRef{കശ്ചിത് കാന്താ വിരഹഗുരുണാ}.


\begin{slokam}{\VSv}{\VKG}{പേർ കാളും കവിമല്ലരെ}
 പേർ കാളും കവിമല്ലരെ പ്രതിമയാൽ ഛായാപടത്താൽ വൃഥാ\\
ലോകം സ്മാരകമേർപ്പെടുത്തിയഭിനന്ദിക്കുന്നതായ്‌ കാണ്മു നാം;\\
പോകുന്നീലതുകാണുവാൻ സഹൃദയന്മാരും, നമുക്കക്ഷര-\\
ശ്ലോകത്തിൽ സ്മരണീയർ തൻ കൃതികളെച്ചൊല്ലാ, മതല്ലേ സുഖം?
\end{slokam}

\Letter{പ}{പ}

\begin{slokam}{\VKm}{\PCM}{പേറ്റുനോവവിടെ നിന്നിടട്ടെ}
 പേറ്റുനോവവിടെ നിന്നിടട്ടെ, രുചിയറ്റു, ദേഹബലശോഷണം\\
കൂട്ടിടേണ്ട, മലമൂത്രശയ്യയിലൊരാണ്ടു നീക്കുവതുമങ്ങനെ\\
ഗർഭമാം ചുമടിനുള്ള കൂലിയതുപോലുമേകുവതിനാവുകി-\\
ല്ലെത്ര യോഗ്യതയെഴുന്ന പുത്രനുമഹോ! മഹാജനനി! കൈ തൊഴാം!
\end{slokam}

\Letter{പ}{ഗ}

\OSlRef{ആസ്താം താവദിയം}

\begin{slokam}{\VSv}{\VNM}{പൈന്തിങ്കൾത്തിരു നെറ്റി}
പൈന്തിങ്കൾ തെളി നെറ്റി ചെറ്റു മറയും മാറായി നീലാഞ്ജന-\\
ച്ചാന്തിൻ ചേലണിവേണി തൻ തലയിലിട്ടിട്ടുള്ള ശുഭ്രാംബരം,\\
കാന്തിപ്പെട്ട നിതംബമണ്ഡല മണൽത്തിട്ടിങ്കലൂടേ മന-\\
ശ്ശാന്തിപ്പാൽപ്പുഴ പോലെ മന്ദമൊഴുകിച്ചുംബിച്ചു കാൽപ്പൂവിനെ.
\end{slokam}

\Letter{പ}{ക}

\Book{ഒരു സന്ധ്യാപ്രണാമം}.

\begin{slokam}{\VSr}{\VNM}{പൊട്ടാക്കിപ്ഫാലവട്ടത്തിരുമിഴി}
 പൊട്ടാക്കിപ്ഫാലവട്ടത്തിരുമിഴി, ജടയെക്കാറൊളിച്ചാരുകൂന്തൽ-\\
ക്കെട്ടാക്കി, ക്കേതകിപ്പൂവതിനുടെ വടിവാക്കിപ്പരം ചന്ദ്രഖണ്ഡം,\\
മട്ടൊക്കെത്തന്നെ മാറി, പ്പൃഥയുടെ സുതനായ്‌ കാട്ടിലുള്‍പ്പുക്കു വൈര-\\
പ്പെട്ടൂക്കാൽ ജന്യമിട്ടാ മഹിതകപടകാട്ടാളനെക്കൈതൊഴുന്നേൻ!
\end{slokam}

\Letter{പ}{മ}


\begin{slokam}{\VSr}{\VNM}{പൊൽപ്പൂവാമമ്പെടുത്തോ}
പൊൽപ്പൂവാമമ്പെടുത്തുള്ളൊരു തിരുവടി തൻ ധർമ്മദാരങ്ങളാണെ-\\
ന്നുൾപ്പൂവിൽ തോന്നുമാറുള്ളിളമൃഗമിഴിമാരുണ്ടു ലക്ഷോപലക്ഷം! \\
ഇപ്പൂരം ഹന്ത കാണുന്നവരുടെ മിഴികൾക്കൊക്കെയും നല്ല പച്ച-\\
ക്കർപ്പൂരം തന്നെയാണെന്നിഹ ചില കവിതക്കാർ പുകഴ്ത്തിത്തുടങ്ങീ.
\end{slokam}

\Letter{പ}{ഇ}

\begin{slokam}{\VSv}{\ARRV}{പോകുണ്ടുന്നിതു നാൾ ശകുന്തള}
പോകുണ്ടുന്നിതു നാൾ ശകുന്തള പിരിഞ്ഞെന്നോർത്തു ഹൃത്തുത്സുകം;\\
തൂകാഞ്ഞശ്രുഗളോദരം കലുഷിതം; ഭാവം ജഡം ചിന്തയാൽ;\\
കാട്ടിൽ പാർക്കുമെനിക്കുമിത്ര കഠിനം സ്നേഹോദിതം കുണ്ഠിതം;\\
നാട്ടിൽപ്പെട്ട ഗൃഹസ്ഥനെത്രയുളവാം പുത്രീവിയോഗവ്യഥ!
\end{slokam}

\Letter{പ}{ക}

\Book{അഭിജ്ഞാനശാകുന്തളം പരിഭാഷ}.
\OSlRef{യാസ്യത്യദ്യ ശകുന്തളേതി}.

\begin{slokam}{\VVt}{\HM}{പോവുന്ന നേരമഴകാർന്നിടതൂർന്നു}
പോവുന്ന നേരമഴകാർന്നിടതൂർന്നു നീളും\\
രാവിന്റെ കുന്തളമഴിഞ്ഞതുമൂലമാവാം\\
ഈ വിശ്വമൊന്നുതരമാക്കിടുവാൻ കൊതിക്കും\\
പൂ വീണലംകൃതമനോഹരിയായി മുറ്റം!
\end{slokam}

\Letter{പ}{ഇ}




\begin{slokam}{\VVt}{\VNM}{പ്രാണാധിഭർത്ത്രി, കരയായ്ക}
 പ്രാണാധിഭർത്ത്രി, കരയാ, യ്കരിമുക്തനാനാ-\\
ബാണാളി താങ്ങുവതിനീയൊരു നെഞ്ചു പോരും;\\
ബാണാത്മജാനയനനീരൊരു തുള്ളി പോലും\\
വീണാൽ സഹിപ്പതനിരുദ്ധനസാദ്ധ്യമത്രേ!
\end{slokam}

\Letter{പ}{ബ}

\Book{ബന്ധനസ്ഥനായ അനിരുദ്ധൻ}.



\begin{slokam}{\VSv}{\ARRV}{പ്രാര്‍ത്ഥിച്ചാല്‍ പദമേകുമെങ്കിലുമഹോ}
 പ്രാര്‍ത്ഥിച്ചാല്‍ പദമേകുമെങ്കിലുമഹോ! മുന്നോട്ടെടുക്കാ ദൃഢം,\\
ക്രോധിച്ചാല്‍ വിറയാര്‍ന്നിടും പുനരുടന്‍ വൈവര്‍ണ്യവും കാട്ടിടും\\
കൂട്ടാക്കാതെ പിടിച്ചിഴച്ചിടുകിലോ സ്തംഭം പിടിച്ചീടുമേ\\
കഷ്ടം! മൂഢനു വാണി, യാര്യസഭയില്‍ കേഴും നവോഢാസമം.
\end{slokam}

\Letter{പ}{ക}


\begin{slokam}{\VSv}{\MPN}{പ്രാലേയാദ്രിസുതേ, തവ സ്‌തനതടം}
പ്രാലേയാദ്രിസുതേ, തവ സ്‌തനതടം മാലേയപങ്കാക്തമെ-\\
ന്താലേപങ്ങൾ പൊഴിഞ്ഞു ഭസ്മകരജസ്സാലേ പരീതാഭമായ്\\
ചാലേ കാണ്മതു, മന്ത്രവാദമിരവിൽ ബാലേ നടന്നോ, വിയർ-\\
ത്തീലേ തുള്ളലി, ലെന്നു തോഴികളുഷഃകാലേ തൊഴും മെയ് തൊഴാം
\end{slokam}

\Letter{പ}{ച}


\end{enumerate}
\subsection{ഫ}
\begin{enumerate}

\begin{slokam}{\VSv}{\Unk}{ഫാലത്തീയിനു വെള്ളമുണ്ടു}
ഫാലത്തീയിനു വെള്ളമുണ്ടു തലയിൽ, ക്കണ്ഠസ്ഥഹാലാഹല-\\
ജ്ജ്വാലയ്ക്കുണ്ടു ശിവാധരാമൃതരസം, മെയ്യിൽപ്പെടും പാമ്പിനും\\
ചേലൊത്തോഷധിനായകൻ തലയിലു, ണ്ടിന്നൊന്നു കൊണ്ടും ഭവാ-\\
നാലസ്യം പിണയാതെ ശങ്കര! ജയിച്ചാലും ജഗന്മണ്ഡലം!
\end{slokam}

\Letter{ഫ}{ച}


\begin{slokam}{\VKm}{\Balendu}{ഫാലനേത്രമതിലുള്ള തീപ്പൊരി}
"ഫാലനേത്രമതിലുള്ള തീപ്പൊരി പടര്‍ന്നു കേറി ജട കത്തിടാം\\
ജ്വാല വേഗമൊടണച്ചിടുന്നതിനു വേണ്ടിയാറു കരുതുന്നതാം"\\
ശൈലപുത്രിയുടെ കോപമാറ്റുവതിനീവിധത്തിലടവോതുമ-\\
ക്കാലകാലനുടെ കാലുതാന്‍ ശരണമേതു വിഘ്നവുമൊഴിക്കുവാന്‍.
\end{slokam}

\Letter{ഫ}{ശ}


\begin{slokam}{\VSv}{\KKK}{ഫാലം ചാരു ലലന്തികാവിലസിതം}
 ഫാലം ചാരു ലലന്തികാവിലസിതം ബാലേന്ദുമൗലിസ്ഥലം\\
ലോലംബാളകചുംബികുങ്കുമലസത്കസ്തൂരികാസുന്ദരം\\
നീലത്താമരലോചനം നിഖിലനിർമ്മാണത്തിൽ നിഷ്ണാതമാം\\
ഭ്രൂലാസ്യങ്ങളുമംബ! കാണണമെനിക്കനന്ദസന്ദായകം.
\end{slokam}

\Letter{ഫ}{ന}

\begin{slokam}{\VSr}{\KND}{ഫാലേ നീലാളകങ്ങൾക്ക്}
ഫാലേ നീലാളകങ്ങള്‍ക്കിടയിലഴകെഴും ചില്ലിതൻ മേൽവശം ത-\\
ന്മാലേയസ്നിഗ്ദ്ധരേഖയ്ക്കിടയിൽ നടുവിൽ നീ തൊട്ടതാം കുങ്കുമാങ്കം\\
കാലേ സഹ്യാചലത്തിൻ കുടിലവലലതാശ്യാമസീമാഞ്ചലത്തിൻ\\
മേലേ പൊന്തും വിഭാതദ്യുമണിയൊടെതിരായ്‌, സുഭ്രു, ശോഭിച്ചിരുന്നു.
\end{slokam}

\Letter{ഫ}{ക}

\begin{slokam}{\VSv}{\VNM}{ഫാലേ വേർപ്പുകള്‍ വറ്റിയില്ല}
 ഫാലേ വേർപ്പുകള്‍ വറ്റിയില്ല മുഴുവൻ, വാർകൂന്തൽ കെട്ടിക്കഴി-\\
ഞ്ഞീ, ലേറ്റം മുലമൊട്ടുലച്ച നെടുവീർപ്പേറ്റെല, യെന്നാകിലും\\
ചാലേ തൽക്ഷണശോഭയിൽ തരളനാം കാന്തന്റെ നൽച്ചുംബന-\\
ത്താലേ പേലവഗാത്രിയാള്‍ക്കപരമെന്തോതാ? മതാന്തൻ സ്മരൻ!
\end{slokam}

\Letter{ഫ}{ച}


\begin{slokam}{\VSv}{\KV}{ഫുല്ലാബ്ജത്തിനു രമ്യതക്കു}
ഫുല്ലാബ്ജത്തിനു രമ്യതക്കു കുറവോ പായല്‍ പതിഞ്ഞീടിലും?\\
ചൊല്ലാര്‍ന്നോരഴകല്ലയോ പനിമതിക്കങ്കം കറുത്തെങ്കിലും?\\
മല്ലാക്ഷീമണിയാള്‍ക്കു വല്‌ക്കലമിതും ഭൂയിഷ്ടശോഭാവഹം;\\
നല്ലാകാരമതിന്നലങ്കരണമാമെല്ലാപ്പദാര്‍ത്ഥങ്ങളും.
\end{slokam}

\Letter{ഫ}{മ}

\Book{അഭിജ്ഞാനശാകുന്തളം പരിഭാഷ}.

\end{enumerate}
\subsection{ബ}
\begin{enumerate}

\begin{slokam}{\VSr}{\VNM}{ബാണൻ തൻ കോട്ട കാത്തൂ}
ബാണൻ തൻ കോട്ട കാത്തൂ ഭഗവതി, ഭുവനാധീശനാം നിൻ മണാളൻ;\\
ബാണം വർഷിച്ചു മെയ്‌ മൂടിയ രണപടുവാം ഫൽഗുനന്നിഷ്ടമേകീ;\\
വേണം തൻ ഭക്തരോടിത്രയുമകമലിവങ്ങെങ്കിൽ നിൻ ഭക്തനാമെൻ\\
ത്രാണത്തിന്നെന്തമാന്തം തവ? സതി പതിസാധർമ്മ്യമേൽക്കേണ്ടതല്ലോ.

\end{slokam}

\Letter{ബ}{വ}


\end{enumerate}

\subsection{ഭ}

\begin{enumerate}

\begin{slokam}{\VSr}{\SVL}{ഭക്തർക്കിഷ്ടം കൊടുക്കും}
ഭക്തർക്കിഷ്ടം കൊടുക്കും ഭുവനജനനി, നിൻ ചെഞ്ചൊടിക്കും, ചൊടിക്കും\\
ദൈത്യന്മാരെപ്പൊടിക്കും വിരുതിനു, മിരുളിൻ പേർ മുടിക്കും മുടിക്കും,\\
അത്താടിക്കും തടിക്കും രുചിയുടെ ലഹരിക്കുത്തടിക്കും തടിക്കും\\
നിത്യം കൂപ്പാമടിക്കും, ഗണപതി വിടുവാനായ്‌ മടിക്കും മടിക്കും.
\end{slokam}

\Letter{ഭ}{അ}

\Topic{അന്ത്യപ്രാസവും യമകവും}.  \PrevSlRef{നിത്യം നശ്ചിത്തപദ്മേ},
\NextSlRef{അശ്വത്ഥത്തിന്നിലയ്ക്കും}


\begin{slokam}{\VSr}{\Unk}{ഭക്ത്യാ കൈക്കൊണ്ടു ചിത്തേ}
ഭക്ത്യാ കൈക്കൊണ്ടു ചിത്തേ ഭഗവതി, ഭവതീം കാമരാജാങ്കശയ്യാ-\\
മധ്യാസീനാം, പ്രസന്നാം, പ്രശിഥിലകബരീസൗരഭാപൂരിതാങ്ഗാം,\\
മെത്തും മാധ്വീമദാന്ധാം, ശ്രവണപരിലസത്‌സ്വർണ്ണതാടങ്കചക്രാ,\\
മുദ്യദ്ബാലാർക്കശോണാ, മുരസി നിഹിതമാണിക്യവീണാ, മുപാസേ.
\end{slokam}

\Letter{ഭ}{മ}

\begin{slokam}{\VSr}{\Mazha}{ഭംഗ്യാ പിംഗേ ഭുജംഗാവലി}
ഭംഗ്യാ പിംഗേ ഭുജംഗാവലിരചിതവിമർദ്ദേ കപർദ്ദേ ദധാനം\\
തുംഗാൻ ഗംഗാതരംഗാൻ, നിടിലഹുതവഹജ്വാലയാ ശോഭമാനം\\
ശൃംഗാരാദ്വൈതവിദ്യാപരിമളലഹരീം വാമഭാഗേ വഹന്തം\\
മംഗല്യം കൈവളർപ്പാൻ ദിനമനു മനമേ, ചന്ദ്രചൂഡം ഭജേഥാഃ
\end{slokam}

\Letter{ഭ}{ശ}

\Book{ഭാഷാനൈഷധചമ്പു}.


\begin{slokam}{\VSr}{\Unk}{ഭംഗ്യാ വാർകൊങ്കയൊന്നും}
ഭംഗ്യാ വാർകൊങ്കയൊന്നും മണികലശയുതം മാരസാമ്രാജ്യലക്ഷ്മീ-\\
മംഗല്യാസ്ഥാനമംസദ്വയമഹിതമണിത്തോരണം മാറിടം തേ\\
ശൃംഗാരാവാസഭൂമേ, പുനരിരുപുറവും തൂക്കുമപ്പുഷ്പമാലാ-\\
ശങ്കാമംകൂരയത്യമ്പൊടു കരയുഗളീ തത്ര കൌണോത്തരേ! തേ.
\end{slokam}

\Letter{ഭ}{ശ}

\Book{പദ്യരത്നം}.


\begin{slokam}{\VDv}{\VCBP}{ഭവനമാ വനമാക്കി}
ഭവനമാ വനമാക്കി വസിച്ചിടു-\\
ന്നവരുമേവരുമേ തിരയും വിഭോ!\\
മഹിതമീ ഹിതമീ വിധമാക്കുകെ-\\
ന്നകമലം, കമലം തൊഴുമക്ഷികള്‍.
\end{slokam}

\Letter{ഭ}{മ}

\Topic{യമകം (ദ്രുതവിളംബിതം, നാലു വരിയിലും)}.  
\NextSlRef{മലയമാലയമായ}

\end{enumerate}


\subsection{മ}
\begin{enumerate}


\begin{slokam}{\VDv}{\Ull}{മലയമാലയമായ}
മലയമാലയമായ തപോധനന്‍\\
തല കുനിച്ചധരത്തിനു താഴെയും\\
ബലമൊടെത്തുമവര്‍ക്കിരു കയ്യിലും\\
വിലസി വേ, ലസി വേറെയുമായുധം.
\end{slokam}

\Letter{ഭ}{ബ}

\Book{ഉമാകേരളം}.

\Topic{യമകം (ദ്രുതവിളംബിതം, രണ്ടു വരികളിൽ)}. \PrevSlRef{ഭവനമാ വനമാക്കി}


\begin{slokam}{\VSv}{\NNM}{മങ്കത്തയ്യൊളിമെയ്‌മിനുപ്പെഴുമിളം}
മങ്കത്തയ്യൊളിമെയ്‌മിനുപ്പെഴുമിളം പത്രപ്പടർപ്പാൽ, തനി-\\
ത്തങ്കത്തൂലികകൊണ്ടു താരണിമണം കൂട്ടും നിറക്കൂട്ടിനാൽ,\\
കൺ കക്കും വിധമാതതഭ്രമറയിൽത്താനിന്നു മായാമയീ-\\
സങ്കൽപത്തെ വരയ്ക്കുമാദിമകലാകൗതൂഹലത്തെത്തൊഴാം!
\end{slokam}

\Letter{മ}{ക}

\begin{slokam}{\VSv}{\VKG}{മച്ചിത്തത്തിലടിച്ചിടും}
 മച്ചിത്തത്തിലടിച്ചിടും നിനവലച്ചാർത്തിങ്കലോരോന്നിലും\\
ത്വഛ്രീമദ്ധരിനീലകോമളമുഖം ബിംബിച്ചുകണ്ടാവു ഞാൻ,\\
കയ്ച്ചാലും മധുരിക്കിലും മധുരിപോ, നിർബ്ബാധമായ്‌ നിൻപദേ\\
വെച്ചാവൂ വിധിപോലെ, കൊച്ചുതുളസിപ്പൂപോലെ, മജ്ജീവിതം.
\end{slokam}

\Letter{മ}{ക}


\begin{slokam}{\VKm}{\PCM}{മണ്ണിലുണ്ടു കരിവിണ്ണിലുണ്ടു}
 മണ്ണിലുണ്ടു കരിവിണ്ണിലുണ്ടു കളിയാടിടുന്ന കലമാനിലും\\
കണ്ണിറുക്കി നറുപാൽ കുടിയ്ക്കുമൊരു പൂച്ച, പൂ, പുഴ, പശുക്കളിൽ\\
കണ്ണിനുള്ള വിഷയങ്ങളായവയിലൊക്കെ രാധികയറിഞ്ഞതാ\\
വെണ്ണ കട്ടവനെ; യന്നു തൊട്ടു ഹരി കണ്ണനെന്ന വിളി കേട്ടുപോൽ!
\end{slokam}

\Letter{മ}{ക}

\begin{slokam}{\VSr}{\VKG}{മണ്ണുണ്ണും, വെണ്ണയുണ്ണും,}
മണ്ണുണ്ണും, വെണ്ണയുണ്ണും, കടുതരവിഷസമ്മിശ്രമാം സ്തന്യമുണ്ണും,\\
മണ്ണും കല്ലും നിറഞ്ഞോരവിലുമരയിലച്ചീരയും തീയുമുണ്ണും,\\
തിണ്ണം ബ്രഹ്മാണ്ഡമങ്ങേക്കുടവയർ, ഗുരുവായൂരെഴും നാഥ, നീയെ-\\
ന്തുണ്ണില്ലുണ്ണീ? നിവേദിക്കുവനടിമലരിൽ കൂപ്പുമെൻ തപ്തബാഷ്പം!
\end{slokam}

\Letter{മ}{ത}

\begin{slokam}{\VSr}{\VKG}{മണ്ണും പെണ്ണും കൊതിക്കും}
 മണ്ണും പെണ്ണും കൊതിക്കും, കവിതയുടെ വളപ്പിന്റെ വേലിക്കൽ നിന്ന-\\
പ്പെണ്ണിൻ നീലക്കടക്കണ്മുന പതിയുവതിന്നാശയാലെത്തി നോക്കും,\\
ഉണ്ണാനുണ്ടെങ്കിലില്ലാത്തൊരു നില നിരുപിച്ചുള്ളുരുക്കും, നൃജന്മം\\
കണ്ണാ, ഞാൻ പാഴിലാക്കിത്തുലയുവതിനു മുമ്പെന്നെ രക്ഷിക്ക വേഗം!
\end{slokam}

\Letter{മ}{ഉ}



\begin{slokam}{\VMk}{\KV}{മദ്ധ്വാസക്തഭ്രമരമുഖരാരാമ}
മദ്ധ്വാസക്തഭ്രമരമുഖരാരാമമധ്യത്തിലുള്ളോ-\\
രദ്ധ്വാവിൽ പുക്കനവഹിതനായങ്ങുമിങ്ങും നടന്നാൽ\\
വദ്ധ്വാ ചേർന്നിട്ടതുവഴി വരും വല്യ സായിപ്പു നിന്നെ-\\
ബ്ബദ്ധ്വാ പാർപ്പിച്ചിടുമദയമായഞ്ജസാ പഞ്ജരത്തിൽ
\end{slokam}

\Letter{മ}{വ}

\Book{മയൂരസന്ദേശം}.


\begin{slokam}{\VSr}{\VCBP}{മന്നിൽക്കോളാർന്നിരമ്പും ജലനിധി}
മന്നിൽക്കോളാർന്നിരമ്പും ജലനിധി, മുകളിൽ ചാരുതാരാ സമൂഹം,\\
ചിന്നിക്കാണും നഭോമണ്ഡല, മതിനു നടുക്കുജ്ജ്വലിക്കുന്ന ചന്ദ്രൻ,\\
എന്നിസ്സർവ്വേശസൃഷ്ടിക്രമമഹിമ കുറിക്കുന്ന വസ്തുക്കളെല്ലാ-\\
മൊന്നിച്ചാഹന്ത കാൺകെക്കരളിടയിലഹംബുദ്ധി നിൽക്കുന്നതാണോ?
\end{slokam}

\Letter{മ}{എ}

\Book{വിശ്വരൂപം}.



\begin{slokam}{\VSv}{\VKG}{മയ്യഞ്ചും തിരുമെയ്യു}
മയ്യഞ്ചും തിരുമെയ്യു ചെന്നു തടവും, നക്കും പദാബ്ജങ്ങള്‍ ഞാൻ,\\
പയ്യാറ്റും മമ യാമുനോദകവുമാ വൃന്ദാവനപ്പുൽകളും,\\
നിയ്യൂതും മുരളീരവം നുകരുമെന്നായർക്കിടാവേ, വെറും\\
പയ്യായാൽ മതിയായിരുന്നു തിരുവമ്പാടിക്കകത്തന്നു ഞാൻ.
\end{slokam}

\Letter{മ}{ന}

\begin{slokam}{\VSr}{\ONN}{മർത്യാകാരേണ ഗോപീ}
മർത്യാകാരേണ ഗോപീവസനനിര കവർന്നോരു ദൈത്യാരിയെത്തൻ\\
ചിത്തേ ബന്ധിച്ച വഞ്ചീശ്വര! തവ നൃപനീതിക്കു തെറ്റില്ല, പക്ഷേ\\
പൊൽത്താർ മാതാവിതാ തൻ കണവനെ വിടുവാനാശ്രയിക്കുന്നു ദാസീ-\\
വൃത്യാ നിത്യം ഭവാനെ, ക്കനിവവളിലുദിക്കൊല്ല കാരുണ്യരാശേ!
\end{slokam}

\Letter{മ}{പ}



\begin{slokam}{\VSv}{ഇക്കാവമ്മ}{മല്ലാരിപ്രിയയായ ഭാമ}
മല്ലാരിപ്രിയയായ ഭാമ സമരം ചെയ്തീലയോ? തേർ തെളി-\\
ച്ചില്ലേ പണ്ടു സുഭദ്ര? പാരിതു ഭരിക്കുന്നില്ലെ വിക്ടോറിയാ?\\
മല്ലാക്ഷീമണികള്‍ക്കു പാടവമിവയ്ക്കെല്ലാം ഭവിച്ചീടുകിൽ\\
ചൊല്ലേറും കവിതയ്ക്കു മാത്രമവരാളല്ലെന്നു വന്നീടുമോ?
\end{slokam}

\Letter{മ}{മ}

\Book{സുഭദ്രാധനഞ്ജയം നാടകം}.


\begin{slokam}{\VMk}{\KV}{മല്ലീജാതിപ്രഭൃതി}
മല്ലീജാതിപ്രഭൃതികുസുമസ്മേരമായുല്ലസിക്കും\\
സല്ലീലാഭിഃ ‍കിസലയകരം കൊണ്ടു നിന്നെത്തലോടും\\
വല്ലീനാം നീ പരിചയരസം പൂണ്ടു കൌതൂഹലത്താ-\\
ലുല്ലീഢാത്മാ ചിരതരമിരുന്നങ്ങമാന്തിച്ചിടൊല്ലാ
\end{slokam}

\Letter{മ}{വ}

\Book{മയൂരസന്ദേശം}.



\begin{slokam}{\VSr}{കെ. പി. സി. അനുജൻ നമ്പൂതിരിപ്പാട്}{മറ്റാരുണ്ടിങ്ങെനിക്കപ്പെരുവനം}
മറ്റാരുണ്ടിങ്ങെനിയ്ക്കപ്പെരുവനമമരുന്നോരിരട്ടപ്പനല്ലാ-\\
തുറ്റാളായ് ജീവിതത്തിൽ തുണയരുളുവതിന്നിന്നുപിന്നെന്നുമെന്നും\\
തെറ്റാവാം മറ്റു ദൈവങ്ങളിലിവനകമേ ഭക്തി തോന്നുന്നതില്ലേ\\
ചെറ്റാരാലെന്തു കൊണ്ടോ പൊരുളരുളുകിലോ ഭാഗ്യമാണെന്റെ ഭാഗ്യം
\end{slokam}

\Letter{മ}{ത}

\begin{slokam}{\VSv}{\UN}{മാടിൻ പാലൊരു തുള്ളിവിട്ടു}
 മാടിൻ പാലൊരു തുള്ളിവിട്ടു മുഴുവൻ തൂവെണ്ണയോ, ടാറ്റിൽ നീ-\\
രാടും ഗോപവധുക്കള്‍ തൻ തുണി ഹൃദന്തത്തോടെ, ദുശ്ചിന്തകള്‍\\
മൂടും മാനസമാർന്നൊരെന്നഴലിതാ പാപങ്ങളോടും ഹരി-\\
ച്ചോടുന്നൂ ഹരി, യെന്തു ചെയ്‌വു തടയാൻ? കാലിൽ പിടിക്കുന്നു ഞാൻ!
\end{slokam}

\Letter{മ}{മ}


\begin{slokam}{\VSv}{\KT}{മാതംഗാനന, മംബ്ജവാസരമണീം}
മാതംഗാനന, മംബ്ജവാസരമണീം, ഗോവിന്ദമാദ്യം ഗുരും,\\
വ്യാസം, പാണിനി ഗർഗനാരദ കണാദാദ്യാൻ മുനീന്ദ്രാൻ ബുധാൻ,\\
ദുർഗാം ചാപി മൃദംഗശൈലനിലയാം ശ്രീപോർക്കലീമിഷ്ടദാം\\
ഭക്ത്യാ നിത്യമുപാസ്മഹേ സപദി നഃ കുർവന്ത്വമീ മംഗളം
\end{slokam}

\Letter{മ}{ദ}

\begin{slokam}{\VSv}{\VNM}{മാനം ചേർന്ന ഭടന്റെ}
മാനം ചേർന്ന ഭടന്റെ മിന്നൽ ചിതറും കൈവാളിളക്കത്തിലും,\\
മാനഞ്ചും മിഴി തൻ മനോരമണനിൽച്ചായുന്ന കൺകോണിലും,\\
സാനന്ദം കളിയാടിടുന്ന ശിശുവിൻ തൂവേർപ്പണിപ്പൂങ്കവിള്‍-\\
സ്ഥാനത്തും, നിഴലിച്ചു കാണ്മു കവിതേ, നിൻ മഞ്ജുരൂപത്തെ ഞാൻ.
\end{slokam}

\Letter{മ}{സ}

\Book{കവിത}.

\begin{slokam}{\VSr}{\VCBP}{മാനം, മര്യാദ, മാന്യ}
മാനം, മര്യാദ, മാന്യപ്രണയമധുരമാം ശീല, മൊക്കുന്ന മട്ടിൽ\\
ദാനം തൊട്ടുള്ള നാനാ ഗുണവിഭവമിണങ്ങീടുമെൻ പ്രാണനാഡി!\\
ജ്ഞാനധ്യാനൈകരൂപാമൃതമണയുവതിന്നുള്ള നിന്നന്ത്യയാത്ര-\\
യ്ക്കാനന്ദം കൈവരട്ടേ, തവ വിമല കഥാവസ്തു ശേഷിച്ചിടട്ടെ!
\end{slokam}

\Letter{മ}{ജ}


\begin{slokam}{\VSv}{\Naduv}{മാന്യൻമാർ പലരും}
മാന്യന്മാർ പലരും നിറഞ്ഞ സഭയിൽ ദുർബുദ്ധി ദുശ്ശാസനൻ\\
ചെന്നാ ദ്രൗപദിദേവിതന്റെ ചികുരം ചുറ്റിപ്പിടിച്ചങ്ങിനേ\\
നിന്നീടാതെ വലിച്ചിഴച്ചതുകിടക്കട്ടേ മഹാകഷ്ടമാ-\\
ത്തന്വംഗീമണിതന്നുടുപ്പുടവ തൻ കൈകൊണ്ടഴിച്ചീലയോ
\end{slokam}

\Letter{മ}{ന}

\Book{ഭഗവദ്ദൂത് നാടകം}.


\begin{slokam}{\VMk}{\KV}{മാരക്രീഡാമഹലഹളയിൽ}
മാരക്രീഡാമഹലഹളയിൽ ജാലമാർഗ്ഗേണ ലീലാ-\\
ഗാരക്രോഡേ നിഭൃതഗതിയായെത്തി നിത്യം നിശായാം\\
വാരസ്ത്രീണാം വപു‍ഷി വിലസും സ്വേദബിന്ദുക്കളാകും\\
ഹാരസ്തോമം ഹരതി വിരുതേറുന്ന ചോരൻ സമീരൻ
\end{slokam}

\Letter{മ}{വ}

\Book{മയൂരസന്ദേശം}.


\begin{slokam}{\VSr}{\Unk}{മാരൻ പൂമെയ്‌ കരിക്കാം}
 മാരൻ പൂമെയ്‌ കരിക്കാ, മരിയ പുരമെരിക്കാ, മെരിക്കും ധരിക്കാം,\\
പാരീരെഴും ഭരിക്കാം, പരിചിനൊടുമുടിക്കാം, നടിക്കാം ചിതായാം,\\
ഗൗരിക്കങ്ഗം പകുക്കാം, ഝടിതി കുടുകുടെക്കാളകൂടം കുടിക്കാ,-\\
മോരോന്നേ വിസ്മയം നിൻ തിരുവുരു തിരുവൈക്കത്തെഴും തിങ്കള്‍മൗലേ!
\end{slokam}

\Letter{മ}{ഗ}


\begin{slokam}{\VSr}{\OKM}{മിന്നൽക്കൊക്കുന്ന പൂമെയ്പ്പൊലിമ}
മിന്നൽക്കൊക്കുന്ന പൂമെയ്പ്പൊലിമയു, മകതാരിട്ടുലയ്ക്കും മുലക്കു,-\\
ന്നന്നപ്പോക്കും, മഴക്കാറെതിർതലമുടിയും, മുല്ലമൊട്ടൊത്ത പല്ലും,\\
കന്നൽക്കണ്ണും, കടുംചോപ്പുടയ ചൊടികളും കാണുകിൽ കൊച്ചുതെക്കൻ-\\
തെന്നൽത്തേരിൽക്കരേറുന്നവനുടെ തറവാട്ടമ്മയോയെന്നു തോന്നും.
\end{slokam}

\Letter{മ}{ക}

\Topic{പച്ചമലയാളശ്ലോകം}.

\begin{slokam}{\VSv}{\VNM}{മീനാങ്കോപമ, കൺകലക്കം}
മീനാങ്കോപമ, കൺകലക്കമവിടേയ്ക്കൊട്ടല്ലുറങ്ങായ്കയാൽ\\
മ്ലാനാപാണ്ഡുരമായ്ച്ചമഞ്ഞിതു മണം വീശുന്ന പൂമേനിയും;\\
ഞാനായിന്നലെ രാത്രിമാത്രമയി, ഹാ, വേർപെട്ടതിൻ മൂലമീ-\\
ദ്ദൂനാവസ്തയിലായ്‌ ഭവാൻ; മയി തവ സ്നേഹം മഹത്തെത്രയും!
\end{slokam}

\Letter{മ}{ഞ}

\begin{slokam}{\VMt}{\SNG}{മീനായതും ഭവതി മാനായതും}
 മീനായതും ഭവതി മാനായതും ജനനി നീ നാഗവും നഗഖഗം\\
താനായതും ധരനദീനാരിയും നരനുമാനാകവും നരകവും\\
നീ നാമരൂപമതിൽ നാനാവിധപ്രകൃതി മാനായി നിന്നറിയുമീ\\
ഞാനായതും ഭവതി ഹേ നാദരൂപിണി, യഹോ! നാടകം നിഖിലവും.
\end{slokam}

\Letter{മ}{ന}

\begin{slokam}{\VSr}{\KKT}{മുച്ചാണ്‍ പൊക്കം കലര്‍ന്നാ മുരരിപു}
മുച്ചാണ്‍ പൊക്കം കലര്‍ന്നാ മുരരിപു ഭഗവാന്‍, മൂന്നടിബ്ഭൂമി വാങ്ങി-\\
സ്സ്വച്ഛന്ദം രണ്ടുകാല്‍ വെച്ചുലകു മുഴുവനും നേടി നേരിട്ടിടുമ്പോള്‍,\\
വെച്ചാലും കാലു മൂന്നാമതു മമ തലയില്‍ തന്നെ,യെന്നങ്ങു ധൈര്യം\\
വെച്ചോതും, വീര വൈരോചനി വചനമതോര്‍ത്തത്ഭുതപ്പെട്ടിടുന്നേന്‍! 
\end{slokam}

\Letter{മ}{വ}

\begin{slokam}{\VSr}{\Vyl}{മുട്ടാതേർപ്പെട്ടു}
മുട്ടാതേർപ്പെട്ടു മുവ്വാണ്ടിടയിൽ മറുകരയ്ക്കെത്തിയാദ്ധീര, നാഴി-\\
ക്കെട്ടാളും കേരളത്തിന്നിനിയൊരു കടലും കൂടി നേടിക്കൊടുത്തൂ \\
കെട്ടാതുണ്ടാപ്പരപ്പിൽ ദ്രുപദതനയ തൻ വേണി, കോപം നുരയ്ക്കും\\
മട്ടാം ഭീമാട്ടഹാസം, ഭവഹരഭഗവത്പാഞ്ചജന്യപ്രണാദം!
\end{slokam}

\Letter{മ}{ക}

\begin{slokam}{\VKm}{\VKG}{മുട്ടുകുത്തി, മണിമണ്ഡന}
മുട്ടുകുത്തി, മണിമണ്ഡനസ്വനമുയർന്നിടാതെ, യതിസാഹസ-\\
പ്പെട്ടിഴഞ്ഞു, കതകൊച്ചയറ്റവിധമായ്‌ തുറന്നു, ചരിതാർത്ഥനായ്‌\\
കട്ടിലിൻ മുകളിലെത്തിനിന്നുറിയിൽ വെച്ച വെണ്ണ മലർവായ്ക്കക-\\
ത്തിട്ടു കട്ടുപുലരുന്ന തസ്കരകലാവിശാരദനു കൈതൊഴാം
\end{slokam}

\Letter{മ}{ക}


\begin{slokam}{\VSr}{\PG}{മുണ്ടാക്കക്ഷത്തു ചുറ്റി}
മുണ്ടാക്കക്ഷത്തു ചുറ്റിദ്ദൃഢമിരുകരവും മാറിലമ്മാറു കെട്ടി-\\
ക്കുണ്ടാളും ചിന്ത മൂലം തല ചെറുതു കുനിച്ചക്കവീന്ദ്രൻ ചിലപ്പോള്‍\\
കണ്ടാൽ കാണാത്ത ഭാവത്തൊടു മെതിയടിമേൽ വീട്ടുമുറ്റത്തുലാത്തു-\\
ന്നുണ്ടാ, മപ്പോളുറയ്ക്കാം പ്രതിഭ കവിത തൻ പേറ്റുനോവേറ്റുവെന്നായ്‌
\end{slokam}

\Letter{മ}{ക}

\Topic{രേഖാചിത്രം}.


\begin{slokam}{\VSv}{\DSN}{മുത്താണെന്നു നിനച്ചു ഞാൻ}
മുത്താണെന്നു നിനച്ചു ഞാൻ നറുമഴത്തുള്ളിയ്ക്കു കൈ നീട്ടു, മുത്-\\
കൃഷ്ടാകർഷകഹേമമെന്നുകരുതിപ്പൂമൊട്ടിനെപുൽകിടും,\\
ശുദ്ധാകർഷകഹേമമെന്നു കരുതിത്തേടും പ്രഭാതക്കതിർ-\\
ത്തൊത്താകാശവുമിസ്ഫുരത്പ്രകൃതിയും മോഹിച്ചിടാതാരു താൻ?
\end{slokam}

\Letter{മ}{ശ}



\begin{slokam}{\VSr}{\ONN}{മുപ്പാരും കാക്കുവാനില്ലപരൻ}
മുപ്പാരും കാക്കുവാനില്ലപര,നൊരു മകൻ ഭുക്തിയിൽ തൃപ്തിയില്ലാ-\\
തെപ്പോഴും വന്നലട്ടും പരിണയമണയാപ്പെൺകിടാവുണ്ടൊരുത്തി,\\
വിൽപ്പാനുള്ളോരു പണ്ടം നഹി, പകലുദധൗ സോദരൻ, തെണ്ടി ഭർത്താ-\\
വിപ്പാടാർക്കുള്ളു വേറേ? തവ \sam{മലമകളേ, ജാതകം ജാതി തന്നെ}!
\end{slokam}

\Letter{മ}{വ}


സമസ്യാപൂരണം. മറ്റു പൂരണങ്ങൾ: \SlRef{എല്ലായ്പോഴും കളിപ്പാൻ ചുടല}, \SlRef{പെറ്റോരാ മക്കളെല്ലാമപകടം}, \SlRef{മെയ്യിൽപ്പാമ്പുണ്ടനേകം}.


\begin{slokam}{\VSr}{കെ ആർ കെ കുറുപ്പ്}{മുറ്റത്തീണത്തിലോടി}
മുറ്റത്തീണത്തിലോടി, ക്കുസൃതികള്‍ പലതും കാട്ടി, ഞാൻ വാടിവീഴ്കെ-\\
ത്തെറ്റെന്നെത്തിക്കരത്താലുടനടി നെടുതായ്‌ താങ്ങി മെയ്യിൽത്തലോടി,\\
മുറ്റും മുത്തങ്ങളേകി, ത്തിറമൊടു മടിയിൽ വെച്ചു, മമ്മിഞ്ഞ തന്നും\\
മറ്റും പാലിച്ചൊരമ്മേ, തവ പദമലർ വിട്ടില്ല മറ്റാശ്രയം മേ
\end{slokam}

\Letter{മ}{മ}

\begin{slokam}{\VSv}{\VRV}{മുറ്റത്തെത്തുളസിത്തറയ്ക്കലെരിയും}
മുറ്റത്തെത്തുളസിത്തറയ്ക്കലെരിയും തൃക്കല്‍വിളക്കല്ല, വെണ്‍- \\
കൊറ്റപ്പൂങ്കുടചൂടി വാണ നൃപവംശത്തിന്‍ വിഴുപ്പല്ല ഞാന്‍. \\
ഒറ്റയ്ക്കല്ല, മനുഷ്യമാനസദിവാസ്വപ്നങ്ങള്‍തന്‍ പൊന്‍കതിര്‍- \\
ക്കറ്റക്കെട്ടു ചുമന്നു വന്നയുഗസതൃത്തിന്‍ വിയര്‍പ്പാണു ഞാന്‍! 
\end{slokam}

\Letter{മ}{ഒ}

\Book{അദ്ധ്വാനത്തിൻ വിയർപ്പാണു ഞാൻ}

\begin{slokam}{\VSr}{\HM}{മുറ്റത്തേയ്ക്കെത്തിനോക്കി}
മുറ്റത്തേയ്ക്കെത്തിനോക്കിക്കതിരവനിരുളിൻ കാടു മൊത്തം വകഞ്ഞും,\\
ചുറ്റും ശ്രദ്ധിച്ചു കൊണ്ടും തറയുടെ നടുവിൽ കൈയു വയ്ക്കുന്ന കാണ്ക!\\
ഒറ്റക്കാര്യത്തിനാവാം, പ്രണയിനി പുലരിപ്പെൺകിടാവാഗ്രഹിക്കേ\\
പറ്റുംമട്ടേകണംപോൽ, ഒരു നറു തുളസിപ്പൊൻകതിർക്കെട്ടു, ചൂടാൻ!
\end{slokam}

\Letter{മ}{ഒ}


\begin{slokam}{\VSv}{\Unk}{മുറ്റാതോരണിതിങ്കൾമങ്ഗലഗുണേ}
മുറ്റാതോരണിതിങ്കൾമങ്ഗലഗുണേ! നിൻനെറ്റിയോടേറ്റുടൻ\\
മുറ്റും തോറ്റതുമൂലമമ്മുകിലിടെപ്പോയ്പ്പുക്കൊളിക്കിൻറതും,\\
മറ്റും ബാഷ്പേചയം തുഷാരസലിലവ്യാജേന വാക്കിൻറതും;\\
പറ്റാ, കാൺ, ബലിനാ വിരോധമെളിയോർക്കേണാക്ഷി, കൌണോത്തരേ !
\end{slokam}

\Letter{മ}{മ}

\Book{പദ്യരത്നം}.


\begin{slokam}{\VSv}{\VRV}{മൂടിക്കെട്ടിയ മൗനമല്ല}
മൂടിക്കെട്ടിയ മൗനമല്ല, നിഴലിൻ നീലത്തടാകങ്ങളിൽ\\
വാടിക്കൂമ്പിയ മോഹഭംഗമലരിൻ മൊട്ടല്ല മുത്തല്ല ഞാൻ\\
കാടിന്നുള്ളിലരിച്ചു വീണ വെയില,ല്ലന്തർമുഖദ്ധ്യാനമാം\\
കൂടിന്നുള്ളിലെ നിദ്രയല്ല, പുലർകാലത്തിൻ ചുവപ്പാണു ഞാൻ.
\end{slokam}

\Letter{മ}{ക}

\Book{അദ്ധ്വാനത്തിൻ വിയർപ്പാണു ഞാൻ}

\begin{slokam}{\VSv}{\Poonth}{മൂടില്ലാത്തൊരു മുണ്ടുകൊണ്ടു}
മൂടില്ലാത്തൊരു മുണ്ടുകൊണ്ടു മുടിയും മൂടീട്ടു വൻ കറ്റയും\\
ചൂടിക്കൊണ്ടരിവാള്‍ പുറത്തു തിരുകി പ്രാഞ്ചിക്കിതച്ചങ്ങിനെ\\
നാടൻ കച്ചയുടുത്തു മേനിമുഴുവൻ ചേറും പുരണ്ടിപ്പൊഴീ-\\
പ്പാടത്തുന്നു വരുന്ന നിൻ വരവു കണ്ടേറെക്കൊതിക്കുന്നു ഞാൻ!
\end{slokam}

\Letter{മ}{ന}



\begin{slokam}{\VSv}{\UN}{മൂടുന്നൂ ഭയവും നിരാശതകളും}
മൂടുന്നൂ ഭയവും നിരാശതകളും വല്മീകമായ്, ജീവിത--\\
ക്കാടിന്നുള്ളിലലഞ്ഞിടു, ന്നപരരെത്തട്ടിപ്പറിക്കുന്നു ഞാൻ,\\
വേടന്മാരൊടു പദ്യമോതു, മൊരുവർ കൂടില്ല പാപം ചുമ--\\
ന്നീടാ, നിന്നു നിഷാദ, നെന്നു കവിയായ് വാല്മീകിയായ് മാറിടും?
\end{slokam}

\Letter{മ}{വ}

\begin{slokam}{\VSr}{\KKT}{മൂഢന്നും പണ്ഡിതന്നും}
 മൂഢന്നും പണ്ഡിതന്നും പെരിയ ധനികനും പിച്ച തെണ്ടുന്നവന്നും\\
പ്രൗഢന്നും പ്രാകൃതന്നും പ്രഭുവിനിടയനും കണ്ട നായ്ക്കും നരിക്കും\\
ബാഢം വ്യാപിക്കുമാറായ്പ്പകലുമിരവിലും ലോകമോർക്കാതെ മായാ-\\
ഗൂഢക്കയ്യാൽ മയക്കും മഹിതമരണമേ! നിന്റെ ഘോഷം വിശേഷം.
\end{slokam}

\Letter{മ}{ബ}

\begin{slokam}{\VSr}{\Ull}{മൂലം നാള്‍ മുറ്റുമാളും}
 മൂലം നാള്‍ മുറ്റുമാളും മുഴുസുകൃതഫലം, മുഖ്യവഞ്ചിക്ഷിതിശ്രീ-\\
മൂലം, മൂർദ്ധാഭിഷിക്തവ്രജമുകുടമിളന്മുഗ്ദ്ധമുക്താകലാപം,\\
പാലംഭോരാശികന്യാപതിഭജനപരാധീന, മന്യൂനകീർത്തിയ്‌-\\
ക്കാലംബം, രാമവർമ്മാഭിധ, മവനകലാലാലസം, ലാലസിപ്പൂ!
\end{slokam}

\Letter{മ}{പ}

\begin{slokam}{\VSv}{\ARRV}{മെയ്യില്‍ പാര്‍വ്വതി പാതി}
മെയ്യില്‍ പാര്‍വ്വതി പാതി, പാതി ഹരിയും പങ്കിട്ടെടുത്തീടവേ\\
പോയല്ലോ ഹരനെന്നു ഗംഗയുടനേ ചെന്നങ്ങു ചേര്‍ന്നാഴിയില്‍\\
വാനത്തമ്പിളിലേഖ, പാമ്പു കുഴിയില്‍, സര്‍വജ്ഞതാധീശതാ-\\
സ്ഥാനം രണ്ടു ഭവാങ്ക, ലെങ്കലുമഹോ ഭിക്ഷാടനം ഭൂപതേ!
\end{slokam}

\Letter{മ}{വ}

പരിഭാഷ.  \OSlRef{അർദ്ധം ദാനവവൈരിണാ}.

\begin{slokam}{\VSr}{\Naduv}{മെയ്യിൽപ്പാമ്പുണ്ടനേകം}
മെയ്യിൽപ്പാമ്പുണ്ടനേകം ഗളമതിൽ വിലസും കാളകൂടം കഠോരം,\\
കയ്യിൽ ശൂലം, കഠാരം, തിരുമിഴിയിതു തീക്കട്ട, വേഷം വിശേഷം,\\
അയ്യോ! നിൻകാന്തനൊത്തുള്ളൊരുപൊറുതി മഹാദുർഘടം തന്നെ, യോർത്താൽ\\
വയ്യേ! മറ്റാർക്കുമില്ലിങ്ങനെ \sam{മലമകളേ, ജാതകം ജാതി തന്നെ}!
\end{slokam}

\Letter{മ}{അ}


സമസ്യാപൂരണം. മറ്റു പൂരണങ്ങൾ: \SlRef{എല്ലായ്പോഴും കളിപ്പാൻ ചുടല}, \SlRef{പെറ്റോരാ മക്കളെല്ലാമപകടം}, \SlRef{മുപ്പാരും കാക്കുവാനില്ലപരൻ}.



\begin{slokam}{\VSr}{\Unk}{മേഘം, വണ്ടിണ്ട, ചന്ദ്രക്കല,}
 മേഘം, വണ്ടിണ്ട, ചന്ദ്രക്കല, മദനധനുർബ്ബാണ, മെള്‍പ്പൂവു, പാശം,\\
ചാദർശം, വീണ, വെണ്മുത്തഴകിയ പവഴം, പങ്കജം, ശംഖു, മാല\\
പൊൽക്കുംഭം, പാമ്പു, നീരിൻ ചെറുതിര, യരയാൽപ്പത്ര, മാവർത്തചക്രം,\\
തുമ്പിക്കൈ, കുപ്പി, കൂർമ്മം, നളിന -- മവയവം നാരണീനന്ദനായാഃ
\end{slokam}

\Letter{മ}{പ}

\Book{ലീലാതിലകം}.

\begin{slokam}{\VSv}{\NKD}{മേലേ വെൺ കുട പോൽ}
മേലേ വെൺ കുട പോൽ വിടർന്ന ഫണചക്രം; തൻ വലം കൈത്തല- \\
ത്താലല്പോന്നമിതം മുഖം; രമ തലോടും പാദപാഥോരുഹം;\\
പാലാഴിത്തിര ചുറ്റിലും; തുയിലിനായയ്യയ്യ! ഭോഗോല്ലസ-\\
ദ്ദോലാതല്പവുമീയനന്തശയനം കാണായ് വരേണം ക്ഷണം! 
\end{slokam}

\Letter{മ}{പ}

\begin{slokam}{\VSv}{\RV}{മേളിച്ചേറെ രസിച്ച}
മേളിച്ചേറെ രസിച്ച തോഴരൊടുവിൽപ്പോകും മടങ്ങാതെ, യി-\\
ക്കേളിക്കൊട്ടിലൊഴിഞ്ഞു മണ്ണിലലിയും മണ്ണായ്‌ക്കളിക്കോപ്പുകള്‍\\
കാളും കാളിമ ചൂഴുവാനണയുമന്നേരം തനിച്ചാകുമെൻ\\
നീളും പാതയിലെന്നെനോക്കി വരികില്ലേ, കാളി, യെന്നമ്മ നീ? 
\end{slokam}

\Letter{മ}{ക}

\end{enumerate}


\subsection{യ}

\begin{enumerate}

\begin{slokam}{\VSv}{\VNM}{യക്ഷാധീശ്വരപട്ടമോ, മഹിതമാം}
യക്ഷാധീശ്വരപട്ടമോ, മഹിതമാം സ്വാരാജ്യസാമ്രാജ്യമോ,\\
ത്ര്യക്ഷാദിത്രിദശാധികാരനിലയോ വേണ്ടാ നമുക്കെൻ വിഭോ!\\
ലക്ഷാദിത്യസമാനമായൊരനഘജ്യോതിസ്സു ചിന്നുന്ന നി-\\
ന്നക്ഷാമാദ്ഭുതചിത്സ്വരൂപമകമേ കാണായ്‌ വരേണം സദാ!
\end{slokam}

\Letter{യ}{ല}

\begin{slokam}{\VSv}{\YK}{യാഗത്തിന്നു വധിച്ചിടാം പശുവിനെ}
യാഗത്തിന്നു വധിച്ചിടാം പശുവിനെ, പ്പയ്യാറ്റുവാൻ കൊൽവതാ-\\
ണാഗസ്സെന്നു വിധിപ്പതും, ലഹരിപാനീയത്തെ വർജ്ജിക്കുവോർ\\
ഭോഗത്താൽ ക്രതുവേദിതന്നിലതിനെസ്സാധൂകരിക്കുന്നതും,\\
യോഗക്ഷേമവിധിജ്ഞരേ! വരമുനിപ്രോക്തങ്ങളെന്നോ മതം?
\end{slokam}

\Letter{യ}{ഭ}



\end{enumerate}


\subsection{ര}

\begin{enumerate}


\begin{slokam}{\VSr}{\Unk}{രക്ഷോനാഥൻ കുലുക്കീ}
രക്ഷോനാഥൻ കുലുക്കീ ശിഖരി പൊടിപെടുത്താൻ കൊടുത്താൻ വരം നീ\\
വിൽക്കോലാൽ തച്ചുപൂജിച്ചരുളിന വിജയന്നസ്ത്രമോർത്തല്ലി നൽകീ\\
തൃക്കാലുൾക്കാമ്പിലാക്കിക്കലിതരുചി ഭജിക്കും നമുക്കേതുമേഹേ\\
മുഷ്കേ നിന്നോടു നല്ലൂ കരുണ തരുവതിന്നാശു ചെല്ലൂർ പിരാനേ!
\end{slokam}

\Letter{ര}{ത}

\Book{ചെല്ലൂർനാഥസ്തവം}. 


\begin{slokam}{\VSv}{\VNM}{രണ്ടായ്‌ നീങ്ങിയകന്നു}
 രണ്ടായ്‌ നീങ്ങിയകന്നു നിന്നിവിടെ നാം വിസ്തീർണ്ണമാർഗ്ഗം ശരി-\\
ക്കുണ്ടാക്കുന്നു വിടേശജർക്കു വിജയപ്രാസാദമുള്‍പ്പൂകുവാൻ;\\
പണ്ടാ പ്രാജ്ഞപിതാക്കള്‍ ചെയ്ത പടി, നാം തോളോടു തോളായ്‌ നില-\\
ക്കൊണ്ടാലോ, മതിൽ വേറെ വേണ്ട, ഭരതക്ഷേത്രത്തെ രക്ഷിക്കുവാൻ!
\end{slokam}

\Letter{ര}{പ}

\begin{slokam}{\VKm}{\VenM}{രണ്ടുകയ്യിലുമുരുണ്ട വെണ്ണ}
രണ്ടുകയ്യിലുമുരുണ്ട വെണ്ണയുമിരുണ്ടു നീണ്ട കചഭാരവും\\
കണ്ഠദേശമതിൽ വണ്ടണഞ്ഞ മലർകൊണ്ടു തീർത്ത വനമാലയും\\
പൂണ്ടു, പായസവുമുണ്ടുകൊണ്ടഴകിലണ്ടർകോൻനദിയിലാണ്ടെഴും\\
കൊണ്ടൽവർണ്ണ ജയ! മണ്ടിവന്നു കുടികൊണ്ടുകൊള്‍ക മനമേറി മേ.
\end{slokam}

\Letter{ര}{പ}


\begin{slokam}{\VMk}{\KV}{രണ്ടും മൂന്നും തവണ കൃഷി}
രണ്ടും മൂന്നും തവണ കൃഷിയേറ്റുന്ന കണ്ടങ്ങളേയും\\
വണ്ടും ഞണ്ടും വടിവൊടു കളിക്കുന്ന കച്ഛങ്ങളേയും\\
തണ്ടും കെട്ടിത്തരമൊടു ചരിക്കുന്ന വള്ളങ്ങളേയും\\
കണ്ടുംകൊണ്ടച്ചറുപുഴകൾതൻ തീരമാർഗ്ഗേണ പോക.
\end{slokam}

\Letter{ര}{ത}

\Book{മയൂരസന്ദേശം}.


\begin{slokam}{\VSv}{\VNM}{രാവിൽ സ്വൈരമനിദ്രയായ്‌}
 രാവിൽ സ്വൈരമനിദ്രയായ്‌, ത്വയി ലയിച്ചാനീലപത്രാഭമാം\\
ദ്യോവിൽ പൊന്മഷി കൊണ്ടു തന്നെ പലതും കുത്തിക്കുറിക്കുന്നു താൻ,\\
ആവില്ലെന്നഥ മായ്ച്ചിടുന്നു, കുതുകാൽ വീണ്ടും തുടങ്ങുന്നു - പേർ-\\
ത്തീ വിശ്വപ്രകൃതിക്കുമത്ര വശയായിട്ടില്ല ദുഷ്‌പ്രാപ നീ!
\end{slokam}

\Letter{ര}{അ}

\begin{slokam}{\VSr}{\YK}{രേതോരൂപത്തിലച്ഛൻ}
രേതോരൂപത്തിലച്ഛൻ ജനനിയിലൊഴുകിച്ചേർന്നു ഞാനായതോർത്താ-\\
ലേതോ മുത്തച്ഛനാദ്യൻ ജനകനി, ലതിനും മുമ്പെനിക്കിപ്പുറത്തും,\\
ചേതോഗുപ്തൻ നിതാന്തൻ ജഗദധിപതി ജീവോർജ്ജമായുജ്ജ്വലിപ്പൂ\\
വീതോൽക്കമ്പം, സ്വദിക്കാം ദ്യുതിയതു കരളിൻ കണ്ണിലെപ്പുണ്ണകന്നാൽ.
\end{slokam}

\Letter{ര}{ച}



\begin{slokam}{\VSv}{\Naduv}{രേ രേ കർണ്ണ, രണത്തിനായ്‌ വരിക}
രേ രേ കർണ്ണ, രണത്തിനായ്‌ വരിക, നിൻ സാമർത്ഥ്യവും ബന്ധുവാ-\\
യോരക്കൗരവരാജനിത്തിരിസഹായിക്കുന്നതും കാണണം\\
മാരാരാതികൃപാവിലാസമിവനുണ്ടെന്നാകിൽ വൈകാതെ നീ\\
ചേരും കാലനികേതനത്തിലതിനീ ലക്ഷ്മീശനും സാക്ഷിയാം
\end{slokam}

\Letter{ര}{മ}

\Book{ഭഗവദ്ദൂത് നാടകം}.
\end{enumerate}

\subsection{ല}

\begin{enumerate}

\begin{slokam}{\VSr}{\VenM}{ലക്ഷ്യം കൂടാതെ ലങ്കാ}
 ലക്ഷ്യം കൂടാതെ ലങ്കാനഗരമതു തകർത്തക്ഷമം രൂക്ഷനാകും\\
രക്ഷോജാലാധിപത്യം തടവിന ദശകണ്ഠന്റെ കണ്ഠം മുറിപ്പാൻ\\
ലക്ഷ്യം വച്ചങ്ങു ചീറി ദ്രുതമണയുമൊരത്യുഗ്രമാം രാമബാണം\\
രക്ഷിച്ചീടട്ടെ നിത്യം കലിമലമകലെപ്പോക്കി നന്നാക്കി നമ്മെ.
\end{slokam}

\Letter{ല}{ല}

\begin{slokam}{\VSv}{\VNM}{ലാവണ്യക്കടലിൽക്കളങ്കം}
 ലാവണ്യക്കടലിൽക്കളങ്കമിയലാതുണ്ടായ വാർതിങ്കളോ,\\
പൂവമ്പന്റെ പുകഴ്ച കാട്ടി വിലസും പുത്തൻ കൊടിക്കൂറയോ,\\
ദൈവത്തിന്റെ വിചിത്രസൃഷ്ടിവിരുതോ, ശൃംഗാരസൂക്തോന്മിഷൽ-\\
കൈവല്യപ്പൊരുളോ, നമുക്കെതിരിലിക്കാണാകുമേണാക്ഷിയാള്‍?
\end{slokam}

\Letter{ല}{ദ}

\Book{വിലാസലതിക}.


\begin{slokam}{\VSr}{\VCBP}{ലാവണ്യം കൊണ്ടിണങ്ങും പുതുമ}
ലാവണ്യം കൊണ്ടിണങ്ങും പുതുമ, കവിതകൊണ്ടുള്ള സത്‌കീർത്തി, വിദ്വദ്‌-\\
ഭാവം കൊണ്ടുള്ള മാന്യസ്ഥിതി, രണപടുതാമൂലമാം വൻ പ്രതാപം\\
ഈവണ്ണം വർണനീയം ഗുണമഖിലമൊരേ വാതിലിൽ തട്ടിമുട്ടി-\\
ജ്ജീവത്താമാദിമൂലപ്രകൃതിയിലൊടുവിൽ ചെന്നുചേരുന്നുവല്ലോ.
\end{slokam}

\Letter{ല}{ഇ}

\Book{ഒരു വിലാപം}.


\begin{slokam}{\VMk}{\KV}{ലീലാരണ്യേ വിഹഗ}
ലീലാരണ്യേ വിഹഗമൃഗയാലോലനായേകദാ ഞാൻ\\
നീലാപാംഗേ! കമപി നിഹനിച്ചീടിനേൻ നീഡജത്തേ\\
മാലാർന്നാരാൽ മരുവുമിണയേക്ക​ണ്ടു നീ താം ച നേതും\\
കാലാഗാരം സപദി കൃപയാ കാതരേ! ചൊല്ലിയില്ലേ?
\end{slokam}

\Letter{ല}{മ}

\Book{മയൂരസന്ദേശം}.


\begin{slokam}{\VSv}{\CKP}{ലോകം ശാശ്വതമല്ല, ജീവിത}
 ലോകം ശാശ്വതമല്ല, ജീവിതസുഖസ്വപ്നങ്ങള്‍ മായും, വരും\\
ശോകം, മായികബുദ്ബുദങ്ങള്‍ മറയും, പായും സരിത്സഞ്ചയം,\\
നാകം കാല്‌പനികോത്സവാങ്കിതലസത്ക്കാനൽജലം - പിന്നെയെ-\\
ന്തേകം, സത്യ, മനശ്വരം? മൃതി - അതേ, മൃത്യോ, ജയിക്കുന്നു നീ!
\end{slokam}

\Letter{ല}{ന}

\begin{slokam}{\VSr}{\VKG}{ലാളിച്ചീടാൻ യശോദാ}
 ലാളിച്ചീടാൻ യശോദാകരലതികകളിൽ പിഞ്ചുകുഞ്ഞായി, ലോകം\\
പാലിച്ചീടാൻ കഠോരാസുരവരനികരധ്വംസിയായ്‌, കംസജിത്തായ്‌,\\
കേളിക്കാടാൻ വ്രജസ്ത്രീജനഹൃദയമണിപ്പൊത്തിലെത്തത്തയായും\\
മേളിച്ചീടുന്ന വാതാലയസുകൃതപതാകയ്ക്കിതാ കുമ്പിടുന്നേൻ.
\end{slokam}

\Letter{ല}{ക}


\begin{slokam}{\VSr}{\Unk}{ലോകാനാമേകനാഥം}
ലോകാനാമേകനാഥം, പദതളിരിൽ വണങ്ങും ജനാനാമശേഷാ-\\
മാകാംക്ഷാം പൂരയന്തം, നയനശിഖിശിഖാലീഢചൂതായുധാംഗം,\\
ഏകീഭാവായ കുന്നിൻമകളെ നിജശരീരാർദ്ധമായ്‌ ചേർത്തു, പേർത്തും\\
ഭോഗോന്മേഷം വളർക്കും വിബുധപരിവൃഢം ചന്ദ്രചൂഡം ഭജേഥാഃ.
\end{slokam}

\Letter{ല}{എ}


\begin{slokam}{\VSv}{\RN}{ലോകാലോകകലാമയീ}
ലോകാലോകകലാമയീ, സകലസമ്പൂർണ്ണേ, യപർണ്ണേ, ജഗ-\\
ന്മാതേ, മംഗളകാരിണീ, യമൃതസംഗീതപ്രഭാവർഷിണീ, \\
ശോകാർത്തിപ്രഹരേ, വിമൂകതിമിരം വാചാലസൂര്യോദയ-\\
ശ്രീയാക്കും മഹിതേ, പ്രണാമമഖിലേ മൂകാംബികേ, അംബികേ! 
\end{slokam}

\Letter{ല}{ശ}

\end{enumerate}

\subsection{വ}

\begin{enumerate}

\begin{slokam}{\VSr}{\VKG}{വക്കത്തുത്കണ്ഠയാലുത്കടരുജ}
 വക്കത്തുത്കണ്ഠയാലുത്കടരുജ തടവും വല്ലവസ്നേഹിതന്മാ-\\
രാക്രന്ദിയ്ക്കെ, ക്കടക്കൺനനവൊടു പശുവൃന്ദങ്ങളങ്ങമ്പരക്കേ\\
അർക്കാപത്യാന്തരാളാദുപരിയുയരുമക്കാളിയപ്പത്തി തന്മേ-\\
ലക്കാർവർണ്ണൻ നടത്തീടിന നടനകലാവിപ്ലവം വെൽവുതാക!
\end{slokam}

\Letter{വ}{അ}



\begin{slokam}{\VSr}{\Unk}{വക്ത്രാംഭോജന്മ കൈലാസവത്}
വക്ത്രാംഭോജന്മ കൈലാസവദിദമളകാലംകൃതം; കൊങ്കയുഗ്മം\\
വൃത്രാരാതേരുദാരം കലിശമിവ പരിച്ഛിന്നസാരം ഗിരാണാം;\\
മദ്ധ്യം മത്തേഭവത്തേ പിടിയിലമുഴുവോ; ൻറെത്രയും ചിത്രമത്രേ\\
മുഗ്ദ്ധേ, കേളുത്രമാതേ, വപുരുദധിരിവാഭാതി ലാവണ്യപൂർണ്ണം.
\end{slokam}

\Letter{വ}{മ}

\Book{പദ്യരത്നം}.


\begin{slokam}{\VSv}{\DSN}{വണ്ടാണെന്ന വിധം ചലിക്കുമളകം}
വണ്ടാണെന്ന വിധം ചലിക്കുമളകം, നീലായതാംഭോരുഹ-\\
ച്ചെണ്ടാണെന്ന വിധം തുളിച്ചു രസഭാവത്തിൽ സ്ഫുരിക്കും മുഖം,\\
കണ്ടാൽ കൗതുകമേറിടും തനു, നിനക്കെല്ലാം സമൃദ്ധം, ജനം\\
കൊണ്ടാടാനിടയായതത്ഭുതവുമല്ലാകർഷകം നൃത്തവും! 
\end{slokam}

\Letter{വ}{ക}


\begin{slokam}{\VSr}{\Unk}{വന്ദിക്കുന്നോരെനിക്കെന്തഭിമതം}
വന്ദിക്കുന്നോരെനിക്കെന്തഭിമതമരുളീടാതെ ചുറ്റിച്ചിടുന്നൂ\\
കുന്നിക്കും ഞാനൊഴിക്കില്ലതിനു പണി നമുക്കുണ്ടു വേറേ പുരാരേ\\
കുന്നിൻകന്യേ കടുപ്പം,തവ പതിപതിവായ് ഗം-പ്രഭോ വേണ്ട ശാഠ്യം\\
തന്നേക്കൂ കാമമിന്നിക്കഥ മുഴുവനുരക്കാതിരിക്കേണമെന്നാൽ!
\end{slokam}

\Letter{വ}{ക}


\begin{slokam}{\VVt}{\Ull}{വന്ധ്യം ശമിക്കരിശം}
വന്ധ്യം ശമിക്കരിശമെന്നറിവാർന്നൊടുക്കം\\
വിന്ധ്യപ്രഭേദി മുനി സഹ്യമിതിൽത്തപിപ്പാൻ\\
സന്ധ്യർത്ഥി കുന്നുകളൊടെന്ന വിധം കടന്നു\\
സന്ധ്യയ്ക്കു ചണ്ഡകിരണൻ ചരമാദൃയിൽപ്പോൽ.
\end{slokam}

\Letter{വ}{സ}

\Book{ഉമാകേരളം}.


\begin{slokam}{\VVt}{\Ull}{വൻ നർമ്മദാനദിയെയും}
 വൻ നർമ്മദാനദിയെയും വഴിമേൽത്തടഞ്ഞ\\
മന്നന്റെ വീര്യ, മവളോതിയറിഞ്ഞൊരാഴി\\
തന്നന്തികത്തിലവനെസ്സകുലം വധിച്ചു\\
വന്നപ്പൊഴബ്ഭൃഗുസുതന്നിതു കാഴ്ചവച്ചു.
\end{slokam}

\Letter{വ}{ത}

\Book{ഉമാകേരളം}.


\begin{slokam}{\VSr}{\ARRV}{വർണ്ണത്തിൻ ധർമ്മമൊപ്പിച്ചനവധി}
വർണ്ണത്തിൻ ധർമ്മമൊപ്പിച്ചനവധി വിധിയങ്ങാഗമാദേശഭംഗ്യാ\\
വർണ്ണിച്ചംഗത്തിൽ വേണ്ടും വിധമിഹ ഗുണവും വൃദ്ധിയും ചേർത്തു ചെമ്മേ\\
സന്ധിയ്ക്കും വിഗ്രഹത്തിന്നപി നിയമമുറപ്പിച്ചു നാനാധികാരം\\
ബന്ധിച്ചിപ്പാർത്ഥിവേന്ദ്രൻ പ്രകൃതിഷു പരമാം പ്രത്യയം ചേർത്തിടുന്നു.
\end{slokam}

\Letter{വ}{സ}


\begin{slokam}{\VSv}{\HM}{വറ്റിപ്പോയൊരു നീർത്തടത്തിലുഴറും}
വറ്റിപ്പോയൊരു നീർത്തടത്തിലുഴറും മത്സ്യം കണക്കായി ഞാൻ \\
പറ്റുന്നില്ലൊരു വാക്കുപോലുമെഴുതാൻ പാരം കിണഞ്ഞീടിലും \\
ഒറ്റയ്ക്കാക്കി നടന്നകന്നു കവിതേ! കൈവിട്ടു നീ പോയതോ? \\
മറ്റുള്ളോർക്കു കൊടുത്തിടുമ്പൊളിവനെപ്പാടേ മറക്കുന്നതോ?
\end{slokam}

\Letter{വ}{ഒ}



\begin{slokam}{\VSv}{\VKG}{വാകച്ചാർത്തിനു വല്ല}
വാകച്ചാർത്തിനു വല്ലവണ്ണവുമുണർന്നെത്തുമ്പൊഴേക്കമ്പലം\\
മാകന്ദാശുഗമാനദണ്ഡമഹിളാമാണിക്യമാലാഞ്ചിതം\\
വാകപ്പൂമൃദുമെയ്യു മെയ്യിലുരസുമ്പോ, ഴെന്റെ ഗോപീജന-\\
ശ്രീകമ്രസ്തനകുങ്കുമാങ്കിത, മനസ്സോടുന്നു വല്ലേടവും!
\end{slokam}

\Letter{വ}{വ}

\begin{slokam}{\VSv}{\VKG}{വാടാമെൻ തനു}
വാടാമെൻ തനു, മങ്ങിടാം മിഴിക, ളെൻ കൂടപ്പിറപ്പാം ഭയം\\
കൂടാം, കോടിജനാപഹാസദഹനച്ചൂടാൽ ദഹിയ്ക്കാം മനം,\\
വീടില്ലെങ്കിലൊരാൽമരത്തണലിൽ ഞാൻ കൂടും, മരിക്കും വരെ-\\
പ്പാടും നിൻ തിരുനാമമന്ത്ര, മിടയഗ്രാമപ്രഭാപുഞ്ജമേ!
\end{slokam}

\Letter{വ}{വ}

\begin{slokam}{\VSv}{\Unk}{വാണീദേവി, സുനീലവേണി}
 വാണീദേവി, സുനീലവേണി, സുഭഗേ, വീണാരവം കൈതൊഴും\\
വാണീ, വൈഭവമോഹിനീ, ത്രിജഗതാം നാഥേ, വിരിഞ്ചപ്രിയേ,\\
വാണീദോഷമശേഷമാശു കളവാനെൻനാവിലാത്താദരം\\
വാണീടേണ, മതിന്നു നിന്നടിയിൽ ഞാൻ വീഴുന്നു മൂകാംബികേ!
\end{slokam}

\Letter{വ}{വ}

\begin{slokam}{\VSr}{\VNM}{വാനത്തെഗ്ഗംഗയെത്തൻ }
വാനത്തെഗ്ഗംഗയെത്തൻ നെറുകയിലണിയുന്നോനു കാമപ്പനിപ്പി-\\
ച്ചൂനം കൂടാതണയ്ക്കും പണിയുടയ പനിപ്പർവ്വതപ്പൈതലാളേ!\\
നൂനം സംസാരഘോരപ്പനിയെഴുമിവരെ സ്വാനുകമ്പാരസത്തിൽ\\
സ്നാനം ചെയ്യിച്ചു സൗഖ്യസ്ഥിതിയരുളിവിടും നിന്റെ വൈദ്യം വിചിത്രം!
\end{slokam}

\Letter{വ}{ന}

\begin{slokam}{\VSv}{\Nalankal}{വായിക്കാൻ കഴിവസ്തമിച്ചു}
വായിക്കാൻ കഴിവസ്തമിച്ചു, കനിവിൽ വാഗ്ദേവി നേത്രാഞ്ചലം\\
പായിക്കാൻ മടികാട്ടിടുന്നി, തിരവിൽ കൈവിട്ടു മേ നിദ്രയും,\\
മായാപാശനിമഗ്നിതൻ, പലതരം രോഗങ്ങളാൽ മർദ്ദിതൻ,\\
\sam{തീയാണെന്നുടെയുള്ളിൽ, നീ വരികയെൻ ചാരത്തു നിസ്സംഗതേ}!
\end{slokam}

\Letter{വ}{മ}


സമസ്യാപൂരണം. 


\begin{slokam}{\VSr}{\KND}{വാരഞ്ചും താരി, ലോമൽക്കുളിരൊളി}
 വാരഞ്ചും താരി, ലോമൽക്കുളിരൊളി തിരളും തിങ്കളിൽ, ചിന്നിമിന്നും\\
താരത്തിൽ, കാന്തികാളും ഖരകരനി, ലിളം തെന്നലിൽ, കന്നിയാറിൽ,\\
സ്ഫാരശ്രീ മാരിവില്ലിൽ, കളമുരളി, ലെന്തിന്നു സർവത്ര നിത്യോ-\\
ദാരം സൽക്കാവ്യസാരം വിതറുമൊരു മഹാകാവ്യകാരൻ ജയിപ്പൂ!
\end{slokam}

\Letter{വ}{സ}



\begin{slokam}{\VVt}{\Ull}{വാരാശി, തന്നൊടുവിലെശ്ശിശു}
 വാരാശി, തന്നൊടുവിലെശ്ശിശു കേരളത്തെ\\
നേരായ്‌ പുലർത്തിടണമെന്നു കരാറു വാങ്ങി\\
ധാരാളമംബുവരുളുന്നതുകൊണ്ടു മേന്മേൽ\\
ധാരാധരങ്ങളിതിൽ മാരി പൊഴിച്ചിടുന്നു.
\end{slokam}

\Letter{വ}{ധ}

\Book{ഉമാകേരളം}.

\begin{slokam}{\VSv}{\VRV}{വാളല്ലെൻ സമരായുധം}
വാളല്ലെൻ സമരായുധം, ഝണഝണദ്ധ്വാനം മുഴക്കീടുവാ-\\
നാള, ല്ലെൻ കരവാളു വിറ്റൊരു മണിപ്പൊൻവീണ വാങ്ങിച്ചു ഞാൻ;\\
താളം, രാഗ, ലയ, ശ്രുതി, സ്വരമിവയ്ക്കല്ലാതെയൊന്നിന്നുമി-\\
ന്നോളക്കുത്തുകള്‍ തീർക്കുവാൻ കഴിയുകില്ലെൻ പ്രേമതീർത്ഥങ്ങളിൽ.
\end{slokam}

\Letter{വ}{ത}


\Book{സർഗ്ഗസംഗീതം}.

\begin{slokam}{\VSv}{\UN}{വാളിൻ ശൂരത മൂക്കിലും}
വാളിൻ ശൂരത മൂക്കിലും മുലയിലും, ബാണം ചതിപ്പോരിൽ നാ--\\
ടാളാൻ ജ്യേഷ്ഠനെയൊറ്റിടും സഹജനെപ്പാലിച്ചിടാ, നീവിധം\\
മേളത്തോടെ ജയിച്ചു വന്ന വരവാ ദീപാവലിയ്ക്കോളമായ്,\\
ആളും തീയിലെയായിരം തിരികളോ സീതയ്ക്കു താപാവലി!
\end{slokam}

\Letter{വ}{മ}

\begin{slokam}{\VSv}{\DSN}{വീടാണെങ്കിൽ വിളക്കു}
വീടാണെങ്കിൽ വിളക്കു വേണ, മതു കത്തിക്കാൻ വധൂരത്ന, മുൾ-\\
ക്കൂടാൻ നന്മ കലർന്ന തോഴ, രറിവെത്തിക്കാൻ മഹദ്ഗ്രന്ഥവും,\\
ചൂടാൻ പൂവുകൾ, മുഗ്ദ്ധഭാഷണമുതിർത്തുല്ലാസമെങ്ങും വിത-\\
ച്ചീടാൻ കൊച്ചുകിടാങ്ങളും - സുഖമിതിൻ മേലെന്തു കൈവന്നിടാൻ?
\end{slokam}

\Letter{വ}{ച}




\begin{slokam}{\VSv}{\RN}{വിദ്വത്പീഠമെനിക്കു വേണ്ട}
വിദ്വത്പീഠമെനിക്കുവേണ്ട, കുടജം കേറേണ്ട, സർവജ്ഞതാ- \\
മുദ്രയ്ക്കും ഭ്രമമില്ലെനിക്കു, ചെറുതേ മോഹം പ്രസാദാത്മികേ!\\
ഒറ്റപ്പെട്ടു വിളിക്കുമെന്റെ പുറകേയുണ്ടാവണം നിന്റെ കാ-\\
ലൊച്ചത്താള, മിടംതിരിഞ്ഞു പുറകിൽ നോക്കില്ല ഞാൻ നിശ്ചയം.
\end{slokam}

\Letter{വ}{ഒ}

\begin{slokam}{\VDv}{\VCBP}{വിമലമാമലമാനിനി}
വിമലമാമലമാനിനി, ബാലനാം\\
മമ ഹിതം മഹി തന്നിൽ വിളങ്ങുവാൻ\\
ഇനി ഭവാനി ഭവാഭിധസിന്ധു തൻ\\
സുതരണം തരണം തവ നോക്കുകള്‍
\end{slokam}

\Letter{വ}{മ}

\Topic{യമകം (ദ്രുതവിളംബിതം, നാലു വരിയിലും)}.  \NextSlRef{ഇവനിതാ വനിതാ}

\begin{slokam}{\VDv}{\RV}{വില പെരുത്തു കൊടുത്തു}
വില പെരുത്തു കൊടുത്തു കിടച്ചിടും\\
പല നിറങ്ങളിലുള്ളൊരു കോളകള്‍\\
ചിലതിലുണ്ടു വിഷാംശ, മതോർക്കയാൽ\\
\sam{കുലമണേ, ലെമണേഡു കുടിച്ചു ഞാൻ}.
\end{slokam}

\Letter{വ}{ച}

\Topic{യമകം (ദ്രുതവിളംബിതം)}.

സമസ്യാപൂരണം. 


\begin{slokam}{\VSv}{\KochT}{വിശ്വാധീശ്വര, രൂപയായിരമെനിക്കീ}
വിശ്വാധീശ്വര, രൂപയായിരമെനിക്കീ റോട്ടിലെങ്ങാൻ കിട-\\
ന്നാശ്വാസത്തൊടു കിട്ടിയെങ്കി, ലവിടെയ്ക്കേകാമതിൽപ്പാതി ഞാൻ\\
വിശ്വാസം കുറവെങ്കിലോ, തിരുവടിക്കുള്ളോരു പങ്കാദ്യമായ്‌\\
ഇച്ഛായോഗ്യമെടുത്തു ബാക്കി തരണേ പിന്നെന്തു പേടിക്കുവാൻ?
\end{slokam}
 
\Letter{വ}{വ}

\begin{slokam}{\VSv}{\VRV}{വിശ്വാമിത്ര, വസിഷ്ഠ, ഗൗതമ,}
 വിശ്വാമിത്ര, വസിഷ്ഠ, ഗൗതമ, ഭരദ്വാജാദികള്‍ നട്ടൊരാ\\
വിശ്വാസച്ചെടി കായ്ച്ചുണങ്ങിയ കനിത്തോടേന്തി വേദാന്തമേ!\\
വിശ്വം, ശക്തിതരംഗചാലിതവിയദ്ഗേഹങ്ങളിൽ, കാലമാ-\\
മശ്വത്തെപ്പുറകേ നടത്തുമിവിടേക്കെന്തിന്നു വന്നെത്തി നീ?
\end{slokam}

\Letter{വ}{വ}


\Book{ഗ്രാമദർശനം}


\begin{slokam}{\VSv}{\CKP}{വീതാശങ്കമഹോ, വിനാശകരമാ}
വീതാശങ്കമഹോ, വിനാശകരമാസ്സാമ്രാജ്യദുര്‍മ്മോഹമാം\\
വേതാളത്തിനു രക്തതര്‍പ്പണമനുഷ്ഠിക്കുന്ന രാഷ്ട്രങ്ങളേ,\\
സ്വാതന്ത്ര്യം ജലരേഖ--മര്‍ത്ത്യരെ വെറും ചെന്നായ്ക്കളാക്കാം, കുറെ\\
പ്രേതങ്ങള്‍ക്കുഴറാം ജഗത്തിലിതിനോ നിങ്ങള്‍ക്കു യുദ്ധഭ്രമം!
\end{slokam}

\Letter{വ}{സ}


\begin{slokam}{\VSv}{\VNM}{വീണക്കമ്പികള്‍ മീട്ടി നിൻ}
 വീണക്കമ്പികള്‍ മീട്ടി നിൻ കരവിരൽക്കെല്ലാമിരട്ടിക്കുമി-\\
ശ്ശോണത്വം ബത കണ്ടു "ഗാനമുടനേ നിർത്തേണ"മെന്നക്ഷിയും\\
"വേണം തെല്ലിടകൂടെ"യെന്നു ദുര കൊണ്ടെൻ കർണ്ണവും തങ്ങളിൽ\\
പ്രാണപ്രേയസി, തർക്കമാ - ണിവിടെ ഞാൻ മദ്ധ്യസ്ഥതയ്ക്കക്ഷമൻ!
\end{slokam}

\Letter{വ}{വ}



\begin{slokam}{\VSv}{\VNM}{വീണക്കമ്പി മുറുക്കിടുന്നു}
വീണക്കമ്പി മുറുക്കിടുന്നു മൃദുകൈത്താരാലൊരാരോമലാള്‍,\\
ചാണക്കല്ലിലൊരുത്തി ചന്ദനമരയ്ക്കുന്നൂ ചലശ്രോണിയായ്‌,\\
ശോണശ്രീചഷകത്തിൽ നന്മധു നിറയ്ക്കുന്നൂ ശരിക്കന്യയാ-\\
മേണപ്പെണ്മിഴി, സർവ്വതോ മധുരമീ മണ്ഡോദരീ മന്ദിരം!
\end{slokam}

\Letter{വ}{ശ}

\Book{രാവണന്റെ അന്തഃപുരഗമനം}.


\begin{slokam}{\VSr}{\VKG}{വീണാൽ വീഴട്ടെ, വീഴാത്തവനുലകിലെവൻ}
വീണാൽ വീഴട്ടെ, വീഴാത്തവനുലകിലെവൻ?  വീണു മണ്ണിൽക്കിടക്കും \\
നാണക്കേടാണൊഴിക്കേണ്ടതു, പൊടി പുരളും മുമ്പെഴുന്നേൽക്ക നല്ലൂ; \\
താണും നൂണും കടക്കാനിടവഴികളിടയ്ക്കുണ്ടു, നേർപാത കാണി-\\
ല്ലാണിക്കല്ലാണു തോല്മയ്ക്കിടയിലിടറിടാതുള്ള കാൽവെയ്പു കുഞ്ഞേ!
\end{slokam}

\Letter{വ}{ത}


\begin{slokam}{\VSv}{\UN}{വീഴാൻ പോവതു കണ്ടു}
വീഴാൻ പോവതു കണ്ടു  താങ്ങിയവർ, ഭീ ചൂഴാതെ കൂടെത്തുവോർ,\\
ആഴിക്കൊത്ത നിരാശ തന്ന ചുഴിയിൽ താഴാതെ രക്ഷിച്ചവർ,\\
കേഴാൻ തോളു കുനിച്ചു  തന്നവ, രിടം വാഴാൻ കനിഞ്ഞേകിയോർ,\\
ഊഴിയ്ക്കായ് ജനിയാർന്ന ദേവതകളാം തോഴർക്കു കൈകൂപ്പിടാം!
\end{slokam}

\Topic{അഷ്ടപ്രാസം}.

\Letter{വ}{ക}

\begin{slokam}{\VSv}{\VKG}{വൃത്തോത്തുംഗകുചങ്ങൾ}
 വൃത്തോത്തുംഗകുചങ്ങ, ളഗ്ഗുരുനിതംബം, കാകളീമാധുരീ-\\
സത്താം ശബ്ദസുഖം, പ്രസാദ, മുചിതാലങ്കാരസമ്പന്നതാ,\\
ഇത്ഥം കോമളിമാവിണങ്ങുമവളെക്കാണുന്ന വിദ്യാർത്ഥികള്‍-\\
ക്കത്യന്തം കവിതയ്ക്കു ശക്തിയുളവാകാഞ്ഞാലതാണത്ഭുതം!
\end{slokam}

\Letter{വ}{ഇ}

\begin{slokam}{\VSv}{\VKG}{വെണ്ണത്തൂമണമാർന്ന വായ്‌മലരിനാൽ}
 വെണ്ണത്തൂമണമാർന്ന വായ്‌മലരിനാൽ ചുംബിക്കെയമ്മയ്ക്കു മൈ-\\
ക്കണ്ണിൽ തിങ്ങിവഴിഞ്ഞിടുന്ന പരമാനന്ദം സമീക്ഷിക്കവേ,\\
വിണ്ണിൽപ്പോലുമലഭ്യമാമമൃതൊലിച്ചീടുംവിധം ചെമ്മലർ-\\
ത്തൊണ്ണിൻ തൂമ വെളിപ്പെടുംപടി ചിരിക്കും കണ്ണ! കാക്കേണമേ.
\end{slokam}

\Letter{വ}{വ}

\begin{slokam}{\VSr}{\Unk}{വെണ്ണീറും, വെള്ളെലിമ്പും,}
 വെണ്ണീറും, വെള്ളെലിമ്പും, വിഷധരവിലസത്‌പാമ്പുമാപാദചൂഡം,\\
തണ്ണീരെപ്പോഴുമോലും, തലയിലെരികനൽക്കട്ട പൊട്ടിന്റ കണ്ണും,\\
എണ്ണേറും ഭൂതയൂഥങ്ങളൊടൊരു കളിയും കണ്ടു നിന്നോടിണങ്ങും\\
പെണ്ണോളം ധൈര്യമുള്ളോരുലകിലൊരുവർ മറ്റില്ല, ചെല്ലൂർപിരാനേ!
\end{slokam}

\Letter{വ}{എ}

\Book{ചെല്ലൂർനാഥസ്തവം}. 

\begin{slokam}{\VSv}{\Unk}{വെയ്ക്കാനന്തിവിളക്കു}
 വെയ്ക്കാനന്തിവിളക്കു, വീടു ശുചിയായ്‌ വെക്കാൻ, രുചിക്കും വിധം\\
വെയ്ക്കാൻ ഭക്ഷണ, മെന്നകത്തു കുടിവെയ്ക്കാൻ പ്രേമ സർവസ്വമായ്‌\\
വെയ്ക്കാൻപങ്കുസുഖാസുഖങ്ങ, ളഖിലം നീ സമ്മതം മൂളുകിൽ\\
വെയ്ക്കാം കൈമലരെന്റെ കയ്യിൽ വിജയിച്ചീടട്ടെ മജ്ജീവിതം!
\end{slokam}

\Letter{വ}{വ}

\begin{slokam}{\VSr}{\CKP}{വെള്ളം ചേർക്കാതെടുത്തോരമൃതിനു}
 വെള്ളം ചേർക്കാതെടുത്തോരമൃതിനു സമമാം നല്ലിളം കള്ളു, ചില്ലിൻ\\
വെള്ളഗ്ലാസ്സിൽ പകർന്നങ്ങനെ രുചികരമാം മത്സ്യമാംസാദി കൂട്ടി\\
ചെല്ലും തോതിൽ ചെലുത്തി, ക്കളിചിരികള്‍ തമാശൊത്തു മേളിപ്പതേക്കാള്‍\\
സ്വർല്ലോകത്തും ലഭിക്കില്ലുപരിയൊരു സുഖം - പോക വേദാന്തമേ നീ!
\end{slokam}

\Letter{വ}{ച}

\begin{slokam}{\VSr}{\SVL}{വെള്ളം, വെണ്ണീർ, വൃഷം,}
 വെള്ളം, വെണ്ണീർ, വൃഷം, വെണ്മഴു, വരകരിതോ, ലാര്യവിത്താധിപൻ തൊ-\\
ട്ടുള്ളോരീ നൽക്കൃഷിക്കോപ്പുകളഖിലമധീനത്തിലുണ്ടായിരിക്കെ\\
പള്ളിപ്പിച്ചയ്ക്കെഴുന്നള്ളരുതു പുരരിപോ! കാടുവെട്ടിത്തെളിച്ചാ\\
വെള്ളിക്കുന്നിൽകൃഷിച്ചെയ്യുക, പണിവതിനും ഭൂതസാർത്ഥം സമൃദ്ധം!
\end{slokam}

\Letter{വ}{പ}


\begin{slokam}{\VKm}{\CVVB}{വേണുവിൻ ശ്രുതിയൊടൊത്തു}
വേണുവിൻ ശ്രുതിയൊടൊത്തു പാടി മധുരസ്വരത്തി, ലതിനൊത്തുടൻ\\
ചേണിയന്ന പടി താളമിട്ടു, തള കൊഞ്ചിടുന്ന പദമൂന്നിയും,\\
പാണി കൊണ്ടു ചുമലിൽപ്പിടിച്ചു, മിളകുന്ന പൊൻവള കിലുങ്ങിയും\\
ശ്രോണി തന്നിലിളകുന്ന ചേലയൊടു ചെയ്തൊരാ നടനമോർക്കുവിൻ!
\end{slokam}

\Letter{വ}{പ}

\Book{നാരായണീയം പരിഭാഷ}. 
\OSlRef{വേണുനാദകൃതതാന}.

\begin{slokam}{\VSv}{\VNM}{വേദം നിന്നുടെ ശാസനക്കുറി}
 വേദം നിന്നുടെ ശാസനക്കുറി, പുരാണൗഘം ഹിതോദ്ബോധനം,\\
സ്വാദത്യന്തമിയന്ന കാവ്യഗണമോ സപ്രേമസംഭാഷണം\\
വൈദഗ്ദ്ധ്യത്തികവാൽ ജഗത്തു മുഴുവൻ താനേ വശത്താക്കി നീ\\
നാദബ്രഹ്മനൃപാസനോപരി വിളങ്ങുന്നൂ മഹാരാജ്ഞിയായ്‌.
\end{slokam}

\Letter{വ}{വ}


\begin{slokam}{\VSv}{\UN}{വേറേ ജന്മമെനിക്കു വേണ്ട}
വേറേ ജന്മമെനിക്കു വേണ്ട, പുറകിൽ പോകേണ്ട, ജീവിച്ചൊരീ\\
നൂറിൽ പാതി കഴിഞ്ഞ യാത്ര, യതിലിന്നൊന്നും തിരുത്തേണ്ട മേ,\\
കേറാനുണ്ടിനി പാതി മാർഗ്ഗ, മതിലെൻ വിജ്ഞാനവും കർമ്മവും,\\
ദാരങ്ങൾ, പ്രണയം, കിനാവിവ മെനഞ്ഞാനന്ദമാർന്നീടണം!
\end{slokam}

\Letter{വ}{ക}


\begin{slokam}{\VVt}{\KA}{വൈരാഗ്യമേറിയൊരു}
 വൈരാഗ്യമേറിയൊരു വൈദികനാട്ടെ, യേറ്റ \\
വൈരിയ്ക്കു മുൻപുഴറിയോടിയ ഭീരുവാട്ടേ,\\
നേരേ വിടർന്നു വിലസീടിന നിന്ന നോക്കി\\
യാരാകിലെന്തു, മിഴിയുള്ളവർ നിന്നിരിക്കാം.
\end{slokam}

\Letter{വ}{ന}

\Book{വീണ പൂവ്}.

\begin{slokam}{\VVt}{\Ull}{വ്യാളം വിഭൂതിയിവ പൂണ്ട്}
 വ്യാളം വിഭൂതിയിവ പൂ, ണ്ടഖിലാഗമങ്ങള്‍-\\
ക്കാലംബമായ്‌, ഭൃതഗുഹത്വമൊടൊത്തുകൂടി,\\
കോലം ശിവാകലിതമാക്കിയുമിഗ്ഗിരീശൻ\\
ശ്രീലദ്വിജാധിപനെ മൗലിയിലേന്തിടുന്നു.
\end{slokam}

\Letter{വ}{ക}


\begin{slokam}{\VSv}{\KA}{വ്യോമത്തിൻ മലിനത്വമേറ്റി}
വ്യോമത്തിൻ മലിനത്വമേറ്റിയവിടെപ്പൊങ്ങുന്നതെന്തോ മഹാ-\\
ഭീമത്വം കലരുന്ന കാലഫണിതൻ ജിഹ്വാഞ്ചലം‌പോലവേ,\\
ശ്രീമദ്ഭാസുര”ശാരദാലയ”മഹാദീപം കലാശിച്ചെഴും\\
ധൂമത്തിൻ നികരുംബമല്ലി? - വസുധേ, കേണിടു കേണിടു നീ!
\end{slokam}

\Letter{വ}{ശ}

\Book{പ്രരോദനം}.




\end{enumerate}

\subsection{ശ}

\begin{enumerate}

\begin{slokam}{\VSr}{\VenM}{ശങ്കാഹീനം ശശാങ്കാമലതരയശസാ}
ശങ്കാഹീനം ശശാങ്കാമലതരയശസാ കേരളോൽപന്നഭാഷാ-\\
വങ്കാട്ടിൽ സഞ്ചരിയ്ക്കും സിതമണി ധരണീദേവഹര്യക്ഷവര്യൻ\\
ഹുങ്കാരത്തോടെതിർക്കും കരിവരനിടിലം തച്ചുടയ്ക്കുമ്പൊള്‍ നിന്ദാ-\\
ഹങ്കാരം പൂണ്ട നീയാമൊരു കുറുനരിയെക്കൂസുമോ കുന്നി പോലും?
\end{slokam}

\Letter{ശ}{ഹ}

\begin{slokam}{\VSr}{\Unk}{ശർമ്മത്തെസ്സൽക്കരിക്കും}
ശർമ്മത്തെസ്സൽക്കരിക്കും ഗതിയെയനുകരിക്കും കരിക്കും, കരിക്കും\\
ദുർമ്മത്തിൻ ധൂർത്തുടയ്ക്കും കചഭരമതുടയ്ക്കും തുടയ്ക്കും തുടയ്ക്കും,\\
നിർമ്മായം സങ്കടത്തെക്കളയുക വികടത്തെക്കടത്തെക്കടത്തി-\\
ന്നമ്മേ കായങ്കലാശേ കലിതതി സകലാശേ കലാശേ കലാശേ!
\end{slokam}

\Letter{ശ}{ന}

\Topic{അന്ത്യപ്രാസവും യമകവും}.  \PrevSlRef{അശ്വത്ഥത്തിന്നിലയ്ക്കും},
\NextSlRef{നിത്യം നശ്ചിത്തപദ്മേ}


\begin{slokam}{\VSr}{\PKV}{ശസ്ത്രത്തെശ്ശൂരനാമെൻ}
ശസ്ത്രത്തെശ്ശൂരനാമെൻ ജനകനിനിയെടുക്കില്ല നന്നെന്നുറച്ചി-\\
ട്ടസ്രസ്തൻ നീയശങ്കം കരമിഹ ഗുരു തൻ മൗലിയിൽ ചേർത്ത നേരം\\
വിശ്വത്തിൽ പാർത്ഥപാഞ്ചാലകനിഖിലചമൂമർദ്ദിയായ്‌ ചാപഭൃത്താ-\\
മശ്വത്ഥാമാവു വാഴുന്നൊരു കഥ വഴിപോലുള്ളിലോർത്തില്ലയോ നീ?
\end{slokam}

\Letter{ശ}{വ}

\Book{വേണീസംഹാരം പരിഭാഷ}. 

\begin{slokam}{\VSr}{\GRT}{ശൃങ്ഗാരത്തിന്റെ നാമ്പോ}
ശൃങ്ഗാരത്തിന്റെ നാമ്പോ, രസികതയൊഴുകിപ്പോകുവാനുള്ള തൂമ്പോ,\\
സൗന്ദര്യത്തിന്റെ കാമ്പോ, മദനരസചിദാനന്ദ പൂന്തേൻകുഴമ്പോ,\\
ബ്രഹ്മാവിൻ സൃഷ്ടിവൻപോ, നയനസുഖലതയ്ക്കൂന്നു നൽകുന്ന കമ്പോ,\\
കന്ദർപ്പൻ വിട്ടൊരമ്പോ, ത്രിഭുവനവിജയത്തിന്നിവൻ? തോഴി! യമ്പോ!
\end{slokam}

\Letter{ശ}{ബ}


\begin{slokam}{\VSr}{\VNM}{ശീട്ടാട്ടം, ശിങ്കമാനക്കുഴൽവിളി}
ശീട്ടാട്ടം, ശിങ്കമാനക്കുഴൽവിളി, ചതുരംഗങ്ങള്‍, ചർവ്വാംഗിമാർ തൻ\\
പാ, ട്ടായം പൂണ്ട തായമ്പക, വകതിരിവുള്ളക്ഷരശ്ലോകപാഠം,\\
കൂട്ടാളിക്കൂട്ടരൊത്തുള്ളൊരു സരസജനത്തിന്റെ സല്ലാപഘോഷം,\\
കേട്ടാലാവി, ല്ലിവണ്ണം പലതുമവിടെയാ രാവിലാവിർഭവിച്ചു.
\end{slokam}

\Letter{ശ}{ക}


\begin{slokam}{\VSr}{\VNM}{ശോണാകാരം നറും തൃച്ചൊടി}
 ശോണാകാരം നറും തൃച്ചൊടി; കുചയുഗളം തുംഗഭദ്രാത്മകം; പൂ-\\
ബാണാരിക്കെപ്പൊഴും നർമ്മദ ഭുജലത; നിൻ വേണിയോ കൃഷ്ണ തന്നേ;\\
ചേണാർന്നോരദൃജാതേ! പ്രഥിതനദനദീരൂപമായുള്ള നിങ്കൽ-\\
ത്താണാരാപൂർണ്ണഭക്ത്യാ മുഴുകു, മവനപങ്കാനുവിദ്ധൻ വിദഗ്ദ്ധൻ!
\end{slokam}

\Letter{ശ}{ച}


\begin{slokam}{\VSr}{\Unk}{ശൗരേരാദ്യാവതാരം തരളമിഴിയുഗം}
ശൗരേരാദ്യാവതാരം തരളമിഴിയുഗം; വാക്കു ഗീർവാണമുഖ്യാ-\\
ഹാരം; വക്ഷോജഭാരം ത്രിപുരഹരകരോല്ലാസിനീ ചാപവല്ലീ;\\
മാരൻപൂമേനിയല്ലോ കൊടിനടുവു; മഹാ മന്ദരാദ്രിം വിരോധം\\
വാരാതേ താങ്ങുവോന്റിപ്പുറവടി പിരളീ നായികേ, താവകീനം.”
\end{slokam}

\Letter{ശ}{മ}


\begin{slokam}{\VSr}{\VNM}{ശ്യാമപ്പൂമെത്ത}
ശ്യാമപ്പൂമെത്ത, ചഞ്ചൽക്കുളിർവിശറി, മണീകീർണ്ണമാം നീലമേലാ-\\
പ്പോമൽത്തങ്കഗ്ഗുളോ, പ്പീ വക വിഭവശതം ചേർന്ന കേളീഗൃഹം മേ\\
പ്രേമത്താലേ സ്വയം തന്നരുളിയ പരമോദാരശീലന്റെ മുന്നിൽ\\
കാമത്താൽ കൊച്ചുകൈക്കുമ്പിളിതഹഹ! മലർത്തുന്ന ഞാനെത്ര ഭോഷൻ!
\end{slokam}

\Letter{ശ}{പ}

\begin{slokam}{\VSv}{\RV}{ശ്യാമാപാംഗ, നിനക്കു}
ശ്യാമാപാംഗ, നിനക്കു പണ്ടു കവിതാസമ്പന്നരിക്കോവിലിൽ\\
പ്രേമത്തോടെ കൊളുത്തി പദ്യനിരയാൽ വാടാത്ത ദീപാവലി\\
പാമോയിൽത്തിരിയാണു,കാന്തികുറവാ,ണീടില്ലയെന്നാകിലും\\
കേമത്തം കലരാത്ത ഞാനുമിവിടെച്ചെയ്തോട്ടെ ദീപാഞ്ജലി!
\end{slokam}

\Letter{ശ}{പ}

\begin{slokam}{\VSr}{\SVL}{ശ്രീപാർക്കും സ്ഥാനമല്ലോ}
ശ്രീപാർക്കും സ്ഥാനമല്ലോ ഗിരിശ! തവ ശരം, തൂണി രത്നാകരം, നൽ-\\
ചാപം പൊൻകുന്നു, സേവൻ നിധിപതി, രജതക്കുന്നിരിക്കും പ്രദേശം,\\
ആപീഡം ചന്ദ്രകാന്തം, തനുവിലണിയുവാൻ ഭൂതി, പിന്നെപ്പുരാരേ!\\
നീ പോയിപ്പിച്ചതെണ്ടുന്നതു തലയിലെഴുത്തിന്റെ തായാട്ടമല്ലോ!
\end{slokam}

\Letter{ശ}{അ}

\begin{slokam}{\VMk}{\KV}{ശ്രീമത്ത്വത്താൽ മദമൊടു}
ശ്രീമത്ത്വത്താൽ മദമൊടു ജഗൽപ്രാണനേത്തിന്നു നന്നായ്\\
സാമർത്ഥ്യത്തോടഹിഭയമുദിപ്പിച്ചു ഭൂമീപതിയ്ക്കും\\
കേമത്വം പൂണ്ടനൃജൂഗതിയാമബ്ഭുജംഗദ്വിജിഹ്വ-\\
സ്തോമത്തോടസ്ത്വയി തവ സഖേ! സാധ ബോധം വിരോധം.
\end{slokam}

\Letter{ശ}{ക}

\Book{മയൂരസന്ദേശം}.


\begin{slokam}{\VSr}{\VNM}{ശ്രീമത്‌ സൂര്യന്നു ശിഷ്യൻ}
ശ്രീമത്‌ സൂര്യന്നു ശിഷ്യൻ, പവനനു തനയൻ, സൂര്യപുത്രന്നമാത്യൻ,\\
രാമസ്വാമിയ്ക്കു ദൂതൻ, ജനകതനുജയാള്‍ക്കാമയം തീർത്ത വൈദ്യൻ,\\
ഭീമന്നണ്ണൻ, നിശാടർക്കകരുണതരനാം കാല, നാലത്തിയൂരെ-\\
ഗ്രാമത്തിന്നിഷ്ടദൈവം, ശ്രിതസുരതരുവാ ശ്രീഹനൂമാൻ സഹായം!
\end{slokam}

\Letter{ശ}{ഭ}

\begin{slokam}{\VSv}{\VNM}{ശ്രീയാ, ണുർവ്വശിയാണു}
 ശ്രീയാ, ണുർവ്വശിയാണു, ശീലവതിയാണെന്നൊക്കെ നാട്ടാർ വെടി-\\
പ്പായാഹന്ത! പുകഴ്ത്തുമീ മൊഴികളാൽ കർണ്ണം തഴമ്പിച്ചു മേ;\\
പ്രേയാനോടൊരുമിപ്പതിന്നു തടവില്ലാത്തോരു സാധാരണ-\\
സ്ത്രീയായാൽ മതിയായിരുന്നു - വിധി താനെന്നെച്ചതിച്ചൂ വൃഥാ!
\end{slokam}

\Letter{ശ}{പ}

\Book{വിലാസലതിക}.

\begin{slokam}{\VMk}{\KV}{ശ്രീലാസ്യത്താലെഴുമഴക്}
ശ്രീലാസ്യത്താലെഴുമഴകുഞ്ചിച്ചുമുൾപ്പുക്കു വാഴും\\
ലോലാക്ഷീണാമണികുഴലതിപ്പീലിപോലുല്ലസിച്ചും\\
നീലാശ്മശ്രീ തഴുകിന തളം നിൻഗളച്ഛായമായും\\
ലീലാസൌധപ്രകരമെതിരാം തത്ര തേ ചിത്രപത്ര!
\end{slokam}

\Letter{ശ}{ന}

\Book{മയൂരസന്ദേശം}.


\end{enumerate}

\subsection{സ}

\begin{enumerate}
\begin{slokam}{\VPv}{\UN}{സദാ സുകൃതവാഹയായ്}
സദാ സുകൃതവാഹയായ്,   സകലലോകസന്തുഷ്ടയായ്,\\
സ്മിതാധരസമേതയായ്,  സിതസമാനസദ്‌വാണിയായ്,\\
നിതാന്തകമനീയയായ്,    നിജസുജീവിതപ്പാതയിൽ\\
പദാന്തരമണയ്ക്കു നീ -  പ്രണയമാധുരീസിന്ധുവായ്!
\end{slokam}

\Letter{സ}{ന}

\begin{slokam}{\VSv}{\VNM}{സദ്വർണ്ണാഞ്ചിതശയ്യ ചേർന്ന്}
 സദ്വർണ്ണാഞ്ചിതശയ്യ ചേർ, ന്നഴകെഴും ഭാവപ്രഭാവത്തൊടും,\\
മൃദ്വംഗാനുഗുണപ്രയുക്തവിവിധാലങ്കാരസമ്പത്തൊടും,\\
വിദ്വല്ലാളിതകാളിദാസകവിതയ്ക്കൊപ്പം വിളങ്ങുന്ന നീ\\
മദ്വക്ഷോമണിമാലികേ, കിമപി കൈക്കൊള്‍കാ പ്രസാദത്തെയും!
\end{slokam}

\Letter{സ}{വ}

\Book{വിലാസലതിക}.

\begin{slokam}{\VSr}{\ARRV}{സമ്പത്തായ്‌ സംയമത്തെ}
 സമ്പത്തായ്‌ സംയമത്തെക്കരുതി മരുവുമീ നമ്മെയും, തൻ കുലത്തിൻ\\
വൻപും, ബന്ധൂക്തി കൂടാതിവള്‍ നിജഹൃദയം നിങ്കലർപ്പിച്ചതും നീ\\
നന്നായോർത്തിട്ടു ദാരപ്പരിഷയിലിവളെക്കൂടി മാനിച്ചിടേണം\\
പിന്നത്തേ യോഗമെല്ലാം വിധിവശ, മതിലിജ്ഞാതികള്‍ക്കില്ല ചോദ്യം.
\end{slokam}

\Letter{സ}{ന}

\Book{അഭിജ്ഞാനശാകുന്തളം പരിഭാഷ}.

\begin{slokam}{\VSv}{\YK}{സം‌വർത്തത്തിലൊരാൽ}
സം‌വർത്തത്തിലൊരാലിലയ്ക്കുപരി താൻ വർത്തിച്ച സാമർത്ഥ്യമേ,\\
ഗർവ്വത്തിൻ തല വീശി സാധുവിജയം സാധിച്ച സാരഥ്യമേ, \\
പാവപ്പെട്ട സതീർത്ഥ്യനെക്കനിവൊടാശ്ലേഷിച്ച ദാക്ഷിണ്യമേ,  \\
കൈവല്യക്കടലായ് മുമുക്ഷുനദിയെപ്പുൽകിത്തലോടുന്നു നീ!
\end{slokam}

\Letter{സ}{പ}

\begin{slokam}{\VSr}{\ARRV}{സംസത്തിൽ സ്വാവമാനോദ്യത}
സംസത്തിൽ സ്വാവമാനോദ്യതനൃപഭടരോടാത്തരോഷൻ, സ്വവീര്യം\\
ശംസിക്കും സാധുപൗരപ്പരിഷയിലലിവാർന്നുന്മിഷന്മന്ദഹാസൻ,\\
അംസത്തിൽച്ചന്ദ്രലേഖാവിമലകുവലയാപീഡദന്തങ്ങളേന്തി-\\
ക്കംസധ്വംസത്തിനോങ്ങും മുരരിപുഭഗവാൻ നിങ്ങളെത്താങ്ങിടട്ടെ!
\end{slokam}

\Letter{സ}{അ}

\Book{ചാരുദത്തൻ നാടകം}.  നാന്ദീശ്ലോകം.


\begin{slokam}{\VSr}{\PCM}{സംസാരത്തിന്‍ കൊളുത്തെന്‍}
സംസാരത്തിന്‍ കൊളുത്തെന്‍ മുതുകിലമരുമീ ഭക്തര്‍ തന്നാര്‍പ്പിനൊത്തെന്‍-\\
മാംസസ്നായ്‌വസ്ഥിമേദോമലകലിതമുടല്‍ക്കെട്ടു മാനത്തു പൊങ്ങും\\
ധ്വംസം ദേഹാത്മഭാവത്തിനു വരണമിവ; ക്കില്ലയെന്നാകില്‍ ഞാനെ-\\
ന്നംസം ഭേദിച്ചു ബീഭത്സത, ജനനി, നിവേദിക്കണോ സത്ത്വരൂപേ?
\end{slokam}

\Letter{സ}{ധ}


\begin{slokam}{\VSv}{\KA}{സംസാരാമയഘോര}
സംസാരാമയഘോരസാഗരതരീഭൂതേ! സമസ്തേശ്വരീ!\\
ഹിംസാപേതഹിരണ്യഗർഭദയിതേ! ഹീരോപലോദ്യത്പ്രഭേ! \\
ഹംസാരാധിതഹംസവാഹനസുതേ! ഹംസാത്മികേ! ഹംസികേ!\\
ഹംസദ്ധ്യാനകലേ! ഹരാംഗനിലയേ! അംബേ! കടാക്ഷിക്ക നീ!
\end{slokam}

\Letter{സ}{ഹ}

\begin{slokam}{\VSv}{\ONN}{സൽപ്പാത്രത്തിലൊഴിച്ചതില്ലൊരു}
സൽപ്പാത്രത്തിലൊഴിച്ചതില്ലൊരു തവിത്തോയം, ഗുരുശ്രീപദ-\\
പ്പൊൽപ്പൂവൊന്നു തലോടിയില്ല, സമയേ ചെയ്തീല സന്ധ്യാർച്ചനം,\\
കെൽപ്പേറും യമരാജകിങ്കരഖരവ്യാപാരഘോരാമയം\\
നിൽപ്പാനുള്ള മരുന്നു ഞാൻ കരുതിയില്ലമ്മേ ഭയം മേ പരം!
\end{slokam}

\Letter{സ}{ക}

\begin{slokam}{\VRt}{\Unk}{സാരമുള്ള വചനങ്ങള്‍}
സാരമുള്ള വചനങ്ങള്‍ കേള്‍ക്കിലും\\
നീരസാർത്ഥമറിയുന്നു ദുർജ്ജനം\\
ക്ഷീരമുള്ളൊരകിടിൻ ചുവട്ടിലും\\
ചോര തന്നെ കൊതുകിന്നു കൗതുകം!
\end{slokam}

\Letter{സ}{ക}

\begin{slokam}{\VSr}{\VCBP}{സാരാനർഘപ്രകാശപ്രചുരിമ}
 സാരാനർഘപ്രകാശപ്രചുരിമ പുരളും ദിവ്യരത്നങ്ങളേറെ-\\
പ്പാരാവാരത്തിനുള്ളിൽപ്പരമിരുള്‍ നിറയും കന്ദരത്തിൽ കിടപ്പൂ\\
ഘോരാരണ്യച്ചുഴൽക്കാറ്റടികളിലിളകും തൂമണം വ്യർത്ഥമാക്കു-\\
ന്നോരപ്പൂവെത്രയുണ്ടാമവകളിലൊരു നാളൊന്നു കേളിപ്പെടുന്നൂ.
\end{slokam}

\Letter{സ}{ഘ}

\Book{ഒരു വിലാപം}.

\begin{slokam}{\VSv}{\ARRV}{സേവിച്ചീടുക പൂജ്യരെ}
 സേവിച്ചീടുക പൂജ്യരെ, പ്രിയസഖിക്കൊപ്പം സപത്നീജനം\\
ഭാവിച്ചീടുക, കാന്തനോടിടയൊലാ ധിക്കാരമേറ്റീടിലും,\\
കാണിച്ചീടുക ഭൃത്യരില്‍ദ്ദയ, ഞെളിഞ്ഞീടായ്ക ഭാഗ്യങ്ങളാല്‍,\\
വാണിട്ടിങ്ങനെ കന്യയാള്‍ ഗൃഹിണിയാ, മല്ലെങ്കിലോ ബാധതാന്‍
\end{slokam}

\Letter{സ}{ക}

\Book{അഭിജ്ഞാനശാകുന്തളം പരിഭാഷ}.  \OSlRef{ശുശ്രൂഷസ്വ ഗുരൂൻ}.


\begin{slokam}{\VSr}{\Unk}{സിന്ദൂരം നീരസം താൻ}
സിന്ദൂരം നീരസം താൻ, തളിർ നിറമിഴിയും കിംശുകം ഗന്ധഹീനം \\
ബിംബം കയ്ക്കും കഠോരം പവിഴമണി ജപാ പുഷ്പമോ വാടുമല്ലോ,\\
സന്ധ്യാമേഘം പൊടിച്ചിട്ടമൃതിലതു കുഴച്ചിട്ടുരുട്ടിക്രമത്താൽ\\
നീട്ടിക്കൽപിച്ചിതെന്റേ കരുതുവനധരം നാരണീനന്ദനായാഃ
\end{slokam}

\Letter{സ}{സ}

\Book{പദ്യരത്നം}.

\begin{slokam}{\VSr}{\Unk}{സോമാര്‍ദ്ധത്തിന്നുദിപ്പാൻ}
സോമാര്‍ദ്ധത്തിന്നുദിപ്പാനുദയഗിരിതടം, ചിത്രകൂടം ഭുജംഗ-\\
സ്തോമാനാം, വൈധസീനാമരിയ പിണമിടും കാടു മൂര്‍ദ്ധാവലീനാം,\\
വാര്‍മേവീടും നറും കാഞ്ചനമണികലശം ദിവ്യഗംഗാജലാനാം,\\
കാമാരേ, നിന്‍ കപര്‍ദ്ദം, ജയതി ഘനകൃപാകല്യ, ചെല്ലൂര്‍പിരാനേ!
\end{slokam}

\Letter{സ}{വ}



\Book{ചെല്ലൂർനാഥസ്തവം}. 

\begin{slokam}{\VSv}{\VKG}{സൗകര്യപ്പെടുമെങ്കിലേതു}
സൗകര്യപ്പെടുമെങ്കിലേതു പകലും രാവും രമിയ്ക്കും,ജനാ-\\
ലോകത്തിൽ ചുളിയില്ല നെറ്റി, തിരുമുറ്റത്താകിലും സമ്മതം\\
പൂകും പൂമണിമച്ചിലും വയലിലും വേണെങ്കിലീയക്ഷര-\\
ശ്ലോകസ്വൈരിണിയായ്‌ രമിയ്ക്കുക ഭവാൻ സന്യാസിയാണെങ്കിലും!
\end{slokam}

\Letter{സ}{പ}


\begin{slokam}{\VSv}{\Balendu}{സ്വത്തിന്നാര്‍ത്തി പെരുത്തതാം}
സ്വത്തിന്നാര്‍ത്തി പെരുത്തതാം, കൊടിയതാം ശസ്ത്രങ്ങളാര്‍ജ്ജിപ്പതാം,\\
ചിത്താവേശമടക്കുവാന്‍ ഹനനവും സംഭോഗവും ചെയ്‌വതാം,\\
ക്ഷുത്തില്ലാതെ ഭുജിപ്പതാം, തനയര്‍ തന്‍ സമ്പാദ്യമിച്ഛിപ്പതാം,\\
മര്‍ത്യന്നന്യമൃഗങ്ങളെക്കവിയുമാ നിസ്തുല്യമാം വൈഭവം!
\end{slokam}

\Letter{സ}{ക}



\begin{slokam}{\VSv}{\KJ}{സ്വായത്തം  മലയാളമാർന്നിടുമിടം}
സ്വായത്തം മലയാളമാർന്നിടുമിടം തോയം തുളുമ്പും ജട-\\
ശ്രേയസ്സാംഗളഭൂഷണം വിഷമഹോ! നീയഗ്നിവാഹിൻ! ദിവി\\
സ്വീയർക്കന്നടനത്തിലാർത്തമിഴു വാനായംവരുത്തുന്നു നിർ-\\
മ്മായം ഭാഷയതെൻ ഗിരീശ! നിയതം ശ്രീയൻപിലേകാവു മേ.
\end{slokam}

\Letter{സ}{സ}

\Topic{അഷ്ടപ്രാസം}.

\end{enumerate}

\subsection{ഹ}

\begin{enumerate}

\begin{slokam}{\VSv}{\VNM}{ഹന്ത, ക്രീഡയിലിക്കരം}
 ഹന്ത! ക്രീഡയിലിക്കരം കടപറിച്ചിട്ടോരു വെള്ളാന തൻ\\
ദന്തത്താൽ, മഘവോപലക്കുമിളയും വെച്ചെൻ പ്രസാദാർത്ഥമായ്‌\\
വിൺതച്ചൻ വിരചിച്ച പാദുകകളെ സ്വൈരം ത്രിലോകീമലർ-\\
പ്പെൺതയ്യത്ര തലോടുമെന്നടികളിൽ ചേർക്കുന്നു ചേടീജനം!
\end{slokam}

\Letter{ഹ}{വ}

\Book{രാവണന്റെ അന്തഃപുരഗമനം}.

\begin{slokam}{\VSv}{\Vyl}{ഹാ കഷ്ടം നരജീവിതം}
ഹാ കഷ്ടം! നരജീവിതം ദുരിത, മീ ശോകം മറക്കാൻ സുഖോ-\\
ദ്രേകം ചീട്ടുകളിക്കയാം ചിലർ, ചിലർക്കാകണ്‌ഠപാനം പ്രിയം,\\
മൂകം മൂക്കിനു നേർക്കു കാണ്മു ചിലരിന്നേകം ശിവം സുന്ദരം,\\
ശ്ലോകം ചൊല്ലിയിരിപ്പു ഞങ്ങള്‍ ചില, രീ ലോകം വിഭിന്നോത്സവം!
\end{slokam}

\Letter{ഹ}{മ}

\Topic{അഷ്ടപ്രാസം}.

\begin{slokam}{\VVt}{\KA}{ഹാ, പുഷ്പമേ, അധിക}
ഹാ, പുഷ്പമേ, അധികതുംഗപദത്തിലെത്ര\\
ശോഭിച്ചിരുന്നിതൊരു രാജ്ഞി കണക്കയേ നീ\\
ശ്രീ ഭൂവിലസ്ഥിര--അസംശയ--മിന്നു നിന്റെ-\\
യാഭൂതിയെങ്ങു പുനരെങ്ങു കിടപ്പിതോർത്താൽ?
\end{slokam}

\Letter{ഹ}{ശ}

\Book{വീണ പൂവ്}.

\begin{slokam}{\VSv}{\KCKP}{ഹാ ഹാ, ഭൂതലവാസമെത്ര}
ഹാ ഹാ, ഭൂതലവാസമെത്ര പരമാനന്ദത്തിനിന്നാസ്പദം! \\
ഹാ ഹാ, സൂര്യസുധാകരാദികൾ വിളങ്ങീടുന്നൊരാകാശവും\\
ഹാ ഹാ, പുഷ്പഫലാദിപൂർണ്ണതരുജാലോല്ലാസിനീഭൂമിയും\\
ഹാ ഹാ, സത്സമസൃഷ്ടരും തരുമൊരസ്സൗഖ്യങ്ങളോർക്കാവതോ?
\end{slokam}

\Letter{ഹ}{ഹ}


\Book{ആസന്നമരണചിന്താശതകം}.

\begin{slokam}{\VDv}{\Unk}{ഹിമകണാഞ്ചിത}
ഹിമകണാഞ്ചിതസൂനഗണോല്ലസത്-\\
കമലിനീകുലമാർന്ന സരിത്തുകൾ\\
വിമലകാന്തിയൊടങ്ങു ലസിച്ചു, ഭൂ-\\
രമണി ഹാ! മണിഹാരമണിഞ്ഞ പോൽ. 
\end{slokam}

\Letter{ഹ}{വ}

\Topic{യമകം (ദ്രുതവിളംബിതം)}.


\begin{slokam}{\VSv}{\TMV}{ഹുങ്കാളുന്ന തിമിങ്ഗിലങ്ങള്‍}
ഹുങ്കാളുന്ന തിമിങ്ഗിലങ്ങള്‍ തലകാണിക്കെ, ത്തിരിഞ്ഞോടുവോ-\\
രെൻ കൈവർത്തക, ചെയ്‌വതെന്തു ചെറുമീൻ വർഗ്ഗത്തൊടിന്നക്രമം?\\
തൻ കയ്യൂക്കിലഹങ്കരിച്ചടിപിടിക്കങ്ങാടിയിൽ ചെന്നു തോ-\\
റ്റങ്കത്തിന്നുടനമ്മയോടണയുമാ വീരൻ ഭവാൻ തന്നെയൊ?
\end{slokam}

\Letter{ഹ}{ത}



\begin{slokam}{\VSv}{\RV}{ഹൃത്തിന്നൊത്ത വപുസ്സെടുത്തു}
ഹൃത്തിന്നൊത്ത വപുസ്സെടുത്തു കരിയായ്‌, ക്കാട്ടാളനായ്‌, ത്തെണ്ടിയാ,-\\
യർത്ഥം പേർത്തറിവാക്കിയും, കലഹമാർന്നും, മെയ്‌ പകുത്തേകിയും\\
മൂർദ്ധാവിങ്കലനർത്ഥവും കപടമായ്ക്കാത്തും ചിരം മാറുവോ-\\
രർത്ഥം വാക്കിനൊടെന്നമട്ടുമയുമൊത്താടും പൊരുൾ കാക്കണം
\end{slokam}

\Letter{ഹ}{മ}


\begin{slokam}{\VSv}{\VNM}{ഹേ മൽത്തോഴി, വിലക്ഷിതാവിലസിതം}
ഹേ മൽത്തോഴി, വിലക്ഷിതാവിലസിതം കൊണ്ടൊന്നു കൂപ്പി, സ്ഫുരത്- \\
പ്രേമത്താൽ വികസിച്ചു, മന്മഥരസത്തേനുൾക്കലർന്നുള്ളതായ്, \\
തൂമന്ദസ്മിതമായ പാലിൽ മുഴുകിച്ചോരീയപാംഗോത്ക്കല- \\
സ്തോമത്താൽ സ്വയമേതൊരീശ്വരനെയാണർച്ചിപ്പതിത്തവ്വിൽ നീ? 
\end{slokam}

\Letter{ഹ}{ത}

\Book{വിലാസലതിക}


\begin{slokam}{\VSr}{\VNM}{ഹ്രീങ്കാരക്ഷീരവാരാന്നിധി}
 ഹ്രീങ്കാരക്ഷീരവാരാന്നിധിപരമസുധേ, പാണിചഞ്ചൽകൃപാണീ-\\
ഭാങ്കാരത്രാസിതാഖണ്ഡലവിമതകലേ, വിശ്വവല്ലിക്കു വേരേ!\\
ഞാൻ കാലിൽ കൂപ്പിടുന്നേൻ, യതിഹൃദയമിളിന്ദാളി മേളിക്കുമോമൽ-\\
പ്പൂങ്കാവേ നിന്റെപേരിൽ ഭഗവതി, ലളിതേ, ഭക്തി സിദ്ധിക്കണം മേ!
\end{slokam}

\Letter{ഹ}{ഞ}


\end{enumerate}

