\section{ശ്ലോകങ്ങൾ: സംസ്കൃതം}
\label{sec:slokams:san}

\subsection{അ, ആ}

\begin{enumerate}

\begin{slokam}{\VBh}{\Sank}{അകണ്ഠേ കളങ്കാദനംഗേ}
അകണ്ഠേ കളങ്കാദനംഗേ ഭുജംഗാ-\\
ദപാണൗ കപാലാദഫാലേ ന ലാക്ഷാത്‌\\
അമൗലൗ ശശാങ്കാദവാമേ കളത്രാ-\\
ദഹം ദേവമന്യം ന മന്യേ ന മന്യേ
\end{slokam}

\Letter{അ}{അ}

\begin{slokam}{\VSn}{പതഞ്ജലി}{അഖണ്ഡവിധുമണ്ഡലമുഖം}
അഖണ്ഡവിധുമണ്ഡലമുഖം, ഡമരുമണ്ഡിതകരം, കനകമണ്ഡപഗൃഹം,\\
സ്വഗണ്ഡതലമുണ്ഡശശികുണ്ഡലിപമണ്ഡനപരം, ജലജഖണ്ഡജടിലം,\\
മൃകണ്ഡുസുതചണ്ഡകരദണ്ഡധരമുണ്ഡനപദം, ഭുജഗകുണ്ഡലധരം,\\
ശിഖണ്ഡശശിഖണ്ഡമണിമണ്ഡനപരം, പരചിദംബരനടം, ഹൃദി ഭജേ!
\end{slokam}

\Letter{അ}{മ}

\begin{slokam}{\VSr}{\Melp}{അഗ്രേ പശ്യാമി തേജോ}
അഗ്രേ പശ്യാമി തേജോനിബിഡതരകളായാവലീ ലോഭനീയം\\
പീയൂഷാപ്ലാവിതോഹം തദനുതദുദരേ ദിവ്യ കൈശോരവേഷം\\
താരുണ്യാരംഭരമ്യം പരമസുഖരസാസ്വാദരോമാഞ്ചിതാംഗൈ-\\
രാവീതം നാരദാദ്യൈര്‍വിലസദുപനിഷത്സുന്ദരീ മണ്ഡലൈശ്ച
\end{slokam}

\Letter{അ}{ത}

\Book{നാരായണീയം}.


\begin{slokam}{\VKm}{\LS}{അങ്കണേ ഘടിതരിംഘണം}
അങ്കണേ ഘടിതരിംഘണം ചലിതകിങ്കിണീകൃതഘണാഘണം\\
ചഞ്ചലാളകകുലാകുലം തരളലോചനം ദുരിതമോചനം\\
അങ്കുരദ്വിശദ ദന്തകുഡ്മളവിലോഭനീയ വദനാംബുജം\\
ശംബരേശമപി ചിന്തയാമി ശിശുമിന്ദ്രനീല മണിമേചകം
\end{slokam}

\Letter{അ}{അ}


\begin{slokam}{\VSr}{\Melp}{അജ്ഞാത്വാ തേ മഹത്ത്വം}
അജ്ഞാത്വാ തേ മഹത്ത്വം യദിഹ നിഗതിതം വിശ്വനാഥ, ക്ഷമേഥാഃ\\
സ്തോത്രം ചൈതത്‌ സഹസ്രോത്തരമധികതരം ത്വത്പ്രസാദായ ഭൂയാത്‌\\
ദ്വേധാ നാരായണീയം ശ്രുതിഷു ച ജാനഷാ സ്തുത്യതാ വര്‍ണ്ണനേന\\
സ്ഫീതം ലീലാവതാരൈരൈദമിഹ കുരുതാമായുരാരോഗ്യസൌഖ്യം!
\end{slokam}

\Letter{അ}{ദ}

\Book{നാരായണീയം}.

\begin{slokam}{\VTd}{\Melp}{അഥ ദിക്ഷു വിദിക്ഷു}
അഥ ദിക്ഷു വിദിക്ഷു പരിക്ഷുഭിത-\\
ഭ്രമിതോദരവാരിനിനാദഭരൈഃ\\
ഉദഗാദുദഗാദുരഗാധിപതി-\\
സ്ത്വദുപാന്തമശാന്തരുഷാന്ധമനഃ
\end{slokam}

\Letter{അ}{ഉ}

\Book{നാരായണീയം}.


\begin{slokam}{\VBh}{\Sank}{അനാദ്യന്തമാദ്യം}
അനാദ്യന്തമാദ്യം പരം തത്വമർത്ഥം\\
ചിദാകാരമേകം തുരീയം ത്വമേയം\\
ഹരിബ്രഹ്മമൃഗ്യം പരബ്രഹ്മരൂപം\\
മനോവാഗതീതം മഹാശൈവമീഡേ
\end{slokam}

\Letter{അ}{ഹ}

\begin{slokam}{\VSv}{\Unk}{അന്തര്യശ്ച ബഹിർ}
അന്തര്യശ്ച ബഹിർവിധൂതതിമിരം ജ്യോതിർമയം ശാശ്വതം\\
സ്ഥാനം പ്രാപ്യ വിരാജതേ വിനമതാമജ്ഞാനമുൻമൂലയൻ\\
പശ്യന്വിശ്വമപീദമുല്ലസതി യോ വിശ്വസ്യ പാരേ പര-\\
സ്തസ്മൈ ശ്രീരമണായ ലോകഗുരവേ ശോകസ്യ ഹന്ത്രേ നമഃ
\end{slokam}

\Letter{അ}{പ}



\begin{slokam}{\VMk}{\KD}{അപ്യന്നസ്മിൻ ജലധര}
അപ്യന്നസ്മിൻ ജലധര, നഹാകാളമാസ്വാദ്യ കാലേ \\
സ്ഥാതവ്യം തേ, നയനവിഷയം യാവദത്യേതി ഭാനുഃ \\
കുർവ്വൻ സന്ധ്യാബലിപടഹതാം ശൂലിനഃ ശ്ലാഘനീയാഃ \\
മാമന്ദ്രാണാം ഫലമവികലം ലപ്സ്യസേ ഗർജ്ജിതാനാം. 
\end{slokam}

\Letter{അ}{ക}

\Book{മേഘസന്ദേശം}.
പരിഭാഷ: \SlRef{ആ ദിക്കിൽ കാത്തു നിൽപ്പൂ ജലധര}.

\begin{slokam}{\VSv}{\CN}{അബ്ദാര്‍ദ്ധേന ഹരിം}
അബ്ദാര്‍ദ്ധേന ഹരിം പ്രസന്നമകരോദൌത്താനപാദിശ്ശിശു,-\\
സ്സപ്താഹേന നൃപഃ പരീക്ഷി, ദബലാ യാമാര്‍ദ്ധതഃ പിംഗളാ\\
ഖട്വാംഗോ ഘടികാദ്വയേന - നവതി പ്രായോപി തന്നവ്യഥേ\\
തം കാരുണ്യനിധിം പ്രപദ്യ ശരണം ശേഷായുഷാ തോഷയേ
\end{slokam}

\Letter{അ}{ഖ}


\begin{slokam}{\VSv}{\Unk}{അംബാ കുപ്യതി താത}
 "അംബാ കുപ്യതി താത, മൂര്‍ദ്ധ്നി വിദ്ധൃതാം ഗംഗേയമുത്സൃജ്യതാം"\\
"വിദ്വന്‍, ഷണ്മുഖ, കാ ഗതിര്‍മ്മയി ചിരാദഭ്യാഗതായാം വദ"\\
രോഷാവേശവശാദശേഷവദനൈഃ പ്രത്യുത്തരം ദത്തവാന്‍\\
"\sam{അംഭോദിര്‍ജ്ജലധിഃ പയോധിരുദധിര്‍വ്വാരാന്നിധിര്‍വാരിധിഃ}"
\end{slokam}

\Letter{അ}{ര}

സമസ്യാപൂരണം.

\SeeAlso{നിത്യം തെണ്ടുവതെത്ര നീചം}. 

\begin{slokam}{\VOth}{\NDK}{അയി ജനമാനസമാനസ}
അയി ജനമാനസമാനസമാനസമാസമഹം സവിദൂരമണേ\\
ദയിതഹിതംകരദീനദയാകരദീർഘവിശാലസുഫാലതനോ\\
ത്വയി കലയേ മഹിതുംഗഹിമാലയസന്നിഭതുംഗസമാധിനിധിം\\
ജയ ജയ ദേശിക, ജാതിനിരാസക, നാണുഗുരോ, ഗുരുദേവ, വിഭോ! 
\end{slokam}

\Letter{അ}{ത}

\begin{slokam}{\VSv}{\KD}{അർദ്ധം ദാനവവൈരിണാ}
അര്‍ദ്ധം ദാനവവൈരിണാ ഗിരിജയാപ്യര്‍ദ്ധം ഹരസ്യാഹൃതം\\
ദേവേത്ഥം ജഗതീതലേ സ്മരഹരാഭാവേ സമുന്മീലിതേ\\
ഗംഗാ സാഗര, മംബരം ശശികലാ, നാഗാധിപഃ ക്ഷ്മാതലം,\\
സര്‍വജ്ഞത്വമധീശ്വരത്വമഗമത് ത്വാം, മാം ച ഭിക്ഷാടനം 
\end{slokam}

\Letter{അ}{ഗ}

പരിഭാഷ:  \SlRef{മെയ്യില്‍ പാര്‍വ്വതി പാതി}.


\begin{slokam}{\VSk}{\Sank}{അവിദ്യാനാമന്തസ്തിമിരമിഹിര}
അവിദ്യാനാമന്തസ്തിമിരമിഹിരദ്വീപനഗരീ\\
ജഡാനാം ചൈതന്യസ്തബകമകരന്ദസ്രുതിഝരീ\\
ദരിദ്രാണാം ചിന്താമണിഗുണനികാ, ജന്മജലധൌ\\
നിമഗ്നാനാം ദംഷ്ട്രാ, മുരരിപുവരാഹസ്യ ഭവതി
\end{slokam}

\Letter{അ}{ദ}


\begin{slokam}{\VSr}{\Melp}{ആനീലശ്ലക്ഷ്മകേശം}
ആനീലശ്ലക്ഷ്മകേശം, ജ്വലിതമകരസത്കുണ്ഡലം മന്ദഹാസ- \\ 
സ്യന്ദാർദ്രം, കൗസ്തുഭശ്രീപരിഗതവനമാലോരുഹാരാഭിരാമം, \\ 
ശ്രീവത്സാങ്കം, സുബാഹും, മൃദുലസദുദരം, കാഞ്ചനച്ഛായചേലം, \\
ചാരുസ്നിഗ്ദ്ധോരുമംഭോരുഹലളിതപദം ഭാവയേഹം ഭവന്തം
\end{slokam}

\Letter{അ}{ശ}

\Book{നാരായണീയം} (95:8). 


\begin{slokam}{\VSv}{\Amar}{ആലോലാമളകാവലീം വിലുളിതാം}
  ആലോലാമളകാവലീം വിലുളിതാം ബിഭ്രച്ചലത്കുണ്ഡലം\\
  കിഞ്ചിന്മൃഷ്ടവിശേഷകം തനുതരൈഃ സ്വേദാംഭസാം ജാലകൈഃ\\
  തമ്പ്യാ യത്  സുരതാന്തതാന്തനയനം വക്ത്രം രതിവ്യത്യയേ\\
  തത്ത്വാം പാതു ചിരായ കിം ഹരിഹരബ്രഹ്മാദിഭിർദൈവതൈഃ
\end{slokam}

\Letter{അ}{ത}


\Book{അമരുകശതകം}.  പരിഭാഷ: \SlRef{ചഞ്ചലങ്ങളളകങ്ങൾ ചിന്നി}.



\begin{slokam}{\VSv}{\Sank}{ആസ്താം താവദിയം}
ആസ്താം താവദിയം പ്രസൂതിസമയേ ദുർവ്വാരശൂലവ്യഥാ\\
നൈരുച്യം തനുശോഷണം മലമയീ ശയ്യാ ച സാംവത്സരീ\\
ഏകസ്യാപി ന ഗർഭഭാര ഭരണക്ലേശസ്യ യസ്യാ ക്ഷമോ\\
ദാതും നിഷ്കൃതിമുന്നതോപി തനയസ്തസ്യൈ ജനന്യൈ നമ
\end{slokam}

\Letter{അ}{എ}

പരിഭാഷകൾ:  \SlRef{നിൽക്കട്ടേ പേറ്റുനോവിൻ കഥ}, \SlRef{പേറ്റുനോവവിടെ നിന്നിടട്ടെ}.

\end{enumerate}
\subsection{ഇ, ഈ}

\begin{enumerate}

\begin{slokam}{\VSv}{\Unk}{ഇന്ദ്രം ദ്വ്യക്ഷ, മമന്ദപൂര്‍വ്വമുദധിം}
ഇന്ദ്രം ദ്വ്യക്ഷ, മമന്ദപൂര്‍വ്വമുദധിം, പഞ്ചാനനം പദ്മജം,\\
ശൈലാന്‍ പക്ഷധരാന്‍, ഹയാനപി ച, തം കാമം ച സദ്വിഗ്രഹം,\\
അബ്ധിം ശുദ്ധജലം, സിതം ശിവഗളം, ലക്ഷ്മീപതിം പിംഗളം,\\
ജാനേ സര്‍വ്വമഹം പ്രഭോ രഘുപതേ ദത്താപഹാരം വിനാ
\end{slokam}

\Letter{ഇ}{അ}

\end{enumerate}
\subsection{ഉ, ഊ}

\begin{enumerate}




\begin{slokam}{\VSr}{\Melp}{ഉത്തുംഗോല്ലാസിനാസം}
ഉത്തുംഗോല്ലാസിനാസം ഹരിമണിമുകുരപ്രോല്ലസദ്ഗണ്ഡപാളീ-\\
വ്യാലോലത്കർണപാശാഞ്ചിതമകരമണീകുണ്ഡലദ്വന്ദ്വദീപ്രം\\
ഉന്മീലദ്ദന്തപങ്ങ്ക്തി സ്ഫുരദരുണതരച്ഛായബിംബാധരാന്തഃ-\\
പ്രീതിപ്രസ്യന്ദിമന്ദസ്മിതമധുരതരം വക്ത്രമുദ്ഭാസതാം മേ
\end{slokam}

\Letter{ഉ}{ഉ}

\Book{നാരായണീയം} (100:4).



\begin{slokam}{\VSv}{\Unk}{ഉത്സര്‍പദ്‌വലിഭങ്ഗഭീഷണ}
ഉത്സര്‍പദ്‌വലിഭങ്ഗഭീഷണഹനുഹ്രസ്വസ്ഥവീയസ്തര-\\
ഗ്രീവം പീവരദോശ്ശതോദ്ഗതനഖ ക്രൂരാംശുദൂരോല്‌ബണം\\
വ്യോമോല്ലങ്ഘിഘനാഘനോപമഘനപ്രദ്ധ്വാനനിര്‍ദ്ധാവിത-\\
സ്പര്‍ദ്ധാലുപ്രകരം നമാമി ഭവതസ്തന്നാരസിംഹം വപുഃ
\end{slokam}

\Letter{ഉ}{വ}

\Book{നാരായണീയം}, 25:4.

\begin{slokam}{\VSr}{\Melp}{ഊരൂ ചാരൂ തവോരൂ}
ഊരൂ ചാരൂ തവോരൂ ഘനമസൃണരുചൗ ചിത്തചോരൗ രമായാഃ \\
വിശ്വക്ഷോഭം വിശങ്ക്യ ധ്രുവമനിശമുഭൗ പീതചേലാവൃതാംഗൗ\\
ആനമ്രാണാം പുരസ്താന്ന്യസനധൃതസമസ്താർത്ഥപാളീസമുദ്ഗ-\\
ച്ഛായം ജാനുദ്വയം ച ക്രമപൃഥുലമനോജ്ഞേ ച ജംഘേ നിഷേവേ
\end{slokam}

\Letter{ഉ}{അ}

\Book{നാരായണീയം} (100:8).


\end{enumerate}
\subsection{ക}

\begin{enumerate}

\begin{slokam}{\VMk}{\KD}{കശ്ചിത് കാന്താ വിരഹഗുരുണാ}
കശ്ചിത് കാന്താ വിരഹഗുരുണാ സ്വാധികാരാത് പ്രമത്തഃ\\
ശാപേനാസ്തംഗമിതമഹിമാ വർഷഭോഗ്യേണ ഭർത്തുഃ\\
യക്ഷശ്ചക്രേ ജനകതനയാസ്നാനപുണ്യോദകേഷു\\
സ്നിഗ്ദച്ഛായാതരുഷു വസതിം രാമഗിര്യാശ്രമേഷു
\end{slokam}

\Letter{ക}{യ}

\Book{മേഘസന്ദേശം}.  പരിഭാഷ: \SlRef{പേരോർക്കുന്നീല, കൃത്യപ്പിഴ}

\begin{slokam}{\VSv}{\LS}{കസ്തൂരീതിലകം ലലാടഫലകേ}
കസ്തൂരീതിലകം ലലാടഫലകേ, വക്ഷസ്ഥലേ കൌസ്തുഭം,\\
നാസാഗ്രേ നവമൌക്തികം, കരതലേ വേണും, കരേ കങ്കണം,\\
സര്‍വ്വാംഗേ ഹരിചന്ദനം ച കലയന്‍ കണ്ഠേ ച മുക്താവലീം\\
ഗോപസ്ത്രീപരിവേഷ്ടിതോ വിജയതേ ഗോപാല ചൂഡാമണി
\end{slokam}

\Letter{ക}{യ}


\begin{slokam}{\VPc}{\Rav}{കരാളഫാലപട്ടികാ}
കരാളഫാലപട്ടികാധഗദ്ധഗദ്ധഗജ്ജ്വല-\\
ദ്ധനഞ്ജയാധരീകൃതപ്രചണ്ഡപഞ്ചസായകേ\\
ധരാധരേന്ദ്രനന്ദിനീകുചാഗ്രചിത്രപത്രക-\\
പ്രകൽപനൈകശിൽപിനി ത്രിലോചനേ മതിർമ്മമ
\end{slokam}

\Letter{ക}{ധ}

\begin{slokam}{\VSv}{\KD}{കാര്യാ സൈകതലീനഹംസമിഥുനാ}
കാര്യാ സൈകതലീനഹംസമിഥുനാ സ്രോതോവഹാ മാലിനീ\\
പാദാസ്താമഭിതോ നിഷണ്ണഹരിണാ ഗൌരീഗുരോഃ പാവനാഃ\\
ശാഖാലംബിതവത്കലസ്യ ച തരോര്‍നിര്‍മാതുമിച്ഛാമ്യധഃ\\
ശൃംഗേ കൃഷ്ണമൃഗസ്യ വാമനയനം കണ്ഡൂയമാനാം മൃഗീം.
\end{slokam}

\Letter{ക}{ശ}

\Book{അഭിജ്ഞാനശാകുന്തളം}. പരിഭാഷ: \SlRef{ചാലേ മാലിനിയും}. 

\begin{slokam}{\VPc}{ഇലന്തൂർ നാരായണൻ വൈദ്യർ}{കൃതാന്തബന്ധബന്ധനൈക}
 കൃതാന്തബന്ധബന്ധനൈകകൃന്തനം മുരാന്തകം\\
നിതാന്തഭാസുരം വരം വരേണ്യമീശ്വരം ഹരിം\\
കൃപാകദംബമാധുരീരസപ്രവാഹനിർഗ്ഗള-\\
ന്മുഖാരവിന്ദമച്യുതം നമാമി ലോകനായകം.
\end{slokam}

\Letter{ക}{ക}



\begin{slokam}{\VSv}{നീലകണ്ഠദീക്ഷിതർ}{കീടഃ കശ്ചന വൃശ്ചികഃ,}
കീടഃ കശ്ചന വൃശ്ചികഃ, കിയദയം പ്രാണീ, കിയച്ചേഷ്ടതേ,\\
കോ ഭാരോ ഹനനേ\prash{}സ്യ, ജീവതി സ വാ കാലം കിയന്തഃ പുനഃ\\
നാമ്‌നാപ്യസ്യ കിയദ്‌ ബിഭേതി ജനതാ ദൂരേ കിയദ്‌ ധാവതി\\
കിം ബ്രൂമോ ഗരളസ്യ ദുർവ്വിഷഹതാം പുച്ഛാഗ്രശൂകസ്പൃശഃ?
\end{slokam}

\Letter{ക}{ന}


\Book{അന്യാപദേശശതകം}. പരിഭാഷ: \SlRef{തേളു തുച്ഛമൊരു}. 


\begin{slokam}{\VMt}{\Sank}{കൂലാതിഗാമിഭയ}
കൂലാതിഗാമിഭയതൂലാവലീജ്വലനകീലാ, നിജസ്തുതിവിധൗ\\
കോലാഹലക്ഷപണകാലാമരീകുശലകീലാലപോഷണനഭാ,\\
സ്ഥൂലാ കുചേ, ജലദനീലാ കചേ, കലിതലീലാ കദംബവിപിനേ,\\
ശൂലായുധപ്രണതിശീലാ, വിഭാതു ഹൃദി, ശൈലാധിരാജതനയാ.
\end{slokam}

\Letter{ക}{സ}

\Topic{ദ്വാദശപ്രാസം}.




\begin{slokam}{\VSv}{\Melp}{കേയൂരാങ്ഗദകങ്കണോത്തമ}
കേയൂരാങ്ഗദകങ്കണോത്തമമഹാരത്നാങ്ഗുലീയാങ്കിത-\\
ശ്രീമത്ബാഹുചതുഷ്കസങ്ഗതഗദാ ശംഖാരിപംകേരുഹാം\\
കാഞ്ചിത് കാഞ്ചന കാഞ്ചിലാഞ്‌ഛിതലസത് പീതാംബരാലംബനീ-\\
മാലംബേ വിമലാംബുജദ്യുതിപദാം മൂർത്തിം തവാർത്തിച്ഛിദം.
\end{slokam}

\Letter{ക}{ക}

\Book{നാരായണീയം}.


\begin{slokam}{\VKm}{\Melp}{കേശപാശധൃതപിഞ്ഛികാ}
കേശപാശധൃതപിഞ്ഛികാവിതതി സഞ്ചലന്മകരകുണ്ഡലം\\
ഹാരജാല വനമാലികാലുളിതമംഗരാഗ ഘനസൗരഭം\\
പീതചേലധൃതകാഞ്ചികാഞ്ചിദമുദഞ്ചദംശു മണിനൂപുരം\\
രാസകേളി പരിഭൂഷിതം തവ ഹി രൂപമീശ കലയാമഹേ!
\end{slokam}

\Letter{ക}{പ}


\Book{നാരായണീയം}.  പരിഭാഷ: \SlRef{പീലി ചാർത്തിയൊരു കുന്തളം}.

\begin{slokam}{\VSv}{\MR}{കേളീലോലമുദാരനാദ}
കേളീലോലമുദാരനാദമുരളീനാളീനിലീനാധരം\\
ധൂളീധൂസരകാന്തകുന്തളഭരവ്യാസങ്ഗിപിഞ്ഛാഞ്ചലം\\
നാളീകായതലോചനം നവഘനശ്യാമം ക്വണത്കിങ്ങിണീ-\\
പാളീദന്ദുരപിങ്ഗളാംബരധരം ഗോപാലബാലം ഭജേ
\end{slokam}

\Letter{ക}{ന}

\Book{കൃഷ്ണഗീതി}.


\end{enumerate}

\subsection{ഗ}

\begin{enumerate}


\begin{slokam}{\VSr}{\Melp}{ഗംഗാ ഗീതാ ച ഗായത്ര്യപി}
ഗംഗാ ഗീതാ ച ഗായത്ര്യപി ച തുളസികാ ഗോപികാചന്ദനം തത്-\\
സാളഗ്രാമാഭിപൂജാ പരപുരുഷ തഥൈകാദശീ നാമവർണാഃ\\
ഏതാന്യഷ്ടാപ്യയത്നാന്യയി കലിസമയേ ത്വത്പ്രസാദപ്രവൃദ്ധ്യാ \\
ക്ഷിപ്രം മുക്തിപ്രദാനീത്യഭിദധുരൃഷയസ്തേഷു മാം സജ്ജയേഥാഃ
\end{slokam}

\Letter{ഗ}{എ}

\Book{നാരായണീയം}.


\begin{slokam}{\VMt}{സ്വാതിതിരുനാൾ}{ഗംഗാധരാദൃത, മസംഗാശയാംബുരുഹ}
ഗംഗാധരാദൃത, മസംഗാശയാംബുരുഹഭൃംഗായിതം, ദിതിഭുവാം\\
ഭംഗാവഹം, വിധൃതതുംഗാചലം, പൃഥുഭുജംഗാധിരാജശയനം,\\
അംഗാനുഷംഗിമൃദുപിംഗാംബരം, പരമനംഗാതിസുന്ദരതനും,\\
ശൃംഗാരമുഖ്യരസരംഗായിതം, ഭജ ത, മംഗാബ്ജനാഭമനിശം.
\end{slokam}

\Letter{ഗ}{അ}

\Topic{ദ്വാദശപ്രാസം}.

\end{enumerate}

\subsection{ച}


\begin{enumerate}

\begin{slokam}{\VSv}{\Sank}{ചന്ദ്രോദ്ഭാസിതശേഖരേ സ്മരഹരേ}
ചന്ദ്രോദ്ഭാസിതശേഖരേ സ്മരഹരേ ഗംഗാധരേ ശങ്കരേ \\
സർപ്പൈർഭൂഷിതകണ്ഠകർണ്ണവിവരേ നേത്രോത്ഥവൈശ്വാനരേ \\
ദന്തിത്വക്കൃതസുന്ദരാംബരധരേ ത്രെയിലോക്യസാരേ ഹരേ \\
മോക്ഷാർത്ഥം കുരു ചിത്തവൃത്തിമഖിലാ, മന്യൈസ്തു കിം കർമ്മദിഃ?
\end{slokam}

\Letter{ച}{ദ}

\begin{slokam}{\VMt}{\Sank}{ചേടീഭവന്നിഖിലഖാടീ}
ചേടീഭവന്നിഖിലഖാടീകദംബതരുവാടീഷു നാകിപടലീ-\\
കോടീരചാരുതരകോടീ മണീകിരണകോടീകരംബിതപദാ\\
പാടീരഗന്ധികുചശാടീ കവിത്വപരിപാടീമഗാധിപസുതാ\\
ഘോടീകുലാദധികധേറ്റെമുദാരമുഖവീടീരസേന തനുതാം.
\end{slokam}

\Letter{ച}{പ}

\Topic{ദ്വാദശപ്രാസം}.

\end{enumerate}

\subsection{ജ}


\begin{enumerate}


\begin{slokam}{\VPc}{\Rav}{ജടാകടാഹസംഭ്രമ}
ജടാകടാഹസംഭ്രമഭ്രമന്നിലിംപനിര്ഝരീ- \\
വിലോലവീചിവല്ലരീ വിരാജമാനമൂർദ്ധനി\\
ധഗദ്ധഗദ്ധഗജ്ജ്വലല്ലലാടപട്ടപാവകേ \\
കിശോരചന്ദ്രശേഖരേ രതിഃ പ്രതിക്ഷണം മമ
\end{slokam}

\Letter{ജ}{ധ}


\begin{slokam}{\VPc}{\Rav}{ജടാടവീഗളജ്ജലപ്രവാഹ}
ജടാടവീഗളജ്ജലപ്രവാഹപാവിതസ്ഥലേ\\
ഗളേऽവലംബ്യ ലംബിതാം ഭുജംഗതുംഗമാലികാം\\
ഡമഡ്ഡമഡ്ഡമന്നിനാദവഡ്ഡമഡ്ഡമർവ്വയം\\
ചകാര ചണ്ഡതാണ്ഡവം തനോതു നഃ ശിവഃ ശിവം.
\end{slokam}

\Letter{ജ}{ഡ}

\begin{slokam}{\VSv}{വ്യാസൻ}{ജന്മാദ് യസ്യ യതോന്വയാത്}
ജന്മാദ് യസ്യ യതോന്വയാദിതരതശ്ചാർതേഷ്വഭിജ്ഞഃ സ്വരാട്\\
തേനേ ബ്രഹ്മ ഹൃദാ യ ആദികവയേ മുഹ്യന്തി യത് സൂരയഃ\\
തേജോവാരിമൃദാം യഥാ വിനിമയോ യത്ര ത്രിസർഗോമൃഷാ\\
ധാമ്നാ സ്വേന സദാ നിരസ്തകുഹകം സത്യം പരം ധീമഹി
\end{slokam}

\Letter{ജ}{ത}

\Book{ഭാഗവതം}.

\end{enumerate}



\subsection{ത}

\begin{enumerate}


\begin{slokam}{\VSr}{\Melp}{തത്തേ പ്രത്യഗ്രധാരാ}
തത്തേ പ്രത്യഗ്രധാരാധരലളിതകളായാവലീകേളികാരം\\
ലാവണ്യസ്യൈകസാരം സുകൃതിജനദൃശാം പൂർ‌ണ്ണപുണ്യാവതാരം\\
ലക്ഷ്മീനിഃശങ്കലീലാനിലയനമമൃതസ്യന്ദസന്ദേഹമന്തഃ\\
സിഞ്ചത് സഞ്ചിന്തകാനാം വപുരനുകുലയേ മാരുതാഗാരനാഥഃ.
\end{slokam}

\Letter{ത}{ല}

\Book{നാരായണീയം} (1:6).



\begin{slokam}{\VMk}{ലക്ഷ്മീദാസൻ}{തത്സേവാർത്ഥം തരുണസഹിതാഃ}
തത്സേവാർത്ഥം തരുണസഹിതാസ്താമ്രപാദാരവിന്ദാ-\\
സ്താമ്യന്മധ്യാസ്തനഭരനതാസ്താരഹാരാവലീകാഃ\\
താരേശാസ്യാസ്തരളനയനാസ്തർജ്ജനീയാളകാഢ്യാ-\\
സ്തത്രസ്യാഃ സ്യുഃ സ്തബകിതകരാസ്താലവൃന്തൈസ്തരുണ്യഃ
\end{slokam}

\Letter{ത}{ത}


\Book{ശുകസന്ദേശം}.  പരിഭാഷ:  \SlRef{കാന്തന്മാരൊത്തു}, \SlRef{ആരാധിപ്പാ, നരുണപദ}.
\Topic{ആദിപ്രാസം}.

\begin{slokam}{\VSk}{\Sank}{തനീയാംസം പാംസും}
തനീയാംസം പാംസും തവചരണപങ്കേരുഹഭവം\\
വിരിഞ്ചിസ്സഞ്ചിന്വൻ വിരചയതി ലോകാനവികലം\\
വഹത്യേനം ശൗരിഃ കഥമപി സഹസ്രേണ ശിരസാ\\
ഹരഃ സംക്ഷുദ്യൈനം ഭജതി ഭസിതോദ്ധൂളനവിധിം
\end{slokam}

\Letter{ത}{വ}

\Book{സൗന്ദര്യലഹരി}.  പരിഭാഷ:  \SlRef{അപ്രജാപതി ഭവത്പദാബ്ജ}.

\begin{slokam}{\VVt}{\KD}{തവ ഹൃതാസ്മി പുരൈവ}
തവ ഹൃതാസ്മി പുരൈവ ഗുണൈരഹം\\
ഹരതി മാം കില ചേദിനൃപോऽധുനാ\\
"അയി കൃപാലയ, പാലയ മാ"മിതി\\
പ്രജഗദേ ജഗദേകപതേ തയാ 
\end{slokam}

\Letter{ത}{അ}

\Book{നാരായണീയം}, 78:7.
\Topic{യമകം (ദ്രുതവിളംബിതം, രണ്ടു വരികളിൽ)}.
\end{enumerate}




\subsection{ദ}

\begin{enumerate}


\begin{slokam}{\VVt}{\KD}{ദർഭാങ്കുരേണ ചരണഃ ക്ഷതഃ}
ദർഭാങ്കുരേണ ചരണഃ ക്ഷത ഇത്യകാണ്ഡേ \\
തന്വീ സ്ഥിതാ കതിചിദേവ പദാനി ഗത്വാ \\
ആസീദ് വിവൃത്തവദനാ ച വിമോദയന്തീ \\
ശാഖാസു വല്ക്കലമസക്തമപി ദ്രുമാണാം. 
\end{slokam}

\Letter{ദ}{അ}

\Book{അഭിജ്ഞാനശാകുന്തളം}.  പരിഭാഷ: \SlRef{കൊണ്ടല്‍വേണിയൊരു രണ്ടുനാലടി}


\end{enumerate}




\subsection{ധ}

\begin{enumerate}

\begin{slokam}{\VPc}{\Rav}{ധരാധരേന്ദ്രനന്ദിനീ}
 ധരാധരേന്ദ്രനന്ദിനീവിലാസബന്ധുബന്ധുര-\\
സ്ഫുരദ്ദിഗന്തസന്തതിഃ പ്രമോദമാനമാനസേ\\
കൃപാകടാക്ഷധോരണീ നിരുദ്ധദുർധരാപതിഃ\\
ക്വചിദ്ദിഗംബരേ മനോവിനോദമേതുവസ്തുനി
\end{slokam}

\Letter{ധ}{ക}

\begin{slokam}{\VVt}{\Melp}{ധിക്‌ പാണ്ഡുപുത്രചരിതം}
 ധിക്‌ പാണ്ഡുപുത്രചരിതം സ്ഥവിരപ്രമാണം\\
ബാലപ്രമാണമപി കഷ്ടമഹോ വിനഷ്ടം!\\
രേ, ധർമ്മജ! ദ്രുപദജാമപി പൃച്ഛ കാര്യം;\\
നാരീപ്രമാണമപി തേऽസ്ത്വിഹ രാജ്യതന്ത്രം!

\end{slokam}

\Letter{ധ}{ര}

\Book{രാജസൂയം ചമ്പു}.

\end{enumerate}




\subsection{ന}

\begin{enumerate}


\begin{slokam}{\VSr}{\KVIT}{നിത്യം നശ്ചിത്തപദ്മേ}
 നിത്യം നശ്ചിത്തപദ്മേ പരിലസതു കപാലീ കപാലീകപാലീ-\\
മാലാധാരീ സമസ്തപ്രമദജനകലാപഃ കലാപഃ കലാപഃ\\
ഭൂത്വാ നിർഭാതി യസ്യാധികമസുസമരീണാമരീണാമരീണാ-\\
മുത്പേഷ്ടാ യശ്ച ദൂരീകൃതകമലമഹസ്തോമഹസ്തോമഹസ്തഃ
\end{slokam}

\Letter{ന}{ഭ}

\Topic{അന്ത്യപ്രാസവും യമകവും}.  \PrevSlRef{ശർമ്മത്തെസ്സൽക്കരിക്കും},
\NextSlRef{ഭക്തർക്കിഷ്ടം കൊടുക്കും}




\begin{slokam}{\VSr}{\Melp}{നിഷ്കമ്പേ നിത്യപൂർ‌ണ്ണേ}
നിഷ്കമ്പേ നിത്യപൂർ‌ണ്ണേ നിരവധിപരമാനന്ദപീയുഷരൂപേ\\
നിർല്ലീനാനേകമുക്താവലി സുഭഗതമേ നിർ‌മ്മലബ്രഹ്മസിന്ധൗ\\
കല്ലോലോല്ലാസതുല്യം ഖലു വിമലതരം സത്ത്വമാഹുഃസ്തദാത്മാ\\
കസ്മാന്നോ നിഷ്കളസ്ത്വം സകള ഇതി വചസ്ത്വത്കലാസ്വേവ ഭൂമൻ!
\end{slokam}

\Letter{ന}{ക}

\Book{നാരായണീയം} (1:4).


\end{enumerate}




\subsection{പ}

\begin{enumerate}


\begin{slokam}{\VSv}{\Amar}{പശ്യാമോ മയി കിം പ്രപദ്യത}
  പശ്യാമോ മയി കിം പ്രപദ്യത ഇതി സ്ഥൈര്യം മയാലംബിതം  \\
  കിം മാം നാലപതീത്യയം ഖലു ശഠഃ കോപസ്തയാപ്യാശ്രിതഃ  \\
  ഇത്യന്യോന്യവിലക്ഷദൃഷ്ടിചതുരേ തസ്മിന്നവസ്ഥാന്തരേ  \\
  സവ്യാജം ഹസിതം മയാ ധൃതിഹരോ മുക്തസ്തു ബാഷ്പസ്തയാ
\end{slokam}

\Letter{പ}{ഇ}

\Book{അമരുകശതകം}. പരിഭാഷ: \SlRef{എന്നോടിന്നിവളെന്തു ചെയ്യും}.



\begin{slokam}{\VVt}{\Unk}{പീതാംബരം കരവിരാജിത}
പീതാംബരം കരവിരാജിതശംഖചക്ര-\\
കൗമോദകീസരസിജം കരുണാസമുദ്രം\\
രാധാസഹായമതിസുന്ദരമന്ദഹാസം\\
വാതാലയേശമനിശം ഹൃദി ഭാവയാമി
\end{slokam}

\Letter{പ}{ര}


\begin{slokam}{\VHr}{\Amar}{പ്രഹരവിരതൌ മദ്ധ്യേ വാഹ്നസ്തതോ}
"പ്രഹരവിരതൌ മദ്ധ്യേ വാഹ്നസ്തതോ\prash{}പി പരേ\prash{}ഥവാ\\
കിമുത സകലേ യാതേ വാഹ്നി പ്രിയ! ത്വമിഹൈഷ്യസി"\\
ഇതി ദിനശതപ്രാപ്യം ദേശം പ്രിയസ്യ യിയാസതോ\\
ഹരതി ഗമനം ബാലാ വാക്യൈസ്സബാഷ്പഗളജ്ജലൈഃ
\end{slokam}

\Letter{പ}{ഇ}

\Book{അമരുകശതകം}. പരിഭാഷ: \SlRef{ചൊല്ലൂ, നിൻ വരവെപ്പൊൾ}.

\end{enumerate}

\subsection{ബ}

\begin{enumerate}

\begin{slokam}{\VSv}{\Unk}{ബാലാർക്കായുതതേജസം, ത്രിഭുവന}
ബാലാർക്കായുതതേജസം, ത്രിഭുവനപ്രക്ഷോഭകം, സുന്ദരം,\\
സുഗ്രീവാദിസമസ്തവാനരഗണൈസ്സംസേവ്യപാദാംബുജം,\\
നാദേനൈവ സമസ്തരാക്ഷസഗണാൻ സന്ത്രാസയന്തം, പ്രഭും,\\
ശ്രീമദ്രാമപദാംബുജസ്മൃതിരതം, ധ്യായാമി വാതാത്മജം.
\end{slokam}

\Letter{ബ}{ന}

\begin{slokam}{\VSv}{\Unk}{ബാലാർക്കായുതതേജസം ധൃതജടാ}
ബാലാർക്കായുതതേജസം ധൃതജടാജൂടേന്ദുഖണ്ഡോജ്വലം\\
നാഗേന്ദ്രൈഃ കൃതഭൂഷണം ജപപടീം ശൂലം കപാലം കരൈഃ\\
ഖട്വാംഗം ദധതം ത്രിനേത്രവിലസത്‌ പഞ്ചാനനം സുന്ദരം\\
വ്യാഘ്രത്വക്‌പരിധാനമബ്ജനിലയം ശ്രീനീലകണ്ഠം ഭജേ.
\end{slokam}

\Letter{ബ}{ഖ}

\begin{slokam}{\VSv}{\ONN}{ബ്രഹ്മാവിന്റെയുമന്തകന്റെയുമഹോ}
 ബ്രഹ്മാവിന്റെയുമന്തകന്റെയുമഹോ ഡിപ്പാർട്ടുമെന്റിൽക്കിട-\\
ന്നമ്മേ ഞാൻ തിരിയുന്നിതെത്ര യുഗമായ്‌, എന്നാണിതിൻ മോചനം?\\
ധർമ്മാധർമ്മ പരീക്ഷണത്തിനിനിമേൽ കാലന്റെ കച്ചേരിയിൽ\\
ചെമ്മേ ഹാജരെനിക്കിളച്ചു തരണേ! തദ്ദർശനം കർശനം!
\end{slokam}

\Letter{ബ}{ധ}



\begin{slokam}{\VSr}{\Unk}{ബ്രഹ്മാവിഷ്ണുർഗിരീശഃ}
ബ്രഹ്മാവിഷ്ണുർഗിരീശസ്സുരപതിരനലഃ പ്രേതരാഡ്യാതുനാഥ-\\
സ്തോയാധീശശ്ച വായുർധനദഗുഹഗണേശാർക്കചന്ദ്രാശ്ച രുദ്രാഃ\\
വിശ്വാദിത്യാശ്വിസാദ്ധ്യാഃ വസപിതൃമരുതസ്സിദ്ധവിദ്യാർത്ഥയക്ഷാഃ \\ 
ഗന്ധർവ്വാഃ കിന്നരാദ്യാഖിലഗഗനചരാഃ മംഗലം മേ ദിശന്തു.
\end{slokam}

\Letter{ബ}{വ}

\end{enumerate}

\subsection{ഭ}

\begin{enumerate}



\begin{slokam}{\VSr}{\Unk}{ഭിക്ഷാർത്ഥീ സ ക്വ യാതഃ}
"ഭിക്ഷാർത്ഥീ സ ക്വ യാതഃ, സുതനു?" -- "ബലിമഖേ"; "താണ്ഡവം ക്വാദ്യ ഭദ്രേ?"\\
"മന്യേ വൃന്ദാവനാന്തേ"; "ക്വ നു സ മൃഗശിശുർ?" -- "നൈവ ജാനേ വരാഹം";\\
"ബാലേ, കച്ചിന്ന ദൃഷ്ടോ ജരഠവൃഷപതിർ?" -- "ഗ്ഗോപ ഏവാത്ര വേത്താ"\\
ലീലാസല്ലാപ ഇത്ഥം ജലനിധിഹിമവത്കന്യയോസ്ത്രായതാം വഃ
\end{slokam}

\Letter{ഭ}{ബ}

\Topic{ഉമാരമാസംവാദം}.  പരിഭാഷ:  \SlRef{പിച്ചക്കാരൻ ഗമിച്ചാനെവിടെ}, \SlRef{തെണ്ടിച്ചാരെങ്ങു പോയ് പാർവ്വതി}.

\begin{slokam}{\VSv}{\KD}{ഭിക്ഷോ, മാംസനിഷേവണം}
 "ഭിക്ഷോ, മാംസനിഷേവണം കിമുചിതം?", "കിം തേന മദ്യം വിനാ?";\\
"മദ്യംചാപി തവപ്രിയം?", "പ്രിയമഹോ വാരാംഗനാഭിസ്സമം.";\\
"വാരസ്ത്രീ രതയേ കുതസ്തവധനം?", "ദ്യൂതേന ചൗര്യേണ വാ.";\\
"ചൗര്യദ്യൂതപരിശ്രമോസ്തി ഭവതഃ?", "ഭ്രഷ്ടസ്യ കാന്യാ ഗതി?"
\end{slokam}

\Letter{ഭ}{വ}

\begin{slokam}{\VSr}{\Melp}{ഭുഞ്ജാനാസ്സാകമേകാം}
 ഭുഞ്ജാനാസ്സാകമേകാ, മഗണിതഗുരവോ, ബ്രഹ്മഹന്തുസ്തനൂജാഃ,\\
മുണ്ഡാപൗത്രാശ്ച, രണ്ഡാജഠരസമുദിതാഃ, പണ്ഡിതാഃ പാണ്ഡുപുത്രാഃ\\
ഭ്രൂണഘ്ന്യാസ്സൂനു, മേനം ദ്വിജനകതനയം, ഭ്രാതരം പീതശീധോഃ,\\
കൃഷ്ണം യന്മാനനീയം ജഗൃഹുരിദമലം വർത്തതേ യുക്തരൂപം!
\end{slokam}

\Letter{ഭ}{ഭ}



\end{enumerate}

\subsection{മ}

\begin{enumerate}

\begin{slokam}{\VMl}{\RV}{മഥിതദനുജജാലം}
മഥിതദനുജജാലം, മംഗളാപ്രാണലോലം,\\
വിധൃതഡമരുശൂലം, വിത്തപാലാനുകൂലം,\\
സകലഭുവനമൂലം, സന്നതാശാധിപാലം,\\
ഭവവിതരണശീലം ഭാവയേ കാലകാലം
\end{slokam}

\Letter{മ}{സ}

\begin{slokam}{\VPv}{\KDBB}{മയൂഖനഖരത്രുട}
മയൂഖനഖരത്രുടത്തിമിരകുംഭികുംഭസ്ഥല-\\
സ്ഖലത്തരളതാരകാവലയകീർണ്ണമുക്താഫലാഃ\\
പുരന്ദരഹരിദ്ദരീകുഹരഗർഭസുപ്തോദ്ധിത-\\
സ്തുഷാരകരകേസരീ ഗഗനകാനനം ഗാഹതേ
\end{slokam}

\Letter{മ}{പ}



\end{enumerate}

\subsection{യ}

\begin{enumerate}

\begin{slokam}{\VSv}{\AD}{യത്ത്വന്നേത്രസമാനകാന്തി}
യത്ത്വന്നേത്രസമാനകാന്തി സലിലേ മഗ്നം തദിന്ദീവരം \\
മേഘൈരന്തയിതഃ പ്രിയേ തവ മുഖച്ഛായാനുകാരീ ശശീ \\
യേപി തദ്ഗമനാനുസാരി ഗതയസ്തേ രാജഹംസാ ഗതാഃ \\
തത്സാമീപ്യമീപ്യവിനോദമാത്രമയി തേ ദേവേന ന ക്ഷമ്യതേ
\end{slokam}

\Letter{യ}{യ}

\Book{കുവയാനന്ദം}.  പരിഭാഷകൾ: \SlRef{നിന്‍ നേത്രത്തിനു തുല്യമാം}, \SlRef{കണ്ണോടൊത്ത കറുത്ത താമരയിതാ}.


\begin{slokam}{\VSv}{\Melp}{യത്‌ത്രൈലോക്യമഹീയസോ∫പി}
യത്‌ത്രൈലോക്യമഹീയസോ∫പി മഹിതം സമ്മോഹനം മോഹനാത്‌\\
കാന്തം കാന്തിനിധാനതോ∫പി മധുരം മാധുര്യധുര്യാദപി\\
സൗന്ദര്യോത്തരതോ∫പി സുന്ദരതരം ത്വദ്രൂപമാശ്ചര്യതോ∫-\\
പ്യാശ്ചര്യം ഭുവനേ ന കസ്യ കുതുകം പുഷ്ണാതി വിഷ്ണോ വിഭോ.
\end{slokam}

\Letter{യ}{സ}

\Book{നാരായണീയം}. 

\begin{slokam}{\VSk}{\KD}{യദാലോകേ സൂക്ഷ്മം}
യദാലോകേ സൂക്ഷ്മം, വ്രജതി സഹസാ തദ്വിപുലതാം;\\
യദര്‍ദ്ധേ വിച്ഛിന്നം, ഭവതി കൃതസന്ധാനമിവ തത്‌;\\
പ്രകൃത്യാ യദ്വക്രം, തദപി സമരേഖം നയനയോര്‍;-\\
ന മേ ദൂരേ കിഞ്ചിത്‌ ക്ഷണമപി, ന പാര്‍ശ്വേ രഥജവാത്‌.
\end{slokam}

\Letter{യ}{പ}


\Book{അഭിജ്ഞാനശാകുന്തളം}. 


\begin{slokam}{\VSv}{\Unk}{യാ കുന്ദേന്ദുതുഷാരഹാരധവളാ}
യാ കുന്ദേന്ദുതുഷാരഹാരധവളാ യാ ശുഭ്രവസ്ത്രാവൃതാ\\
യാ വീണാവരദണ്ഡമണ്ഡിതകരാ യാ ശ്വേതപദ്മാസനാ\\
യാ ബ്രഹ്മാച്യുതശങ്കരപ്രഭൃതിഭിർദ്ദേവൈസ്സദാ പൂജിതാ\\
സാ മാം പാതു സരസ്വതീ ഭഗവതീ നിശ്ശേഷജാഡ്യാപഹാ.
\end{slokam}

\Letter{യ}{യ}

\begin{slokam}{\VSr}{\KD}{യാ സൃഷ്ടിഃ സ്രഷ്ടുരാദ്യാഃ}
യാ സൃഷ്ടിഃ സ്രഷ്ടുരാദ്യാഃ, വഹതി വിധിഹുതം യാ ഹവിര്‍, യാ ച ഹോത്രീ,\\
യേ ദ്വേ കാലം വിധത്തഃ, ശ്രുതിവിഷയഗുണാ യാ സ്ഥിതാ വ്യാപ്യവിശ്വം,\\
യാമാഹുഃ സര്‍വ്വഭൂതപ്രകൃതിരിതി, യയാ പ്രാണിനഃ പ്രാണവന്തഃ\\
പ്രത്യക്ഷാഭിഃ പ്രപന്നസ്തനുഭിരവതു വസ്താഭിരഷ്ടാഭിരീശഃ
\end{slokam}

\Letter{യ}{യ}


\Book{അഭിജ്ഞാനശാകുന്തളം}. നാന്ദീശ്ലോകം.

\begin{slokam}{\VSv}{\KD}{യാസ്യത്യദ്യ ശകുന്തളേതി}
യാസ്യത്യദ്യ ശകുന്തളേതി ഹൃദയം സംസ്പൃഷ്ടമുത്കണ്ഠയാ\\
കണ്ഠഃ സ്തംഭിതബാഷ്പവൃഷ്ടികലുഷ, ശ്ചിന്താജഡം ദർശനം \\
വൈക്ലബ്യം മമ താവദീദൃശമഹോ സ്നേഹാദരണ്യൗകസഃ \\
പീഡ്യന്തേ ഗൃഹിണഃ കഥം നു തനയാവിശ്ലേഷദുഃഖൈർനവൈഃ
\end{slokam}

\Letter{യ}{വ}


\Book{അഭിജ്ഞാനശാകുന്തളം}. പരിഭാഷകൾ: \SlRef{ഇന്നാകുന്നു ശകുന്തളാ},
\SlRef{പോകുണ്ടുന്നിതു നാൾ ശകുന്തള}.


\begin{slokam}{\VSr}{\Melp}{യോഗീന്ദ്രാണാം ത്വദംഗേഷ്വധിക}
യോഗീന്ദ്രാണാം ത്വദംഗേഷ്വധികസുമധുരം മുക്തിഭാജാം നിവാസോ\\
ഭക്താനാം കാമവർഷദ്യുതരുകിസലയം നാഥ തേ പാദമൂലം\\
നിത്യം ചിത്തസ്ഥിതം മേ പവനപുരപതേ കൃഷ്ണ കാരുണ്യസിന്ധോ\\
ഹൃത്വാ നിഃശേഷതാപാൻപ്രദിശതു പരമാനന്ദസന്ദോഹലക്ഷ്മീം
\end{slokam}

\Letter{യ}{ന}

\Book{നാരായണീയം}.



\end{enumerate}

\subsection{ര}

\begin{enumerate}

\begin{slokam}{\VSk}{\Unk}{രമാകാന്തം കാന്തം}
രമാകാന്തം കാന്തം ഭവഭവഭയാന്തം ഭവസുഖം\\
ദുരാശാന്തം ശാന്തം സകലഹൃദി ഭാന്തം ഭുവനപം\\
വിവാദാന്തം ദാന്തം ദനുജനിചയാന്തം സുചരിതം\\
സദാ തം ഗോവിന്ദം പരമസുഖകന്ദം ഭജത രേ!
\end{slokam}

\Letter{ര}{വ}

\Book{ഗോവിന്ദാഷ്ടകം}.

\end{enumerate}

\subsection{വ}

\begin{enumerate}

\begin{slokam}{\VSv}{\BH}{വിദ്യാ നാമ നരസ്യ രൂപമധികം}
 വിദ്യാ നാമ നരസ്യ രൂപമധികം പ്രച്ഛന്നഗുപ്തം ധനം\\
വിദ്യാ ഭോഗകരീ യശഃ സുഖകരീ വിദ്യാ ഗുരൂണം ഗുരുഃ\\
വിദ്യാ ബന്ധുജനോ വിടേശഗമനേ വിദ്യാ പരാ ദേവതാ\\
വിദ്യാ രാജസു പൂജ്യതേ ന തു ധനം വിദ്യാവിഹീനഃ പശുഃ
\end{slokam}

\Letter{വ}{വ}

\Book{നീതിശതകം}.

\begin{slokam}{\VDv}{\Melp}{വിവിധനര്‍മ്മഭിരേവം}
വിവിധനര്‍മ്മഭിരേവമഹര്‍ന്നിശം\\
പ്രമദമാകലയന്‍ പുനരേകദാ\\
ഋജുമതേഃ കില വക്രഗിരാ ഭവാന്‍\\
വരതനോരതനോരതിലോലതാം
\end{slokam}

\Letter{വ}{ഋ}

\Book{നാരായണീയം}. 
\Topic{യമകം (ദ്രുതവിളംബിതം)}. \NextSlRef{ഋതുവിലംഗജദീപനമാം}.


\begin{slokam}{\VIv}{\Unk}{വീടീകരാഗ്രാ വിരഹാതുരാ}
 വീടീകരാഗ്രാ വിരഹാതുരാ സാ\\
ചേടീമവാദീദിഹ -- ചിത്തജന്മാ\\
പ്രാണേശ്വരോ ജീവിതമർദ്ധരാത്രം\\
\sam{ആയാതി നായാതി ന യാതി യാതി}
\end{slokam}

\Letter{വ}{പ}

സമസ്യാപൂരണം.

\begin{slokam}{\VKm}{\Melp}{വേണുനാദകൃതതാന}
വേണുനാദകൃതതാനഗാനകളഗാനരാഗഗതിയോജനാ-\\
ലോഭനീയമൃദുപാദപാതകൃതതാളമേളനമനോഹരം\\
പാണിസംക്വണിതകങ്കണം ച മുഹുരംസലംബിതകരാംബുജം\\
ശ്രോണിബിംബചലദംബരം ഭജത രാസകേളിരസഡംബരം
\end{slokam}

\Letter{വ}{പ}

\Book{നാരായണീയം}. പരിഭാഷ: \SlRef{വേണുവിൻ ശ്രുതിയൊടൊത്തു}.


\end{enumerate}

\subsection{ശ}

\begin{enumerate}

\begin{slokam}{\VSv}{\KD}{ശുശ്രൂഷസ്വ ഗുരൂൻ}
ശുശ്രൂഷസ്വ ഗുരൂന്‍, കുരു പ്രിയസഖീവൃത്തിം സപത്നീജനേ\\
ഭര്‍ത്തുര്‍വിപ്രകൃതാപി രോഷണതയാ മാ സ്മ പ്രതീപം ഗമഃ\\
ഭൂയിഷ്ഠം ഭവ ദക്ഷിണാ പരിജനേ, ഭാഗ്യേഷ്വനുത്സേകിനീ,\\
യാന്ത്യേവം ഗൃഹിണീപദം യുവതയോ, വാമാ കുലസ്യാധയഃ
\end{slokam}

\Letter{ശ}{ഭ}

\Book{അഭിജ്ഞാനശാകുന്തളം}.  പരിഭാഷ: \SlRef{സേവിച്ചീടുക പൂജ്യരെ}.

\begin{slokam}{\VSr}{വേങ്കടാധ്വരി}{ശ്രീരാജീവാക്ഷവക്ഷസ്ഥല}
ശ്രീരാജീവാക്ഷവക്ഷസ്ഥലനിലയരമാഹസ്തവാസ്തവ്യലോല-\\
ല്ലീലാബ്ജാന്നിഷ്പതന്തീമധുരമധുരഝരീ നാഭിപദ്മേ മുരാരേഃ\\
അസ്തോകം ലോകമാത്രാദ്വിയുഗമുഖശിശോരാനനേഷ്വര്‍പ്യമാണം\\
ശംഖപ്രാന്തേനദിവ്യമ്പയ ഇതി വിബുധൈശ്ശങ്ക്യ മാനാപുനാതു.
\end{slokam}

\Letter{ശ}{അ}

\Book{വിശ്വഗുണാദർശചമ്പു}.  പരിഭാഷ: \SlRef{നാളീകാക്ഷന്റെ മാറത്തരുളിന}.

\end{enumerate}

\subsection{സ}

\begin{enumerate}


\begin{slokam}{\VMl}{\KD}{സരസിജമനുവിദ്ധം}
സരസിജമനുവിദ്ധം ശൈവലേനാപിരമ്യം\\
മലിനമപി ഹിമാംശോര്‍ലക്ഷ്മ ലക്ഷ്മീം തനോതി\\
ഇയമധികമനോജ്ഞാ വല്‌ക്കലേനാപി തന്വീ\\
കിമിവ ഹി മധുരാണാം മണ്ഡനം നാകൃതീനാം
\end{slokam}

\Letter{സ}{ഇ}

\Book{അഭിജ്ഞാനശാകുന്തളം}. പരിഭാഷ: \SlRef{ഫുല്ലാബ്ജത്തിനു രമ്യതക്കു}

\begin{slokam}{\VSr}{\Melp}{സാന്ദ്രാനന്ദാവബോധാത്മകമനുപമിതം}
സാന്ദ്രാനന്ദാവബോധാത്മകമനുപമിതം കാലദേശാവധിഭ്യാം\\
നിർമുക്തം നിത്യമുക്തം നിഗമശതസഹസ്രേണ നിർഭാസ്യമാനം\\
അസ്പഷ്ടം ദൃഷ്ടമാത്രേ പുനരുരു പുരുഷാർഥാത്മകം ബ്രഹ്മതത്ത്വം\\
തത്താവത്ഭാതി സക്ഷാദ്ഗുരുപവനപുരേ ഹന്ത ഭാഗ്യം ജനാനാം!
\end{slokam}

\Letter{സ}{അ}

\begin{slokam}{\VSv}{\Unk}{സിന്ദൂരാരുണവിഗ്രഹാം}
സിന്ദൂരാരുണവിഗ്രഹാം, ത്രിണയനാം, മാണിക്യമൗലീസ്ഫുരത്‌-\\
താരാനായകശേഖരാം, സ്മിതമുഖീ, മാപീനവക്ഷോരുഹാം,\\
പാണിഭ്യാമളിപൂർണരക്തചഷകം രക്തോൽപലം ബിഭ്രതീം,\\
സൗമ്യാം, രത്നഘടസ്ഥരക്തചരണാം, ധ്യായേത്‌ പരാമംബികാം
\end{slokam}

\Letter{സ}{പ}


\begin{slokam}{\VSv}{\Melp}{സൂര്യസ്പർദ്ധികിരീട, മൂർദ്ധ്വ}
സൂര്യസ്പർദ്ധികിരീട, മൂർദ്ധ്വതിലകപ്രോദ്‌ഭാസിഫാലാന്തരം\\
കാരുണ്യാകുലനേത്രമാർദ്രഹസിതോല്ലാസം സുനാസാപുടം\\
ഗണ്ഡോദ്യന്മകരാഭകുണ്ഡലയുഗം കണ്ഠോജ്ജ്വലത്കൗസ്തുഭം\\
ത്വദ്രൂ‍പം വനമാല്യഹാരപടല ശ്രീവത്സദീപ്രം ഭജേ.
\end{slokam}

\Letter{സ}{ഗ}

\Book{നാരായണീയം} (2:1).


\end{enumerate}

\subsection{ഹ}

\begin{enumerate}

\begin{slokam}{\VSr}{\Unk}{ഹത്വാ യുദ്ധേ ദശാസ്യം}
ഹത്വാ യുദ്ധേ ദശാസ്യം ത്രിഭുവനവിഷമം, വാമഹസ്തേന ചാപം\\
ഭൂമൗ വിഷ്ടഭ്യ തിഷ്ഠ, ന്നിതരകരധൃതം ഭ്രാമയൻ ബാണമേകം,\\
ആരക്തോപാന്തനേത്രഃ, ശരദളിതവപുഃ, കോടിസൂര്യപ്രകാശോ,\\
വീരശ്രീബന്ധുരാംഗ, സ്ത്രിദശപതിനുതഃ, പാതു മാം വീരരാമഃ
\end{slokam}

\Letter{ഹ}{അ}


\begin{slokam}{\VPc}{\RV}{ഹിമാദ്രിതുംഗശേഖരാം}
ഹിമാദ്രിതുംഗശേഖരാം, സമുദ്രഭംഗനൂപുരാം,\\
തമാലനീലവാസസാം, സുമാകരൈർസുവാസിതാം,\\
ഉമാ, ശചീ, സരസ്വതീ, രമാമുഖാംഗനാനുതാം\\
നമാമി ഭാരതാംബികാം തമോ\prash{}രി കോടി ഭാസ്വരാം.
\end{slokam}

\Letter{ഹ}{ഉ}

\end{enumerate}

