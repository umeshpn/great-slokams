\documentclass{article}
\usepackage[normalem]{ulem}
\usepackage{float}
\makeatletter
\DeclareRobustCommand{\em}{%
  \@nomath\em \color{gray}}
\makeatother


\usepackage[no-math]{fontspec}
\usepackage{polyglossia}

\usepackage{xcolor}

\usepackage{dingbat}

\usepackage{array}

\newenvironment{rslokam}{
  %% \color{blue}
  \begin{verse}
}{
  \end{verse}
}

\usepackage{imakeidx}
% \usepackage[columns=1]{idxlayout}


\indexsetup{level=\subsection*,toclevel=subsection}

%% Slokam Index
\makeindex[name=slindex,title=ശ്ലോകസൂചിക,columns=1,intoc,options={-s malayalam.ist}]

%% Vruththam Index
\makeindex[name=vrindex,title=വൃത്തസൂചിക,columns=1,intoc,options={-s malayalam.ist}]

%% Slokam Index
\makeindex[name=kaindex,title=കവിസൂചിക,columns=1,intoc,options={-s malayalam.ist}]

%% Slokam Index
\makeindex[name=viindex,title=വിഷയസൂചിക,columns=1,intoc,options={-s malayalam.ist}]

%% Slokam Index
\makeindex[name=saindex,title=സമസ്യാസൂചിക,columns=1,intoc,options={-s malayalam.ist}]

%% Slokam Index
\makeindex[name=akindex,title=അക്ഷരസൂചിക,columns=1,intoc,options={-s malayalam.ist}]

%% Slokam Index
\makeindex[name=krindex,title=കൃതിസൂചിക,columns=1,intoc,options={-s malayalam.ist}]

\usepackage{fdsymbol}
\newcommand{\La}{{$\cup$}}
\newcommand{\Gu}{{$\minus$}}
\newcommand{\Ya}{~{\color{red}\textbf{|}}~}
\newcommand{\Li}{~~~~{\color{red}\textbf{//}}~~~~}
\newcommand{\Yo}{{\color{blue}\textbf{|}}}
\newcommand{\Ivr}[1]{#1}


\newcommand{\prash}{$\int$}
\newcounter{slcount}
\setcounter{slcount}{0}
%% \newcommand{\beginning}{Start}

%% Meter, Poet, First part
\newenvironment{slokam}[3]{

  \gdef\beginning{#3}
  \gdef\thepoet{#2}
  \refstepcounter{slcount}\label{#3}
  \filbreak
  \item \textbf{ശ്ലോകം~\arabic{slcount}: (#2)} 
  \index[slindex]{#3...~(#2)}
  \index[vrindex]{#1!#3...}
  \index[kaindex]{#2!#3...}
  \begin{verse}
    \marginpar{\vspace{2.5em}\fbox{#1}}
    \bfseries
    \color{teal}
    %% \mfonts
    \small
  }{
  \end{verse}
}

%% 1 - 1 (2)
%% 2 - 3 (4)
%% 3 - 5 (6)
\newenvironment{IndexedSlokam}[6]{

  \gdef\beginning{#5}
  \gdef\thepoet{#3}
  \gdef\IndexedBeginning{#6}
  \gdef\IndexedPoet{#4}
  \refstepcounter{slcount}\label{#5}
  \filbreak
  \item \textbf{ശ്ലോകം~\arabic{slcount}: (#3)} 
  \index[slindex]{#6...~(#4)@#5...~(#3)}
  \index[vrindex]{#2@#1!#6@#5...}
  \index[kaindex]{#4@#3!#6@#5...}
  \begin{verse}
    \marginpar{\vspace{2.5em}\fbox{#1}}
    \bfseries
    \color{teal}
    %% \mfonts
    \small
  }{
  \end{verse}
}


\newcommand{\sam}[1]{\uline{#1}{\index[saindex]{#1!\beginning...}\index[viindex]{സമസ്യാപൂരണം!\beginning...}}}
\newcommand{\samd}[2]{\uline{#1}{\index[saindex]{#2!\beginning...}\index[viindex]{സമസ്യാപൂരണം!\beginning...}}}

\newcommand{\Topic}[1]{#1\index[viindex]{#1!\beginning...}}
\newcommand{\IndexedTopic}[2]{#1\index[viindex]{#2@#1!\IndexedBeginning@\beginning...}}
\newcommand{\Book}[1]{\textbf{കൃതി:} #1\index[krindex]{#1~(\thepoet)!\beginning...}}
\newcommand{\IndexedBook}[2]{\textbf{കൃതി:} #1\index[krindex]{#2~(\IndexedPoet)@#1~(\thepoet)!\IndexedBeginning@\beginning...}}

% \newcommand{\SlokRef}[1]{{\mfontl #1\ldots}~(പേജ് \pageref{#1})}
\newcommand{\SlRef}[1]{{\color{magenta} \mfontl #1\ldots}~(ശ്ലോകം
  \ref{#1})}
\newcommand{\SeeAlso}[1]{\leftpointright~\SlRef{#1}}
\newcommand{\OSlRef}[1]{\textbf{മൂലശ്ലോകം}: {\color{magenta} \mfontl #1\ldots}~(ശ്ലോകം \ref{#1})}
\newcommand{\PrevSlRef}[1]{{$\Leftarrow$ \color{magenta} \mfontl #1\ldots}~(ശ്ലോകം \ref{#1})}
\newcommand{\NextSlRef}[1]{{$\Rightarrow$ \color{magenta} \mfontl #1\ldots}~(ശ്ലോകം \ref{#1})}

\newcommand{\mfont}{Gayathri}
\newcommand{\sfont}{Gayathri}
\newcommand{\thinmfont}{Rachana}

\setdefaultlanguage{malayalam}
\setotherlanguages{english}
\setmainfont[Script=Malayalam, HyphenChar="00AD]{\mfont}
\newfontfamily{\malayalamfonttt}{Monaco}
%% \newfontfamily{\mfontl}[Script=Malayalam, HyphenChar="00AD, UprightFont={* Thin}]{\thinmfont}
\newfontfamily{\mfontl}[Script=Malayalam, HyphenChar="00AD]{\thinmfont}
\newfontfamily{\mfonts}[Script=Malayalam, HyphenChar="00AD]{\sfont}
\newcommand{\Comm}[1]{{\small\mfontl #1}}
\usepackage{fancyhdr}
\setlength{\headheight}{15.2pt}
\pagestyle{fancy}

\setlength{\parindent}{0pt}               %% First line of paragraph is not indented
\addtolength{\parskip}{0.5\baselineskip}  %% Increase vertical space between paragraphs
\usepackage{verse}

\newcommand{\Vruththam}[1]{#1}

\newcommand{\VAn}{\Vruththam{അനുഷ്ടുപ്പ്}}
\newcommand{\VAr}{\Vruththam{ആര്യ}}
\newcommand{\VAv}{\Vruththam{അപരവക്ത്രം}}
\newcommand{\VCm}{\Vruththam{ചമ്പകമാല}}
\newcommand{\VDv}{\Vruththam{ദ്രുതവിളംബിതം}}
\newcommand{\VBh}{\Vruththam{ഭുജംഗപ്രയാതം}}
\newcommand{\VGt}{\Vruththam{ഗീതി}}
\newcommand{\VHr}{\Vruththam{ഹരിണി}}
\newcommand{\VIv}{\Vruththam{ഇന്ദ്രവജ്ര}}
\newcommand{\VIvd}{\Vruththam{ഇന്ദുവദന}}
\newcommand{\VIvs}{\Vruththam{ഇന്ദ്രവംശ}}
\newcommand{\VKm}{\Vruththam{കുസുമമഞ്ജരി}}
\newcommand{\VMb}{\Vruththam{മഞ്ജുഭാഷിണി}}
\newcommand{\VMk}{\Vruththam{മന്ദാക്രാന്ത}}
\newcommand{\VMl}{\Vruththam{മാലിനി}}
\newcommand{\VMlk}{\Vruththam{മല്ലിക}}
\newcommand{\VMt}{\Vruththam{മത്തേഭം}}
\newcommand{\VPc}{\Vruththam{പഞ്ചചാമരം}}
\newcommand{\VPu}{\Vruththam{പുഷ്പിതാഗ്ര}}
\newcommand{\VPr}{\Vruththam{പ്രഹർഷിണി}}
\newcommand{\VPv}{\Vruththam{പൃത്ഥ്വി}}
\newcommand{\VRt}{\Vruththam{രഥോദ്ധത}}
\newcommand{\VSk}{\Vruththam{ശിഖരിണി}}
\newcommand{\VSl}{\Vruththam{ശാലിനി}}
\newcommand{\VSm}{\Vruththam{സമ്മത}}
%% \newcommand{\VSn}{\Vruththam{ശങ്കരനടനം}}
\newcommand{\VSr}{\Vruththam{സ്രഗ്ദ്ധര}}
\newcommand{\VSu}{\Vruththam{സുമുഖി}}
\newcommand{\VSv}{\Vruththam{ശാർദ്ദൂലവിക്രീഡിതം}}
\newcommand{\VSn}{\Vruththam{ശംഭുനടനം}}
\newcommand{\VSw}{\Vruththam{സ്വാഗത}}
\newcommand{\VTd}{\Vruththam{തോടകം}}
\newcommand{\VUj}{\Vruththam{ഉപജാതി}}
\newcommand{\VUv}{\Vruththam{ഉപേന്ദ്രവജ്ര}}
\newcommand{\VVs}{\Vruththam{വംശസ്ഥം}}
\newcommand{\VVm}{\Vruththam{വസന്തമാലിക}}
\newcommand{\VVt}{\Vruththam{വസന്തതിലകം}}
\newcommand{\VVy}{\Vruththam{വിയോഗിനി}}
\newcommand{\VOth}{\Vruththam{മറ്റുള്ളവ}}

\newcommand{\AD}{അപ്പയ്യദീക്ഷിതർ}
\newcommand{\Amar}{അമരുകൻ}
\newcommand{\ARRV}{ഏ. ആർ. രാജരാജവർമ്മ}
\newcommand{\ARSK}{ഏ. ആർ. ശ്രീകൃഷ്ണൻ}
\newcommand{\AR}{ആത്മാരാമൻ}
\newcommand{\AUK}{അരിയന്നൂർ ഉണ്ണിക്കൃഷ്ണൻ}
\newcommand{\Balendu}{ബാലേന്ദു}
\newcommand{\BB}{ഭവഭൂതി}
\newcommand{\BH}{ഭർത്തൃഹരി}
\newcommand{\CGN}{ചെറുവറ്റ ഗോവിന്ദൻ നമ്പൂതിരി}
\newcommand{\CKP}{ചങ്ങമ്പുഴ}
\newcommand{\CUV}{ചുനക്കര ഉണ്ണിക്കൃഷ്ണവാരിയർ}
\newcommand{\CN}{ചേലപ്പറമ്പു നമ്പൂതിരി}
\newcommand{\CVVB}{സി. വി. വാസുദേവഭട്ടതിരി}
\newcommand{\DSN}{ഡി. ശ്രീമാൻ നമ്പൂതിരി}
\newcommand{\EP}{ഏവൂർ പരമേശ്വരൻ}
\newcommand{\GRT}{ഗ്രാമത്തിൽ രാമവർമ്മത്തമ്പുരാൻ}
\newcommand{\GSK}{ജി. ശങ്കരക്കുറുപ്പ്}
\newcommand{\HM}{ഹരിദാസ് മംഗലപ്പിള്ളി} 
\newcommand{\KAM}{കാത്തുള്ളിൽ അച്യുതമേനോൻ}
\newcommand{\KA}{കുമാരനാശാൻ}
\newcommand{\KCKP}{കെ. സി. കേശവപിള്ള}
\newcommand{\KD}{കാളിദാസൻ}
\newcommand{\KDBB}{കാളിദാസൻ/ഭവഭൂതി}
\newcommand{\KJ}{കൈതയ്ക്കൽ ജാതവേദൻ}
\newcommand{\KKK}{കുട്ടമത്തു കുഞ്ഞമ്പുക്കുറുപ്പ്}
\newcommand{\KKNARRV}{കുറ്റിപ്പുറം/ഏ. ആർ.}
\newcommand{\KKN}{കുറ്റിപ്പുറത്തു കേശവൻ നായർ}
\newcommand{\KKT}{കൊടുങ്ങല്ലൂർ കുഞ്ഞിക്കുട്ടൻ തമ്പുരാൻ}
\newcommand{\KND}{കെ. എൻ. ഡി.}
\newcommand{\KKR}{കെ. എൻ. ഡി.}
\newcommand{\KN}{കുഞ്ചൻ നമ്പ്യാർ}
\newcommand{\KT}{കോട്ടയത്തു തമ്പുരാൻ}
\newcommand{\KVIT}{കൊടുങ്ങല്ലൂർ വിദ്വാൻ ഇളയതമ്പുരാൻ}
\newcommand{\KVPR}{കേരളവർമ്മ പഴശ്ശിരാജാ}
\newcommand{\KV}{കേരളവർമ്മ വലിയകോയിത്തമ്പുരാൻ}
\newcommand{\KochT}{കൊച്ചുണ്ണിത്തമ്പുരാൻ}
\newcommand{\KothJ}{കോതനല്ലൂർ ജോസഫ്}
\newcommand{\Kund}{കുണ്ടൂർ നാരായണമേനോൻ}
\newcommand{\LS}{ലീലാശുകൻ}
\newcommand{\MPN}{മഠം പരമേശ്വരൻ നമ്പൂതിരി}
\newcommand{\Mazha}{മഴമംഗലം}
\newcommand{\Melp}{മേൽപ്പത്തൂർ നാരായണഭട്ടതിരി}
\newcommand{\MR}{മാനവേദരാജാ}
\newcommand{\NDK}{എൻ. ഡി. കൃഷ്ണനുണ്ണി}
\newcommand{\NKD}{എൻ. കെ. ദേശം}
\newcommand{\NNM}{നാലാപ്പാട്ടു നാരായണമേനോൻ}
\newcommand{\NV}{എൻ. വി. കൃഷ്ണവാര്യർ}
\newcommand{\Naduv}{നടുവത്തച്ഛൻ}
\newcommand{\Nalankal}{നാലാങ്കൽ}
\newcommand{\OKM}{ഒടുവിൽ കുഞ്ഞിക്കൃഷ്ണമേനോൻ}
\newcommand{\ONN}{ഒറവങ്കര}
\newcommand{\Ottoor}{ഓട്ടൂർ}
\newcommand{\PCM}{പി. സി. മധുരാജ്}
\newcommand{\PG}{പ്രേംജി}
\newcommand{\PKV}{പന്തളം കേരളവർമ്മ}
\newcommand{\Poonthanam}{പൂന്താനം}
\newcommand{\PVRI}{പെട്ടരഴിയം വലിയ രാമനിളയത്‌}
\newcommand{\Poonth}{പൂന്തോട്ടത്തു നമ്പൂതിരി}
\newcommand{\Punam}{പുനം നമ്പൂതിരി}
\newcommand{\RN}{രമേശൻ നായർ}
\newcommand{\RV}{രാജേഷ് വർമ്മ}
\newcommand{\Rav}{രാവണൻ}
\newcommand{\SNG}{ശ്രീനാരായണഗുരു}
\newcommand{\SVL}{ശീവൊള്ളി}
\newcommand{\Sank}{ശങ്കരാചാര്യർ}
\newcommand{\TMV}{ടി. എം. വി.}
\newcommand{\UN}{ഉമേഷ് നരേന്ദ്രൻ}
\newcommand{\UV}{ഉണ്ണായി വാര്യർ}
\newcommand{\Ull}{ഉള്ളൂർ}
\newcommand{\Unk}{അജ്ഞാതം}
\newcommand{\VCBP}{വി. സി. ബാലകൃഷ്ണപ്പണിക്കർ}
\newcommand{\VKG}{വി. കെ. ജി.}
\newcommand{\VNM}{വള്ളത്തോൾ}
\newcommand{\VRV}{വയലാർ}
\newcommand{\VenA}{വെണ്മണി അച്ഛൻ}
\newcommand{\VenM}{വെണ്മണി മഹൻ}
\newcommand{\VenV}{വെണ്മണി വിഷ്ണു നമ്പൂതിരി}
\newcommand{\Vyl}{വൈലോപ്പിള്ളി}
\newcommand{\YK}{യൂസഫലി കേച്ചേരി}


\newcommand{\Letter}[2]{\index[akindex]{#1 $\Rightarrow$ #2!\beginning}}
\newcommand{\IndexedLetter}[4]{\index[akindex]{#2 $\Rightarrow$ #4@#1 $\Rightarrow$ #3!\IndexedBeginning@\beginning}}

\title{ശ്ലോകങ്ങള്‍}
\author{{\small സമ്പാദനം:} ഉമേഷ് നരേന്ദ്രന്‍}
\date{Last updated: \today}


  \usepackage{hyperref}
  \hypersetup{
    colorlinks=true,       % false: boxed links; true: colored links
    urlcolor=blue,          % color of external links
    colorlinks=true,       % false: boxed links; true: colored links
    linkcolor=blue,          % color of internal links
    citecolor=cyan,        % color of links to bibliography
    filecolor=black,      % color of file links
  }

\input{slokam-count}

\begin{document}

\maketitle

\clearpage

\section*{ആമുഖം}

\begin{verse}
    \bfseries
    \color{teal}
    \small
നാകമാക്കിടുവാൻ വിശ്വം \\
ശോകമൊക്കെ മറക്കുവാൻ \\
ലോകർക്കാനന്ദമേകീടാൻ \\
ശ്ലോകം ചൊല്ലാം നമുക്കെടോ!
\end{verse}


മലയാളത്തില്‍ ശ്ലോകസമാഹാരങ്ങള്‍ ധാരാളം പ്രസിദ്ധീകൃതമായിട്ടുണ്ട്. അവയിൽ പലതും
അച്ചടിയിലില്ലാത്തതിനാല്‍ അക്ഷരശ്ലോകപ്രേമികൾക്ക് പല ശ്ലോകങ്ങളും ലഭ്യമല്ല.  കർത്താവിനെയും കൃതിയെയും അറിയാത്ത ധാരാളം ശ്ലോകങ്ങളും പ്രചാരത്തിലുണ്ട്.  

കിട്ടാവുന്ന നല്ല ശ്ലോകങ്ങളുടെ ഒരു സമാഹാരം ഉണ്ടാക്കുക എന്ന ആഗ്രഹത്തിന്റെ സാഫല്യമാണ് ഈ
പുസ്തകം.  ഇതിനു പുറകില്‍ താഴെപ്പറയുന്ന ലക്ഷ്യങ്ങളാണ് ഉള്ളത്.

\begin{enumerate}
\item ആധുനികസാങ്കേതികവിദ്യ ഉപയോഗിച്ച് ഏറ്റവും ഭംഗിയായി തെറ്റു തിരുത്തി ശ്ലോകങ്ങളെ
  അവതരിപ്പിക്കുക. ടൈപ്സെറ്റിഗിംന് \LaTeX/XeTeX{}, സ്വതന്ത്ര മലയാളം കമ്പ്യൂട്ടിംഗ്
  (SMC) ഉണ്ടാക്കിയ മലയാളം ഫോണ്ടുകള്‍ എന്നിവയാണ്  ഇതില്‍
  ഉപയോഗിച്ചിരിക്കുന്നത്. 
\item എല്ലാവര്‍ക്കും സൗജന്യമായി ഇന്റര്‍നെറ്റില്‍ ലഭ്യമാക്കുക.  ഈ പി. ഡി. എഫും അതിന്റെ
  കൂടെയുള്ള വെബ് പേജും പൂര്‍ണ്ണമായും സൗജന്യമാണ്.
\item അടിയ്ക്കടി തെറ്റു തിരുത്തി നവീകരിക്കുകയും കൂടുതല്‍ ശ്ലോകങ്ങള്‍ ചേര്‍ക്കുകയും ചെയ്യുക.
  മാസത്തില്‍ ഒരിക്കല്‍ പുതിയ അപ്‌ഡേറ്റ് പ്രസിദ്ധീകരിക്കാനാണ് ഉദ്ദേശിക്കുന്നത്.
\item കമ്പ്യൂട്ടറില്‍ വായിക്കുകയോ അച്ചടിച്ച് ഉപയോഗിക്കുകയോ ചെയ്യാവുന്ന രീതിയിലുള്ള
  രൂപകല്പന.  പി. ഡി. എഫ്. ഫോര്‍മാറ്റ് അതിനു യോജിച്ചതാണ്.  കമ്പ്യൂട്ടറില്‍ വായിക്കാനായി
  ക്ലിക്കബിള്‍ ക്രോസ്-റെഫറന്‍സുകള്‍ ഉണ്ട്.  പ്രിന്റിനു വേണ്ടി പേജ് നമ്പരുകളും. 
\item വിശദമായ ഇന്‍ഡക്സും ക്രോസ് റെഫറന്‍സുകളും സ്റ്റാറ്റിസ്റ്റിക്സും.  ഒരു കമ്പ്യൂട്ടര്‍ പ്രോഗ്രാമിന്റെ
  സഹായത്തോടെ തയ്യാറാക്കുന്നതിനാല്‍ ഓരോ അപ്‌ഡേറ്റിലും ഇതു കൃത്യമായി ചേര്‍ക്കാന്‍
  കഴിയും. ``സ്റ്റാറ്റിസ്റ്റിക്സും സൂചികയും'' (പേജ് \pageref{sec:Statistics}) എന്ന
  അദ്ധ്യായത്തില്‍ ഇവയുടെ വിശദവിവരങ്ങള്‍ ഉണ്ട്. 
\item അക്ഷരശ്ലോകപ്രേമികള്‍ക്കുതകുന്ന മറ്റു വിവരങ്ങള്‍ ചേര്‍ക്കുക.  ഉദാഹരണമായി, ഒരു
  ശ്ലോകത്തിനു മുമ്പും പിമ്പും ചൊല്ലാന്‍ പറ്റിയ അതേ ശൈലിയിലുള്ള മറ്റു ശ്ലോകങ്ങളെ
  $\Leftarrow$, $\Rightarrow$ എന്ന ചിഹങ്ങള്‍ വഴി കാണിച്ചിരിക്കുന്നു.  അതു പോലെ,
  പരിഭാഷകളുടെ മൂലശ്ലോകങ്ങളിലേക്കും, അതേ ആശയമുള്ള മറ്റു ശ്ലോകങ്ങളിലേക്കും
  ക്രോസ്-റെഫറന്‍സുകള്‍ ഉണ്ട്. 
\end{enumerate}

മലയാളം, സംസ്കൃതം എന്നു രണ്ടു വിഭാഗങ്ങളായി തിരിച്ചിരിക്കുന്നു.   അക്ഷരശ്ലോകത്തിനു ചൊല്ലാറില്ലാത്ത
അനുഷ്ടുപ്പു പോലെയുള്ള വൃത്തങ്ങളിലുള്ള ശ്ലോകങ്ങളെയും, കാവ്യങ്ങളിലെയും മറ്റും
മുക്തകസ്വഭാവമില്ലാത്ത ശ്ലോകങ്ങളെയും നാല്‍ക്കാലികളെയും കഴിയുന്നത്ര ഒഴിവാക്കിയിട്ടുണ്ട്. 

ഈ പുസ്തകം
\url{https://bit.ly/slokam-pdf}
എന്ന ലിങ്കില്‍ നിന്നു ഡൗണ്‍ലോഡ് ചെയ്യാന്‍ സാധിക്കും.  സേര്‍ച്ചു ചെയ്യാന്‍ സഹായകമായി ഇതിലെ എല്ലാ ശ്ലോകങ്ങളും ഒരു
പേജിലായി
\url{https://bit.ly/slokam-web}
എന്ന ലിങ്കില്‍ ലഭ്യമാണ്.

ഈ പുസ്തകത്തിലെ പിശകുകള്‍, ഇതു നന്നാക്കാനുള്ള നിര്‍ദ്ദേശങ്ങള്‍, ഇതില്‍ ചേര്‍ക്കാനുള്ള ശ്ലോകങ്ങള്‍
ഇവ \texttt{umesh.p.narendran@gmail.com} എന്ന വിലാസത്തില്‍ അയയ്ക്കുക.  ഇതിലെ
ഏതെങ്കിലും ശ്ലോകങ്ങള്‍ക്ക് കോപ്പിറൈറ്റ് ഉള്ളവര്‍ അവ ഒഴിവാക്കാന്‍ ആവശ്യപ്പെട്ടാല്‍ അവ
ഒഴിവാക്കുന്നതാണ്. 

അക്ഷരശ്ലോകപ്രേമികള്‍ക്ക് ഈ പുസ്തകം വളരെ ഉപകാരപ്രദമായിരിക്കും എന്നു പ്രതീക്ഷിക്കുന്നു.

\begin{flushright}
\textbf{ഉമേഷ് നരേന്ദ്രന്‍}\\
2021 ഏപ്രില്‍. 
\end{flushright}


\clearpage

\tableofcontents

\clearpage
\section{ക്രമീകരണം}
ശ്ലോകങ്ങൾ: മലയാളം/മണിപ്രവാളം (പേജ്~\pageref{sec:slokams:mal}), ശ്ലോകങ്ങൾ: സംസ്കൃതം (പേജ്~\pageref{sec:slokams:san}) എന്ന
അദ്ധ്യായങ്ങളിലായി അകാരാദിക്രമത്തിൽ \NTotalSlokams{} ഇതിൽ
ക്രോഡീകരിച്ചിരിക്കുന്നു. ഓരോ ശ്ലോകത്തിന്റീയും കൂടെ അതിന്റെ വൃത്തം, കവി (ലഭ്യമെങ്കിൽ),
കൃതി (ലഭ്യമെങ്കിൽ) എന്നിവ സൂചിപ്പിച്ചിരിക്കുന്നു.

പേജ് \pageref{sec:index} തൊട്ട് ശ്ലോകങ്ങൾ, കവികൾ (ഓരോ കവിയുടെയും ശ്ലോകങ്ങളും) വൃത്തങ്ങൾ (ഓരോ വൃത്തത്തിലെയും
ശ്ലോകങ്ങളും), കൃതികൾ (ലഭ്യമായവ), വിഷയം, സമസ്യാപൂരണങ്ങളായ ശ്ലോകങ്ങളുടെ സമസ്യകൾ,
ഒന്നാം വരിയിലെയും മൂന്നാം വരിയിലെയും അക്ഷരങ്ങൾ എന്നിവയുടെ സൂചികകൾ ചേർത്തിരിക്കുന്നു.
ഓരോന്നിന്റെയും താഴെ അതാതിന്റെ ശ്ലോകങ്ങളും അവയിലേക്കുള്ള പേജ് നമ്പരും ലിങ്കും
കൊടുത്തിട്ടുണ്ട്. 

സൂചികയ്ക്ക് മലയാളത്തിൽ സാധാരണയായ അകാരാദിക്രമം ദീക്ഷിച്ചിരിക്കുന്നു.  താഴെപ്പറയുന്നവ
ശ്രദ്ധിക്കുക.

\begin{enumerate}
\item വർഗ്ഗാക്ഷരങ്ങൾക്കു മുമ്പുള്ള അനുസ്വാരം ആ വർഗ്ഗത്തിലെ അനുനാസികമായി കണക്കാക്കുന്നു.  ഉദാ:
ഗംഗ (ഗങ്ഗ) എന്നത് ഗഗനം, ഗജം എന്നിവയ്ക്ക് ഇടയിൽ വരുന്നു. 

\item ബാക്കിയുള്ള അനുസ്വാരങ്ങളെ മ് ആയി കണക്കാക്കുന്നു.  ഉദാ: സംരംഭം എന്നത് സമ്യക്, സമ്രാട്ട്
എന്നിവയ്ക്കു വരുന്നു.

\item വിസർഗ്ഗം ഹ് എന്നതിനു ശേഷം വരുന്നു. 

\item റ്റ-യെ റ-യുടെ കൂട്ടക്ഷരമായി കണക്കാക്കിയിരിക്കുന്നു. 
\end{enumerate}

ചില ശ്ലോകങ്ങളുടെ കൂടെ താഴെപ്പറയുന്ന ചിഹ്നങ്ങൾ ചേർത്തിരിക്കുന്നു.

\begin{enumerate}
\item [\leftpointright]: ഇതിനോടു സാദൃശ്യമുള്ള മറ്റൊരു ശ്ലോകം. 
\item [$\Leftarrow$]: അക്ഷരശ്ലോകത്തിൽ ഇതിനു മുമ്പു വരാവുന്ന ഇതേ രീതിയിലുള്ള
  ശ്ലോകം. 
\item [$\Rightarrow$]: അക്ഷരശ്ലോകത്തിൽ ഇതിനു ശേഷം വരാവുന്ന ഇതേ രീതിയിലുള്ള
  ശ്ലോകം. 
\end{enumerate}

കൂടാതെ, പ്രസിദ്ധസമസ്യകളെ \uline{അടിവരയിട്ടു} സൂചിപ്പിച്ചിരിക്കുന്നു.

\clearpage
\input{great-slokams-mal-tex}

\clearpage
\input{great-slokams-san-tex}

\clearpage

\section{വൃത്തങ്ങൾ}
{
\renewcommand{\arraystretch}{1.2}
\begin{tabular}{|l|rc|c|c|}
\hline
\textbf{വൃത്തം} & \multicolumn{2}{c|}{\textbf{അക്ഷരം}} & \multicolumn{2}{c|}{\textbf{ലക്ഷണം}}  \\
\hline
\Ivr{ശംഭുനടനം} & 26 & & \La\Gu\La\La\La\Gu\La\La\La\Gu\La\La\La\Gu\La\La\La\Gu\La\La\La\Gu\La\La\La\Gu & ജസനഭജസനഭ ലഗു\\

\Ivr{മത്തേഭം} & 22 &  & \Gu\Gu\La\Gu\La\La\La\Yo\Gu\Gu\La\Gu\La\La\La\Yo\Gu\Gu\La\Gu\La\La\La\Gu & തഭയജസരന ഗു \\

\Ivr{കുസുമമഞ്ജരി} & 21 & & \Gu\La\Gu\La\La\La\Yo\Gu\La\Gu\La\La\La\Yo\Gu\La\Gu\La\La\La\Yo\Gu\La\Gu & രനരനരനര \\

\Ivr{സ്രഗ്ദ്ധര} & 21 & (7/7/7) & \Gu\Gu\Gu\Gu\La\Gu\La\Ya\La\La\La\La\La\La\Gu\Ya\Gu\La\Gu\Gu\La\Gu\Gu  & മരഭനയയയ \\

\Ivr{ശാർദ്ദൂലവിക്രീഡിതം} & 19 & (12/7) & \Gu\Gu\Gu\La\La\Gu\La\Gu\La\La\La\Gu\Ya\Gu\Gu\La\Gu\Gu\La\Gu  & മസജസതത ഗു\\

\Ivr{പൃത്ഥ്വി} & 17 & (8/9) & \La\Gu\La\La\La\Gu\La\Gu\Ya\La\La\La\Gu\La\Gu\Gu\La\Gu &  ജസജസയ ലഗു\\

\Ivr{മന്ദാക്രാന്ത} & 17 & (4/6/7) & \Gu\Gu\Gu\Gu\Ya\La\La\La\La\La\Gu\Ya\Gu\La\Gu\Gu\La\Gu\Gu  & മഭനതത ഗുഗു\\

\Ivr{ശിഖരിണി} & 17 & (6/11) & \La\Gu\Gu\Gu\Gu\Gu\Ya\La\La\La\La\La\Gu\Yo\Gu\La\La\Gu  & യമനസഭ ലഗു\\

\Ivr{ഹരിണി} & 17 & (6/4/7) & \La\La\La\La\La\Gu\Ya\Gu\Gu\Gu\Gu\Ya\La\Gu\La\La\Gu\La\Gu & നസമരസ ലഗു \\

\Ivr{പഞ്ചചാമരം} & 16 & & \La\Gu\La\Gu\La\Gu\La\Gu\La\Gu\La\Gu\La\Gu\La\Gu  & ജരജരജ ഗ\\

\Ivr{മാലിനി} & 15 & (8/7) & \La\La\La\La\La\La\Gu\Gu\Ya\Gu\La\Gu\Gu\La\Gu\Gu  & നനമയയ\\

\Ivr{വസന്തതിലകം} & 14 & & \Gu\Gu\La\Gu\La\La\La\Gu\La\La\Gu\La\Gu\Gu  & തഭജജ ഗുഗു\\

%% \Ivr{ഇന്ദുവദന} & 14 & \Gu\La\La\La\Gu\La\La\La\Gu\La\La\Gu\Gu  \\

%% \Ivr{പ്രഹർഷിണി} & 13 & \Gu\Gu\Gu\Ya\La\La\La\La\Gu\La\Gu\La\Gu\Gu  \\

\Ivr{മഞ്ജുഭാഷിണി} & 13 & & \La\La\Gu\La\Gu\La\La\La\Gu\La\Gu\La\Gu\Gu  & സജസജ ഗു\\

%% \Ivr{വംശസ്ഥം} & 12 & \La\Gu\La\Gu\Gu\La\La\Gu\La\Gu\La\Gu  \\


\Ivr{തോടകം} & 12 & & \La\La\Gu\La\La\Gu\La\La\Gu\La\La\Gu & സസസസ\\

\Ivr{ദ്രുതവിളംബിതം} & 12 & & \La\La\La\Gu\La\La\Gu\La\La\Gu\La\Gu  & നഭഭര\\

\Ivr{ഭുജംഗപ്രയാതം} & 12 & & \La\Gu\Gu\La\Gu\Gu\La\Gu\Gu\La\Gu\Gu  & യയയയ\\

\Ivr{ഇന്ദ്രവജ്ര} & 11 & & \Gu\Gu\La\Gu\Gu\La\La\Gu\La\Gu\Gu  & തതജ ഗുഗു\\

\Ivr{ഉപേന്ദ്രവജ്ര} & 11 & & \La\Gu\La\Gu\Gu\La\La\Gu\La\Gu\Gu  & ജതജ ഗുഗു\\

\Ivr{രഥോദ്ധത} & 11 & & \Gu\La\Gu\La\La\La\Gu\La\Gu\La\Gu  & രനര ലഗു\\

%% \Ivr{സമ്മത} & 11 & \La\La\La\Gu\La\Gu\Yo\Gu\La\Gu\La\Gu  \\

%% \Ivr{ചമ്പകമാല} & 10 & \Gu\La\La\Gu\Gu\Gu\La\La\Gu\Gu  \\

%% \hline

%% \Ivr{പുഷ്പിതാഗ്ര} & 12/13 & \La\La\La\La\La\La\Gu\La\Gu\La\Gu\Gu \Li \La\La\La\La\Gu\La\La\Gu\La\Gu\La\Gu\Gu  \\

%% \Ivr{വസന്തമാലിക} & 11/12 & \La\La\Gu\La\La\Gu\La\Gu\La\Gu\Gu \Li \La\La\Gu\Gu\La\La\Gu\La\Gu\La\Gu\Gu  \\

%% \Ivr{വിയോഗിനി} & 10/11 & \La\La\Gu\La\La\Gu\La\Gu\La\Gu \Li \La\La\Gu\Gu\La\La\Gu\La\Gu\La\Gu  \\


\hline
\end{tabular}
}


\clearpage
\section{സ്റ്റാറ്റിസ്റ്റിക്സ്}
\label{sec:Statistics}

%  ഈ പുസ്തകത്തിലെ
% ശ്ലോകങ്ങളുടെ പൊതുസ്വഭാവം മനസ്സിലാക്കാന്‍ ഉതകുന്ന സ്റ്റാറ്റിസ്റ്റിക്സ് ആണ് ഈ അദ്ധ്യായത്തില്‍.
% വിശദമായ സൂചിക പേജ് \pageref{sec:index} മുതല്‍ തുടങ്ങുന്നു.  

% പട്ടിക 1-ല്‍ ഇതിലെ ശ്ലോകങ്ങളുടെ വൃത്തങ്ങള്‍ അവയുടെ എണ്ണത്തിന്റെ ക്രമത്തില്‍
% ചേര്‍ത്തിരിക്കുന്നു. 70 ശതമാനത്തില്‍ അധികം ശ്ലോകങ്ങളും ശാര്‍ദ്ദൂലവിക്രീഡിതം, സ്രഗ്ദ്ധര എന്നീ
% വൃത്തങ്ങളിലാണ് എന്നു കാണാം.  ഓരോ വൃത്തവും അതിലെ ശ്ലോകങ്ങളും അകാരാദിക്രമത്തില്‍ കാണാന്‍
% സൂചികയിലെ ``(B) വൃത്തങ്ങള്‍'' എന്ന ഭാഗം നോക്കുക. 

% പട്ടിക 2, 3, 4 എന്നിവയില്‍ ഇതിലെ ശ്ലോകങ്ങളുടെ രചയിതാക്കളെ അവരുടെ ശ്ലോകങ്ങളുടെ
% എണ്ണത്തിന്റെ ക്രമത്തില്‍ കൊടുത്തിരിക്കുന്നു.  ഓരോ കവിയെയും ശ്ലോകങ്ങളും 
% അകാരാദിക്രമത്തില്‍ കാണാന്‍ സൂചികയിലെ ``(C) കവികള്‍'' എന്ന ഭാഗം നോക്കുക. 

% പട്ടിക 5-ല്‍ ഒന്നാം വരിയിലും മൂന്നാം വരിയിലും വരുന്ന ആദ്യാക്ഷരങ്ങള്‍ അവയുടെ എണ്ണത്തിന്റെ
% ക്രമത്തില്‍ കൊടുത്തിരിക്കുന്നു.  ഏതൊക്കെ അക്ഷരങ്ങളാണു കൂടുതല്‍ പഠിക്കേണ്ടത് എന്നതു
% തീരുമാനിക്കാന്‍ ഇതു സഹായിക്കും. 

% 6 മുതല്‍ 11 വരെയുള്ള പട്ടികകളില്‍ ഒന്നാം വരിയിലെയും മൂന്നാം വരിയിലെയും അക്ഷരങ്ങളുടെ
% യോഗം അവയുടെ എണ്ണത്തിന്റെ ക്രമത്തില്‍ കൊടുത്തിരിക്കുന്നു.  ഇതിലെ ഓരോന്നിലുമുള്ള ശ്ലോകങ്ങള്‍
% അകാരാദിക്രമത്തില്‍ കാണാന്‍ സൂചികയിലെ ``(G) അക്ഷരത്തുടര്‍ച്ച'' എന്ന ഭാഗം നോക്കുക. 

% ഇവയെക്കൂടാതെ, സൂചികയില്‍ എല്ലാ ശ്ലോകങ്ങളും അകാരാദിക്രമത്തില്‍ (``(A) ശ്ലോകങ്ങള്‍'' എന്ന
% ഭാഗം), കൃതികളും അവയിലെ ശ്ലോകങ്ങളും അകാരാദിക്രമത്തില്‍ (``(D) കൃതികള്‍'' എന്ന ഭാഗം),
% വിഷയങ്ങളും അവ അടങ്ങിയ ശ്ലോകങ്ങളും (``(E) വിഷയങ്ങള്‍'' എന്ന ഭാഗം), സമസ്യകള്‍
% (``(F) സമസ്യാപൂരണങ്ങള്‍'' എന്ന ഭാഗം)  എന്നിവയും സൂചികയില്‍ ഉണ്ട്. 
\newcommand{\aksharam}{അക്ഷരം}
\newcommand{\kramam}{ക്രമം}
\newcommand{\ennam}{എണ്ണം}
\newcommand{\FirstLine}{ഒന്നാം വരി}
\newcommand{\ThirdLine}{മൂന്നാം വരി}
\newcommand{\Letters}{അക്ഷരങ്ങള്‍}
\newcommand{\aksharakramam}{അക്ഷരക്രമം}
\newcommand{\vruththam}{വൃത്തം}
\newcommand{\vruththams}{വൃത്തങ്ങള്‍}
\newcommand{\kavi}{കവി}
\newcommand{\kavis}{കവികള്‍}
\newcommand{\TotalSlokams}{മൊത്തം ശ്ലോകങ്ങള്‍}
\newcommand{\MalSlokams}{(മലയാളം)}
\newcommand{\SanSlokams}{(സംസ്കൃതം)}
\newcommand{\TotalCount}{മൊത്തം}

\input{stats}


\clearpage
\label{sec:index}
\section{സൂചികകൾ}
\printindex[slindex]
\printindex[kaindex]
\printindex[vrindex]
\printindex[krindex]
\printindex[viindex]
\printindex[saindex]
\printindex[akindex]



\end{document}
